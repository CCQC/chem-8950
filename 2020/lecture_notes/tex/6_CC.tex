\documentclass{article}
\usepackage{amsmath,mathtools,amssymb}
\usepackage{graphicx}
\usepackage{booktabs}
\usepackage{blkarray}
\usepackage{gensymb}
\usepackage{verbatim}
\usepackage{mathrsfs}
\usepackage{bbm}
\usepackage{braket}
\usepackage{hyperref}
\usepackage{verbatim}
\usepackage{cancel}
\usepackage{xcolor}
\usepackage[margin=1.0in]{geometry}
\newcommand{\ol}{\overline}
\newcommand{\lp}{\left(}
\newcommand{\rp}{\right)}
\newcommand{\eps}{\varepsilon}
\newcommand{\lam}{\lambda}
\newcommand{\h}{\circ}
\newcommand{\p}{\bullet}

\newcommand{\Ezero}{E^{(0)}}
\newcommand{\Rz}{\mathcal{R}_{0}}

\newcommand{\Phizero}{\Phi^{(0)}}
\newcommand{\Eone}{E^{(1)}}
\newcommand{\En}{E^{(n)}}
\newcommand{\Phione}{\Phi^{(1)}}
\newcommand{\Phin}{\Phi^{(n)}}

\newcommand{\Ecorr}{E_{\mathrm{corr}}}
\newcommand{\Ec}{E_{\mathrm{c}}}
\newcommand{\Hc}{H_{\mathrm{c}}}
\newcommand{\dg}{\ensuremath{^\dagger} }
\def\*#1{\mathbf{#1}}
\DeclarePairedDelimiter\floor{\lfloor}{\rfloor}

\title{Lecture 6: Coupled Cluster Theory}
\date{April 6, 2020}
\begin{document}
\maketitle
\noindent

\section{Quantum Chemistry Wavefunctions}
In the last set of notes, we introduced the concept of a wave operator $\Omega$, which transforms 
    our reference determinant $\Phi$ into the exact wavefunction $\Psi$
\[\ket{\Psi} = \Omega \ket{\Phi} \]
Different valid choices of the wave operator  $\Omega$ constitute different \textit{ansatz},  
    which each arrive at valid form for the same exact wavefunction $\ket{\Psi}$.
In perturbation theory (PT), our wave operator is of the form,
$\Omega = P + \sum_{n=0}^{\infty} (\Rz V')^n \Rz V P$, so that our wavefunction is
\[\ket{\Psi_{\mathrm{PT}}} =  \ket{\Phi_0} + \sum_{n=0}^{\infty} (\Rz V')^n \Rz V \ket{\Phi} \]
In configuration interaction (CI), our wave operator (though we did not call it that at the time)
takes the form $\Omega = (1 + C_1 + C_2 + \cdots + C_n)$ so that 
\[\ket{\Psi_{\mathrm{CI}}} = (1 + C_1 + C_2 + \cdots + C_n) \ket{\Phi} \]
In coupled cluster theory (CC), our wave operator, and therefore our wavefunction, 
is an exponential form,
\[\ket{\Psi_{\mathrm{CC}}} = \exp(T_1 + T_2 + \cdots + T_n) \ket{\Phi} \]
Or simply, 
\[\ket{\Psi_{\mathrm{CC}}} = \exp(T) \ket{\Phi} \]

Each of the wavefunctions, in their limit, approach the exact same wavefunction.
However, perturbation theory is an odd duck, since it always has an infinite number of
    contributions to the wavefunction and energy, 
    whereas the CC and CI wavefunctions for a given $n$-electron system 
    terminate at $n$ terms (you can't have a quadruple excitation in a 3-electron system). 

\section{Coupled Cluster Theory}
Our coupled cluster wavefunction from before is 
\[\ket{\Psi_{\mathrm{CC}}} = \exp(T) \ket{\Phi} \]
\[\ket{\Psi_{\mathrm{CC}}} = \exp(T_1 + T_2 + \cdots + T_n) \ket{\Phi} \]
where $T$ is the ``cluster operator'' which is a sum over
$k$-electron excitation operators
\[T = T_1 + T_2 + T_3 + \cdots \]
\[T_k = \left(\frac{1}{k!}\right)^2 t_{a_1 \cdots a_k}^{i_1 \cdots i_k } \tilde{a}_{i_1 \cdots i_k}^{a_1 \cdots a_k } \]
where $t_{a_1 \cdots a_k}^{i_1 \cdots i_k }$ are the coupled cluster \textit{amplitudes}.
Solving for the coupled cluster energy is done by doing the same thing we always do:
project our reference determinant $\bra{\Phi}$ on the left of both sides 
of the Schr{\"o}dinger equation.
Assuming intermediate normalization, we obtain
\[\braket{\Phi | \Hc | \Psi_{\mathrm{CC}}} = E_c \braket{\Phi|\Psi_{\mathrm{CC}}} \]
\[\braket{\Phi | \Hc \exp(T) | \Phi} = \Ec \]

Equations for the cluster amplitudes $t_{a b \cdots }^{i j \cdots}$
can be found by left-projecting the Schr{\"o}dinger equation by excited determinants
\[ \braket{ \Phi_{i j \cdots }^{a b \cdots} | \Hc \exp(T) | \Phi} = 
   \Ec \braket{\Phi_{i j \cdots }^{a b \cdots} | \exp(T) | \Phi} \]
where an equation involving $t_i^a$ would be obtained by projecting by $\bra{\Phi_{i}^{a}}$,
an equation for $t_{ij}^{ab}$ by projecting by $\bra{\Phi_{ij}^{ab}}$, and so on. 
To see this, one can take the right side of the above equation and expand the exponential $\exp(T) = 1 + T + \frac{T^2}{2!} + \cdots $, and only one term survives:
\[ \Ec \braket{\Phi_{i j \cdots }^{a b \cdots} | \exp(T) | \Phi} = \Ec t_{a b \cdots }^{i j \cdots}\]

Plugging in the exponential expansion of $\exp(T)$ into the energy and amplitude equations 
yields 
\[ \braket{\Phi_{i j \cdots }^{a b \cdots} | \Hc | \Phi} + \braket{\Phi_{i j \cdots }^{a b \cdots}| \Hc T| \Phi} + \braket{\Phi_{i j \cdots }^{a b \cdots} | \Hc \frac{T^2}{2!} | \Phi} 
   = \Ec t_{a b \cdots }^{i j \cdots}\]
\[ \braket{\Phi | \Hc | \Phi} + \braket{\Phi| \Hc T| \Phi} + \braket{\Phi | \Hc \frac{T^2}{2!} | \Phi} = \Ec \]

These equations are not easily solvable. The trick used in traditional coupled cluster 
theory to make these equations more workable is to multiply by $\exp(-T)$ on both sides,
which effectively transforms our Hamiltonian $\bar{\Hc} = \exp(-T) H \exp(T)$.
\[\braket{\Phi | \Hc | \Psi_{\mathrm{CC}}} = E_c \braket{\Phi|\Psi_{\mathrm{CC}}} \]
\[ \braket{\Phi | \exp(-T) \Hc \exp(T) | \Phi } = E \braket{\Phi | \exp(-T) \exp(T) | \Phi} \]
\[ \color{blue} \braket{\Phi | \exp(-T) \Hc \exp(T) | \Phi } = \Ec \]
and 
\\
\[ \braket{ \Phi_{i j \cdots }^{a b \cdots} | \Hc \exp(T) | \Phi} = 
    \Ec \braket{\Phi_{i j \cdots }^{a b \cdots} | \exp(T) | \Phi}  \]
\[ \braket{ \Phi_{i j \cdots }^{a b \cdots} | \exp(-T) \Hc \exp(T) | \Phi} = 
    \Ec \braket{\Phi_{i j \cdots }^{a b \cdots} | \exp(-T) \exp(T) | \Phi}  \]
\[ \braket{ \Phi_{i j \cdots }^{a b \cdots} | \exp(-T) \Hc \exp(T) | \Phi} = 
    \Ec \braket{\Phi_{i j \cdots }^{a b \cdots} | \Phi}  \]
\[ \color{blue} \braket{ \Phi_{i j \cdots }^{a b \cdots} | \exp(-T) \Hc \exp(T) | \Phi} 
   =  0  \]

This transformation decouples the amplitude equation from the energy equation
by setting the right side of the amplitude equation to zero. 
In the language of diagrams, this transformation removes ``disconnected'' diagrams,
and $\Ec t_{ab \cdots}^{ij \cdots}$ is one example. 





%TODO hausdorff
%Our CC equations are infinite series due to the exponential form. 
%However, at a given truncation of excitations (CCSD, CCSDT, CCSDTQ, ...),
%the series always truncates to a finite number of terms, even though
%each $T_1$, $T_2$, $T_3$ are infinite expansions.
%
%We can figure out how these series converge using the Hausdorf expansion. 
The Hausdorff

Accoridng to Wick's theorem, the commutators evaluate to (andreas prop 5.1)

The only nonzero terms in the Hausdorff expansion are those
in which the Hamiltonian has at least one contraction with every cluster
operator on its right. 
In the language of diagrams, this translates to the Hausdorff expansion 
containing only \textit{connected} diagrams,
so that $\bar{\Hc} = \exp(-T) \Hc \exp(T)$ can be evaluated as $\bar{\Hc} = (\Hc \exp(T))_{\mathrm{C}}$,
where the subscript $\mathrm{C}$ denotes a restriction to connected diagrams.

\[\bar{\Hc} = (\Hc \exp(T))_{\mathrm{C}} = (\Hc + \Hc T + \frac{1}{2!} \Hc T^2 + \frac{1}{3!} \Hc T^3 + \frac{1}{4!} \Hc T^4)_{\mathrm{C}}\]
This expansion ends at $T^4$ because $\Hc$ consists of one and two-electron operators,
which can contract with at most two and four $T$'s, where $T = T_1 + T_2 + \cdots$.

This gives us our final results for the coupled cluster energy and amplitude equations,
which only consist of ``connected'' diagrams, or in algebraic language, terms in which
the Hamiltonian has a contraction with the cluster operators.

\section{On the connection between Coupled Cluster and Configuration Interaction}
Our CI wavefunction is  
\[ \ket{\Psi_{\mathrm{CI}}} = (1  + C_1 + C_2 + \cdots) \ket{\Phi}  \]
Our CC wavefunction is 
\[ \ket{\Psi_{\mathrm{CC}}} = \exp(T_1 + T_2 + \cdots) \ket{\Phi}  \]
\[ \ket{\Psi_{\mathrm{CC}}} = \left[ 1 + (T_1 + T_2 + T_3 + T_4 + \cdots ) + \frac{1}{2} (T_1 + T_2 + T_3 + T_4 + \cdots)^2 +  \frac{1}{6} (T_1 + T_2 + T_3 + T_4 + \cdots)^3 \cdots \right] \ket{\Phi}  \]

Noting that each $T_k$ ($C_k$) is just made up of scalar cluster amplitudes 
$t_{ab \cdots}^{ij \cdots}$ (CI coefficients $c_{ab \cdots}^{ij \cdots}$) 
and a Phi-normal-ordered excitation operator,  we can compare these wavefunction 
expansions term by term and find the following:

\begin{align*}
C_1 &= T_1 \\
C_2 &= T_2 + \frac{1}{2} T_1^2 \\
C_3 &= T_3 + T_1 T_2 + \frac{1}{6} T_1^3 \\
C_4 &= T_4 + \frac{1}{2}T_2^2 + T_1 T_3 +  \frac{1}{2} T_1^2 T_2 + \frac{1}{24} T_1^4 \\
\end{align*}

CI does not discriminate between different kinds of $n$-tuple excitations,
but in coupled cluster thoery, we can sub-classify various $n$-tuple excitations 
in different ways. 
Consider the quadruple excitation terms.
The physical interpretation of $T_4$ as an excitation event on some 
quantum mechanical state is 4 electrons coming together at the same location
and, upon interacting, they all have to simultaneously excite to a new state.
$\frac{1}{2}T_2^2$, on the other hand, corresponds to an excitation event 
in which a pair of 2 electrons which are close enough to locally interact
each excite to a new state, while simultaneously another distinct pair of 
    electrons does the same.
Which of these physical events ($T_4$  or $\frac{1}{2}T_2^2$) are more likely 
    to occur in a molecule? 
Almost certainly the latter, since electrons tend to avoid each other, but they also
    tend to form electron pairs! 
We conclude that $\frac{1}{2}T_2^2$ is a more important contribution to our wavefunction
than $T_4$.

Now consider the CCSD wavefunction,
\[ \ket{\Psi_{\mathrm{CCSD}}} = \exp(T_1 + T_2) \ket{\Phi} \] 
\[ \ket{\Psi_{\mathrm{CCSD}}} = (1 + T_1 + T_2 + \frac{1}{2} T_1^2 + T_1 T_2 + 
 \frac{1}{6} T_1^2 + \frac{1}{2} T_2^2 + \frac{1}{2} T_1^2 T_2 + \frac{1}{24} T_1^4 + ...)    \ket{\Phi} \] 

We obtain all single and double excitations according to CI, but we also
obtain some triple excitations ($T_1 T_2$, $\frac{1}{6} T_1^3$) and some quadruple 
excitations ($\frac{1}{2} T_2^2 + \frac{1}{2} T_1^2 T_2 + \frac{1}{24} T_1^4 $).
This is why its called coupled cluster; higher order excitations 
(quadruple, $\frac{1}{2}T_2^2$) are included even in low-order truncation
of the scheme (CCSD) via \textit{coupling} two double excitations together.
To put it another way, in CCSD, we obtain some information about quadruple excitations 
by considering two clusters which each doubly excite simultaneously. 
Not only that, but CCSD is giving us quadruple excitation of relatively high importance,
according to our physical argument earlier. 

\subsection{Deriving CC from CI}
Starting from full-CI,
\[ \ket{\Psi} = \lp 1 + C_1 + C_2 + C_3 + \cdots + C_n \rp \ket{\Phi} \]
\[ \ket{\Psi} = \lp 1 + C \rp \ket{\Phi} \]
We can define the cluster operator $T = T_1 + T_2 + \cdots$ as $T = \ln(1+C)$ where $C = C_1 + C_2 + \cdots$.
This way, $1 + C = e^T$, so it must be the case 
\[\ket{\Psi} = e^T \ket{\Phi} \]
which is our coupled cluster wavefunction.

We can compare coupled cluster and CI operators using the Taylor expansion for $\ln (1 + z)$
\[ \ln (1 + z) = \sum_{m=1}^{\infty} \frac{ (-1)^{m+1}  z^m}{ m} \]
This implies that for $T$ we have 
\[ T = \ln (1 + C) = \sum_{m=1}^{\infty} \frac{(-1)^{m+1}  C^m}{m} \]
This series always converges, since we always have a finite number of electrons in a system.
Expanding the series we get,
\[T = C - \frac{C^2}{2} + \frac{C^3}{3} + \cdots \]
\[T = C_1 + C_2 + C_3 + \cdots -  \frac{1}{2} ( C_1 + C_2 + C_3 + \cdots)^2 +  
  \frac{1}{3} (C_1 + C_2 + C_3 + \cdots)^3 +  \cdots
\]



\section{On the connection between Coupled Cluster and Perturbation Theory}
TODO; diagrams pre-requisite.




\end{document}
