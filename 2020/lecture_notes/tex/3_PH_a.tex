\documentclass{article}
\usepackage{simplewick}
\usepackage{amsmath,mathtools,amssymb}
\usepackage{graphicx}
\usepackage{gensymb}
\usepackage{verbatim}
\usepackage{cancel}
\usepackage{mathrsfs}
\usepackage{bbm}
\usepackage{braket}
\usepackage{xcolor}
\definecolor{orange}{RGB}{255,127,0}
\usepackage{verbatim}
\usepackage[margin=1.0in]{geometry}
\newcommand{\ol}{\overline}
\newcommand{\ctr}{\bcontraction}
\newcommand{\fctr}{\contraction}
\newcommand{\ve}{\varepsilon}
\newcommand{\kphi}{\ensuremath{\ket{\Phi}} }
\newcommand{\ap}{\ensuremath{a_p} }
\newcommand{\dg}{\ensuremath{^\dagger} }
\newcommand{\cd}{\ensuremath{\cdots} }
\newcommand{\apd}{\ensuremath{a_p^\dagger} }
\def\*#1{\mathbf{#1}}
\def\cb#1{{\color{blue}#1}}
\def\co#1{{\color{orange}#1}}
\def\cre#1{{\color{red}#1}}
\def\cg#1{{\color{green}#1}}
\def\cp#1{{\color{purple}#1}}
\DeclarePairedDelimiter\floor{\lfloor}{\rfloor}

\title{Lecture 3.5: Particle-Hole Formalism}
\date{February 24$^{\text{th}}$, 2020}
\begin{document}
\maketitle
\noindent

\section{Evaluating matrix elements in the true vacuum}
So far, we have seen that we can evaluate an arbitrary expectation value $\braket{\Phi_{p_1 \cd p_N} | H | \Phi_{q_1 \cd q_N}}$ by expressing everything as 
a product of operators in the true vacuum. 
In the case where $\Phi_{p_1 \cd p_N} =  \Phi_{q_1 \cd q_N}$, we can use Slater's rules to help us quickly evaluate matrix elements. 
However, for arbitrary determinants $\Phi_{p_1 \cd p_N}$ and $\Phi_{q_1 \cd q_N}$, the steps we would take to evaluate $\braket{\Phi_{p_1 \cd p_N} | H | \Phi_{q_1 \cd q_N}}$ would be:
\begin{enumerate}
\item Write everything in terms of a string of operators in the true vacuum
\item Evaluate using Wick's theorem
\end{enumerate}
For example,
\[\braket{\Phi_{tu}|\hat{O}_1| \Phi_{vw}} = h_{pq} \braket{0| a_u a_t a_p\dg a_q a_v\dg a_w\dg | 0}  \] 
%
\begin{align*}
\braket{\Phi_{tu}|\hat{O}_1| \Phi_{vw}} &= h_{pq} \braket{0| n[
\ctr{a_u}{a}{{}_t}{a}
\ctr{a_u a_t a_p\dg }{a}{{}_q}{a}
\ctr[2ex]{}{a}{{}_u a_t a_p\dg a_q a_v\dg }{a}
a_u a_t a_p\dg a_q a_v\dg a_w\dg] | 0}  
+ h_{pq} \braket{0| n[
\ctr{a_u}{a}{{}_t}{a}
\ctr{a_u a_t a_p\dg }{a}{{}_q a_v\dg }{a}
\ctr[2ex]{}{a}{{}_u a_t a_p\dg a_q }{a}
a_u a_t a_p\dg a_q a_v\dg a_w\dg] | 0}  
\\
&+ h_{pq} \braket{0| n[
\ctr{}{a}{{}_u a_t}{a}
\ctr{a_u a_t a_p\dg }{a}{{}_q}{a}
\ctr[2ex]{a_u}{a}{{}_t a_p\dg a_q a_v\dg }{a}
a_u a_t a_p\dg a_q a_v\dg a_w\dg] | 0}  
+ h_{pq} \braket{0| n[
\ctr{}{a}{{}_u a_t}{a}
\ctr{a_u a_t a_p\dg }{a}{{}_q a_v\dg }{a}
\ctr[2ex]{a_u}{a}{{}_t a_p\dg a_q }{a}
a_u a_t a_p\dg a_q a_v\dg a_w\dg] | 0}  
\\
&= h_{pq} [\delta_{pt}\delta_{qv}\delta_{uw} - \delta_{pt}\delta_{qw}\delta_{uv} - \delta_{pu}\delta_{qv}\delta_{tw} +  \delta_{pu}\delta_{qw}\delta_{vt} ]
\end{align*}
Evaluating $\braket{\Phi_{tu}|\hat{O}_2| \Phi_{vw}}$ would produce another 4 nonzero terms to consider. 
Even for determinants with 2 spin orbitals, we have to consider 8 terms. 
\\ \\
For a determinant with 100 spin orbitals, we have a string of more than 200 operators to work on, and many many terms to consider.
We need a different approach. 
\\ \\
To find a more convenient approach, we have to identify the problem with our current approach.
Our problem is that we are expressing our determinants as products of creation operators acting on the vacuum:
\[\ket{\Phi_{p_1 \cd p_\mu \cd p_N}} = a_{p_1}\dg \cd a_{p_\mu}\dg \cd a_{p_N}\dg \ket{0} \]
We made a choice to do everything with respect to the true vacuum. 
\\ \\ 
How can we simply our problem? 
\section{Re-expressing the wavefunction expansion}
We previously showed that a wavefunction can be expanded in a basis of Slater determinants: 
\[\ket{\Psi} =  \sum_{\nu} c_\nu \ket{\Phi_{\nu}} \]
If the basis set $\nu$ approaches the complete basis set limit, then the full expansion approaches the exact wavefunction.
We can express this linear combination of Slater determinations in multiple ways.
One way is to choose a reference determinant \kphi and write the other Slater determinants with respect to the difference to the reference determinant. 
\\ \\
From now on, we will use the following meanings for different set of indices: 
\begin{itemize}
\item $i, j, k \cd$  denotes spin orbitals occupied in \kphi
\item  $a, b, c \cd$  denotes spin orbitals unoccupied in \kphi
\item  $p, q, r, s \cd$  denotes generic spin orbitals
\end{itemize}
For example, a Slater determinant that is different from the reference determinant \kphi by one orbital can be written as: 
\[a_a\dg a_i \kphi = \ket{\Phi_i^a} \]
We can think of this as a single excitation from orbital $i$ to $a$. 
The full wavefunction expansion can thus be written in terms of single, double, up to n-tuple excitations: 
\\
\[\ket{\Psi} = c_0 \kphi \sum_{ia} c_a^i \ket{\Phi_i^a} + \sum_{\substack{ij \\ ab}} c_{ab}^{ij} \ket{\Phi_{ij}^{ab}} + \cd + \sum_{\substack{{i_1 \cd i_n} \\ {a_1 \cd a_n}}}  c_{a_1 \cd a_n}^{i_1 \cd i_n} \ket{\Phi_{i_1 \cd i_n}^{a_1 \cd a_n}} \]
where $c_{a_1 \cd a_n}^{i_1 \cd i_n}$ are the expansion coefficients. 
\\ \\
In our standard second quantization formalism, we would have to express every Slater determinant in terms of creation operators acting on the true vacuum $\ket{0}$. 
We can see, though, that our problem would be greatly simplified if we could have a formalism that works with respect to the reference determinant. 
The number of operators one would have to work with would be greatly reduced.
For the terms in the wavefunction expansion, one would only have to write down operators associated with the excitation from the reference determinant. 
\\ \\
Our next topic will be to develop a formalism in which the reference determinant acts as our vacuum. 
\section{Arriving at a quasiparticle framework}
\subsection{Particle framework}
Recall that we started our discussion of second quantization by uniquely representing a determinant with an occupation vector. 
For example, a determinant that has the first 4 spin orbitals occupied in a basis of 10 spin orbitals can be expressed as: 
\[\kphi = \ket{1111000000} \]
In the occupation number formalism, $n_p = 1$ if a spin orbital is occupied, and $n_p = 0$ if a spin orbital is unoccupied. 
You could say we are working in a``particle" frame of reference, where the concept of occupation is defined by whether there is a particle present. 
A spin orbital is \textit{occupied} when there is a particle in that spin orbital.
A spin orbital is \textit{unoccupied} when there is no particle present. 
\\ \\
Under the ``particle framework", a vacuum state is one in which there are no particles: $\ket{0}$
\\ \\
Apart from the fact that this intuitively makes sense to us, there is no reason why we need to work in this framework. 
There is no reason why we need to describe states by following the presence of particles. 
\subsection{Hole framework}
Instead of keeping track of particles, and working in a ``particle framework", we could work in a ``hole framework". 
What is a ``hole"? It is simply the absence of a particle. 
Instead of keeping track of whether a particle is in a spin orbital, we can keep track of whether a hole is in a spin orbital. 
\\ \\
If a particle is in a spin orbital $\phi_p$, in the  ``particle framework", $n_p = 1$ because we are keeping track of particles and we give an occupation number of 1 to 
something that is occupied with a particle.
In the ``hole framework", if a particle is in a spin orbital $\phi_p$, then there is no hole in $\phi_p$! 
In this framework, we would say $\phi_p$ has an occupation number of $\bar{0}$, because the hole is unoccupied. 
We use the bar ( $\bar{\,}$ ) to indicate that we are working in the ``hole framework". 
\\ \\
What would $\bar{n}_p = \bar{1}$ mean?
This means spin orbital $\phi_p$ is occupied with a hole.
Translated into the ``particle framework", this means there is no particle in $\phi_p$.  
\\ \\ 
We can see the mapping from the particle frame to the hole framework for \kphi is: 
\[ \ket{1111000000} \rightarrow  \ket{\bar{0}\bar{0}\bar{0}\bar{0}\bar{1}\bar{1}\bar{1}\bar{1}\bar{1}\bar{1}} \]
The presence of a particle means the absence of a hole. 
\\ \\
What is a vacuum state in a hole framework? 
It is a state where there are no holes: $\ket{\bar{0}}$
\subsection{Quasiparticle framework}
Now, let's take another step of abstraction. 
We can define another frame to work in, the ``quasiparticle framework". 
What is a ``quasiparticle"? 
It is something that can only be created when we excite out of a reference determinant.
It does not have have meaning without a reference determinant, since it only comes into existence when
there is a excitation from the reference determinant. 
\\ \\
By the definition of quasiparticles, the reference determinant \kphi is a state in which there are \textit{no quasiparticles}.
\kphi is thus a vacuum in the quasiparticle framework because there are no quasiparticles. 
Remember, a vacuum is a state in which no spin-orbital is occupied.
How can we \kphi in such a way in which no spin-orbital is occupied?
\\ \\ 
We can do this by combining the particle and hole frameworks:
\[\kphi =  \ket{\bar{0}\bar{0}\bar{0}\bar{0} 0 0 0 0 0 0} \]
We talk about the originally particle occupied spin-orbitals in terms of the hole framework:
there are no holes.
We talk about the originally particle unoccupied spin-orbitals in terms of the particle framework:
there are no particles.
In this way, \kphi serves as a vacuum in the quasiparticle frame. 
This is exactly what we wanted!
We wanted to develop a framework where \kphi acts like a vacuum.
In the quasiparticle framework, we can do so.
We call \kphi the Fermi vacuum.
\\ \\
The mapping between the particle framework and the quasiparticle framework is: 
\[ \ket{1111000000} \rightarrow  \ket{\bar{0}\bar{0}\bar{0}\bar{0}000000} \]
where we switch from a particle framework to a hole framework \textit{only} for the particle occupied spin-orbitals in our reference determinant. 
This mapping is called particle-hole isomorphism.
\\ \\
We now have to reframe our mathematics under the quasiparticle framework using particle-hole isomorphism.
We can prove that the mathematics in this new framework is actually equivalent to the mathematics in the old frame, and convince ourselves that is it okay
to work in different frames. 
This new formalism that we develop is called the particle-hole formalism.
\section{Particle-hole formalism} 
In our original treatment of the second quantization formalism, we introduced the concepts of 
\begin{enumerate}
\item vacuum
\item creation/annihilation operators
\item normal product
\item	contraction
\item	normal product with contraction 
\item Wick's theorem
\item Hamiltonian in second quantization
\end{enumerate}
The quantities above were all formulated with respect to the true vacuum.
Our next task is to reformulate these quantities in the particle-hole formalism.
We need to know what these quantities look like with respect to the Fermi vacuum. 

\subsection{Quasiparticle creation and annihilation operators}
How do we create a quasiparticle? 
In the particle framework, we create a quasiparticle by exciting from an occupied spin-orbital in our reference determinant
to an unoccupied spin-orbital: 
\[\ket{\Phi_i^a} = a_a\dg a_i \kphi \]
What does this action look like in the occupation number formalism?
\[a_a\dg a_i \ket{11\cb{1_i}100\co{0_a}000} = \ket{11\cb{0_i}100\co{1_a}000}\]
In words, we are annihilating a particle in i, and creating a particle in a. 
Let's map this in the quasiparticle frame: 
\[\ket{11\cb{0_i}100\co{1_a}000} \rightarrow  \ket{\bar{0}\bar{0}\cb{\bar{1}_i}\bar{0}00\co{1_a}000} \]
In words, for the quasiparticle frame, we are creating a hole in i, and creation a particle in a. 
If we use $b$ to denote creation and annihilation operators in the quasiparticle framework, 
we see there is a mapping between the operators.
\\ \\
Annihilating particle in i (particle framework) is the same as creating a hole in i (quasiparticle framework): 
\[a_i \rightarrow b_i\dg\]
Creating particle in a (particle framework) is the same as creating a particle in a(quasiparticle framework): 
\[a_a\dg \rightarrow b_a\dg\]
We can take the adjoint of the above for two additional mappings:
\[a_i\dg \rightarrow b_i\]
\[a_a \rightarrow b_a\]
The quasiparticle operators can be generally defined as: 
\[
  b_p =
    \begin{cases}
      a_p\dg & \text{p = i (occupied)}\\
      a_p & \text{p = a (unoccupied)}
    \end{cases}       
\]

\[
  b_p\dg =
    \begin{cases}
      a_p & \text{p = i (occupied)}\\
      a_p\dg & \text{p = a (unoccupied)}
    \end{cases}       
\]
The definition for creation and annihilation operators stays the same for the set of unoccupied spin-orbitals, but flips for the set of occupied orbitals.
This is consistent with the mapping 
\[ \ket{1111000000} \rightarrow  \ket{\bar{0}\bar{0}\bar{0}\bar{0}000000} \]
for which we switched from tracking particles to holes only for the set of occupied orbitals. 
\\ \\
We need to now prove that the mathematics of quasiparticle operators in the particle-hole formalism 
is consistent with the mathematics of operators $a_p$ and $a_p\dg$ in the old formalism. 
\subsection{Annihilation operators in the vacuum}
In the particle framework, an annihliation operator acting on the vacuum gives us zero:
\[a_p \ket{0} = 0 \]
In the quasiparticle framework, we have to consider two cases for $b_p \kphi$: 
\\ \\
In the case that p = i:
\[ b_i \ket{\Phi} = a_i\dg  \ket{\Phi}  = 0     \]
In the case that p = a: 
\[ b_a \ket{\Phi} = a_a  \ket{\Phi}  = 0     \]
Thus, $b_p$ acting on the Fermi vacuum is zero. 

\subsection{Anticommutation relations}

The anticommutation rules for quasiparticle operators is the same as for $a_p$ and $a_p\dg$:

\begin{enumerate}
\item  $\{b_p, b_q\} = 0$
\item $\{b_p^\dagger, b_q^\dagger \} = 0$
\item $\{b_p, b_q\dg \} = \delta_{pq}$
\end{enumerate}
We can prove these commutation relations by considering how the quasiparticle operators map back to $a_p$ and $a_b\dg$ in 4 cases:
\\ \\
p = i, q = j: 
\[ \{b_i, b_j\} =  \{a_i\dg, a_j\dg\} = 0 \]
\[ \{b_i^\dagger, b_j^\dagger \}  = \{a_i, a_j \} = 0 \]
\[ \{b_i, b_j\dg \} =  \{a_i\dg, a_j \} = \delta_{ij} \]
p = i, q = a: 
\[ \{b_i, b_a\} =  \{a_i\dg, a_j\} = \delta_{ia} = 0 \]
\[ \{b_i^\dagger, b_a^\dagger \}  = \{a_i, a_a\dg \}  =  \delta_{ia} = 0 \]
\[ \{b_i, b_a\dg \} =  \{a_i\dg, a_a\dg \} = 0 = \delta_{ia} \]
In the above we use the equation $\delta_{ia} = 0$, which is true because $i$ can never be the same as $a$. 
\\ \\ 
p = a, q = i: 
\[ \{b_a, b_i\} =  \{a_a, a_i\dg\} =  \delta_{ia} = 0 \]
\[ \{b_a^\dagger, b_i^\dagger \}  = \{a_a\dg, a_i \} =  \delta_{ia} = 0 \]
\[ \{b_a, b_i\dg \} =  \{a_a, a_i \} =  0 = \delta_{ia}  \]
p = a, q = b: 
\[ \{b_a, b_b\} =  \{a_a, a_b\}  = 0 \]
\[ \{b_a^\dagger, b_b^\dagger \}  = \{a_a\dg, a_b\dg \} = 0 \]
\[ \{b_a, b_b\dg \} =  \{a_a, a_b\dg\}  = \delta_{ab}  \]

\subsection{Normal products with respect to the Fermi vacuum }
For a normal product with respect to the Fermi vacuum, we write: 
\[N[y_1 \cd y_m]= (-1)^R b_{R_1}\dg \cd b_{R_\alpha}\dg b_{R_{\alpha+1}} \cd b_{R_m} \]
\[
R = 
\begin{pmatrix}
1 & 2 &\cd & \alpha & \alpha+1 & \cd & m  \\
R_1 & R_2 &\cd & R_\alpha& R_{\alpha+1} & \cd & R_m  \\
\end{pmatrix}
\] \\
Notice that we use a capital N instead of a small n, and we use $y$ for denoting a general quasiparticle operator $b_p$ or $b_p\dg$. 
We refer to normal ordering with respect to the Fermi vacuum as $\Phi$-normal ordering. 
\\ \\
For example,
\[N[b_p\dg b_q b_r\dg] = -b_p\dg b_r\dg b_q = b_r\dg b_p\dg b_q\]
\\ \\
As with the normal product in the true vacuum, 
\[N[\varnothing] = 1\]
\\ \\
Properties of $\Phi$-normal products: 
\begin{enumerate}
\item A normal product of a normal ordered string of operators is just that string of operators:
\[N[b_{1}\dg \cd b_{\mu}\dg b_{\mu+1} \cd b_{m} ] = b_{1}\dg \cd b_{\mu}\dg b_{{\mu+1}} \cd b_{m}  \]
We note that a string of only creation operators or only annihilation operators are already in normal order.
\item A normal product of a normal product is just the normal product:
\[N[N[y_1 \cd y_m]] = N[y_1 \cd y_m] \]
This is simply a manifestation of property 1.
\item You can freely permute a string of operators, with an appropriate phase factor, as long as it is within the normal ordered form:
\[N[y_{R_1} \cd y_{R_m} ] = (-1)^R N[ y_1 \cd y_m]\]
For example,
\[N[y_2 y_1] = -N[y_1 y_2] \]
\end{enumerate}

\subsection{Contractions with respect to the Fermi vacuum }

For contractions with respect to the Fermi vacuum, we write the contraction line on top instead of the botton: 
$$
\fctr{}{y}{{}_1}{y}
y_1 y_2 = y_1y_2 - N[y_1y_2]
$$

\begin{enumerate}
\item $y_1 = b_p$, $y_2 = b_q$ \\
$$
\fctr{}{b}{{}_p}{b}
b_p b_q = b_p b_q  - N[b_p b_q] = b_p b_q  - b_p b_q  = 0 
$$
\item $y_1 = b_p$, $y_2 = b_q\dg$ \\
$$
\fctr{}{b}{{}_p}{b}
b_p b_q\dg = b_p b_q\dg  - N[b_p b_q\dg] = b_p b_q\dg  +  b_q\dg b_p = \delta_{pq}
$$
\item $y_1 = b_p\dg$, $y_2 = b_q$ \\
$$
\fctr{}{b}{{}_p}{b}
b_p\dg b_q = b_p\dg b_q  - N[b_p\dg b_q] = b_p\dg b_q  - b_p\dg b_q  = 0 
$$
\item $y_1 = b_p\dg$, $y_2 = b_q\dg$ \\
$$
\fctr{}{b}{{}_p}{b}
b_p\dg b_q\dg = b_p\dg b_q\dg  - N[b_p\dg b_q\dg] = b_p\dg b_q\dg  - b_p\dg b_q\dg  = 0 
$$
\end{enumerate}

\subsection{$\Phi$-normal products with contractions}
A normal product with contractions in the Fermi vacuum,
\[ 
\ctr{N[y_1 \cd }{x}{{}_{\nu_1} \cd y_{\nu_\lambda} \cd }{x}
\ctr[2ex]{N[y_1 \cd y_{\nu_1} \cd}{ x}{{}_{\nu_\lambda} \cd y_{\mu_1} \cd }{x} 
N[y_1 \cd y_{\nu_1} \cd y_{\nu_\lambda} \cd y_{\mu_1} \cd y_{\mu_\lambda} \cd y_m] 
\] 
can be rewritten as: 
\[ 
\ctr{N[y_1 \cd }{x}{{}_{\nu_1} \cd y_{\nu_\lambda} \cd }{x}
\ctr[2ex]{N[y_1 \cd y_{\nu_1} \cd}{ x}{{}_{\nu_\lambda} \cd y_{\mu_1} \cd }{x} 
N[y_1 \cd y_{\nu_1} \cd y_{\nu_\lambda} \cd y_{\mu_1} \cd y_{\mu_\lambda} \cd y_m] 
= 
\ctr{(-1)^R}{x}{{}_{\nu_1} }{x}
\ctr{(-1)^R y_{\nu_1}y_{\mu_1} \cd}{x}{{}_{\nu_\lambda} }{x}
(-1)^R y_{\nu_1}y_{\mu_1} \cd y_{\nu_\lambda} y_{\mu_\lambda} N[y_{k_1} \cd y_{k_\rho}]
\] 
where 
\[
R = 
\begin{pmatrix}
1 & 2 & \cd & 2\lambda -1 & 2\lambda & 2\lambda + 1&\cd & m  \\
\nu_1 & \mu_1 & \cd& \nu_\lambda & \mu_\lambda  &k_1 & \cd & k_\rho \\
\end{pmatrix}
\] \\
and the indices $2\lambda + \rho = m$.
\\ \\
The two tips used for normal products with contractions in the true vacuum still apply:
First, neighboring contractions can be moved out of the normal product with no sign change.
Second, a fully contracted normal product, or a group within a normal product that is fully contracted, 
can be moved out of the normal product with a sign change equal to $(-1)^\nu$ where $\nu$ is the number of intersecting contraction lines 
in that group. 
\subsubsection{Example 1}
\begin{align*}
N[\fctr{}{b}{{}_p b_q}{b}
\fctr[2ex]{ b_p}{b}{{}_q b_r\dg}{b}
 b_p b_q b_r\dg b_s\dg b_t b_u\dg ]  &= 
 - \fctr{}{b}{{}_p}{b}b_p b_r\dg \fctr{}{b}{{}_q}{b}b_q b_s\dg N[b_t b_u\dg] \\
 &= - \delta_{pr}\delta_{ps}N[b_t b_u\dg] \\
 &= \delta_{pr}\delta_{ps} b_u\dg b_t 
\end{align*}
%do here or later: normal product in the fermi for a operators)
%do here or later: normal product w/ contraction in fermi for a operators)

\subsection{Wick's Theorem }

The statement of Wick's theorem remains the same for quasiparticle operators in the Fermi Vacuum: 

\[y_1 \cd y_m = N[y_1\cd y_m] + \sum_{a.c} N\ol{[y_1 \cd y_m]} \]
Wick's theorem in the Fermi vacuum with $b_p$ and $b_p\dg$ 
can be proved in the same way that we proved Wick's theorem in the true vacuum with 
$a_p$ and $a_p\dg$.
\\ \\
The generalized Wick's theorem for quasiparticle operators states 
that for a string of quasiparticle operators which have some substring which is already in normal ordering (w.r.t the Fermi vacuum),
you do not have to consider contractions between quasiparticle operators in that substring.\\
Formally stated,
$$y_1 \cd N[b_{p_1}\dg \cd b_{p_\alpha}\dg b_{p_{\alpha + 1}} b_{p_\beta}] \cd y_m = N[y_1 \cd y_m ]  + \sum_{a.c.} {}' N\ol{[y_1 \cd y_m ]} $$
where $\sum_{a.c.} {}'$ means skipping contractions that originate from the same normal ordered group.

\section{Solving for expectation values}
Since our formalism is consistent with the formalism in the true vacuum,
the rules we previously developed for expectation values of $a_p$ and $a_p\dg$ in the true vacuum
 will apply for quasiparticle operators $b_p$ and $b_p\dg$ in the Fermi vacuum: 
\begin{itemize}
\item Rule 1: $N[y_1 \cd y_m]\kphi = 0$ unless all $y_1 \cd y_m$ are creation operators $b_p\dg$\\
\item Rule 2: $\braket{\Phi | N[y_1 \cd y_m] | \Phi} = 0$ for  $m \geq 1$ \\
\item Rule 3. $\fctr{N[y_1 \cd }{y}{{}_{i_1} \cd y_{i_\lambda} \cd }{y}
\fctr[2ex]{N[y_1 \cd y_{i_1} \cd}{ y}{{}_{i_\lambda} \cd y_{j_1} \cd }{y} 
N[y_1 \cd y_{i_1} \cd y_{i_\lambda} \cd y_{j_1} \cd y_{j_\lambda} \cd y_m] \kphi= 0$ if there is at least 1 uncontracted annihilator among operators $y_1 \cd y_m$.  \\
\item Rule 4. $\braket{\Phi | \fctr{N[y_1 \cd }{y}{{}_{i_1} \cd y_{i_\lambda} \cd }{y}
\fctr[2ex]{N[y_1 \cd y_{i_1} \cd}{ y}{{}_{i_\lambda} \cd y_{j_1} \cd }{y} 
N[y_1 \cd y_{i_1} \cd y_{i_\lambda} \cd y_{j_1} \cd y_{j_\lambda} \cd y_m] |\Phi} = 0$ unless all operators are contracted.
A vacuum expectation value of any normal product with contractions will be zero if there are a odd number of operators. \\

\item Rule 5 $\braket{\Phi | y_1 \cd y_{2m + 1} | \Phi} = 0 $ \\

\item Rule 6 $\braket{\Phi | y_1 \cd y_{2m } | \Phi} = \sum_{f.c.} \braket{\Phi | N\ol{\ol{ [y_1 \cd y_{2m }]}}| \Phi} = \sum_{f.c.} (-1)^R \fctr{y}{{}_{i_1}}{y}  y_{i_1} y_{j_1} \cd \fctr{y}{{}_{i_m}}{y} y_{i_m} y_{i_m}$\\

\begin{equation*}
R = 
\begin{pmatrix}
1 & 2 &\cd  & 2m - 1 & 2m  \\
i_1 & j_1 & \cd  & i_m & j_m \\
\end{pmatrix}
\end{equation*}

\item Rule 7: $y_1 \cd y_m \kphi = \sum\limits_{annihilator f.c.} n\ol{[y_1 \cd y_m] }\ket{\Phi} $
where ``annihilator f.c" denotes all contractions with the annihilation operator fully contracted.
\end{itemize}
We now have a formalism to perform computations in which everything is a quasiparticle operator acting on a reference determinant \kphi. 
Now, we can evaluate matrix elements like the following very easily: 

\begin{align*}
\braket{\Phi| \fctr{}{b}{{}_p b_q}{b}
\fctr[2ex]{ b_p}{b}{{}_q b_r\dg}{b}
 b_p b_q b_r\dg b_s\dg \fctr{}{b}{{}_t}{b}b_t b_u\dg |\Phi} &= (-) \delta_{pr} \delta_{qs} \delta_{tu} 
\end{align*}
\\
However, our Hamiltonian is still in terms of particle operators $a_p$ and $a_p\dg$. 
How do we practically perform computations using our formalism, for example, on 
\[\braket{\Phi | \hat{h} |\Phi} = \sum_{pq} h_{pq} \braket{\Phi |a_p\dg a_q |\Phi} \]
\subsection{Translate everything in terms of $b_p$ and $b_p\dg$ }
One strategy we can take is to translate every $a_p$ and $a_p\dg$ into $b_p$ and $b_p\dg$ and then perform the computation in terms of 
$b_p$ and $b_p\dg$. 
\subsubsection{Example 1} 
\begin{align*}
 \braket{\Phi_b^j | \Phi_a^i} &= \braket{\Phi | a_j\dg a_b a_a\dg a_i |  \Phi} \\
 &= \braket{\Phi | b_j b_b b_a\dg b_i\dg |  \Phi}   \\
 &= \braket{\Phi | \fctr{}{b}{{}_j b_b}{b} b_j \fctr[2ex]{}{b}{{}_b b_a\dg}{b}b_b b_a\dg b_i\dg |  \Phi}   + \braket{\Phi | \fctr[2ex]{}{b}{{}_j b_b b_a\dg }{b} b_j \fctr{}{b}{{}_b}{b}b_b b_a\dg b_i\dg |  \Phi}  \\
 &= -\delta_{ja}\delta_{ib} + \delta_{ab} \delta_{ij} \\
 &= \delta_{ab} \delta_{ij}
\end{align*}

\subsubsection{Example 2} 
\[\braket{\Phi | \hat{h} |\Phi} = \sum_{pq} h_{pq} \braket{\Phi |a_p\dg a_q |\Phi} \]
Because $a_p$ and $a_p\dg$ are general operators, we have to split them into 4 cases: 

\begin{align*}
\braket{\Phi | \hat{h} |\Phi} &= \sum_{pq} h_{pq} \braket{\Phi |a_p\dg a_q |\Phi} \\
&= \sum_{ij} h_{ij} \braket{\Phi |a_i\dg a_j |\Phi} +  \sum_{ia} h_{ia} \braket{\Phi |a_i\dg a_a |\Phi} +  \sum_{ai} h_{ai} \braket{\Phi |a_a\dg a_i |\Phi} +  \sum_{ab} h_{ab} \braket{\Phi |a_a\dg a_b |\Phi} \\
&= \sum_{ij} h_{ij} \braket{\Phi |b_i b_j\dg |\Phi} +  \sum_{ia} h_{ia} \braket{\Phi |b_i b_a |\Phi} +  \sum_{ai} h_{ai} \braket{\Phi |b_a\dg b_i\dg |\Phi} +  \sum_{ab} h_{ab} \braket{\Phi |b_a\dg b_b |\Phi} \\
&=  \sum_{ij} h_{ij} \braket{\Phi | \fctr{}{b}{{}_i}{b} b_i b_j\dg |\Phi} +  \sum_{ia} h_{ia} \braket{\Phi | \fctr{}{b}{{}_i}{b} b_i b_a |\Phi} +  \sum_{ai} h_{ai} \braket{\Phi | \fctr{}{b}{{}_a}{b}b_a\dg b_i\dg |\Phi} +  \sum_{ab} h_{ab} \braket{\Phi | \fctr{}{b}{{}_a}{b} b_a\dg b_b |\Phi} \\
&=  \sum_{ij} h_{ij} \braket{\Phi | \fctr{}{b}{{}_i}{b} b_i b_j\dg |\Phi}  \\
&= \sum_{ij} h_{ij} \delta_{ij}  \\
&= \sum_{i} h_{ii} \\
\end{align*}
This is not the most efficient way to proceed.
For example, for the two-electron operator, we would have to split it into 16 terms and compute contractions on all the terms. 
We need a different strategy, namely, seeing how $a_p$ and $a_p\dg$ directly acts in the Fermi vacuum.
This is where we are heading.


%(at the end have a change comparing quantities in the true vs fermi vacuum
% 3_Particle_Hole_b: 
%CI stuff - relate the grid to the overlap of SD is SD of overlaps stuff?
%\section{$\Phi$--Normal ordered Hamiltonian}
%\section{Derivation of Slater's first rule in the Fermi vacuum}
%\section{Derivation of Slater's second rule in the Fermi vacuum}
%\section{Derivation of Slater's third rule in the Fermi vacuum}

%(note about second and third rules - can prove those as well! - what the nonzero tells us about implications of spin orbitals differing)! 
%math with expectation value inform which specific value for integral. 

%Examples for class lecture: explicitly write wick's thrm for 3, 4 operators 
%Keep in mind these are all for true vacuum! 

\begin{comment}
Types of problems:
1. Slater's rules
Types of ways to solve:
1. First quantization
2. 2nd quantization in the true vacuum without Wick's theorem
3. 2nd quantization in the true vacuum with Wick's theorem
4. PH formalism with Wick's theorem 
5. PH formalism with Wick's theorem and $\Phi$-normal ordered Hamiltonian
\end{comment}
\end{document}
