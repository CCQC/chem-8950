\documentclass{article}
\usepackage{amsmath,mathtools,amssymb}
\usepackage{graphicx}
\usepackage{blkarray}
\usepackage{gensymb}
\usepackage{verbatim}
\usepackage{mathrsfs}
\usepackage{bbm}
\usepackage{braket}
\usepackage{hyperref}
\usepackage{verbatim}
\usepackage{cancel}
\usepackage[margin=1.0in]{geometry}
\newcommand{\ol}{\overline}
\newcommand{\lp}{\left(}
\newcommand{\rp}{\right)}
\newcommand{\eps}{\varepsilon}
\newcommand{\lam}{\lambda}
\newcommand{\h}{\circ}
\newcommand{\p}{\bullet}
\newcommand{\Ecorr}{E_{\mathrm{corr}}}
\newcommand{\Hc}{H_{\mathrm{c}}}
\newcommand{\dg}{\ensuremath{^\dagger} }
\def\*#1{\mathbf{#1}}
\DeclarePairedDelimiter\floor{\lfloor}{\rfloor}

\title{Lecture 5: Perturbation Theory}
\date{March 27, 2020}
\begin{document}
\maketitle
\noindent
To remind you of the current task at hand, we are trying to find good-quality approximations 
    to electronic wavefunctions in order to obtain the electronic energy of a molecular system. 
That is, we seek to solve the Schr{\"o}dinger equation in the best way possible.
\[H \ket{\Psi} = E \ket{\Psi} \]
We already learned that HF theory is a decent and cheap approximation, and full-CI is exact but very expensive. 
Perturbation theory is one of many approaches to finding wavefunctions and energies which are of 
    reasonable cost and reasonable accuracy.
Recall, however, that the HF wavefunction $\Phi$ gives us a very good starting point, so we are really just interested in solving for the correlation energy,
which is obtained by shifting our Hamiltonian by the HF energy.
\[\Hc \ket{\Psi} = \Ecorr \ket{\Psi} \]
where 
\[\Hc = H - \braket{\Phi| H | \Phi} \]
\[\Ecorr = E_{\mathrm{exact}} - E_{\mathrm{HF}}\]
In this set of notes, we will cover what is known as M{\o}ller-Plesset perturbation theory methods (MP$n$) 
    for obtaining correlation energies.

\subsection{Rayleigh-Schr{\"o}dinger Perturbation Theory}
Perturbation theory (PT), in the most general sense, is applicable in nearly every domain of physics. 
If a problem can be separated into some easily solvable piece A and not-so-easily solvable piece B 
    (``the perturbation''), one can approach the exact solution of the problem in terms of the solution to A 
    plus a series of corrections to account for the influence of perturbation.
Rayleigh-Schr{\"o}dinger Perturbation Theory (RSPT) applies the ideas of PT to the time-independent 
    Schr{\"o}dinger equation.
The Hamiltonian is decomposed into a sum of a simple (easy to solve) problem $H_0$ and a perturbation $V$:
\[H = H_0 + V \]
If the perturbation is ``small'' (i.e., does not wildly shift the solutions of the problem) the 
    energy levels/eigenstates associated with the \textit{perturbed system} described by $H$ can be expressed 
    as sum of the energy levels/eigenstates of  the \textit{simple system} (described by $H_0$) and a series 
    of corrections obtained by considering the influence of the perturbation $V$ on the 
    energy levels/eigenstates of the simple system.
Each of these corrections contain eigenvalues of $H_0$ and matrix elements of the perturbation 
    between the eigenfunctions of $H_0$, as we will see.
In summary, RSPT is applying age-old ideas of general perturbation theory to the
    time independent Schr{\"o}dinger equation.
MP$n$ methods can be understood as applying nondegenerate RSPT to the problem
    of describing electron correlation in fermionic systems. 

The eigenvalue problem we wish to solve is the following:
\[H \ket{\Psi_i} = (H_0 + V) \ket{\Psi_i} = E_i \ket{\Psi_i}\]
We are assuming $H_0$ is a simple, easy-to-solve system, for which we have access to all the eigenstates 
    and eigenvalues.
\[H_0 \ket{\Phi_i^{(0)}} = \eps_i^{(0)} \ket{\Phi_i^{(0)}} \]
These eigenstates $\ket{\Phi_i^{(0)}}$ and eigenvalues $\eps_i$ for the simple system are hopefully very close
    to the exact results  $\ket{\Psi_i}$ and  $E_i$, that is, the perturbation $V$ is small.
Now, we do something a bit weird.
We introduce a parameter $\lam$ in front of $V$, which allows us to toggle the influence of the
    perturbation on the energy levels and eigenstates. 
Before saying much else about $\lam$, let's write it down and see what it allows us to do.
\[(H_0 +  \lam V) \ket{\Psi_i} = E_i \ket{\Psi_i}\]
When $\lam = 0$, our Hamiltonian is just $H_0$, and our eigenstates and energy levels become that 
    of the simple system $\ket{\Phi_i^{(0)}}$ and $ \eps_i^{(0)}$.
When $\lam = 1$, our Hamiltonian is the full Hamiltonian $H$, and our eigenstates and energy levels 
    are the perturbed system $\ket{\Psi_i^{(0)}}$ and $E_i$.
By doing this, \textit{the energy levels and eigenstates of the perturbed system are a function
    of }$\lam$.
To put it another way, our energy levels and eigenstates depend on lambda; 
    we might express this mathematically as  $E_i(\lam)$ and $\Psi_i(\lam)$.
Since these quantities are \textit{functions of} $\lam$, we can write down a Taylor series
expansion of the energy levels and eigenstates of $H$ in terms of $\lam$:
\[E_i =  \eps_i^{(0)} +  \lam \eps_i^{(1)} + \lam^2 \eps_i^{(2)}  + \cdots \]
\[\ket{\Psi_i} =  \Phi_i^{(0)} +  \lam \Phi_i^{(1)} + \lam^2 \Phi_i^{(2)} + \cdots \]
where each term is,  
\[ \eps_i^{(k)} = \left. \frac{1}{k!} \frac{d^k \eps_i}{d \lam^k} \right\rvert_{\lam = 0} \]
\[ \Psi_i^{(k)} = \left. \frac{1}{k!} \frac{d^k \ket{\Psi_i}}{d \lam^k} \right\rvert_{\lam = 0} \]

We call $\eps_i^{(0)}$ the 0$^{th}$ order energy, which is the energy of the simple system. 
Every additional term in the energy expansion is the $n^{th}$ order energy correction, 
    and these are what we must solve for.
Likewise terminology is used for the eigenstates $\ket{\Psi_i^{(k)}}$


Many textbooks and other resources simply say $\lam$ is a 
    ``fictitious dummy parameter'' or ``book-keeping device'' that is 
    used to keep track of the  ``orders'' of the terms when working with the perturbation expansion.
This is a bit of an oversimplification. The parameter $\lam$ has the utility of turning
the perturbation on/off, and motivates the use of a Taylor expansion in the first place.
Without $\lam$, you have no Taylor expansion, and no perturbation theory. 
While it does serve as a useful label for the orders of each term, it's not just a ``dummy'' parameter. 
It's the key to all of perturbation theory.


We have a form for the wavefunction and the energy. Plugging these 
into the time-independent Schr{\"o}dinger equation, we get 
\[
(H_0 + \lam V) (\Phi_i^{(0)} +  \lam \Phi_i^{(1)} + \lam^2 \Phi_i^{(2)} + \cdots)
=  (\eps_i^{(0)} +  \lam \eps_i^{(1)} + \lam^2 \eps_i^{(2)}  + \cdots ) (\Phi_i^{(0)} +  \lam \Phi_i^{(1)} + \lam^2 \Phi_i^{(2)} + \cdots)
 \]

Subtracting from each side so that the left is 0 and grouping like powers of $\lam$ gives
\begin{align*}
&  \left(H_0 \ket{\Phi_i^{(0)}} - \eps_i^{(0)} \ket{\Phi_i^{(0)}}   \right) \\
&+ \lam \left( H_0\ket{\Phi_i^{(1)}} + V\ket{\Phi_i^{(0)}} - \eps_i^{(0)}\ket{\Phi_i^{(1)}} - \eps_i^{(1)}\ket{\Phi_i^{(0)}} \right)  \\
&+ \lam^2 \left( H_0 \ket{\Phi_i^{(2)}} + V \ket{\Phi_i^{(1)}} - \eps_i^{(0)} \ket{\Phi_i^{(2)}} - \eps_i^{(1)} \ket{\Phi_i^{(1)}} - \eps_i^{(2)} \ket{\Phi_i^{(0)}}  \right)  \\
&+ \cdots \\ 
&= 0
\end{align*}

The only way the above equation can be satisfied for any arbitrary value of $\lam$ is 
    if the coefficient in parentheses for each power of $\lam$ is equal to 0.
Therefore, since each term for each power of $\lam$ is equal to 0,
    this equation is fully separable into an infinite series of simultaneous equations
\begin{align*}
&  H_0 \ket{\Phi_i^{(0)}} = \eps_i^{(0)} \ket{\Phi_i^{(0)}}   \\
&  H_0\ket{\Phi_i^{(1)}} + V\ket{\Phi_i^{(0)}} = \eps_i^{(0)}\ket{\Phi_i^{(1)}} + \eps_i^{(1)}\ket{\Phi_i^{(0)}}  \\
&  H_0 \ket{\Phi_i^{(2)}} + V \ket{\Phi_i^{(1)}} = \eps_i^{(0)} \ket{\Phi_i^{(2)}} + \eps_i^{(1)} \ket{\Phi_i^{(1)}} + \eps_i^{(2)} \ket{\Phi_i^{(0)}}  \\
& \cdots \\ 
\end{align*}
Notice we have now dropped the $\lam$'s since each of the above equations hold whether or not $\lam^n$ is applied to each term.








 
% TODO finish Szabo 6.1, up to second order PT.
% then start up at 6.5 which applies HF and derives MP2. Then
% Give expression for MP3, say how complicated things get, motivate Piecuch style


\end{document}
