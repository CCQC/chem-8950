
\documentclass{article}
\usepackage{simplewick}
\usepackage{amsmath,mathtools,amssymb}
\usepackage{graphicx}
\usepackage{gensymb}
\usepackage{verbatim}
\usepackage{cancel}
\usepackage{mathrsfs}
\usepackage{bbm}
\usepackage{braket}
\usepackage{xcolor}
\definecolor{orange}{RGB}{255,127,0}
\usepackage{verbatim}
\usepackage[margin=1.0in]{geometry}
\newcommand{\ol}{\overline}
\newcommand{\ctr}{\contraction}
\newcommand{\ve}{\varepsilon}
\newcommand{\ap}{\ensuremath{a_p} }
\newcommand{\dg}{\ensuremath{^\dagger} }
\newcommand{\cd}{\ensuremath{\cdots} }
\newcommand{\apd}{\ensuremath{a_p^\dagger} }
\def\*#1{\mathbf{#1}}
\DeclarePairedDelimiter\floor{\lfloor}{\rfloor}

\title{Lecture 3.3: Wick's Theorem}
\date{February 17$^{\text{th}}$, 2020}
\begin{document}
\maketitle
\noindent
\section{A note on permutations} 
We will use the following to denote permutations on a set of indices: 
\[ x_{R_1} \cd x_{R_N} = (-1)^R x_1 \cd x_N \]
R is the permutation of the set $\{x_1, \cd, x_N\}$, and belongs to the permutation group $S_N$ ($R \in S_N$). 
It can be represented as: 
\begin{equation*}
R = 
\begin{pmatrix}
1 & 2 & \cdots & N  \\
R_1& R_2& \cdots & R_N \\
\end{pmatrix}
\end{equation*}
The first row corresponds to the original indices while the second row corresponds to the permuted indices. 
R can also be written in cyclic notation to easily determine the sign of the permutation. It is easiest to show this with an example: \\ \\
\[ x_2 x_3 x_1 x_5 x_4  = (-1)^R x_1 x_2 x_3 x_4 x_5 \]
Permutation R can be expressed as:
\begin{equation*}
R = 
\begin{pmatrix}
1 & 2 &3 & 4 & 5  \\
2 & 3 & 1 & 5 & 4 \\
\end{pmatrix}
\end{equation*}
where the permuted indices $R_1$ corresponds to index 2 in the original string and so on. \\ \\
To determine the cyclic permutations, we pick one index and follow the permutations until we get back to that same index. 
We can see that here, index 1 turns to 2, index 2 turns to 3, and index 3 turns to 1.
Thus, (123) represents a cyclic permutation. 
In the same way, index 4 becomes index 5, and index 5 becomes index 4, and (45) represents another cyclic permutation. 
\\ \\
In cyclic notation, we can write: 
\begin{equation*}
R = 
\begin{pmatrix}
1 & 2 &3 & 4 & 5  \\
2 & 3 & 1 & 5 & 4 \\
\end{pmatrix}
= (123)(45)
\end{equation*}
An odd number of elements in a particular cyclic permutation corresponds to a positive sign (+) for the permutation.
An even number of elements in a particular cyclic permutation corresponds to a negative sign ($-$) for the permutation. 
In the above example, we have one cyclic permutation with a positive sign and one cyclic permutation with a negative sign, so the overall sign of the permutation is negative.
We thus have: 
\[ x_2 x_3 x_1 x_5 x_4  = -x_1 x_2 x_3 x_4 x_5 \]
\\
Here is another example where one index does not change its place: 
\[ x_1 x_3 x_2 x_5 x_4  = (-1)^R x_1 x_2 x_3 x_4 x_5 \]
\begin{equation*}
R = 
\begin{pmatrix}
1 & 2 &3 & 4 & 5  \\
1 & 3 & 2 & 5 & 4 \\
\end{pmatrix}
= (1)(23)(45) = (+)(-)(-) = (+)
\end{equation*}
\[ x_1 x_3 x_2 x_5 x_4  =  x_1 x_2 x_3 x_4 x_5 \]

\section{Prerequisites to Wick's Theorem}

Last time, we determined that much of the math we have to solve boils down to evaluating expectation values of 
strings of creation and annihilation operators in the true vacuum.
Using anti-communtation rules to do this was tedious, and we wanted to find a more efficient way. 
Our answer is to use Wick's Theorem. 
We first need to introduce three concepts needed to present Wick's Theorem: normal products, contractions, and normal products with contractions.

\subsection{Normal Products}
A normal product is a product of creation and annihilation operators where all the creation operators are to the left of the expression.
We say that a product in which all creation operators are to the left is ``normal ordered". 
We denote a normal product of a string of operators $x_1 \cd x_m$ (where $x_i$ can represent a creation or annihilation operator) with the notation ``$n[x_1 \cd x_m]$'', and formally define it as: 
\[n[x_1 \cd x_m]= (-1)^R a_{R_1}\dg \cd a_{R_j}\dg a_{R_{j+1}} \cd a_{R_m} \]
\[
R = 
\begin{pmatrix}
1 & 2 &\cd & j & j+1 & \cd & m  \\
R_1 & R_2 &\cd & R_j & R_{j+1} & \cd & R_m  \\
\end{pmatrix}
\] \\
The normal product of an empty set is defined to be one:
\[ n[\varnothing] = 1\]

\subsubsection{Example 1}
\[n[a_p\dg a_q a_r ] = (-)^R a_p\dg a_r\dg a_q \]\\
\[
R = 
\begin{pmatrix}
p & q & r  \\
p & r & q  \\
\end{pmatrix}
= (p)(qr) = (+)(-) = (-)
\] \\
\[n[a_p\dg a_q a_r ] = - a_p\dg a_r\dg a_q \]
Equivalently, we can write: 
\[n[a_p\dg a_q a_r ] = (-)^R a_r\dg a_p\dg a_q \]\\
\[
R = 
\begin{pmatrix}
p & q & r  \\
r & p & q  \\
\end{pmatrix}
= (prq) = (+) 
\] \\
\[n[a_p\dg a_q a_r ] =  a_r\dg a_q\dg a_q \]
Thus we see normal products can be written in different ways. 
\subsubsection{Example 2}
\[ n[a_p\dg a_q\dg a_r\dg] = a_p\dg a_q\dg a_r\dg \]
Here, the string is already normal ordered, so we can just drop the normal product notation. 
\subsubsection{Properties of Normal Products}
There are several properties of normal products that are useful to us: 
\begin{enumerate}
\item A normal product of a normal ordered string of operators is just that string of operators:
\[n[a_{R_1}\dg \cd a_{R_j}\dg a_{R_{j+1}} \cd a_{R_m} ] = a_{R_1}\dg \cd a_{R_j}\dg a_{R_{j+1}} \cd a_{R_m}  \]
We note that a string of only creation operators or only annihilation operators are already in normal order. 
\item A normal product of a normal product is just the normal product: 
\[n[n[x_1 \cd x_m]] = n[x_1 \cd x_m] \]
This is simply a manifestation of property 1. 
\item You can freely permute a string of operators, with an appropriate phase factor, as long as it is within the normal ordered form:
\[n[x_{R_1} \cd x_{R_m} ] = (-1)^R n[ x_1 \cd x_m]\]
For example,
\[n[x_2 x_1] = -n[x_1 x_2] \]
\end{enumerate}

\subsection{Contractions of Operators}

We \textit{define} a contraction between 2 operators $x_1$ and $x_2$ where $x_i$ can be either a creation or an annihilation operator, as the quantity: 
$$
\contraction{}{x}{{}_1}{x}
x_1 x_2 = x_1x_2 - n[x_1x_2]
$$
We can derive the value of the contraction for various combinations of creation and annihilation operators by evaluating the expression. 
We will make use of the properties listed above as well as remembering our anticommutation relations.
We have four cases to consider: 
\begin{enumerate}
\item $x_1 = a_p$, $x_2 = a_q$ \\
$$
\contraction{}{a}{{}_p}{a}
a_p a_q = a_p a_q  - n[a_p a_q] = a_p a_q  - a_p a_q  = 0 
$$
\item $x_1 = a_p$, $x_2 = a_q\dg$ \\
$$
\contraction{}{a}{{}_p}{a}
a_p a_q\dg = a_p a_q\dg  - n[a_p a_q\dg] = a_p a_q\dg  +  a_q\dg a_p = \delta_{pq}
$$
\item $x_1 = a_p\dg$, $x_2 = a_q$ \\
$$
\contraction{}{a}{{}_p}{a}
a_p\dg a_q = a_p\dg a_q  - n[a_p\dg a_q] = a_p\dg a_q  - a_p\dg a_q  = 0 
$$
\item $x_1 = a_p\dg$, $x_2 = a_q\dg$ \\
$$
\contraction{}{a}{{}_p}{a}
a_p\dg a_q\dg = a_p\dg a_q\dg  - n[a_p\dg a_q\dg] = a_p\dg a_q\dg  - a_p\dg a_q\dg  = 0 
$$

\end{enumerate}
\subsection{Normal Products with Contractions}
A normal products with contractions inside
\[ 
\contraction{n[x_1 \cd }{x}{{}_{i_1} \cd x_{i_\lambda} \cd }{x}
\contraction[2ex]{n[x_1 \cd x_{i_1} \cd}{ x}{{}_{i_\lambda} \cd x_{j_1} \cd }{x} 
n[x_1 \cd x_{i_1} \cd x_{i_\lambda} \cd x_{j_1} \cd x_{j_\lambda} \cd x_m] 
\] 
can be rewritten as: 
\[ 
\contraction{n[x_1 \cd }{x}{{}_{i_1} \cd x_{i_\lambda} \cd }{x}
\contraction[2ex]{n[x_1 \cd x_{i_1} \cd}{ x}{{}_{i_\lambda} \cd x_{j_1} \cd }{x} 
n[x_1 \cd x_{i_1} \cd x_{i_\lambda} \cd x_{j_1} \cd x_{j_\lambda} \cd x_m] 
= 
\contraction{(-1)^R}{x}{{}_{i_1} }{x}
\contraction{(-1)^R x_{i_1}x_{j_1} \cd}{x}{{}_{i_\lambda} }{x}
(-1)^R x_{i_1}x_{j_1} \cd x_{i_\lambda} x_{j_\lambda} n[x_{k_1} \cd x_{k_\mu}]
\] 
where 
\[
R = 
\begin{pmatrix}
1 & 2 & \cd & x_m  \\
i_1 & j_1 & \cd& k_\mu \\
\end{pmatrix}
\] \\
and the indices $2\lambda + \mu = m$.
In words, a normal product with contractions can be written as a product of contractions outside the normal product multiplied by the uncontracted terms and a phase factor. 
It is important to note that when the contractions are taken out, you cannot flip the order of the contraction.
In other words, 
\[ \contraction{}{x}{{}_p}{x}
x_p x_q \neq \pm x_q x_p  \]
\subsubsection{Example 1}
\[\contraction{n[}{a}{{}_p a_q }{a}
\contraction{n[a_p a_q a_r\dg] = (-)^R a}{a}{_p}{a}
n[a_p a_q a_r\dg] = (-1)^R a_p a_r\dg n[a_q] \] \\
\[
R = 
\begin{pmatrix}
p & q & r  \\
p & r& q\
\end{pmatrix}
= (p)(qr) = (-)
\] \\
\[\contraction{n[}{a}{{}_p a_q }{a}
\contraction{n[a_p a_q a_r\dg] = -}{a}{{}_p}{a}
n[a_p a_q a_r\dg] = - a_p a_r\dg n[a_q] = -\delta_{pr}a_q \]
\subsubsection{Two tips}
There are two tips for faster evaluation of normal products with contractions. \\ \\
First, for a \textit{fully contracted} term, meaning that all operators are involved in a contraction, 
the phase factor can be obtained by counting the number of intersecting contraction lines $\nu$: 
\[(-1)^R = (-1)^\nu\]
For example, the term 
$\contraction{n[}{a}{{}_p}{a}
\contraction{n[a_p a_q  }{a}{{}_r\dg }{a}
\contraction{n[a_p a_q a_r\dg a_s }{a}{{}_t\dg}{a}
n[a_p a_q a_r\dg a_s a_t\dg a_u\dg] $
has 0 intersecting contraction lines, and will have an overall phase factor of (+1).  \\ \\
The term 
$\contraction{n[}{a}{{}_p a_q}{a}
\contraction[2ex]{n[a_p }{a}{{}_q a_r\dg }{a}
\contraction{n[a_p a_q a_r\dg a_s  }{a}{{}_t\dg}{a}
n[a_p a_q a_r\dg a_s a_t\dg a_u\dg] $ has 1 intersection, and will have an overall phase factor of (-1).  \\ \\
The term
$\contraction{n[}{a}{{}_p a_q}{a}
\contraction[2ex]{n[a_p }{a}{{}_q a_r\dg a_s }{a}
\contraction{n[a_p a_q a_r\dg }{a}{{}_s a_t\dg }{a}
n[a_p a_q a_r\dg a_s a_t\dg a_u\dg] $ has 2 intersections, and will have an overall phase factor of (+1). \\ \\
We will use the following notation to denote a normal product which is fully contracted: 
\[n\ol{\ol{[x_1 \cd x_m]}} \]
The above three examples can all be expressed with this notation. \\ \\
If you have a group of fully contracted operators within your normal product, you can take the full group out and use the number of intersecting contraction lines
to determine the permutation associated with taking the group out.  \\ \\
For example, in the term 
\[\contraction{n[}{a}{{}_p a_q}{a}
\contraction[2ex]{n[a_p }{a}{{}_q a_r\dg a_s }{a}
\contraction{n[a_p a_q a_r\dg }{a}{{}_s a_t\dg }{a}
\contraction{n[a_p a_q a_r\dg a_s a_t\dg a_u\dg a_v\dg}{a}{{}_w a_x}{a}
n[a_p a_q a_r\dg a_s a_t\dg a_u\dg a_v\dg a_w a_x a_y\dg] \]
the first six operators are fully contracted, and has an overall phase factor of (+) because there are 2 intersections.
As a result, we can take it out directly of the normal product: 
\[\contraction{n[}{a}{{}_p a_q}{a}
\contraction[2ex]{n[a_p }{a}{{}_q a_r\dg a_s }{a}
\contraction{n[a_p a_q a_r\dg }{a}{{}_s a_t\dg }{a}
\contraction{n[a_p a_q a_r\dg a_s a_t\dg a_u\dg a_v\dg}{a}{{}_w a_x}{a}
n[{\color{blue} a_p a_q a_r\dg a_s a_t\dg a_u\dg} a_v\dg a_w a_x a_y\dg] =  
\contraction{}{a}{{}_p}{a}
\contraction{a_p a_r\dg }{a}{{}_q }{a}
\contraction{a_p a_r\dg a_q a_t\dg  }{a}{{}_s}{a}
\contraction{a_p a_r\dg a_q a_t\dg a_s  a_u\dg n[a_v\dg}{a}{{}_w a_x}{a}
{\color{blue} a_p a_r\dg a_q a_t\dg a_s  a_u\dg }n[a_v\dg a_w a_x a_y\dg]\]
\\ \\
The second tip is that neighboring contractions can be taken out of the normal product without considering the permutation.
By ``neighboring contraction", I mean contractions between operators that are next to each other.
This is because the permutation associated with a neighboring pair is always positive. 
For example,
\[
\contraction[2ex]{n[}{a}{{}_p{\color{blue} a_q a_r\dg } a_s}{a}
\contraction{n[a_p }{a}{{}_q}{a}
n[a_p {\color{blue} a_q a_r\dg } a_s a_t\dg a_u\dg] = 
\contraction{}{a}{{}_q}{a}
\contraction{{\color{blue} a_q a_r\dg }  n[}{a}{{}_p a_s}{a}
{\color{blue} a_q a_r\dg }  n[a_p a_s a_t\dg a_u\dg] \]
\subsection{Matrix elements of normal products with contractions}
Remember that our ultimate goal is to know how to quickly evaluate matrix elements of strings of operators in the vacuum. 
We are now ready to prove some rules for determining how a normal ordered string of operators acts in the true vacuum.

\begin{itemize}
\item Rule 1: $n[x_1 \cd x_m]\ket{)} = 0$ unless all $x_1 \cd x_m$ are creation operators \\

Proof: If all operators are creation operators, 
\[n[a_{p_1}\dg \cd a_{p_m}\dg]\ket{0} = a_{p_1}\dg \cd a_{p_m}\dg\ket{0} = \ket{\Phi_{p_1 \cd p_m}} \neq 0\]

If at least one $x_i$ is an annihilation operator, it will be the rightmost term by the definition of the normal product, and will annihilate the vacuum.
The overall result is zero.  
\item Rule 2: $\braket{0 | n[x_1 \cd x_m] | 0} = 0$ for  $m \geq 1$ \\

Proof: If all operators are creation operators, they will annihilate the vacuum because 
\[(a_p\ket{0})\dg = \bra{0}a_p\dg = 0\]
If one or more operators are annihilation operators, they will be on the right by the definition of the normal product and annihilate the vacuum. 
\item Rule 3. $\contraction{n[x_1 \cd }{x}{{}_{i_1} \cd x_{i_\lambda} \cd }{x}
\contraction[2ex]{n[x_1 \cd x_{i_1} \cd}{ x}{{}_{i_\lambda} \cd x_{j_1} \cd }{x} 
n[x_1 \cd x_{i_1} \cd x_{i_\lambda} \cd x_{j_1} \cd x_{j_\lambda} \cd x_m] \ket{0} = 0$ if there is at least 1 uncontracted annihilator among operators $x_1 \cd x_m$.  \\

Proof:
\[ 
\contraction{n[x_1 \cd }{x}{{}_{i_1} \cd x_{i_\lambda} \cd }{x}
\contraction[2ex]{n[x_1 \cd x_{i_1} \cd}{ x}{{}_{i_\lambda} \cd x_{j_1} \cd }{x} 
n[x_1 \cd x_{i_1} \cd x_{i_\lambda} \cd x_{j_1} \cd x_{j_\lambda} \cd x_m] \ket{0}
= 
\contraction{(-1)^R}{x}{{}_{i_1} }{x}
\contraction{(-1)^R x_{i_1}x_{j_1} \cd}{x}{{}_{i_\lambda} }{x}
(-1)^R x_{i_1}x_{j_1} \cd x_{i_\lambda} x_{j_\lambda} n[x_{k_1} \cd x_{k_\mu}] \ket{0}
\] 
By Rule 1, $n[x_{k_1} \cd x_{k_\mu}] \ket{0} = 0$ unless all operators $x_{k_1} \cd x_{k_\mu}$ are creation operators. 
\item Rule 4. $\braket{0| \contraction{n[x_1 \cd }{x}{{}_{i_1} \cd x_{i_\lambda} \cd }{x}
\contraction[2ex]{n[x_1 \cd x_{i_1} \cd}{ x}{{}_{i_\lambda} \cd x_{j_1} \cd }{x} 
n[x_1 \cd x_{i_1} \cd x_{i_\lambda} \cd x_{j_1} \cd x_{j_\lambda} \cd x_m] |0} = 0$ unless all operators are contracted. 
In other words, only a fully contracted normal product ($n\ol{\ol{[x_1 \cd x_m] }}$) gives a nonzero expectation value in the true vacuum. \\

Proof: This is easily see as an application of Rule 2. Contractions of operators reduces to a number which can be pulled out of the expectation value.The only case where the expectation value is nonzero is one in which there is no operators left in the normal product, $\braket{0 | n[\varnothing] | 0} \neq 0$ \\

As a consequence of Rule 4, a vacuum expectation value of any normal product with contractions will be zero if there are a odd number of operators. 
\end{itemize}
We now know how to quickly evaluate matrix elements of a normal ordered string, and a normal ordered string with contractions.
Next, we need to know how to quickly put any string in normal order so we can take advantage of these shortcuts. 
This is what Wick's theorem provides us. 

\section{Wick's Theorem}
We will first start by seeing what happens if we put specific strings of operators in normal order just by applying anticommutation relations.  \\ 
Putting a product of 2 operators in normal order is a direct application of anticommuntation relations: 
\[a_p a_q\dg = \delta_{pq} - a_q\dg a_p \]
The first term, $\delta_{pq}$, can be re-expressed as $\contraction{}{a}{{}_p}{a}a_p a_q\dg$, as shown in Section 2.3. 
We can equivalently write $n[\contraction{}{a}{{}_p}{a}a_p a_q\dg]$.
You will show that $\contraction{}{a}{{}_p}{a}a_p a_q\dg = n[\contraction{}{a}{{}_p}{a}a_p a_q\dg]$ in the exercises. 
The second term is simply the normal product $n[a_p a_q\dg]$. 
Thus, we can write (switching the order of the first and second terms): 
\[a_p a_q\dg = n[a_p a_q\dg] + n[\contraction{}{a}{{}_p}{a}a_p a_q\dg] \]
\\
Let's try with a product of 3 operators, $a_p a_q\dg a_r\dg$: 
\begin{align*}
a_p a_q\dg a_r\dg &= (\delta_{pq} - a_q\dg a_p)a_r\dg \\
&= \delta_{pq}a_r\dg  - a_q\dg a_p a_r\dg \\
& = \delta_{pq}a_r\dg - a_q\dg (\delta_{pr} - a_r\dg a_p)  \\
& = \delta_{pq}a_r\dg - \delta_{pr} a_q\dg +  a_q\dg a_r\dg a_p  \\
\end{align*}
Rewriting the terms in a similar way as for the product of 2 operators, we obtain:
\[a_p a_q\dg a_r\dg  = n[a_p a_q\dg a_r\dg ] + n[\contraction{}{a}{{}_p}{a} a_p a_q\dg a_r\dg ]  + n[\contraction{}{a}{{}_p a_q\dg}{a} a_p a_q\dg a_r\dg ] \]
In these two examples, we observe that the products of the creation and annihilations operators could be expressed as its normal product plus all possible contractions of its normal product form.
Wick's theorem simply generalizes this observation to any string of creation and annihilation operators. 
Formally defined, Wick's theorem states: 
\begin{align*}
x_1 \cd x_m &= n[x_1 \cd x_m ]  \\
&+ \sum_{i < j} \contraction{n[x_1 \cd}{x}{{}_i \cd }{x} n[x_1 \cd x_i \cd x_j x_2] \\
&+ \sum_{\substack{i_1 < j_1,  i_2 < j_2 \\ i_1 < j_2, j_1 \neq j_2}} 
\contraction{n[x_1 \cd}{x}{{}_{i_1} \cd x_{i_2} \cd}{x}
\contraction[2ex]{n[x_1 \cd x_{i_1} \cd }{x}{{}_{i_2} \cd x_{j_1} \cd}{x}
n[x_1 \cd x_{i_1} \cd x_{i_2} \cd x_{j_1} \cd x_{j_2} \cd x_m ]  \\
&+ \cd \\
&+ \sum_{f.c.} n \ol{\ol{[x_1 \cd x_m]  }}
\end{align*}
where ``f.c." stands for fully contracted. 
In words, a string of operators can be written as its normal product plus all contractions of the normal product. 
We will adoption a short term notation for expressing all possible contractions of the normal product and write Wick's theorem as: 
\[x_1 \cd x_m = n[x_1 \cd x_m ]  + \sum_{a.c.} n\ol{[x_1 \cd x_m ]} \]\\
where ``a.c." stands for all contractions.
Let's explicitly write out Wick's theorem for a general product of 4 operators for a more concrete example: 
\begin{align*}
x_1x_2x_3x_4 &= n[x_1x_2x_3x_4 ] \\ 
&+n[ \ctr{}{x}{{}_1}{x} x_1x_2x_3x_4 ] + n[ \ctr{}{x}{{}_1x_2}{x} x_1x_2x_3x_4 ] + n[ \ctr{}{x}{{}_1x_2 x_3}{x} x_1x_2x_3x_4 ] 
+ n[ \ctr{x_1}{x}{{}_2}{x} x_1x_2x_3x_4 ] +  n[ \ctr{x_1}{x}{{}_2 x_3}{x} x_1x_2x_3x_4 ] + n[ \ctr{x_1 x_2}{x}{{}_3}{x} x_1x_2x_3x_4 ] \\
&+ n[ \ctr{}{x}{{}_1}{x}  \ctr{x_1 x_2}{x}{{}_3}{x} x_1x_2x_3x_4 ]
+ n[ \ctr{}{x}{{}_1x_2}{x}  \ctr[2ex]{x_1}{x}{{}_2 x_3}{x} x_1x_2x_3x_4 ] 
+ n[ \ctr{x_1}{x}{{}_2}{x} \ctr[2ex]{}{x}{{}_1x_2 x_3}{x}      x_1x_2x_3x_4 ] 
\end{align*}
\\
Wick's theorem is proved generally in the next set of lecture notes.

\subsection{Generalized Wick's Theorem}
In some cases, subgroups of operator strings are already in the normal product form.
This will allow us to simplify the terms we need to consider for Wick's theorem. \\ \\
Let's begin with an example: 
\[x_1 n[x_2 x_3] \] 
Wick's theorem for $x_2 x_3$ is
\[ x_2 x_3 = n[x_2 x_3] +   n[\contraction{}{x}{{}_2}{x} x_2 x_3] \]
We can rearrange this as: 
\[n[x_2 x_3]  = x_2 x_3  - n[\contraction{}{x}{{}_2}{x} x_2 x_3]  \]
We can substitute this into our original expression: 
\begin{align*}
x_1 n[x_2 x_3] &= x_1 (x_2 x_3  - n[\contraction{}{x}{{}_2}{x} x_2 x_3] ) \\
&= {\color{blue} x_1 x_2 x_3} - x_1n[\contraction{}{x}{{}_2}{x} x_2 x_3] \\
&= {\color{blue} x_1 x_2 x_3} - x_1\contraction{}{x}{{}_2}{x} x_2 x_3 \\
&= {\color{blue} n[x_1x_2x_3] + n[\contraction{}{x}{{}_1}{x} x_1 x_2 x_3] +n[\contraction{}{x}{{}_1 x_2}{x} x_1 x_2 x_3] + n[\contraction{x_1}{x}{{}_2}{x} x_1 x_2 x_3] }  - x_1\contraction{}{x}{{}_2}{x} x_2 x_3 \\
&=  {\color{blue} n[x_1x_2x_3] + n[\contraction{}{x}{{}_1}{x} x_1 x_2 x_3] +n[\contraction{}{x}{{}_1 x_2}{x} x_1 x_2 x_3] + \contraction{}{x}{{}_2}{x} x_2 x_3 n[x_1] }  - x_1\contraction{}{x}{{}_2}{x} x_2 x_3 \\
&= {\color{blue} n[x_1x_2x_3] + n[\contraction{}{x}{{}_1}{x} x_1 x_2 x_3] +n[\contraction{}{x}{{}_1 x_2}{x} x_1 x_2 x_3] + x_1\contraction{}{x}{{}_2}{x} x_2 x_3  }  - x_1\contraction{}{x}{{}_2}{x} x_2 x_3 \\
&= \color{blue} n[x_1x_2x_3] + n[\contraction{}{x}{{}_1}{x} x_1 x_2 x_3] +n[\contraction{}{x}{{}_1 x_2}{x} x_1 x_2 x_3] 
\end{align*}
We observe that the contraction between the operators originally in normal ordered form cancel out in the final expression.
\\ \\
This is an example of the generalized Wick's theorem, which formally states: 
$$x_1 \cd x_m = n[x_1 \cd x_m ]  + \sum_{a.c.} {}' n\ol{[x_1 \cd x_m ]} $$
where $\sum_{a.c.} {}' $ denotes skipping contractions of operators that originated from the same normal ordered group. 
\\ \\
Thus, we see it is to our benefit to have as many operators as we can in normal ordered form because then we do not have to consider contractions between the normal ordered operators. 
The generalized Wick's theorem is proved in the next set of lecture notes.

\subsection{Expectation values with Wick's theorem }
In Section 2.4 we proved some rules for quickly evaluate normal products or normal products with contractions acting in the true vacuum. 
We have now introduced Wick's theorem (WT), which is a clever bookkeeping method to express any product of operators as normal products and normal products with contractions. 
We will now apply WT with the rules developed in Section 2.4 to start seeing how we can take advantage of WT when evaluating matrix elements of operator strings. 
\\ \\
We list the four previous rules here for easy reference: 
\begin{itemize}
\item Rule 1: $n[x_1 \cd x_m]\ket{)} = 0$ unless all $x_1 \cd x_m$ are creation operators \\
\item Rule 2: $\braket{0 | n[x_1 \cd x_m] | 0} = 0$ for  $m \geq 1$ \\
\item Rule 3. $\contraction{n[x_1 \cd }{x}{{}_{i_1} \cd x_{i_\lambda} \cd }{x}
\contraction[2ex]{n[x_1 \cd x_{i_1} \cd}{ x}{{}_{i_\lambda} \cd x_{j_1} \cd }{x} 
n[x_1 \cd x_{i_1} \cd x_{i_\lambda} \cd x_{j_1} \cd x_{j_\lambda} \cd x_m] \ket{0} = 0$ if there is at least 1 uncontracted annihilator among operators $x_1 \cd x_m$.  \\
\item Rule 4. $\braket{0| \contraction{n[x_1 \cd }{x}{{}_{i_1} \cd x_{i_\lambda} \cd }{x}
\contraction[2ex]{n[x_1 \cd x_{i_1} \cd}{ x}{{}_{i_\lambda} \cd x_{j_1} \cd }{x} 
n[x_1 \cd x_{i_1} \cd x_{i_\lambda} \cd x_{j_1} \cd x_{j_\lambda} \cd x_m] |0} = 0$ unless all operators are contracted.
A vacuum expectation value of any normal product with contractions will be zero if there are a odd number of operators. \\
\end{itemize}

We can use Rules 1--4 in conjunction with Wick's theorem to develop some more rules: 

\begin{itemize}
\item Rule 5 $\braket{0 | x_1 \cd x_{2m + 1} | 0} = 0 $ \\

Proof: 
\[\braket{0 | x_1 \cd x_{2m + 1} | 0}   =  \braket{0 | n[x_1 \cd x_{2m + 1}] | 0}  +  \sum_{a.c.} \braket{0 | n\ol{[x_1 \cd x_{2m + 1}]} | 0}  \]
The first term is zero by Rule 2 and the second term is zero by Rule 4

\item \item Rule 6 $\braket{0 | x_1 \cd x_{2m } | 0} = \sum_{f.c.} \braket{0 | \ol{\ol{ n[x_1 \cd x_{2m }]}}|0} = \sum_{f.c.} (-1)^R x_{i_1} x_{j_1} \cd x_{i_m} x_{i_m}$\\

\begin{equation*}
R = 
\begin{pmatrix}
1 & 2 &\cd  & 2m - 1 & 2m  \\
i_1 & j_1 & \cd  & i_m & j_m \\
\end{pmatrix}
\end{equation*}

Proof: 
\[\braket{0 | x_1 \cd x_{2m} | 0}   =  \braket{0 | n[x_1 \cd x_{2m}] | 0}  +  \sum_{a.c.} \braket{0 | n\ol{[x_1 \cd x_{2m}]} | 0}  \]
The first term is 0 by Rule 2 and the second term is $ \sum_{f.c.} \braket{0 | \ol{\ol{ n[x_1 \cd x_{2m }]}}|0}$ by Rule 4. 
\[ \sum_{f.c.} \braket{0 | \ol{\ol{ n[x_1 \cd x_{2m }]}}|0} =  \sum_{f.c.} (-1)^R x_{i_1} x_{j_1} \cd x_{i_m} x_{i_m} \braket{0 | n[\varnothing] | 0} = \sum_{f.c.} (-1)^R x_{i_1} x_{j_1} \cd x_{i_m} x_{i_m}  \]

\item Rule 7: $x_1 \cd x_m \ket{0} = \sum\limits_{annihilator f.c.} n\ol{[x_1 \cd x_m] }\ket{0} $
where ``annihilator f.c" denotes all contractions with the annihilation operator fully contracted.\\

Proof: 
\[x_1 \cd x_m \ket{0} = n[x_1 \cd x_m] \ket{0} + \sum_{a.c.} n\ol{[x_1 \cd x_m ]} \ket{0} \]
The first term is 0 unless it is all creation operators by Rule 1, and the second is 0 unless all annihilation operators are contracted by Rule 3. 
\end{itemize}
It is not important to remember which rule you are using when applying Wick's theorem to solve problems. 
Rather, we are simply listing the rules to document the important consequences of Wick's theorem that allows us to take shortcuts.
As long as you understand Wick's theorem and how we obtained the rules, you should be able to naturally apply them to problems. 

\section{Derivation of Slater's first rule using Wick's Theorem}

We showed in class that using anticommutation relations to derive Slater's first rule was too much work. 
We could only show that Slater's rule is true for Slater determinants with a few spin orbitals, but could not prove it generally without spending a lot of tedious time. 
Now, with Wick's theorem, we can prove Slater's first rule generally. 
\[\braket{\Phi_{p_1 \cd p_N} | \sum_{pq} h_{pq} a_p^\dagger a_q  +  \frac{1}{2}  \sum_{pqrs} \braket{pq | rs} a\dg_{p}a\dg_{q} a_s a_r  | \Phi_{p_1 \cd p_N}} = \sum_i h_{ii} + \frac{1}{2}\sum_{ij} \braket{ij || ij} \]

\subsection{One-body operator}
\[\braket{{\color{orange} \Phi_{p_1 \cd p_N} } | \sum_{pq}   h_{pq}{\color{blue}  a_p^\dagger a_q } | {\color{red} \Phi_{p_1 \cd p_N}}} = 
\sum_{pq}   h_{pq} \braket{ 0| {\color{orange} a_{p_N} \cd a_{p_1} } {\color{blue} a_p^\dagger a_q } {\color{red} a_{p_1}\dg \cd a_{p_N}\dg }| 0} \]
We know from Wick's theorem that the only nonzero terms will be the fully contracted terms. 
Furthermore, we know that we do not have to consider contractions within normal ordered groups. 
Thus, we do not consider contractions within the terms in blue, orange, or red, but only contractions between the different colors.  \\ \\ 
$ {\color{blue} a_p^\dagger}$ will only form nonzero contractions with ${\color{orange} a_{p_N} \cd a_{p_1} } $, and 
$ {\color{blue} a_q }$ will only form nonzero contractions with ${\color{red} a_{p_1}\dg \cd a_{p_N}\dg }$. 
After forming those two contractions, the remaining terms in orange and red have to form contractions with each other. 
However, the only nonzero contractions that they can form are between terms with the same indices ($\contraction{}{a}{{}_p}{a} a_p a_q\dg =  \delta_{pq}$)
This means that $ {\color{blue} a_p^\dagger}$ and  $ {\color{blue} a_q }$ can only form nonzero contractions if their respective orange and red contraction partners have the same indices as each other. 
\\ \\
To summarize, the only terms in the expectation value that does not vanish are ones in which the operators in blue are contracted with orange and red operators with the same indices (${\color{orange} a_{p_i} } $ and 
${\color{red} a_{p_i}\dg}$), and the remaining orange and red operators form contractions between operators with the same indices. There will be N contractions that can be formed, since the operators in blue can form 
contractions with any operators in the red and orange set as long as the red and orange operators have the same indices.  \\ \\
We can express this in the following way: 
\[\braket{\Phi | \hat{O}_1 | \Phi} = \sum_{pq}   h_{pq} \sum_i^N \braket{0| 
\contraction{a_{p_N} \cd}{a}{{}_{p_i} \cd  a_{p_1} }{a}  
\contraction{{\color{orange} a_{p_N} \cd a_{p_i} \cd  a_{p_1} } a_p^\dagger}{a}{{}_q a_{p_1}\dg \cd}{a}
\contraction[2ex]{ a_{p_N} \cd a_{p_i} \cd }{a}{{}_{p_1}  a_p^\dagger a_q }{a}
\contraction[3ex]{}{a}{{}_{p_N} \cd a_{p_i} \cd  a_{p_1}{\color{blue} a_p^\dagger a_q } a_{p_1}\dg \cd a_{p_i}\dg \cd}{a}
{\color{orange} a_{p_N} \cd a_{p_i} \cd  a_{p_1} } {\color{blue} a_p^\dagger a_q } {\color{red} a_{p_1}\dg \cd a_{p_i}\dg \cd a_{p_N}\dg } |0} \]
We notice 2 things: 
\begin{enumerate}
\item The number of intersecting contraction lines will always be even, and so the phase factor to move all contractions out of the expectation value is positive.
\item The contractions $\contraction{}{a}{{}_p}{a} {\color{orange} a_{j}}{\color{red} a_j\dg }=  \delta_{jj}$ will collapse to products of 1 for all j. 
\end{enumerate}
This means we are left with: 
\begin{align*}
\braket{\Phi | \hat{O}_1 | \Phi} &= \sum_i^N  \sum_{pq}   h_{pq} \delta_{p p_i}   \delta_{q p_i}  \braket{0| 0} \\
&= \sum_i^N h_{p_i p_i}  \quad \text{or just }\\
&= \sum_i^N h_{i i} 
\end{align*}

\subsection{Two-body operator }
\[\braket{{\color{orange} \Phi_{p_1 \cd p_N} } |  \frac{1}{2}  \sum_{pqrs} \braket{pq | rs} {\color{blue}  a_p\dg a_q\dg a_s a_r } | {\color{red} \Phi_{p_1 \cd p_N}}} = 
 \frac{1}{2}  \sum_{pqrs} \braket{pq | rs} \braket{ 0| {\color{orange} a_{p_N} \cd a_{p_1} } {\color{blue} a_p\dg a_q\dg a_s a_r} {\color{red} a_{p_1}\dg \cd a_{p_N}\dg }| 0} \]
In a similar manner as before, we are limited in the nonzero contractions we can form. 
Operators  ${\color{blue}  a_s}$ and ${\color{blue}  a_r}$ can form contractions with ${\color{red} a_{p_i}\dg } $ and ${\color{red} a_{p_j}\dg}$,
and operators $ {\color{blue}  a_p\dg}$ and ${\color{blue}  a_q\dg}$ can form contractions with  $ {\color{orange} a_{p_i} }$ and ${\color{orange} a_{p_j}}$.
There are four total combinations of contractions between these 8 operators.
The remaining uncontracted operators will form contractions between operators with the same indices and collapse to products of 1. 
We will write the four types of contractions that can formed (contractions between red and orange operators assumed and not written explicitly): 
\begin{align*}
\braket{{\color{orange} \Phi_{p_1 \cd p_N} } | \hat{O}_2  | {\color{red} \Phi_{p_1 \cd p_N}}} &= 
 \frac{1}{2}  \sum_{pqrs} \braket{pq | rs} \sum_{i < j} \braket{ 0| 
 \contraction{  {\color{orange} \cd a_{p_j} \cd a_{p_i} \cd } a_p\dg a_q\dg a_s }{a}{{}_r \cd }{a}
 \contraction{ \cd a_{p_j} \cd }{a}{{}_{p_i}  \cd  }{ }
 \contraction[2ex]{ {\color{orange} \cd a_{p_j} \cd a_{p_i} \cd } a_p\dg a_q\dg}{a}{{}_s a_r \cd  a_{p_i}\dg \cd}{a}
 \contraction[2ex] {\cd}{a}{{}_{p_j} \cd a_{p_i} \cd a_p\dg}{}
 {\color{orange} \cd a_{p_j} \cd a_{p_i} \cd } {\color{blue} a_p\dg a_q\dg a_s a_r} {\color{red}\cd  a_{p_i}\dg \cd a_{p_j}\dg \cd }| 0} \\ 
 &+  \frac{1}{2}  \sum_{pqrs} \braket{pq | rs} \sum_{i < j} \braket{ 0| 
 \contraction{  {\color{orange} \cd a_{p_j} \cd a_{p_i} \cd } a_p\dg a_q\dg a_s }{a}{{}_r \cd }{a}
  \contraction[2ex]{ {\color{orange} \cd a_{p_j} \cd a_{p_i} \cd } a_p\dg a_q\dg}{a}{{}_s a_r \cd  a_{p_i}\dg \cd}{a}
 \contraction{ \cd a_{p_j} \cd }{a}{{}_{p_i} \cd a_p\dg  }{ }
 \contraction[2ex] {\cd}{a}{{}_{p_j} \cd a_{p_i}  \cd }{}
 {\color{orange} \cd a_{p_j} \cd a_{p_i} \cd } {\color{blue} a_p\dg a_q\dg a_s a_r} {\color{red}\cd  a_{p_i}\dg \cd a_{p_j}\dg \cd }| 0} \\
 &+  \frac{1}{2}  \sum_{pqrs} \braket{pq | rs} \sum_{i < j} \braket{ 0| 
 \contraction{  {\color{orange} \cd a_{p_j} \cd a_{p_i} \cd } a_p\dg a_q\dg a_s }{a}{{}_r \cd  a_{p_i}\dg \cd }{a}
  \contraction[2ex]{ {\color{orange} \cd a_{p_j} \cd a_{p_i} \cd } a_p\dg a_q\dg}{a}{{}_s a_r \cd }{a}
 \contraction{ \cd a_{p_j} \cd }{a}{{}_{p_i}  \cd  }{ }
 \contraction[2ex] {\cd}{a}{{}_{p_j} \cd a_{p_i} \cd a_p\dg}{}
 {\color{orange} \cd a_{p_j} \cd a_{p_i} \cd } {\color{blue} a_p\dg a_q\dg a_s a_r} {\color{red}\cd  a_{p_i}\dg \cd a_{p_j}\dg \cd }| 0} \\ 
 &+  \frac{1}{2}  \sum_{pqrs} \braket{pq | rs} \sum_{i < j} \braket{ 0| 
 \contraction{  {\color{orange} \cd a_{p_j} \cd a_{p_i} \cd } a_p\dg a_q\dg a_s }{a}{{}_r \cd  a_{p_i}\dg \cd }{a}
  \contraction[2ex]{ {\color{orange} \cd a_{p_j} \cd a_{p_i} \cd } a_p\dg a_q\dg}{a}{{}_s a_r \cd }{a}
 \contraction{ \cd a_{p_j} \cd }{a}{{}_{p_i} \cd a_p\dg  }{ }
 \contraction[2ex] {\cd}{a}{{}_{p_j} \cd a_{p_i}  \cd }{}
 {\color{orange} \cd a_{p_j} \cd a_{p_i} \cd } {\color{blue} a_p\dg a_q\dg a_s a_r} {\color{red}\cd  a_{p_i}\dg \cd a_{p_j}\dg \cd }| 0} \\ 
 &=  \frac{1}{2}  \sum_{pqrs} \braket{pq | rs} \sum_{i < j} \left[ \delta_{p p_i} \delta_{q p_j} \delta_{r p_i} \delta_{s p_j} 
 - \delta_{p p_j} \delta_{q p_i} \delta_{r p_i} \delta_{s p_j} 
 - \delta_{p p_i} \delta_{q p_j} \delta_{r p_j} \delta_{s p_i} 
 + \delta_{p p_j} \delta_{q p_i} \delta_{r p_j} \delta_{s p_i} 
 \right]  \\
 &=  \frac{1}{2}\sum_{i < j} \left[  \braket{p_i p_j | p_i p_j}  - \braket{p_j p_i | p_i p_j}  - \braket{p_i p_j | p_j p_i} + \braket{p_j p_i | p_j p_i}  \right] \\
 &= \frac{1}{2}  \sum_{i < j} 2 \left[  \braket{p_i p_j | p_i p_j}  - \braket{p_i p_j | p_j p_i} \right] \\
  &= \frac{1}{2} \sum_{i j} \frac{1}{2} (2) \left[  \braket{p_i p_j | p_i p_j}  - \braket{p_i p_j | p_j p_i} \right] \\
  &=  \frac{1}{2} \sum_{i j}  \braket{p_i p_j || p_i p_j}   \quad \text{or just }\\
&=  \frac{1}{2} \sum_{i j}  \braket{ij || ij} 
\end{align*}

%We are still not quite where we want to be in terms of quickly and efficiently evaluating matrix elements of operator strings. 
%The derivation of Slater's rules still took us two pages. 
%Next, we will find ways to 
% 3_Wicks_Theorem_b: 
%\subsection{Proof of 2 Lemmas}
%\subsection{Proof of Wick's Theorem}
%\subsection{Proof of Generalized Wick's Theorem}


% 3_Particle_Hole_a: 
%\section{Particle-Hole Formalism}
%\subsection{Motivating the particle-hole formalism}
%\subsection{Normal Products}
%\subsection{Contractions of Operators}
%\subsection{Normal Products with Contractions}
%\subsection{Wick's Theorem and generalized Wick's Theorem}

%(at the end have a change comparing quantities in the true vs fermi vacuum
%CI stuff - relate the grid to the overlap of SD is SD of overlaps stuff?
% 3_Particle_Hole_b: 
%\section{$\Phi$--Normal ordered Hamiltonian}
%\section{Derivation of Slater's first rule in the Fermi vacuum}
%\section{Derivation of Slater's second rule in the Fermi vacuum}
%\section{Derivation of Slater's third rule in the Fermi vacuum}

%(note about second and third rules - can prove those as well! - what the nonzero tells us about implications of spin orbitals differing)! 
%math with expectation value inform which specific value for integral. 

%Examples for class lecture: explicitly write wick's thrm for 3, 4 operators 
%Keep in mind these are all for true vacuum! 

\begin{comment}
Types of problems:
1. Slater's rules
Types of ways to solve:
1. First quantization
2. 2nd quantization in the true vacuum without Wick's theorem
3. 2nd quantization in the true vacuum with Wick's theorem
4. PH formalism with Wick's theorem 
5. PH formalism with Wick's theorem and $\Phi$-normal ordered Hamiltonian
\end{comment}
\end{document}
