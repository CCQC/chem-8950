\documentclass{article}
\usepackage{simplewick}
\usepackage{amsmath,mathtools,amssymb} \usepackage{graphicx}
\usepackage{gensymb}
\usepackage{verbatim}
\usepackage{cancel}
\usepackage{mathrsfs}
\usepackage{bbm}
\usepackage{braket}
\usepackage{xcolor}
\definecolor{orange}{RGB}{255,127,0}
\definecolor{green}{RGB}{34,139,34}
\definecolor{purple}{RGB}{153,50,204}
\definecolor{blue}{RGB}{70,130,180}
\usepackage{verbatim}
\usepackage[margin=1.0in]{geometry}
\newcommand{\ol}{\overline}
\newcommand{\ctr}{\bcontraction}
\newcommand{\ve}{\varepsilon}
\newcommand{\ap}{\ensuremath{a_p} }
\newcommand{\dg}{\ensuremath{^\dagger} }
\newcommand{\cd}{\ensuremath{\cdots} }
\newcommand{\apd}{\ensuremath{a_p^\dagger} }
\DeclarePairedDelimiter\floor{\lfloor}{\rfloor}
\def\*#1{\mathbf{#1}}
\def\cb#1{{\color{blue}#1}}
\def\co#1{{\color{orange}#1}}
\def\cre#1{{\color{red}#1}}
\def\cg#1{{\color{green}#1}}
\def\cp#1{{\color{purple}#1}}
\usepackage{simpler-wick}
\newif\ifWickCut
\WickCutfalse
\pgfkeys{
  /simplerwick/.cd,
  cut/.code={\WickCuttrue},
  no cut/.code={\WickCutfalse},
  cut ratio/.initial=0.5
}
\makeatletter
\def\swick@end#1#2{
  \swick@setfalse@#1
  \tikzexternaldisable
  \begin{tikzpicture}[remember picture, baseline=(swick-close#1.base)]
    \node[use as bounding box, inner sep=0pt, outer sep=0pt] (swick-close#1) {$\displaystyle #2$};
  \end{tikzpicture}
  \tikz[remember picture, overlay]
{\ifWickCut 
    \draw ($(swick-open#1.south) + (0, -1pt)$) 
          -- ($(swick-open#1.base) + (0,3pt -1*\swick@offset) + #1*(0, -1*\swick@sep)$) 
          -- ($($(swick-open#1.base)!\pgfkeysvalueof{/simplerwick/cut ratio}!(swick-close#1.base)$) + (0,3pt -1*\swick@offset) + #1*(0, -1*\swick@sep)$) 
\else         
    \draw ($(swick-open#1.south) + (0, -1pt)$) 
          -- ($(swick-open#1.base) + (0,3pt -1*\swick@offset) + #1*(0, -1*\swick@sep)$) 
          -- ($(swick-close#1.base) + (0,3pt -1*\swick@offset) + #1*(0, -1*\swick@sep)$) 
          -- ($(swick-close#1.south) + (0,-1pt)$);
\fi}
  \tikzexternalenable}
\makeatother

\title{Lecture 3.4: Deriving Wick's Theorem}
\date{}
\begin{document}
\maketitle
\noindent
\section{Induction Proofs}
We will use proofs by induction in order to prove Wick's theorem. 
An induction proof has two steps: 
\begin{enumerate}
\item Prove statement is true for n=1
\item Assuming that the statement is true for n=k, prove it is true for k+1
\end{enumerate}
The idea is that if you can show a statement is true for the first case n=1, 
and show that it is true if you add one to n, then it is generally true. 

\section{Proof of Wick's Theorem}

To prove Wick's Theorem, we will first prove two lemmas:\\ \\
Lemma 1: 
\[n[x_1 \cd x_m]x_{m+1} = n[x_1 \cd x_m x_{m+1}] + \sum_i^m n[\ctr{x_1 \cd}{x}{{}_i \cd x_m}{x} x_1 \cd x_i \cd x_m x_{m+1}]\]
Lemma 2: 
\[n[\ctr{x_1 \cd}{x}{{}_i \cdot}{\cdot}
\ctr[2ex]{x_1 \cd x }{{}_i }{\cdot \cdot}{\cdot}
x_1 \cd x_i \cdot \cdot \cdot x_m]x_{m+1}  = 
n[\ctr{x_1 \cd}{x}{{}_i \cdot}{\cdot}
\ctr[2ex]{x_1 \cd x }{{}_i }{\cdot \cdot}{\cdot}
x_1 \cd x_i \cdot \cdot \cdot x_m x_{m+1} ] + 
\sum_{\substack{j = 1 \\ j \notin C}}^m n[\ctr{x_1 \cd}{x}{{}_i \cdot}{\cdot}
\ctr[2ex]{x_1 \cd x }{{}_i }{\cdot \cdot}{\cdot}
\ctr{x_1 \cd x_i \cdot \cdot \cdot }{x}{{}_j \cd  x_m }{x}
x_1 \cd x_i \cdot \cdot \cdot x_j \cd x_m x_{m+1} ] 
\]
Then we will use these lemmas to prove Wick's Theorem by induction.
\subsection{Lemma 1}
\[n[x_1 \cd x_m]x_{m+1} = n[x_1 \cd x_m x_{m+1}] + \sum_i^m n[\ctr{x_1 \cd}{x}{{}_i \cd x_m}{x} x_1 \cd x_i \cd x_m x_{m+1}]\]
In words, this lemma says that if you multiply a creation or annihilation operator $x_{m+1}$ with a normal ordered string of creation and annihilation operators,
it is equal to the normal ordered string of creation and annihilation operators, including $x_{m+1}$,
plus all possible contractions involving $x_{m+1}$ and the rest of the operators. 
\\ \\
There are two cases to look at for this proof: 
A,  $x_{m+1}$ is an annihilation operator $a_r$ and 
B,  $x_{m+1}$ is a creation operator $a_r\dg$. 
\\ \\
Case A: 
\\ \\
The case of  $x_{m+1} = a_r$ is easy to prove. 
If $x_{m+1} = a_r$, then $a_r$ can simply be absorbed into the normal product because it is on the right side of the expression: 
\begin{align*}
n[x_1 \cd x_m]a_r = n[x_1 \cd x_m x_{m+1}a_r] \\
\end{align*}
Furthermore, all contractions involving $a_r$ has $a_r$ on the right side, and will go to 0. 
We have thus proved Lemma 1 for $x_{m+1} = a_r$. 
\\ \\
Case B: 
\\ \\
There are two subcases for case B, $x_{m+1} = a_r\dg$. \\ \\
Case B1: all $x_1 \cd x_m$ are annihilation operators, $a_{p_1} \cd a_{p_m}$. \\
Case B2: some $x_1 \cd x_m$ are creation operators.
\\ \\
We will first prove case B1 by induction. \\ \\
For m = 1: 
\begin{align*}
n[a_{p_1}]a_r\dg &= a_{p_1}a_r\dg\\
&= \cg{\delta_{p_1r} } \cb{-  a_r\dg a_{p_1}} \\
  &=  \cg{n[ \ctr{}{a}{{}_{p_1}}{a} a_{p_1}a_r\dg]} \cb{+ n[a_{p_1}a_r\dg] } \\
  &= n[a_{p_1}a_r\dg + n[ \ctr{}{a}{{}_{p_1}}{a} a_{p_1}a_r\dg]
\end{align*} 
Lemma 1 holds for m = 1. Now, we will take the induction step.
Assume that Lemma 1 is true for $m = l$, 
\[n[a_{p_1} \cd a_{p_l}]a_r\dg = n[a_{p_1} \cd a_{p_l} a_r\dg] + \sum_i^l n[\ctr{a_{p_1} \cd}{a}{{}_{p_i} \cd a_{p_l}}{a} a_{p_1} \cd a_{p_i} \cd a_{p_l} a_r\dg]\]
show that it is true for $m=l+1$:  \\ \\
We will first start by multiplying $a_{p_{l+1}}$ from the left on both sides: 
\[\cb{a_{p_{l + 1}}} n[a_{p_1} \cd a_{p_l}]a_r\dg = \cb{a_{p_{l + 1}}} n[a_{p_1} \cd a_{p_l} a_r\dg] + \sum_i^l \cb{a_{p_{l + 1}}}  n[\ctr{a_{p_1} \cd}{a}{{}_{p_i} \cd a_{p_l}}{a} a_{p_1} \cd a_{p_i} \cd a_{p_l} a_r\dg]\]
We can now rewrite the right side of the equation by realizing that $a_{p_1} \cd a_{p_l}$ is already in normal order and can be taken out of the normal product. 
After some rearrangement, we get: 
\begin{align*}
\cb{a_{p_{l + 1}}} n[a_{p_1} \cd a_{p_l}]a_r\dg &= \cb{a_{p_{l + 1}}} n[a_{p_1} \cd a_{p_l} a_r\dg] + \sum_i^l \cb{a_{p_{l + 1}}}  n[\ctr{a_{p_1} \cd}{a}{{}_{p_i} \cd a_{p_l}}{a} a_{p_1} \cd a_{p_i} \cd a_{p_l} a_r\dg] \\
\cb{a_{p_{l + 1}}a_{p_1} \cd a_{p_l} a_r\dg } &= ``\quad " \\%a_{p_{l + 1}} n[a_1 \cd a_l a_r\dg] +\sum_i^l a_{p_{l + 1}}  n[\ctr{a_1 \cd}{a}{{}_i \cd a_l}{a} a_1 \cd a_i \cd a_l a_r\dg] \\
\cb{(-1)^l a_{p_1} \cd a_{p_l} a_{p_{l + 1}} a_r\dg } &=  ``\quad "\\%a_{p_{l + 1}} n[a_1 \cd a_l a_r\dg] + \sum_i^l a_{p_{l + 1}}  n[\ctr{a_1 \cd}{a}{{}_i \cd a_l}{a} a_1 \cd a_i \cd a_l a_r\dg] \\
\cb{(-1)^l n[a_{p_1} \cd a_{p_l} a_{p_{l + 1}} ]a_r\dg } &=  ``\quad "  \\
\end{align*}
We now rearrange the first term on the righthand side of the equation.
We first write the normal product of $n[a_{p_1} \cd a_{p_l} a_r\dg]$, and then rewrite $a_{p_{l + 1}} a_r\dg$:
\begin{align*}
(-1)^l n[a_{p_1} \cd a_{p_l} a_{p_{l + 1}} ]a_r\dg &= \co{a_{p_{l + 1}} n[a_{p_1} \cd a_{p_l} a_r\dg] }+ \sum_i^l a_{p_{l + 1}}  n[\ctr{a_{p_1} \cd}{a}{{}_{p_i} \cd a_{p_l}}{a} a_{p_1} \cd a_{p_i} \cd a_{p_l} a_r\dg] \\
&= \co{ a_{p_{l + 1}} (-1)^l  a_r\dg a_{p_1} \cd a_{p_l}} +  ``\quad" \\
&= \co{  (-1)^l  a_{p_{l + 1}} a_r\dg a_{p_1} \cd a_{p_l}} +  ``\quad" \\
&= \co{  (-1)^l ( n[a_{p_{l + 1}} a_r\dg] + \ctr{}{a}{{}_{p_{l + 1}}}{a} a_{p_{l + 1}} a_r\dg)  a_{p_1} \cd a_{p_l}} +  ``\quad" \\
&= \co{  (-1)^l  n[a_{p_{l + 1}} a_r\dg] a_{p_1} \cd a_{p_l}+  (-1)^l \ctr{}{a}{{}_{p_{l + 1}}}{a} a_{p_{l + 1}} a_r\dg a_{p_1} \cd a_{p_l}} +  ``\quad" \\
\end{align*} 
We can now reorder the terms and manipulate them to be in normal ordering: 
\begin{align*}
&= \co{  (-1)^{l+1}  a_r\dg a_{p_{l + 1}} a_{p_1} \cd a_{p_l}+  (-1)^l n[\ctr{}{a}{{}_{p_{l + 1}}}{a} a_{p_{l + 1}} a_r\dg a_{p_1} \cd a_{p_l}}] +  ``\quad" \\
&= \co{(-1)^{l+1}(-1)^l a_r\dg  a_{p_1} \cd a_{p_l} a_{p_{l + 1}}  + (-1)^l n[a_{p_1} \cd a_{p_l}  \ctr{}{a}{{}_{p_{l + 1}}}{a} a_{p_{l + 1}} a_r\dg ]} +  ``\quad"  \\
&= \co{(-1)^{l+1}(-1)^l n[a_r\dg  a_{p_1} \cd a_{p_l} a_{p_{l + 1}} ] + (-1)^l n[a_{p_1} \cd a_{p_l}  \ctr{}{a}{{}_{p_{l + 1}}}{a} a_{p_{l + 1}} a_r\dg ]} +``\quad" \\
&= \co{(-1)^l n[ a_{p_1} \cd a_{p_l} a_{p_{l + 1}} a_r\dg  ] + (-1)^l n[a_{p_1} \cd a_{p_l}  \ctr{}{a}{{}_{p_{l + 1}}}{a} a_{p_{l + 1}} a_r\dg ]} +``\quad"  \\
\end{align*} 
Finally, we will rearrange the last term to put $a_{p_{l+1}}$ in the normal product: 
\begin{align*}
(-1)^l n[a_{p_1} \cd a_{p_l} a_{p_{l + 1}} ]a_r\dg 
&= (-1)^l n[ a_{p_1} \cd a_{p_l}a_{p_{l + 1}} a_r\dg  ] + (-1)^l n[a_{p_1} \cd a_{p_l}  \ctr{}{a}{{}_{p_{l + 1}}}{a} a_{p_{l + 1}} a_r\dg ] + \cre{\sum_i^l a_{p_{l + 1}}  n[\ctr{a_{p_1} \cd}{a}{{}_{p_i} \cd a_{p_l}}{a} a_{p_1} \cd a_{p_i} \cd a_{p_l} a_r\dg]} \\
&= ``\quad" +  \cre{\sum_i^l   n[a_{p_{l + 1}} \ctr{a_{p_1} \cd}{a}{{}_{p_i} \cd a_{p_l}}{a} a_{p_1} \cd a_{p_i} \cd a_{p_l} a_r\dg]} \\
&= ``\quad"  + \cre{\sum_i^l (-1)^l n[ \ctr{a_{p_1} \cd}{a}{{}_{p_i} \cd a_{p_l} a_{p_{l+1}}}{a} a_{p_1} \cd a_{p_i} \cd a_{p_l} a_{p_{l+1}} a_r\dg]} \\
\end{align*}
Our final expression is thus:
\begin{align*}
\cb{(-1)^l} n[a_{p_1} \cd a_{p_l} a_{p_{l + 1}} ]a_r\dg    &=  \cb{(-1)^l}n[ a_{p_1} \cd a_{p_l}a_{p_{l + 1}} a_r\dg  ]+  \cb{(-1)^l} n[a_{p_1} \cd a_{p_l}  \ctr{}{a}{{}_{p_{l + 1}}}{a} a_{p_{l + 1}} a_r\dg ] +  \cb{(-1)^l} \sum_i^l n[ \ctr{a_{p_1} \cd}{a}{{}_{p_i} \cd a_{p_l} a_{p_{l+1}}}{a} a_{p_1} \cd a_{p_i} \cd a_{p_l} a_{p_{l+1}} a_r\dg] \\
n[a_{p_1} \cd a_{p_l} a_{p_{l + 1}} ]a_r\dg    &= n[ a_{p_1} \cd a_{p_l}a_{p_{l + 1}} a_r\dg  ] +  n[a_{p_1} \cd a_{p_l}  \ctr{}{a}{{}_{p_{l + 1}}}{a} a_{p_{l + 1}} a_r\dg ]  +   \sum_i^l n[ \ctr{a_{p_1} \cd}{a}{{}_{p_i} \cd a_{p_l} a_{p_{l+1}}}{a} a_{p_1} \cd a_{p_i} \cd a_{p_l} a_{p_{l+1}} a_r\dg] \\
n[a_{p_1} \cd a_{p_l} a_{p_{l + 1}} ]a_r\dg  &= n[ a_{p_1} \cd a_{p_l}a_{p_{l + 1}} a_r\dg  ] +  
\sum_i^{l+1} n[ \ctr{a_{p_1} \cd}{a}{{}_{p_i} \cd a_{p_l} a_{p_{l+1}}}{a} a_{p_1} \cd a_{p_i} \cd a_{p_l} a_{p_{l+1}} a_r\dg]
\end{align*}  
This proves Lemma 1 for case B1. 
\\ \\  \\ 
Case B2: Prove 
\[n[x_1 \cd x_m]a_r\dg = n[x_1 \cd x_m a_r\dg] + \sum_i^m n[\ctr{x_1 \cd}{x}{{}_i \cd x_m}{x} x_1 \cd x_i \cd x_m a_r\dg]\]
where some $x_1 \cd x_m$ are creation operators. \\ \\
First, we can rewrite $n[x_1 \cd x_m]$ such that all the creation operators are to the left in the normal product:
\begin{align*}
n[x_1 \cd x_m]a_r\dg &= n[x_1 \cd x_m a_r\dg] + \sum_i^m n[\ctr{x_1 \cd}{x}{{}_i \cd x_m}{x} x_1 \cd x_i \cd x_m a_r\dg] \\ 
(-1)^R n[a_{p_{R_1}}\dg \cd a_{p_{R_k}}\dg a_{p_{R_{k+1}}} \cd a_{p_{R_m}}] a_r\dg &= 
(-1)^R n[a_{p_{R_1}}\dg \cd a_{p_{R_k}}\dg a_{p_{R_{k+1}}} \cd a_{p_{R_m}} a_r\dg ]   \\
&\quad+ \cb{(-1)^R  \sum_i^m  n[\ctr{ a_{p_{R_1}}\dg \cd a_{p_{R_k}}\dg a_{p_{R_{k+1}}} \cd}{a}{{}_{R_i}\cd a_{p_{R_m}} }{a} a_{p_{R_1}}\dg \cd a_{p_{R_k}}\dg a_{p_{R_{k+1}}} \cd a_{R_i}\cd a_{p_{R_m}} a_r\dg ] }\\
\end{align*}
Next, we can split the final term between contractions of $a_r\dg$ with creations operators $a_{p_{R_1}}\dg \cd a_{p_{R_k}}\dg $, and $a_r\dg$ with annihilation operators $a_{p_{R_{k+1}}} \cd a_{p_{R_m}}$
\begin{align*}
(-1)^R n[a_{p_{R_1}}\dg \cd a_{p_{R_k}}\dg a_{p_{R_{k+1}}} \cd a_{p_{R_m}}] a_r\dg
&= (-1)^R n[a_{p_{R_1}}\dg \cd a_{p_{R_k}}\dg a_{p_{R_{k+1}}} \cd a_{p_{R_m}} a_r\dg ]   \\
&\quad+ \cb{(-1)^R  \sum_{i=1}^k  n[\ctr{ a_{p_{R_1}}\dg \cd}{a}{{}_{R_i}\dg \cd a_{p_{R_k}}\dg a_{p_{R_{k+1}}} \cd a_{p_{R_m}}}{a} a_{p_{R_1}}\dg \cd a_{R_i}\dg \cd a_{p_{R_k}}\dg a_{p_{R_{k+1}}} \cd a_{p_{R_m}} a_r\dg ]} \\
&\quad+ \cb{(-1)^R  \sum_{i=k+1}^m  n[\ctr{ a_{p_{R_1}}\dg \cd a_{p_{R_k}}\dg a_{p_{R_{k+1}}} \cd}{a}{{}_{R_i}\cd a_{p_{R_m}} }{a} a_{p_{R_1}}\dg \cd a_{p_{R_k}}\dg a_{p_{R_{k+1}}} \cd a_{R_i}\cd a_{p_{R_m}} a_r\dg ] }\\
\end{align*}
All contractions between 2 creation operators is zero, so we just get: 
\begin{align*}
(-1)^R n[a_{p_{R_1}}\dg \cd a_{p_{R_k}}\dg a_{p_{R_{k+1}}} \cd a_{p_{R_m}}] a_r\dg
&= (-1)^R n[a_{p_{R_1}}\dg \cd a_{p_{R_k}}\dg a_{p_{R_{k+1}}} \cd a_{p_{R_m}} a_r\dg ]   \\
&\quad+ \cb{(-1)^R  \sum_{i=k+1}^m  n[\ctr{ a_{p_{R_1}}\dg \cd a_{p_{R_k}}\dg a_{p_{R_{k+1}}} \cd}{a}{{}_{R_i}\cd a_{p_{R_m}} }{a} a_{p_{R_1}}\dg \cd a_{p_{R_k}}\dg a_{p_{R_{k+1}}} \cd a_{R_i}\cd a_{p_{R_m}} a_r\dg ] }\\
\end{align*}
We can take $a_{p_{R_1}}\dg \cd a_{p_{R_k}}\dg$ out of the normal product because this substring is already in normal product form:
\begin{align*}
(-1)^R (a_{p_{R_1}}\dg \cd a_{p_{R_k}}\dg)  n[a_{p_{R_{k+1}}} \cd a_{p_{R_m}}] a_r\dg 
&= (-1)^R  (a_{p_{R_1}}\dg \cd a_{p_{R_k}}\dg)  n[a_{p_{R_{k+1}}} \cd a_{p_{R_m}}  a_r\dg]  \\
&\quad+ (-1)^R  (a_{p_{R_1}}\dg \cd a_{p_{R_k}}\dg) \sum_{i=k+1}^m n[\ctr{ a_{p_{R_{k+1}}} \cd}{a}{{}_{R_i}\cd a_{p_{R_m}} }{a} a_{p_{R_{k+1}}} \cd a_{R_i}\cd a_{p_{R_m}} a_r\dg ]  \\
 (a_{p_{R_1}}\dg \cd a_{p_{R_k}}\dg)  \co{ n[a_{p_{R_{k+1}}} \cd a_{p_{R_m}}] a_r\dg }
&= (a_{p_{R_1}}\dg \cd a_{p_{R_k}}\dg) \co{ n[a_{p_{R_{k+1}}} \cd a_{p_{R_m}}  a_r\dg] } \\
&\quad+  (a_{p_{R_1}}\dg \cd a_{p_{R_k}}\dg) \co{\sum_{i=k+1}^m n[\ctr{ a_{p_{R_{k+1}}} \cd}{a}{{}_{R_i}\cd a_{p_{R_m}} }{a} a_{p_{R_{k+1}}} \cd a_{R_i}\cd a_{p_{R_m}} a_r\dg ] }
\end{align*}
We see that what we have in orange is a statement of Lemma 1 in the B1 case.
Thus, in the B2 case, we just have a factor multiplied by a statement of Lemma 1 in the B1 case, which we have already proved to be true.
\\ \\
We have therefore proved Lemma 1 to be true. 
\subsection{Lemma 2}
Now, let's prove Lemma 2:
\[n[\ctr{x_1 \cd}{x}{{}_i \cdot}{\cdot}
\ctr[2ex]{x_1 \cd x }{{}_i }{\cdot \cdot}{\cdot}
x_1 \cd x_i \cdot \cdot \cdot x_m]x_{m+1}  = 
n[\ctr{x_1 \cd}{x}{{}_i \cdot}{\cdot}
\ctr[2ex]{x_1 \cd x }{{}_i }{\cdot \cdot}{\cdot}
x_1 \cd x_i \cdot \cdot \cdot x_m x_{m+1} ] + 
\sum_{\substack{j = 1 \\ j \notin C}}^m n[\ctr{x_1 \cd}{x}{{}_i \cdot}{\cdot}
\ctr[2ex]{x_1 \cd x }{{}_i }{\cdot \cdot}{\cdot}
\ctr{x_1 \cd x_i \cdot \cdot \cdot }{x}{{}_j \cd  x_m }{x}
x_1 \cd x_i \cdot \cdot \cdot x_j \cd x_m x_{m+1} ] 
\]
In words, Lemma 2 says that a normal product with some contractions multiplied by another operator $x_{m+1}$ is equal to 
a normal product including $x_{m+1}$ with the same contractions, plus a sum over j of the normal product including $x_{m+1}$ with the same contraction as before 
and an additional contraction between $x_{m+1}$ and a previously uncontracted operator $x_j$. 
Operators that are contracted belong to set $C$. 
Here, j does not include any operators that are already contracted $(j \notin C)$. 
\\
We will use the following index definitions in the proof: 
\[2\lambda + \mu = m\]
\[(i, j, \cd, i_\lambda, j\lambda) \in C\]
\[k_1 \cd k_\mu \notin C\]
First, a normal product with some contractions multiplied by an operator $x_{m+1}$ can be written
with the contractions taken out:
\[n[x_1 \cd \ctr{}{x}{{}_{i_1} \cd }{x}   x_{i_1} \cd  x_{j_1} \cd   \ctr{}{x}{{}_{i_\lambda} \cd }{x} x_{i_\lambda} \cd   x_{j_\lambda} \cd x_m]x_{m + 1} 
= (-1)^R \ctr{}{x}{{}_{i_1}}{x}  x_{i_1} x_{j_1}  \cd \ctr{}{x}{{}_{i_\lambda}}{x} x_{i_\lambda}  x_{j_\lambda} n[x_{k_1} \cd x_{k_\mu}] x_{m+1} \]
We can then use Lemma 1 to expand the expression: 
\begin{align*}
(-1)^R \ctr{}{x}{{}_{i_1}}{x}  x_{i_1} x_{j_1}  \cd \ctr{}{x}{{}_{i_\lambda}}{x} x_{i_\lambda}  x_{j_\lambda} \cb{n[x_{k_1} \cd x_{k_\mu}] x_{m+1}}
&= (-1)^R \ctr{}{x}{{}_{i_1}}{x}  x_{i_1} x_{j_1}  \cd \ctr{}{x}{{}_{i_\lambda}}{x} x_{i_\lambda}  x_{j_\lambda} \cb{n[x_{k_1} \cd x_{k_\mu}x_{m+1}] } \\
&\quad + (-1)^R \ctr{}{x}{{}_{i_1}}{x}  x_{i_1} x_{j_1}  \cd \ctr{}{x}{{}_{i_\lambda}}{x} x_{i_\lambda}  x_{j_\lambda}\cb{ \sum_{j=1}^k n[x_{k_1} \cd \ctr{}{x}{{}_{k_j} \cd  x_{k_\mu}}{x} x_{k_j} \cd x_{k_\mu}x_{m+1} ] } \\
\end{align*}
We can now put the contractions back in their original places and see that this is the statement of Lemma 2:
\begin{align*}
n[x_1 \cd \ctr{}{x}{{}_{i_1} \cd }{x}   x_{i_1} \cd  x_{j_1} \cd   \ctr{}{x}{{}_{i_\lambda} \cd }{x} x_{i_\lambda} \cd   x_{j_\lambda} \cd x_m]x_{m + 1} 
&= n[x_1 \cd \ctr{}{x}{{}_{i_1} \cd }{x} x_{i_1} \cd  x_{j_1} \cd   \ctr{}{x}{{}_{i_\lambda} \cd }{x} x_{i_\lambda} \cd   x_{j_\lambda} \cd x_m x_{m + 1} ] \\
&\quad + \sum_{j=1}^m  n[x_1 \cd \ctr{}{x}{{}_{i_1} \cd }{x} x_{i_1} \cd  x_{j_1} \cd \ctr[2ex]{}{x}{{}_{k_j} \cd  \ctr{}{x}{{}_{i_\lambda} \cd }{x} x_{i_\lambda} \cd   x_{j_\lambda} \cd x_m }{x}   x_{k_j} \cd  
\ctr{}{x}{{}_{i_\lambda} \cd }{x} x_{i_\lambda} \cd   x_{j_\lambda} \cd x_m x_{m + 1} ]
\end{align*}
We have thus proved Lemma 2 to be true. 

\subsection{Wick's Theorem}

We are now ready to prove Wick's theorem by induction. \\ \\
Formally defined, Wick's theorem states: 
\begin{align*}
x_1 \cd x_m &= n[x_1 \cd x_m ]  \\
&+ \sum_{i < j} \ctr{n[x_1 \cd}{x}{{}_i \cd }{x} n[x_1 \cd x_i \cd x_j x_2] \\
&+ \sum_{\substack{i_1 < j_1,  i_2 < j_2 \\ i_1 < j_2, j_1 \neq j_2}} 
\ctr{n[x_1 \cd}{x}{{}_{i_1} \cd x_{i_2} \cd}{x}
\ctr[2ex]{n[x_1 \cd x_{i_1} \cd }{x}{{}_{i_2} \cd x_{j_1} \cd}{x}
n[x_1 \cd x_{i_1} \cd x_{i_2} \cd x_{j_1} \cd x_{j_2} \cd x_m ]  \\
&+ \cd \\
&+ \sum_{f.c.} n \ol{\ol{[x_1 \cd x_m]  }}
\end{align*}
where ``f.c." stands for fully contracted. 
Note that in cases where $m$ is odd, the last terms will not be fully contracted but have one uncontracted operator. 
In words, Wick's theorem states that a string of operators can be written as its normal product plus all possible contractions of the normal product. 
We will adopt a shorthand notation for expressing all possible contractions of the normal product and write Wick's theorem as: 
\[x_1 \cd x_m = n[x_1 \cd x_m ]  + \sum_{a.c.} n\ol{[x_1 \cd x_m ]} \]\\
where `a.c.' stands for `all possible contractions' and $n\ol{[x_1 \cd x_m ]}$ is just representative of some general normal order product with contractions. 
\\ \\
We will first prove Wick's Theorem for the case m=1: 
\\
One operator $x_1$ can just be written as a normal product, and cannot form contractions:
\[x_1 = n[x_1] \]
Wick's theorem is trivially proved. 
We can prove the more interesting case of m=2 as well: 
We can begin by rewriting $x_1x_2$ using the definition of a contraction:
\[x_1x_2 = n[x_1x_2] + \ctr{}{x}{{}_1}{x} x_1 x_2\]
$\ctr{}{x}{{}_1}{x} x_1 x_2$ is equivalently: 
\[x_1x_2 = n[x_1x_2] + \ctr{}{x}{{}_1}{x} x_1 x_2 n[\varnothing]\]
\[ x_1x_2   = n[x_1x_2] + n[ \ctr{}{x}{{}_1}{x} x_1 x_2] \]
And Wick's theorem is proved by simply applying the definition of a contraction.  \\ \\
We will now take the induction step. Assuming Wick's theorem is true for $m=l$, we will prove that it is true for $m=l+1$: 
We can multiply the statement of Wick's theorem on the right by $x_{l+1}$:
\begin{align*}
x_1 \cd x_l x_{l + 1} 
&= \cre{n[x_1 \cd x_l ] x_{l + 1} } \\
& \quad +  \co{\sum_{1 \leq i_1 <  j_1}^l n[x_1 \cd \ctr{}{x}{{}_{i_1}\cd}{x} x_{i_1}  \cd x_{j_1}  \cd x_l ] x_{l + 1} } \\
& \quad +  \cg{ \sum_{\substack{i_1 < j_1,  i_2 < j_2 \\ i_1 < i_2, j_1 \neq j_2}}^l
n[x_1 \cd \ctr{}{x}{{}_{i_1}\cd}{x} x_{i_1}  \cd x_{j_1} \cd \ctr{}{x}{{}_{i_2}\cd}{x} x_{i_2}  \cd x_{j_2} \cd x_l ] x_{l + 1} } \\
 & \quad  +  \cd \\
& \quad + \cb{ \sum_{\substack{i_1 <  j_1 \cd  i_{\lambda - 1} < j_{\lambda - 1} \\ i_1 < \cd < i_{\lambda - 1}, j_1 \neq \cd \neq j_{\lambda-1}}}^l
n[x_1 \cd \ctr{}{x}{{}_{i_1}\cd}{x} x_{i_1}  \cd x_{j_1} \cd \ctr{}{x}{{}_{i_{\lambda -1} }\cd}{x} x_{i_{\lambda-1}}  \cd x_{j_{\lambda - 1}} \cd x_l ] x_{l + 1} } \\
& \quad + \cp{ \sum_{\substack{i_1 <  j_1 \cd  i_\lambda < j_\lambda \\ i_1 < \cd < i_\lambda, j_1 \neq \cd \neq j_\lambda}}^l
n[x_1 \cd \ctr{}{x}{{}_{i_1}\cd}{x} x_{i_1}  \cd x_{j_1} \cd \ctr{}{x}{{}_{i_\lambda}\cd}{x} x_{i_\lambda}  \cd x_{j_\lambda} \cd x_l ] x_{l + 1} } \\
\end{align*}
If $l$ is even, we have for the last term: 
\[ \cd + \sum_{f.c.} n\ol{\ol{[x_1\cd x_l]}}x_{l+1} \]
If $l$ is odd, we have for the last term: 
\[\sum_{k} n[  
\ctr{}{x}{{}_1 \cdot}{\cdot}
\ctr[2ex]{x}{{}_1 }{\cdot \cdot\cdot}{}
x_1 \cdot \cdot \cdot  x_k 
\ctr{}{x}{{}_{k+1} \cdot}{\cdot}{ {}}
\ctr[2ex]{x_{k + 1}  \cdot}{}{\cdot}{x}
x_{k + 1}  \cdot \cdot \cdot  x_l] x_{l+1} \]
where in the normal product only $x_k$ is uncontracted.\\ \\
We can expand each of the terms using Lemma 1 and 2: 
\begin{align*}
\cre{n[x_1 \cd x_l ] x_{l + 1} = n[x_1 \cd x_l  x_{l + 1} ] + n[x_1 \cd \ctr{}{x}{{}_k \cd x_l }{x} x_k \cd x_l  x_{l + 1} ]
} 
\end{align*}
\begin{align*}
\co{\sum_{1 \leq i_1 <  j_1}^l n[x_1 \cd \ctr{}{x}{{}_{i_1}\cd}{x} x_{i_1}  \cd x_{j_1}  \cd x_l ] x_{l + 1} } 
& \quad \co{ = \sum_{1 \leq i_1 <  j_1}^l n[x_1 \cd \ctr{}{x}{{}_{i_1}\cd}{x} x_{i_1}  \cd x_{j_1}  \cd x_l x_{l + 1}]  }\\
& \quad \co{ + \sum_{1 \leq i_1 <  j_1}^l \sum_{k \neq i_1, j_1}^l n[x_1 \cd \ctr{}{x}{{}_{i_1}\cd}{x} x_{i_1}  \cd x_{j_1}  \cd \ctr{}{x}{{}_k \cd  x_l}{x} x_k \cd  x_l x_{l + 1}]  }
\end{align*}
\begin{align*}
\cg{ \sum_{\substack{i_1 < j_1,  i_2 < j_2 \\ i_1 < i_2, j_1 \neq j_2}}^l
n[x_1 \cd \ctr{}{x}{{}_{i_1}\cd}{x} x_{i_1}  \cd x_{j_1} \cd \ctr{}{x}{{}_{i_2}\cd}{x} x_{i_2}  \cd x_{j_2} \cd x_l ] x_{l + 1} } 
& \quad \cg{ =\sum_{\substack{i_1 < j_1,  i_2 < j_2 \\ i_1 < i_2, j_1 \neq j_2}}^l
n[x_1 \cd \ctr{}{x}{{}_{i_1}\cd}{x} x_{i_1}  \cd x_{j_1} \cd \ctr{}{x}{{}_{i_2}\cd}{x} x_{i_2}  \cd x_{j_2} \cd x_l x_{l + 1}  ] }   \\
& \quad \cg{ + \sum_{\substack{i_1 < j_1,  i_2 < j_2 \\ i_1 < i_2, j_1 \neq j_2}}^l \sum_{\substack{k\neq i_1, i_2 \\ k\neq j_1, j_2} }^{l}
n[x_1 \cd \ctr{}{x}{{}_{i_1}\cd}{x} x_{i_1}  \cd x_{j_1} \cd \ctr{}{x}{{}_{i_2}\cd}{x} x_{i_2}  \cd x_{j_2} \cd \ctr{}{x}{{}_k \cd  x_l}{x}  x_k \cd x_l x_{l + 1}  ] }   \\
\end{align*}
\begin{align*}
& \cb{ \sum_{\substack{i_1 <  j_1 \cd  i_{\lambda - 1} < j_{\lambda - 1} \\ i_1 < \cd < i_{\lambda - 1}, j_1 \neq \cd \neq j_{\lambda-1}}}^l
n[x_1 \cd \ctr{}{x}{{}_{i_1}\cd}{x} x_{i_1}  \cd x_{j_1} \cd \ctr{}{x}{{}_{i_{\lambda -1} }\cd}{x} x_{i_{\lambda-1}}  \cd x_{j_{\lambda - 1}} \cd x_l ] x_{l + 1} }  \\
& \quad \cb{ =  \sum_{\substack{i_1 <  j_1 \cd  i_{\lambda - 1} < j_{\lambda - 1} \\ i_1 < \cd < i_{\lambda - 1}, j_1 \neq \cd \neq j_{\lambda-1}}}^l
n[x_1 \cd \ctr{}{x}{{}_{i_1}\cd}{x} x_{i_1}  \cd x_{j_1} \cd \ctr{}{x}{{}_{i_{\lambda -1} }\cd}{x} x_{i_{\lambda-1}}  \cd x_{j_{\lambda - 1}} \cd x_l x_{l + 1} ]  } \\
& \quad \cb{ + \sum_{\substack{i_1 <  j_1 \cd  i_{\lambda - 1} < j_{\lambda - 1} \\ i_1 < \cd < i_{\lambda - 1}, j_1 \neq \cd \neq j_{\lambda-1}}}^l
\sum_{\substack{k\neq i_1 \cd i_{\lambda-1} \\ k\neq j_1\cd j_{\lambda - 1}} }^{l}
n[x_1 \cd \ctr{}{x}{{}_{i_1}\cd}{x} x_{i_1}  \cd x_{j_1} \cd \ctr{}{x}{{}_{i_{\lambda -1} }\cd}{x} x_{i_{\lambda-1}}  \cd x_{j_{\lambda - 1}} \cd \ctr{}{x}{{}_k \cd  x_l}{x}  x_k \cd x_l x_{l + 1} ]  } \\
\end{align*}
\begin{align*}
& \cp{ \sum_{\substack{i_1 <  j_1 \cd  i_{\lambda} < j_{\lambda} \\ i_1 < \cd < i_{\lambda}, j_1 \neq \cd \neq j_{\lambda}}}^l
n[x_1 \cd \ctr{}{x}{{}_{i_1}\cd}{x} x_{i_1}  \cd x_{j_1} \cd \ctr{}{x}{{}_{i_{\lambda } }\cd}{x} x_{i_{\lambda}}  \cd x_{j_{\lambda}} \cd x_l ] x_{l + 1} }  \\
& \quad \cp{ =  \sum_{\substack{i_1 <  j_1 \cd  i_{\lambda } < j_{\lambda } \\ i_1 < \cd < i_{\lambda }, j_1 \neq \cd \neq j_{\lambda}}}^l
n[x_1 \cd \ctr{}{x}{{}_{i_1}\cd}{x} x_{i_1}  \cd x_{j_1} \cd \ctr{}{x}{{}_{i_{\lambda} }\cd}{x} x_{i_{\lambda}}  \cd x_{j_{\lambda }} \cd x_l x_{l + 1} ]  } \\
& \quad \cp{ + \sum_{\substack{i_1 <  j_1 \cd  i_{\lambda } < j_{\lambda } \\ i_1 < \cd < i_{\lambda }, j_1 \neq \cd \neq j_{\lambda}}}^l
\sum_{\substack{k\neq i_1 \cd i_{\lambda} \\ k\neq j_1\cd j_{\lambda }} }^{l}
n[x_1 \cd \ctr{}{x}{{}_{i_1}\cd}{x} x_{i_1}  \cd x_{j_1} \cd \ctr{}{x}{{}_{i_{\lambda } }\cd}{x} x_{i_{\lambda}}  \cd x_{j_{\lambda }} \cd \ctr{}{x}{{}_k \cd  x_l}{x}  x_k \cd x_l x_{l + 1} ]  } \\
\end{align*}
If $l$ is even, the last term is just:
\[ \sum_{f.c.} n\ol{\ol{[x_1\cd x_l]}}x_{l+1} = n\ol{\ol{[x_1\cd x_l}} x_{l+1}]  \]
where only $x_{l+1}$ is uncontracted. \\ \\ 
If $l$ is odd, the last term becomes:
\begin{align*}
\sum_{k}^l n[  
\ctr{}{x}{{}_1 \cdot}{\cdot}
\ctr[2ex]{x}{{}_1 }{\cdot \cdot\cdot}{}
x_1 \cdot \cdot \cdot  x_k 
\ctr{}{x}{{}_{k+1} \cdot}{\cdot}{ {}}
\ctr[2ex]{x_{k + 1}  \cdot}{}{\cdot}{x}
x_{k + 1}  \cdot \cdot \cdot  x_l] x_{l+1} &= 
\sum_{k}^l n[  
\ctr{}{x}{{}_1 \cdot}{\cdot}
\ctr[2ex]{x}{{}_1 }{\cdot \cdot\cdot}{}
x_1 \cdot \cdot \cdot  x_k 
\ctr{}{x}{{}_{k+1} \cdot}{\cdot}{ {}}
\ctr[2ex]{x_{k + 1}  \cdot}{}{\cdot}{x}
x_{k + 1}  \cdot \cdot \cdot  x_l x_{l+1}] \\
&+ \sum_{f.c.} n\ol{\ol{[  x_1  \cd  x_k x_{k + 1} \cd  x_l x_{l+1} ] }} \\
\end{align*}
Putting all the terms together, we get:
\begin{align*}
x_1 \cd x_l x_{l + 1} 
&= \cre{n[x_1 \cd x_l  x_{l + 1} ] + n[x_1 \cd \ctr{}{x}{{}_k \cd x_l }{x} x_k \cd x_l  x_{l + 1} ]} \\
& \quad \co{ + \sum_{1 \leq i_1 <  j_1}^l n[x_1 \cd \ctr{}{x}{{}_{i_1}\cd}{x} x_{i_1}  \cd x_{j_1}  \cd x_l x_{l + 1}]  }\\
& \quad \co{ + \sum_{1 \leq i_1 <  j_1}^l \sum_{k \neq i_1, j_1}^l n[x_1 \cd \ctr{}{x}{{}_{i_1}\cd}{x} x_{i_1}  \cd x_{j_1}  \cd \ctr{}{x}{{}_k \cd  x_l}{x} x_k \cd  x_l x_{l + 1}]  } \\
& \quad \cg{ +\sum_{\substack{i_1 < j_1,  i_2 < j_2 \\ i_1 < i_2, j_1 \neq j_2}}^l
n[x_1 \cd \ctr{}{x}{{}_{i_1}\cd}{x} x_{i_1}  \cd x_{j_1} \cd \ctr{}{x}{{}_{i_2}\cd}{x} x_{i_2}  \cd x_{j_2} \cd x_l x_{l + 1}  ] }   \\
& \quad \cg{ + \sum_{\substack{i_1 < j_1,  i_2 < j_2 \\ i_1 < i_2, j_1 \neq j_2}}^l \sum_{\substack{k\neq i_1, i_2 \\ k\neq j_1, j_2} }^{l}
n[x_1 \cd \ctr{}{x}{{}_{i_1}\cd}{x} x_{i_1}  \cd x_{j_1} \cd \ctr{}{x}{{}_{i_2}\cd}{x} x_{i_2}  \cd x_{j_2} \cd \ctr{}{x}{{}_k \cd  x_l}{x}  x_k \cd x_l x_{l + 1}  ] }   \\
& \quad \cb{ +  \sum_{\substack{i_1 <  j_1 \cd  i_{\lambda - 1} < j_{\lambda - 1} \\ i_1 < \cd < i_{\lambda - 1}, j_1 \neq \cd \neq j_{\lambda-1}}}^l
n[x_1 \cd \ctr{}{x}{{}_{i_1}\cd}{x} x_{i_1}  \cd x_{j_1} \cd \ctr{}{x}{{}_{i_{\lambda -1} }\cd}{x} x_{i_{\lambda-1}}  \cd x_{j_{\lambda - 1}} \cd x_l x_{l + 1} ]  } \\
& \quad \cb{ + \sum_{\substack{i_1 <  j_1 \cd  i_{\lambda - 1} < j_{\lambda - 1} \\ i_1 < \cd < i_{\lambda - 1}, j_1 \neq \cd \neq j_{\lambda-1}}}^l
\sum_{\substack{k\neq i_1 \cd i_{\lambda-1} \\ k\neq j_1\cd j_{\lambda - 1}} }^{l}
n[x_1 \cd \ctr{}{x}{{}_{i_1}\cd}{x} x_{i_1}  \cd x_{j_1} \cd \ctr{}{x}{{}_{i_{\lambda -1} }\cd}{x} x_{i_{\lambda-1}}  \cd x_{j_{\lambda - 1}} \cd \ctr{}{x}{{}_k \cd  x_l}{x}  x_k \cd x_l x_{l + 1} ]  } \\
& \quad \cp{ +  \sum_{\substack{i_1 <  j_1 \cd  i_{\lambda } < j_{\lambda } \\ i_1 < \cd < i_{\lambda }, j_1 \neq \cd \neq j_{\lambda}}}^l
n[x_1 \cd \ctr{}{x}{{}_{i_1}\cd}{x} x_{i_1}  \cd x_{j_1} \cd \ctr{}{x}{{}_{i_{\lambda} }\cd}{x} x_{i_{\lambda}}  \cd x_{j_{\lambda }} \cd x_l x_{l + 1} ]  } \\
& \quad \cp{ + \sum_{\substack{i_1 <  j_1 \cd  i_{\lambda } < j_{\lambda } \\ i_1 < \cd < i_{\lambda }, j_1 \neq \cd \neq j_{\lambda}}}^l
\sum_{\substack{k\neq i_1 \cd i_{\lambda} \\ k\neq j_1\cd j_{\lambda }} }^{l}
n[x_1 \cd \ctr{}{x}{{}_{i_1}\cd}{x} x_{i_1}  \cd x_{j_1} \cd \ctr{}{x}{{}_{i_{\lambda } }\cd}{x} x_{i_{\lambda}}  \cd x_{j_{\lambda }} \cd \ctr{}{x}{{}_k \cd  x_l}{x}  x_k \cd x_l x_{l + 1} ]  } \\
& \quad + \cd
\end{align*}
We notice that on the righthand side we can combine terms 2 and 3 (all single contractions), terms 4 and 5 (all double contractions), and so on, to 
recover the statement of Wick's theorem for $m = l+1$: 
\begin{align*}
x_1 \cd x_l x_{l + 1} 
&= n[x_1 \cd x_l x_{l + 1}  ]  \\
& \quad + \sum_{1 \leq i_1 <  j_1}^{l+1} n[x_1 \cd \ctr{}{x}{{}_{i_1}\cd}{x} x_{i_1}  \cd x_{j_1}  \cd x_l x_{l + 1} ]   \\
& \quad +   \sum_{\substack{i_1 < j_1,  i_2 < j_2 \\ i_1 < i_2, j_1 \neq j_2}}^{l+1}
n[x_1 \cd \ctr{}{x}{{}_{i_1}\cd}{x} x_{i_1}  \cd x_{j_1} \cd \ctr{}{x}{{}_{i_2}\cd}{x} x_{i_2}  \cd x_{j_2} \cd x_l x_{l + 1}  ] \\
 & \quad  +  \cd \\
& \quad +  \sum_{\substack{i_1 <  j_1 \cd  i_{\lambda - 1} < j_{\lambda - 1} \\ i_1 < \cd < i_{\lambda - 1}, j_1 \neq \cd \neq j_{\lambda-1}}}^{l+1}
n[x_1 \cd \ctr{}{x}{{}_{i_1}\cd}{x} x_{i_1}  \cd x_{j_1} \cd \ctr{}{x}{{}_{i_{\lambda -1} }\cd}{x} x_{i_{\lambda-1}}  \cd x_{j_{\lambda - 1}} \cd x_l  x_{l + 1}]   \\
& \quad +  \sum_{\substack{i_1 <  j_1 \cd  i_\lambda < j_\lambda \\ i_1 < \cd < i_\lambda, j_1 \neq \cd \neq j_\lambda}}^{l+1} 
n[x_1 \cd \ctr{}{x}{{}_{i_1}\cd}{x} x_{i_1}  \cd x_{j_1} \cd \ctr{}{x}{{}_{i_\lambda}\cd}{x} x_{i_\lambda}  \cd x_{j_\lambda} \cd x_l x_{l + 1} ] \\
& \quad + \cd
\end{align*}
Finally, we note that for the case $l+1$,
the last terms have one uncontracted operator if $l$ is even and are fully contracted if $l$ is odd. 

\section{Proof of Generalized Wick's Theorem}
In some cases, substring of operator strings are already in the normal product form, and we can use the generalized Wick's theorem, which states: 
$$x_1 \cd x_m = n[x_1 \cd x_m ]  + \sum_{a.c.} {}' n\ol{[x_1 \cd x_m ]} $$
where $\sum_{a.c.} {}' $ denotes skipping contractions of operators that originated from the same normal ordered group. 
\\ \\
Proof: \\ 
\\
We write a string of operators where a substring is already in normal order:
\[x_1 \cd x_{k_{\mu-1}} n[x_{k_{\mu-1}+1} \cd x_{k_\mu}] x_{k_\mu+1} \cd x_m\]
We can permute indices inside the normal-ordered string and take it out of the normal product:   
\[(-1)^R x_1 \cd x_{k_{\mu-1}} a_{p_1}\dg \cd a_{p_\alpha}\dg a_{p_{\alpha + 1}} \cd a_{p_{\beta}} x_{k_\mu+1} \cd x_m\]
Using Wick's theorem, we can expand the string as: 
\begin{align*} 
(-1)^R x_1 \cd x_{k_{\mu-1}} a_{p_1}\dg \cd a_{p_\alpha}\dg & a_{p_{\alpha + 1}} \cd a_{p_{\beta}} x_{k_\mu+1} \cd x_m  = \\ \\
& (-1)^R n[x_1 \cd x_{k_{\mu-1}} a_{p_1}\dg \cd a_{p_\alpha}\dg  a_{p_{\alpha + 1}} \cd a_{p_{\beta}} x_{k_\mu+1} \cd x_m] \\ \\
&+ (-1)^R \sum_{a.c.} n\ol{[x_1 \cd x_{k_{\mu-1}} a_{p_1}\dg \cd a_{p_\alpha}\dg  a_{p_{\alpha + 1}} \cd a_{p_{\beta}} x_{k_\mu+1} \cd x_m] }
\end{align*}
\newpage
\noindent
Let's consider all contractions involving only $a_{p_1}\dg \cd a_{p_\alpha}\dg  a_{p_{\alpha + 1}} \cd a_{p_{\beta}}$:
\begin{itemize}
\item If $\alpha = 0$ and there are no creation operators in the string, there are only annihilation operators in the normal-ordered group,
and any $\ctr{}{a}{{}_r}{a} a_ra_s = 0$. 
\item  If $\beta = 0$ and there are no annihilation operators in the string, there are only creation operators in the normal-ordered group,
and any $\ctr{}{a}{{}_r\dg}{a} a_r\dg a_s\dg = 0$. 
\item If $1 \leq \alpha < \beta$, we have both annihilation and creation operators, and within the normal ordered group will encounter contractions of type: 
$\ctr{}{a}{{}_r}{a} a_ra_s = 0$, $\ctr{}{a}{{}_r\dg}{a} a_r\dg a_s\dg = 0$, and $\ctr{}{a}{{}_r\dg}{a} a_r\dg a_s = 0$
\end{itemize}
Thus, we see that any contractions within a normal ordered group will always be zero, and we can leave them out in the final Wick expansion.
This proves the generalized Wick's theorem. 

\end{document}
