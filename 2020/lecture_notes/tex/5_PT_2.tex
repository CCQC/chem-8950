\documentclass{article}
\usepackage{amsmath,mathtools,amssymb}
\usepackage{graphicx}
\usepackage{booktabs}
\usepackage{blkarray}
\usepackage{gensymb}
\usepackage{verbatim}
\usepackage{mathrsfs}
\usepackage{bbm}
\usepackage{braket}
\usepackage{hyperref}
\usepackage{verbatim}
\usepackage{cancel}
\usepackage[margin=1.0in]{geometry}
\newcommand{\ol}{\overline}
\newcommand{\lp}{\left(}
\newcommand{\rp}{\right)}
\newcommand{\eps}{\varepsilon}
\newcommand{\lam}{\lambda}
\newcommand{\h}{\circ}
\newcommand{\p}{\bullet}

\newcommand{\Ezero}{E^{(0)}}
\newcommand{\Rz}{\mathcal{R}_{0}}

\newcommand{\Phizero}{\Phi^{(0)}}
\newcommand{\Eone}{E^{(1)}}
\newcommand{\En}{E^{(n)}}
\newcommand{\Phione}{\Phi^{(1)}}
\newcommand{\Phin}{\Phi^{(n)}}

\newcommand{\Ecorr}{E_{\mathrm{corr}}}
\newcommand{\Hc}{H_{\mathrm{c}}}
\newcommand{\dg}{\ensuremath{^\dagger} }
\def\*#1{\mathbf{#1}}
\DeclarePairedDelimiter\floor{\lfloor}{\rfloor}

\title{Lecture 5: Perturbation Theory II}
\date{April 6, 2020}
\begin{document}
\maketitle
\noindent

In perturbation theory, we split our Hamiltonian into parts $H = H_0 + V$, and our ground 
state wavefunction and energy are expressed as an infinite series, where the first term is the unperturbed
quantity of some simpler model system described by $H_0$, and the rest of the terms
are corrections at various ``orders'':
\[\ket{\Psi_0} = \ket{\Phi_0} + \sum_{n=1}^{\infty} \ket{\Phin_0} \]
\[E_0 = \Ezero_0 + \sum_{n=1}^{\infty} \En_0\]

As we saw in the previous set of notes, the expression for each wavefunction and energy correction 
    needs to be derived from a system of equations.
This way of doing things is very inefficient at high orders, so in this set of notes 
    we will use a new scheme, which was primarily popularized and championed by Per-Olav L{\"o}wdin
    in his highly influential series of papers \textit{Studies in Perturbation Theory} in the 1960s.
The majority of the content in these notes can be found in the 9th (IX) paper in the series.
This new methodology will allow you to derive each order of PT using the same exact procedure. 
Though the higher order wavefunctions and energy expressions get more and more complex,
the procedure for deriving them does not change at higher orders.
This methodology also beautifully translates into clean diagrams, which we will see later on.

%\section{L{\"o}wdin and his Resolvents}
\section{The Wave Operator}
Suppose there was some magical operator $\Omega$ which
acts on the unperturbed state $\ket{\Phi_0}$ corresponding to $H_0$ 
and transforms it into the exact state $\ket{\Psi_0}$ described by $H_0 + V$
\[\ket{\Psi_0} = \Omega \ket{\Phi_0} \]
We will call $\Omega$ the \textit{wave operator}.
For now, we have no idea what it is, but we do know what it does.
Suppose further that we are using an intermediate normalized $\Psi_0$ and normalized $\Phi_0$ such that
\[ \braket{\Phi_0|\Phi_0} = 1 \]
\[ \braket{\Phi_0 | \Psi_0} = \braket{\Phi_0 | \Omega | \Phi_0} = 1 \]
Under the above conditions, we will now see what happens to the Schr{\"o}dinger equation,
\[(H_0 + V) \ket{\Psi_0} = E_0 \ket{\Psi_0} \]
Projecting both sides of this equation by $\bra{\Phi_0}$ gives
\[\braket{\Phi_0 | H_0 | \Psi_0} + \braket{\Phi_0 | V | \Psi_0} = E_0 \braket{\Phi_0|\Psi_0} \]
The first term reduces to $\Ezero_0$ by acting $H_0$ to the left. The second term 
can be re-expressed in terms of $\Omega$, and we obtain an expression for the 
perturbed eigenvalue $E_0$ in terms of a \textit{shift} from the unperturbed eigenvalue $\Ezero_0$.  
\[ E_0 = \Ezero_0 + \braket{\Phi_0 | V \Omega | \Phi_0}  \]
If we were to know what $\Omega$ is, at this point the eigenstates and eigenvalues would be completely
determined.
L{\"o}wdin derived that such an $\Omega$ can be constructed from an operator called a \textit{resolvent},
and under this definition, $\Omega$ yields the same perturbation expansion for $E_0$ and $\ket{\Psi_0}$.

\section{Resolvents}
A resolvent is a mathematical structure which is used to study the eigenspectrum
of operators, among other things.
The concept of a \textit{reduced resolvent} arises when one considers an eigenvalue problem \[A \phi = a \phi \]
and divides the basis of the eigenstates into two partitions $p$ and $q$. 
If the $p$ partition is just a single basis function and the $q$ partition 
    is all other basis functions,
    one finds with some manipulation that the eigenvalues are defined by the relation 
\[ a =  A_{11} + \mathbf{A}_{1q}( a \cdot \mathbf{1}_{qq} - \mathbf{A}_{qq})^{-1} \mathbf{A}_{q1} \]
The above is the ``reduced'' characteristic equation, and it can be used for finding the eigenvalues.
It arises in the particular case of our partitioning of our basis.
The above equation can be thought of as an alternative to the standard characteristic equation $\mathrm{det}(A - aI) = 0$.

If we replace the eigenvalues $a$ in the expression above with some continuous, variable parameter $\kappa$, we obtain the following: 
\[f(\kappa) =  A_{11} + \mathbf{A}_{1q}( \kappa \cdot \mathbf{1}_{qq} - \mathbf{A}_{qq})^{-1} \mathbf{A}_{q1} \]

Now, whenever this function is such that $f(\kappa) = \kappa $, we have found an eigenvalue $a = \kappa$.
If we vary this $\kappa$ until $f(\kappa) - \kappa = 0$, we pull out a discrete eigenvalue which is a solution to the eigenvalue problem. 

The key quantity above 
\[\mathcal{R}_{\kappa}(A) = ( \kappa \cdot \mathbf{1}_{qq} - \mathbf{A}_{qq})^{-1}\]
is the \textbf{reduced resolvent} operator of $\mathbf{A}$ for a particular $\kappa$.
\footnote{Here, $\mathbf{1}_{qq}$ is the identity matrix of size $q \times q$ and $\mathbf{A}_{qq}$ is 
 the matrix representation of the operator in just the $q$-space, (over just $q$ basis functions)}
In words, the reduced resolvent is an operator that is assigned to another operator $\mathbf{A}$.
It is only defined in $q$-space part of the operator, and different values of $\kappa$ give different reduced resolvents.
As abstract as that is, it is tremendously useful, as we will see later.

Notice how the resolvent is dependent on just the matrix elements of the operator $\mathbf{A}$
which correspond to the $q$ partition; it lives in the $q$-space partition of our basis.
Expressed pictorially as a matrix representation divided into $pp$, $pq$, $qp$, $qq$ blocks,
\[\mathcal{R}_{\kappa}(A) =
\left( \begin{array}{c|c}
   0 & \mathbf{0} \\
   \midrule
   \mathbf{0} & (\kappa \mathbf{1}_{qq} - A_{qq})^{-1} \\
\end{array}\right)
  \]

\section{L{\"o}wdin's big idea: using reduced resolvents to define perturbation theory and solve the Schr{\"o}dinger equation}
Back to the world of quantum chemistry and the Schrodinger equation, 
we can define two projection operators $P$ and $Q$ which partition our $n$-electron Hilbert space into two parts (note: if that sounds like gobbledygook to you, see the Appendix for details).
It is this partitioning which birthed the idea of a reduced resolvent above.

$P$ is the ``model space projection operator'', which can act on a state 
    and project it onto some reference determinant $\Phi_0$.
\[P = \ket{\Phi_0}\bra{\Phi_0}  \]
$Q$ is the \textit{orthogonal complement} of $P$, which projects out every determinant other than the reference $\Phi_0$
\[Q = 1 - P = \sum_{n \neq 0}  \ket{\Phi_n}\bra{\Phi_n}\]
$P$ and $Q$ basically just split our basis into two parts, and the sum of $P$ and $Q$ yield the resolution of the identity, $P + Q = 1$.

In terms operators, the definition of the reduced resolvent for some operator $H$ is
\[\mathcal{R}_{\kappa} = (\kappa - H)^{-1} Q  \]

L{\"o}wdin's idea, in short, was to generalize the Schr{\"o}dinger equation
to a continous spectrum in $\kappa$, and use the $\kappa$-dependent reduced resolvent in the definition of 
a trial wavefunction to solve for the eigenvalues. Read that last sentence a few more times. 
Hopefully, it will soon become clear what I mean by this.
Starting from the re-arranged Schr{\"o}dinger equation,
\[(E_0 - H) \ket{\Psi_0} = 0 \]
suppose we choose a form for the wavefunction which has a continuous dependence on some parameter
$\kappa$: 
\[(\kappa - H) \ket{\Psi_{\kappa}} = ... \]
The above equation \textit{becomes} the Schrodinger equation whenever the right side is 0.
If this is the case, $\kappa$ happens to be equal to a true eigenvalue $E_0$.
For this reason, we refer to it as a \textit{generalization} of the Schr{\"o}dinger equation.

%%%HERE TODO
L{\"o}wdin derived that a \textbf{trial wavefunction} $\ket{\Psi_{\kappa}}$ 
    which is of the form $\ket{\Psi_\kappa} = \Omega_\kappa \ket{\Phi_0}$
    can be expressed in terms of the reduced resolvent. 
\[\ket{\Psi_\kappa} = \Omega_\kappa \ket{\Phi_0}\]
\[\Omega_{\kappa} = P + \mathcal{R}_{\kappa} H P \]
\[\ket{\Psi_{\kappa}} = \ket{\Phi_0} + \mathcal{R}_{\kappa} H \ket{\Phi_0} \]
% maybe footnote here

What happens when we apply $(\kappa - H)$ to our trial wavefunction $\ket{\Psi_{\kappa}}$? 
as in the above expression $(\kappa - H) \ket{\Psi_{\kappa}} = ... $?
Applying the resolution of the identity $P + Q = 1$, we find
\[ (\kappa - H) \ket{\Psi_{\kappa}} =  P (\kappa - H) \ket{\Psi_{\kappa}} + Q (\kappa - H) \ket{\Psi_{\kappa}}  \]
Looking at just the $Q$ term, we find that
\begin{align*}
Q (\kappa - H) \ket{\Psi_{\kappa}} &= Q (\kappa - H) \ket{\Phi_0} + Q (\kappa - H) \mathcal{R}_{\kappa} H \ket{\Phi_0}  \\
&= \kappa Q \ket{\Phi_0} - Q H \ket{\Phi_0} + Q H \ket{\Phi_0}  \\
&= \kappa Q \ket{\Phi_0}  \\
&= 0
\end{align*}
Above we have used the fact that $ Q (\kappa - H) \mathcal{R}_{\kappa}  = Q$ , which follows
clearly from the definition of the reduced resolvent, and that $Q$ acting on $\ket{\Phi_0}$ is 0, since 
$\ket{\Phi_0}$ lives in the $P$ space. 
We conclude the $Q$ term is 0, so we are left with:
\[ (\kappa - H) \ket{\Psi_{\kappa}} =  P (\kappa - H) \ket{\Psi_{\kappa}} \]
Using the definition of $P = \ket{\Phi_0}\bra{\Phi_0}$,
\begin{align*}
 (\kappa - H) \ket{\Psi_{\kappa}} &=  \ket{\Phi_0}\bra{\Phi_0} (\kappa - H) \ket{\Psi_{\kappa}} \\
 &=  \kappa \ket{\Phi_0} \braket{\Phi_0|\Psi_\kappa} - \ket{\Phi_0} \braket{\Phi_0| H | \Psi_\kappa} \\
 &= (\kappa - \braket{\Phi_0 | H | \Psi_\kappa}) \ket{\Phi_0} \\
 &= (\kappa - \braket{\Phi_0 | H \Omega_\kappa | \Phi_0}) \ket{\Phi_0} \\
 &= (\kappa - \braket{\Phi_0 | H + H \mathcal{R}_\kappa H  | \Phi_0} ) \ket{\Phi_0} \\
 &= (\kappa - f(\kappa)) \ket{\Phi_0}
\end{align*}

where we have defined the function of $\kappa$, denoted  $f(\kappa)$ as
\[ f(\kappa) =  \braket{\Phi_0 | H | \Psi_\kappa} = \braket{\Phi_0 | H + H \mathcal{R}_\kappa H  | \Phi_0}\]
L{\"o}wdin calls this the \textit{bracketing function}, since it enables one to obtain both lower and upper bounds to true eigenvalues.
Restating the above in one compact expression, we have 
\[ (\kappa - H) \ket{\Psi_{\kappa}} = (\kappa - f(\kappa)) \ket{\Phi_0} \]
\[ (\kappa - H) \Omega_\kappa \ket{\Phi_0} = (\kappa - f(\kappa)) \ket{\Phi_0} \]
This is a monumental result.
\textbf{When $\kappa$ is equal to $f(\kappa)$, the right side above goes to zero. If this occurs,
we conclude that our trial function $\ket{\Psi_{\kappa}}$
is a solution to the Schr{\"o}dinger equation, and the eigenvalue $\kappa$ is the corresponding 
energy eigenvalue.}  
Thus, our chosen trial function (derived by L{\"o}wdin) gives us access to the solutions of 
%the Schrodinger equation, by looking for zeros of $F(\kappa) = f(\kappa) - \kappa$.
%When we find a zero, we have the following well-defined, obtainable quantities
%\[ \kappa = E_0 \quad \quad (E_0 - H) \ket{\Psi_{\kappa=E_0}} = 0 \quad \quad \ket{\Psi_{\kappa=E_0}} = \ket{\Phi_0} + \mathcal{R}_{\kappa = E_0} H \ket{\Phi_0} \]
%
%The above equation is always applicable so long as the wavefunction is not orthogonal to $\ket{\Phi_0}$
%and the resolvent exists.

%We conclude from the above that our wave operator $\Omega$ such that
%\[\ket{\Psi_\kappa} = \Omega_\kappa \ket{\Phi_0} \]
%\[\kappa = \braket{\Phi_0|H \Omega_{\kappa} | \Phi_0} \]
%is equal to
%\[\Omega_{\kappa} = 1 + \mathcal{R}_{\kappa} H \]
%For brevity, it is more proper to express $\Omega$ in terms of $P$ to ensure it behaves properly when acted
%on any state
%\[\Omega_{\kappa} = P + \mathcal{R}_{E_0} H P \]
%As defined in the case of $\ket{\Psi_0} = \Omega_{\kappa} \ket{\Phi_0} $, replacing $P$ with unity does not have an effect.
%$\Omega$ has the following properties, which we will not prove:
%\[ P \Omega = P \quad \quad \Omega P = \Omega \quad \quad \Omega^2 = \Omega \]
%These conditions classify $\Omega$ as a Bloch wave operator.
%
%Having defined $\Omega$, we can now re-express our main result above in terms of just our reference
%state $\ket{\Phi_0}$,
%\[ (\kappa - H) \Omega_\kappa \ket{\Phi_0} = (\kappa - f(\kappa)) \ket{\Phi_0} \]
%\[ f(\kappa) = \braket{\Phi_0 | H \Omega_\kappa | \Phi_0}\]

\subsection{L{\"o}wdin's Rayleigh-Schr{\"o}dinger Perturbation Theory}
We can now take our generalized Schr{\"o}dinger equation from above 
\[ (\kappa - H) \Omega_\kappa \ket{\Phi_0} = (\kappa - f(\kappa)) \ket{\Phi_0} \]
and consider how it behaves when our Hamiltonian is split into two parts $H = H_0 + V$. 

\paragraph{\textbf{The Wave Operator}}
Our wave operator becomes
\[\Omega_\kappa = P + \mathcal{R}_{\kappa} H_0 P + \mathcal{R}_{\kappa} V P \]
\[\Omega_\kappa = P + \mathcal{R}_{\kappa} H_0 \ket{\Phi_0}\bra{\Phi_0} + \mathcal{R}_{\kappa} V P \]
\[\Omega_\kappa = P + E_0 \mathcal{R}_{\kappa} P + \mathcal{R}_{\kappa} V P \]
\[\Omega_\kappa = P + \mathcal{R}_{\kappa} V P \]
where the reduced resolvent acting on $P$ is zero, since the resolvent entirely resides in the $Q$ space, which is orthogonal to $P$.

\paragraph{\textbf{The Bracketing Function}}
After a few manipulations which are left as an exercise, 
the bracketing function $f(\kappa) = \braket{\Phi_0 | H \Omega_\kappa | \Phi_0}$ becomes 
\[f(\kappa) = \Ezero_0 + \braket{\Phi_0 | V \Omega_\kappa | \Phi_0 } \]

\paragraph{\textbf{The Reduced Resolvent}}
We have one last quantity to fully define: the reduced resolvent. 
How does the resolvent behave for our Hamiltonian $H = H_0 + V$? 
\[\mathcal{R}_{\kappa} = (\kappa - H)^{-1}Q \]
We are about to do a lot of tricky manipulations, but in doing so,
we will arrive at perhaps the most important result in this set of notes. 
Let's do it.
First we will expand $H$ and add $0 = \Ezero_0 - \Ezero_0$
\[\mathcal{R}_{\kappa} = (\kappa - \Ezero_0 + \Ezero_0 - H_0 - V)^{-1}Q \]
Rearranging and letting $V' = V - (\kappa - \Ezero_0)$:
\[\mathcal{R}_{\kappa} = (\Ezero_0 - H_0 - V')^{-1}Q \]
To move further, we use of the following identity for two operators $A$ and $B$,
\[ (A- B)^{-1} = A^{-1} + A^{-1} B (A - B)^{-1} \]
If we let  $A = (\Ezero_0 - H_0)$ and $B = V'$, we have
\[\mathcal{R}_{\kappa} = \left[ (\Ezero_0 - H_0)^{-1} +  (\Ezero_0 - H_0)^{-1} V' (\Ezero_0 - H_0 - V')^{-1} \right] Q\]
\[\mathcal{R}_{\kappa} = (\Ezero_0 - H_0)^{-1}Q +  (\Ezero_0 - H_0)^{-1} V' (\Ezero_0 - H_0 - V')^{-1}Q\]
Since $Q^2 = Q$ and Q is Hermitian we can insert a $Q$ in the second term:
\[\mathcal{R}_{\kappa} = (\Ezero_0 - H_0)^{-1}Q +  (\Ezero_0 - H_0)^{-1}Q V' (\Ezero_0 - H_0 - V')^{-1}Q\]
Expanding the definition of $V'$ in the parentheses we obtain
\[\mathcal{R}_{\kappa} = (\Ezero_0 - H_0)^{-1}Q +  (\Ezero_0 - H_0)^{-1}Q V' (\kappa - H)^{-1}Q\]
Now, we define the quantity $\Rz =  (\Ezero_0 - H_0)^{-1} Q$, the \textbf{unperturbed reduced resolvent}, 
which is the resolvent defined for $\kappa = \Ezero_0 $ and unperturbed Hamiltonian $H_0$.
\[\mathcal{R}_{\kappa} = \Rz +  \Rz V' (\kappa - H)^{-1}Q\]
Noting the last part of the second term is just our original reduced resolvent $\mathcal{R}_{\kappa} = (\kappa - H)^{-1}Q $,
we finally arrive at our result:
\[\mathcal{R}_{\kappa} = \Rz +  \Rz V' \mathcal{R}_{\kappa} \]
\textbf{The resolvent is dependent on itself. 
Because of this, we can iteratively construct an expansion of the resolvent. 
This will ultimately be the source of our perturbation expansion.}
In an exercise, you will explicitly iterate the resolvent, and show that it generalizes to the following
\[\mathcal{R}_{\kappa} = \sum_{n=0}^{\infty} (\Rz V')^n \Rz \]

Up to this point, we have defined everything with the subscript $\kappa$.
From here on, we will focus on just finding solutions for which $\kappa$ is our ground 
state energy $\kappa = E_0$, so we will drop the $\kappa$ subscripts.
Since the only part of the resolvent above which depends on $\kappa$ is 
$V' = V - (\kappa - \Ezero_0)$, we will now have 
\[V' = V - (E_0 - \Ezero_0)\]
\[\mathcal{R} = \sum_{n=0}^{\infty} (\Rz V')^n \Rz \]
We will likewise drop the $\kappa$ from our definition of the wave operator 
and the wave function
\[\Omega = P + \sum_{n=0}^{\infty} (\Rz V')^n \Rz V P \]
\[\ket{\Psi_0} = \Omega \ket{\Phi_0}\]
\[\ket{\Psi_0} =  \ket{\Phi_0} + \sum_{n=0}^{\infty} (\Rz V')^n \Rz V \ket{\Phi_0} \]
\textbf{This is our perturbation expansion for the wavefunction.}
From section 1, ``The Wave Operator'', we derived our energy expression to be, 
\[ E_0 = \Ezero_0 + \braket{\Phi_0 | V \Omega | \Phi_0}  \]
Our energy expansion is therefore,
\[E_0 = \Ezero_0 + \braket{\Phi_0|V|\Phi_0} + \sum_{n=0}^{\infty} \braket{\Phi_0|V (\Rz V')^n \Rz V | \Phi_0}\]
\textbf{This is our perturbation expansion for the energy.}

The operator $V'$ introduces some complexity that is worth discussing. 
It contains $E_0$, which is the exact energy, 
    so it involves a sum over all orders of energy corrections.
\[V' = V - (E_0 - \Ezero_0) \]
\[V' = V - \sum_{m=1} E_0^{(m)} \]
\[V' = V - E_0^{(1)} - E_0^{(2)} - E_0^{(3)} - \cdots \]

To conclude, we have obtained expressions for the $n^{th}$ order wavefunction 
and energy corrections in perturbation theory.
Looking at the expression for $\ket{\Psi_0}$, each order wavefunction correction
at higher orders is just a string of operators $(\Rz, V, V')$ acting on our reference determinant.
Each energy is just an expectation value of a string of operators $(\Rz, V, V')$.
Using these expansions in practice can be a bit tricky, as we will see in the next section.

\section{Deriving Perturbation Theory Orders}
Before we get started, we need to know how to identify the ``order'' of a term in our
wavefunction expansion. The reason is that if we want the 
$m$th order wavefunction correction, we need to pull out all $m$th order
terms in $\ket{\Psi_0} =  \ket{\Phi_0} + \sum_{n=0}^{\infty} (\Rz V')^n \Rz V \ket{\Phi_0}$.

\paragraph{\textbf{How to identify the order of a term?}}
Some tips for finding the order of a given term:
\begin{itemize} 
\item Each $V$ contributes an order of 1
\item Each $E_0^{(m)}$ contributes an order of $m$
\item Instead of counting orders from $V$ and $E_0^{(m)}$, you can instead count
the number of resolvents $\Rz$. Before doing so, you must convert all $E_0^{(m)}$ to the 
corresponding analytic expression.
\end{itemize}
As an example of the above rules, when determining the order of the term 
\[ \Rz E_0^{(2)} \Rz V \ket{\Phi_0} \]
which appears in the third-order wavefunction, 
we could reason that $V$ and $E_0^{(2)}$ is $1 + 2 = 3 \implies$ third order.
Alternatively, we could recognize that $E_0^{(2)} = \braket{\Phi_0 |V \Rz V|\Phi_0}$ (for now,
just accept that this is true, we will derive it later),
so therefore our expression is 
\[ \Rz \braket{\Phi_0 |V \Rz V|\Phi_0} \Rz V \ket{\Phi_0} \]
The above expression has 3 resolvents, so we conclude it is third order.
OR, we could count that the above expression has 3 $V$'s, and conclude it is third order.
Everything checks out.
Let's derive some perturbation theory orders.

\paragraph{\textbf{Zero order}}
The 0th order wavefunction correction is just the first term of the respective pertubation expansion
fro above, which is of course just our reference determinant $\ket{\Phi_0}$. 
The 0th order energy is just $\Ezero_0$.

\paragraph{\textbf{First order}}
\[ \ket{\Phi_0^{(1)}} = \Rz V \ket{\Phi_0}\]
\[ E_0^{(1)} = \braket{ \Phi_0 | V | \Phi_0} \]

\paragraph{\textbf{Second order}}
We read off from the expansion for $n=1$,
\[ \ket{\Phi_0^{(2)}} = \Rz V' \Rz V \ket{\Phi_0}\]
Here is the only tricky thing with this formalism: $V'$ contains 
higher order contributions to the energy 
\[V' = V - E_0^{(1)} - E_0^{(2)} - E_0^{(3)} - \cdots\]
\textbf{Since we are building the second order wavefunction, it must be composed of only 
second order pieces. The prefactor $\Rz V' \Rz V$ above should therefore only be second order. 
The V is first order, so multiplying by V' should only increase the order by 1. 
This means only terms in the V' expansion which are first order contribute 
to the second order wavefunction.}
This means our $V'$ in this case is truncated to just $V + E_0^{(1)}$,
so we have,
\[\ket{\Phi_0^{(2)}} = \Rz (V - \Eone_0) \Rz V \ket{\Phi_0}\]
\[\ket{\Phi_0^{(2)}} = \Rz V \Rz V \ket{\Phi_0} - \Rz \Eone_0 \Rz V \ket{\Phi_0}\]
That's it! 
The energy correction is,
\[ E_0^{(2)} = \braket{ \Phi_0 | V \Rz V | \Phi_0} \]

\paragraph{\textbf{Third order}}
Here we have track of the orders of various terms
more carefully. 
To remind you, our wavefunction perturbation expansion is 
\[\ket{\Psi_0} =  \ket{\Phi_0} + \sum_{n=0}^{\infty} (\Rz V')^n \Rz V \ket{\Phi_0} \]
At $n=2$, we have a $V$ and two $V'$ in our term.
To make the term third order overall (this is our target for constructing $\ket{\Phi_0^{(3)}}$),
we require that each $V'$ is 1st order, so we only take out the first order
energy and $V$ in our $V'$ expansion $V' = V - E_0^{(1)}$.
Let's now write this $n=2$ term down before moving on, 
\[\ket{\Phi_0^{(3)}} = (\Rz (V - \Eone_0))^2 \Rz V \ket{\Phi_0} + (... \mathrm{more})\]

There is also a third-order term which appears when $n=1$,
$(\Rz V')^1 \Rz V \ket{\Phi_0}$. That $V'$ has a second order contribution from $E_0^{(2)}$,
therefore we set $V' = -E_0^{(2)}$ and 
pick up an additional term $ \Rz E_0^{(2)} \Rz V \ket{\Phi_0}$ for $\ket{\Phi_0^{(3)}}$
\[\ket{\Phi_0^{(3)}} = (\Rz (V - \Eone_0))^2 \Rz V \ket{\Phi_0} - \Rz E_0^{(2)} \Rz V \ket{\Phi_0} \]
Expanding this and simplifying yields
\[\ket{\Phi_0^{(3)}} = \Rz V \Rz V \Rz V \ket{\Phi_0} - \Rz \Eone_0 \Rz V \Rz V \ket{\Phi_0} 
  - \Rz V \Rz \Eone_0 \Rz V \ket{\Phi_0} + \Rz E_0^{(1)} \Rz E_0^{(1)} \Rz V \ket{\Phi_0} 
 - \Rz E_0^{(2)} \Rz V \ket{\Phi_0} \]

Our energy expansion is 
\[E_0 = \Ezero_0 + \braket{\Phi_0|V|\Phi_0} + \sum_{n=0}^{\infty} \braket{\Phi_0|V (\Rz V')^n \Rz V | \Phi_0}\]

The third order contribution to the energy expression occurs for $n=1$ and is 
\[ E_0^{(3)} = \braket{\Phi_0 | V ( \Rz (V - E_0^{(1)})) \Rz V | \Phi_0} \]
Alternatively, we could have also found the above by noting that generally the $(m+1)$ order energy is found
by the $m$th order wavefunction,
\[ E^{(m+1)} = \braket{\Phi_0 | V | \Phi_0^{(m)}} \]
\[ E^{(3)} = \braket{\Phi_0 | V | \Phi_0^{(2)}} \]
and just plug in our derived $\ket{\Phi_0^{(2)}}$.

\section{The Energy Substitution Theorem}
There is a pattern for finding equations for $\Phi_0^{(m)}$, so you don't have to
go through the tedious process of analyzing where to get terms of a certain order for different
values of $n$ in the expansion, or plucking out the proper terms from $V'$
(but you should know how to do this anyway).

\paragraph{\textbf{Energy Substitution Theorem}}
 $\ket{\Phi_0^{(m)}}$ is equal to the sum of a ``principal term'' 
 $(\Rz V)^m \ket{\Phi_0} $ plus all unique substitutions of 
adjacent factors $(\Rz V)^r$ with $(\Rz E_0^{(r)})$. Each term
in the sum is waited by a sign factor $(-1)^k$ where $k$ is the number of substitutions.
The pair of operators $\Rz V$ closest to the ket $\ket{\Phi_0}$ is not considered for substitutions. \\

We will not prove this, but hopefully by looking at the expressions for  $\ket{\Phi_0^{(2)}}$ and $\ket{\Phi_0^{(3)}}$ that we just derived will convince you:

\[\ket{\Phi_0^{(2)}} = \Rz V \Rz V \ket{\Phi_0} - \Rz \Eone_0 \Rz V \ket{\Phi_0}\]

\[\ket{\Phi_0^{(3)}} = \Rz V \Rz V \Rz V \ket{\Phi_0} - \Rz \Eone_0 \Rz V \Rz V \ket{\Phi_0} 
  - \Rz V \Rz \Eone_0 \Rz V \ket{\Phi_0} + \Rz E_0^{(1)} \Rz E_0^{(1)} \Rz V \ket{\Phi_0} 
 - \Rz E_0^{(2)} \Rz V \ket{\Phi_0} \]


Now we barely even need to think when we write down our perturbative wavefunction corrections.
Keeping in mind the energy corrections follow form
\[ E^{(m+1)} = \braket{\Phi_0 | V | \Phi_0^{(m)}} \]
our wavefunction correction expressions determine our energy expression corrections.

\section{Concluding Remarks}
I should note that we have not actually translated the energy corrections into 
programmable expressions. 
To do this, we would have to actually evaluate the expectation values, and it would 
be a lot of work.
Later on in the course, after learning diagrams, we will revisit PT and learn 
how to diagrammatically derive programmable expressions for the various energy orders.
When we do this, we will cleverly choose $H_0$ and $V$ such that $E_0^{(0)} = E_0^{(1)} = 0$,
which simplifies things a good bit, and also makes it so that each energy
correction is purely a correlation energy contribution. 
For a sneak peak, you can look at handout 6 from the 2017 edition of this course.


That was a long and arduous mathematical climb. But we have finally reached the summit,
and now hopefully you can see that all of perturbation theory is obtainable by just 
smacking your reference state $\ket{\Phi_0}$ with operator strings consisting 
of resolvents the pertubed part of the Hamiltonian $V$. 
Compare this to how we derived MP2 in the previous set of notes; we had to do a bunch of 
algebraic manipulations to get to the end result. 
As you can imagine, deriving MP3 and MP4 under that framework only gets worse. 
In L{\"o}wdin's resolvent formalism, we can just read off the result, 
    and then evaluate it to an algebraic expression using diagrams. 

\section{Appendix}
\subsection{Hilbert Space}
Recall that many of the objects we have been talking about (spin orbitals, Slater determinants, wavefunctions)
``live'' in some Hilbert space. This means many things, but for our purposes, we just need to know
there is some ``space'' of \textit{abstract vectors} which have a defined inner product with a certain set of properties.
Every abstract vector in the space can be described in terms of the orthonormal basis which makes up the space. 

\begin{itemize}
\item The set of all spin orbitals form a basis for what you might call the one-electron Hilbert space $\mathcal{H}$
\item $n$ tensor products of $\mathcal{H}$ with itself forms an $n$-electron Hilbert space $\mathcal{H}^n$. 
        A member of the basis of such a space is a product of spin orbitals (Hartree products)
\item Denote the \textit{antisymmetric subspace} of $\mathcal{H}^n$ as $\mathscr{H}^n$. The basis of this space
    is the set of $n$-electron Slater determinants. All possible Slater determinants for an $n$-electron system live in $\mathscr{H}^n$ 
    The full CI wavefunction for an $n$-electron system can be constructed from a linear combination of basis vectors in $\mathscr{H}^n$
\end{itemize}

\subsection{Partitioning}
Consider the following \textit{partition} of this $n$-electron antisymmetric Hilbert space $\mathscr{H}^n$ into
a ``$P$'' space and a ``$Q$'' space.
\[ \mathscr{H}^n = \mathscr{H}^n_P \oplus  \mathscr{H}^n_Q \]
where $P$ is the \textit{projection operator} onto some reference determinant 
\[P = \ket{\Phi_0}\bra{\Phi_0}  \]
and $Q$ is the \textit{orthogonal complement} of $P$, which projects onto every determinant other than the reference $\Phi_0$
\[Q = 1 - P = \sum_{n \neq 0}  \ket{\Phi_n}\bra{\Phi_n} \]
You can think of this as just dividing your Slater determinant basis into two parts, $P$ and $Q$.
The sum $P + Q$ is equal to unity. This can be most easily seen by thinking of $P$ and $Q$ as two distinct chunks of the resolution of the identity. 





\end{document}
