\documentclass{article}
\usepackage{simplewick}
\usepackage{amsmath,mathtools,amssymb}
\usepackage{graphicx}
\usepackage{gensymb}
\usepackage{verbatim}
\usepackage{cancel}
\usepackage{mathrsfs}
\usepackage{bbm}
\usepackage{braket}
\usepackage{xcolor}
\definecolor{orange}{RGB}{255,127,0}
\definecolor{green}{RGB}{34,139,34}
\definecolor{purple}{RGB}{153,50,204}
\definecolor{blue}{RGB}{70,130,180}
\usepackage{verbatim}
\usepackage[margin=1.0in]{geometry}
\newcommand{\ol}{\overline}
\newcommand{\ctr}{\bcontraction}
\newcommand{\fctr}{\contraction}
\newcommand{\ve}{\varepsilon}
\newcommand{\kphi}{\ensuremath{\ket{\Phi}} }
\newcommand{\ap}{\ensuremath{a_p} }
\newcommand{\dg}{\ensuremath{^\dagger} }
\newcommand{\cd}{\ensuremath{\cdots} }
\newcommand{\apd}{\ensuremath{a_p^\dagger} }
\def\*#1{\mathbf{#1}}
\def\cb#1{{\color{blue}#1}}
\def\co#1{{\color{orange}#1}}
\def\cre#1{{\color{red}#1}}
\def\cg#1{{\color{green}#1}}
\def\cp#1{{\color{purple}#1}}
\DeclarePairedDelimiter\floor{\lfloor}{\rfloor}

\title{Lecture 3.6:  $a_p$ and $a_p\dg$ in the Particle-Hole Formalism}
\date{February 26$^{\text{th}}$, 2020}
\begin{document}
\maketitle
\noindent
Previously, we saw that to work in the Fermi vacuum, we could first translate 
particle operators $a_p$ and $a_p\dg$ into quasiparticle operators $b_p$ and $b_p\dg$. 
This was not efficient because for each $p$ there are two cases to consider: occupied and unoccupied.
For a string of 4 operators, $a_p\dg a_q\dg a_s a_r$, this would mean we have 16 terms in the particle-hole formalism to deal with. 
Instead, we want to work directly with $a_p$ and $a_p\dg$ in the particle-hole formalism. 
Our next task is to see how to do that. 
\section{ Working with $a_p$ and $a_p\dg$ in the particle-hole formalism }
\subsection{Anticommuntation relations}
The anticommuntation relations for $a_p$ and $a_p\dg$ stays the same.
Strictly speaking, we see that the definitions depend on what the index $p$ is,
for example:
\[{a_i, a_a} = {b_i\dg, b_a} = \delta_{ia}\]
However, this quantity ends up being 0 because $\delta_{ia}$ is always 0. 
We can make this sort of argument for all mixed cases of occupied and unoccupied indices to see that 
the anticommuntation relations remains:
\begin{enumerate}
\item  $\{a_p, a_q\} = 0$
\item $\{a_p^\dagger, a_q^\dagger \} = 0$
\item $\{a_p, a_q\dg \} = \delta_{pq}$
\end{enumerate}
\subsection{$\Phi$-Normal products using $a_p$ and $a_p\dg$}
How do we evaluate $N[a_p\dg a_q a_r\dg]$?
We need to translate back to quasiparticle operators. 
In order to do that, we have to specify occupation for the indices.  \\ \\
In the case where $p=i$, $q=j$, and $r=a$, 
\begin{align*}
N[a_i\dg a_j a_a\dg] &= N[b_i b_j\dg b_a\dg] \\
&= b_j\dg b_a\dg b_i \\
&= a_j a_a\dg a_i\dg \\
&\neq n[a_i\dg a_j a_a\dg]
\end{align*}
Thus, we see that $N[x_1\cd x_m] \neq n[x_1 \cd x_m]$, and the relationship between the two depends on the indices. \\  \\
A formal definition for the normal product is: 
\[N[x_1 \cd x_m]  = (-1)^R b_{R_1}\dg \cd b_{R_\alpha}\dg b_{R_{\alpha+1}} \cd b_{R_m} \]
Thus, a $\Phi$-normal product using $a_p$ and $a_p\dg$ still means that all the quasiparticle operators $b_p\dg$ are to the left and $b_p$ are to the right. 
In order to explicitly write a form for $N[x_1 \cd x_m]$, we would need to know the occupation of the indices. 
\subsection{Contractions in the Fermi vacuum using $a_p$ and $a_p\dg$}
To figure out what contractions between $a_p$ and $a_p\dg$ look like with respect to the Fermi vacuum, we first introduce the step functions $\chi$ and 
$\pi$: 
\[
\chi (p) =
    \begin{cases}
      1  & p \in occupied \\
      0 & p \in unoccupied
    \end{cases}       
\]

\[
  \pi(p) =
    \begin{cases}
      0  & p \in occupied \\
      1  & p \in unoccupied
    \end{cases}       
\]
For example, 
\[\chi(i) = 1 \quad  \quad \chi(a) = 0 \quad  \quad \pi(i) = 0 \quad   \quad \pi(a) = 1\]
\\ \\
The step functions have the following properties: 
 \[\chi(p) + \pi(p) = 1 \]
 \[\chi(p)\pi(p) = 0 \]
\[\chi^2(p) = \chi(p) \]
\[\pi^2(p) = \pi(p) \]
We can use the step functions to inform how to map between particle operators and quasiparticle operators.
For example, $\chi(p)a_p = b_p\dg$.\\ \\ 
Let's see how we can figure out how to compute a contraction in the Fermi vacuum for $a_p$ and $a_q$: 
\begin{align*}
 \fctr{}{a}{{}_p}{a}
 a_p a_q  &=(1)(1) \fctr{}{a}{{}_p}{a} a_p a_q \\
  &= [\chi(p) + \pi(p)][\chi(q) + \pi(q)] \fctr{}{a}{{}_p}{a} a_p a_q \\
 &= \chi(p)\chi(q)\fctr{}{a}{{}_p}{a} a_p a_q + \chi(p) \pi(q) \fctr{}{a}{{}_p}{a} a_p a_q  +  \pi(p)\chi(q) \fctr{}{a}{{}_p}{a} a_p a_q  + 
 \pi(p) \pi(q) \fctr{}{a}{{}_p}{a} a_p a_q 
\end{align*}
Now, the step functions will inform what b operator to translate to: 
\begin{align*}
 \fctr{}{a}{{}_p}{a}a_p a_q 
 &= \chi(p)\chi(q)\fctr{}{a}{{}_p}{a} a_p a_q + \chi(p) \pi(q) \fctr{}{a}{{}_p}{a} a_p a_q  +  \pi(p)\chi(q) \fctr{}{a}{{}_p}{a} a_p a_q  + 
 \pi(p) \pi(q) \fctr{}{a}{{}_p}{a} a_p a_q  \\
&= \chi(p)\chi(q)\fctr{}{b}{{}_p\dg}{b} b_p\dg b_q\dg + \chi(p) \pi(q) \fctr{}{b}{{}_p\dg}{b} b_p\dg b_q  +  \pi(p)\chi(q) \fctr{}{b}{{}_p}{b} b_p b_q\dg  + 
 \pi(p) \pi(q) \fctr{}{b}{{}_p}{b} b_p b_q \\ \\
&  \text{we can get rid of all contractions that go to zero} \\
&= \cre{\cancelto{0}{\chi(p)\chi(q)\fctr{}{b}{{}_p\dg}{b} b_p\dg b_q\dg}} + \cre{\cancelto{0}{\chi(p) \pi(q) \fctr{}{b}{{}_p\dg}{b} b_p\dg b_q}}  +  \pi(p)\chi(q) \fctr{}{b}{{}_p}{b} b_p b_q\dg  + 
\cre{\cancelto{0}{ \pi(p) \pi(q) \fctr{}{b}{{}_p}{b} b_p b_q }}\\
&=  \pi(p)\chi(q) \delta_{pq} \\
&=  \pi(p)\chi(p) \\
&= 0 
\end{align*}
We can take the same steps to find $\fctr{}{a}{{}_p\dg}{a} a_p\dg a_q $: 
\begin{align*}
 \fctr{}{a}{{}_p\dg}{a}
 a_p\dg a_q  &= [\chi(p) + \pi(p)][\chi(q) + \pi(q)] \fctr{}{a}{{}_p\dg}{a} a_p\dg a_q \\
 &= \chi(p)\chi(q)\fctr{}{a}{{}_p\dg}{a} a_p\dg a_q + \chi(p) \pi(q) \fctr{}{a}{{}_p\dg}{a} a_p\dg a_q  +  \pi(p)\chi(q) \fctr{}{a}{{}_p\dg}{a} a_p\dg a_q  + 
 \pi(p) \pi(q) \fctr{}{a}{{}_p\dg}{a} a_p\dg a_q \\
 &= \chi(p)\chi(q)\fctr{}{b}{{}_p}{b} b_p b_q\dg + \chi(p) \pi(q) \fctr{}{b}{{}_p}{b} b_p b_q  +  \pi(p)\chi(q) \fctr{}{b}{{}_p\dg}{b} b_p\dg b_q\dg  + 
 \pi(p) \pi(q) \fctr{}{b}{{}_p\dg}{b} b_p\dg b_q \\
 &= \chi(p)\chi(q)\fctr{}{b}{{}_p}{b} b_p b_q\dg + \cre{\cancelto{0}{\chi(p) \pi(q) \fctr{}{b}{{}_p}{b} b_p b_q }} +  \cre{\cancelto{0}{ \pi(p)\chi(q) \fctr{}{b}{{}_p\dg}{b} b_p\dg b_q\dg}}  + 
 \cre{\cancelto{0}{\pi(p) \pi(q) \fctr{}{b}{{}_p\dg}{b} b_p\dg b_q }}\\
 &= \chi(p)\chi(q) \delta_{pq} \\
 &= \chi(p)\chi(p) \delta_{pq} \\
 &= \chi^2(p)\delta_{pq} \\
 &= \chi(p)\delta_{pq} \\
  &= \gamma_{pq}\\
\end{align*}
We can do the same type of derivation for $ \fctr{}{a}{{}_p}{a}a_p a_q\dg$ and  $\fctr{}{a}{{}_p}{a} a_p\dg a_q\dg$.
You will practice this in your homework.
The end result is: 
 $$
 \fctr{}{a}{{}_p}{a}
 a_p a_q  = 0 
 $$

 $$
 \fctr{}{a}{{}_p}{a}
 a_p a_q\dg = \pi(p)\delta_{pq} = \eta_{pq}
 $$

 $$
 \fctr{}{a}{{}_p}{a}
 a_p\dg a_q =   \chi(p)\delta_{pq} = \gamma_{pq}
 $$

 $$
 \fctr{}{a}{{}_p}{a}
 a_p\dg a_q\dg = 0
 $$
where we have defined $ \eta_{pq} = \pi(p)\delta_{pq}$ and $ \gamma_{pq} =  \chi(p)\delta_{pq} $. 

\subsection{$\Phi$-Normal products using $a_p$ and $a_p\dg$}
From what we know so far, we see that a $\Phi$-normal product with contractions inside for particle operators
\[ 
\fctr{N[x_1 \cd }{x}{{}_{i_1} \cd x_{i_\lambda} \cd }{x}
\fctr[2ex]{N[x_1 \cd x_{i_1} \cd}{ x}{{}_{i_\lambda} \cd x_{j_1} \cd }{x} 
N[x_1 \cd x_{i_1} \cd x_{i_\lambda} \cd x_{j_1} \cd x_{j_\lambda} \cd x_m] 
\] 
can be rewritten as: 
\[ 
\fctr{N[x_1 \cd }{x}{{}_{i_1} \cd x_{i_\lambda} \cd }{x}
\fctr[2ex]{N[x_1 \cd x_{i_1} \cd}{ x}{{}_{i_\lambda} \cd x_{j_1} \cd }{x} 
N[x_1 \cd x_{i_1} \cd x_{i_\lambda} \cd x_{j_1} \cd x_{j_\lambda} \cd x_m] 
= 
\fctr{(-1)^R}{x}{{}_{i_1} }{x}
\fctr{(-1)^R x_{i_1}x_{j_1} \cd}{x}{{}_{i_\lambda} }{x}
(-1)^R x_{i_1}x_{j_1} \cd x_{i_\lambda} x_{j_\lambda} N[x_{k_1} \cd x_{k_\mu}]
\] 
where 
\[
R = 
\begin{pmatrix}
1 & 2 & \cd & 2\lambda -1 & 2\lambda & 2\lambda + 1&\cd & m  \\
i_1 & j_1 & \cd& i_\lambda & j_\lambda  &k_1 & \cd & k_\mu \\
\end{pmatrix}
\] \\
and the indices $2\lambda + \mu = m$.
\\ \\
For example: 
\begin{align*}
N[ \fctr[2ex]{}{a}{{}_p\dg a_q a_r\dg}{a}
\fctr{ a_p\dg}{a}{{}_q }{a}
 a_p\dg a_q a_r\dg a_s \fctr{}{a}{{}_t}{a}a_t a_u\dg ] 
 &= \fctr{}{a}{{}_p\dg}{a} a_p\dg a_s \fctr{}{a}{{}_q}{a}a_q a_r\dg\fctr{}{a}{{}_t}{a}a_t a_u\dg \\
 &= [\chi(p)\delta_{ps}] [\pi(p)\delta_{qr}] [\pi(t) \delta_{tu}]  \\
 &= \gamma_{ps} \eta_{qr} \eta_{tu}
\end{align*}

\subsection{Wick's Theorem}
Now that we have addressed what $\Phi$-normal ordering and $\Phi$-normal ordering with contractions mean for operators $a_p$ and $a_p\dg$, we are ready to state Wick's theorem for 
$a_p$ and $a_p\dg$ in the particle-hole formalism: 
\[x_1 \cd x_m = N[x_1\cd x_m] + \sum_{a.c.} N\ol{[x_1 \cd x_m]} \]
\\ \\
The generalized Wick's Theorem states: 
\[x_1 \cd N[ x_\nu \cd x_\mu] \cd  x_m = N[x_1\cd x_m] + \sum_{a.c.}{}' N\ol{[x_1 \cd x_m]} \]
Where $ \sum_{a.c.}{}'$ stands for all possible contractions except the ones between the operators within the $\Phi$-normal order. 
\subsection{Rules for expectation values}
We can now see what the rules imply for $a_p$ and $a_p\dg$ in the Fermi vacuum. 
\begin{itemize}
\item Rule 1: $N[x_1 \cd x_m]\kphi = 0$ unless $x_1 \cd x_m$ only contains $a_p$ where $p \in occ$ and $a_q\dg$ where $q \in unocc$.
\item Rule 2: $\braket{\Phi | N[x_1 \cd x_m] | \Phi} = 0$ for  $m \geq 1$ \\
\item Rule 3. $\fctr{N[x_1 \cd }{x}{{}_{\nu_1} \cd x_{\nu_\lambda} \cd }{x}
\fctr[2ex]{N[x_1 \cd x_{\nu_1} \cd}{ x}{{}_{\nu_\lambda} \cd x_{\mu_1} \cd }{x} 
N[x_1 \cd x_{\nu_1} \cd x_{\nu_\lambda} \cd x_{\mu_1} \cd x_{\mu_\lambda} \cd x_m] \kphi= 0$ if there is at least 1 uncontracted 
operator $a_p\dg$ where $p \in occ$ or $a_q$ where $q \in unocc$  \\
\item Rule 4. $\bra{\Phi}\fctr{N[x_1 \cd }{x}{{}_{\nu_1} \cd x_{\nu_\lambda} \cd }{x}
\fctr[2ex]{N[x_1 \cd x_{\nu_1} \cd}{ x}{{}_{\nu_\lambda} \cd x_{\mu_1} \cd }{x} 
N[x_1 \cd x_{\nu_1} \cd x_{\nu_\lambda} \cd x_{\mu_1} \cd x_{\mu_\lambda} \cd x_m] \kphi= 0$ unless all operators are contracted.
A vacuum expectation value of any normal product with contractions will be zero if there are a odd number of operators. \\

\item Rule 5 $\braket{\Phi | x_1 \cd x_{2m + 1} | \Phi} = 0 $ \\

\item Rule 6 $\braket{\Phi | x_1 \cd x_{2m } | \Phi} = \sum\limits_{f.c.} \braket{\Phi | N\ol{\ol{ [x_1 \cd x_{2m }]}}| \Phi} = \sum_{f.c.} (-1)^R 
\fctr{}{x}{{}_{\nu_1}}{x} x_{\nu_1} x_{\mu_1} \cd \fctr{}{x}{{}_{\nu_m}}{x} x_{\nu_m} x_{\mu_m}$\\

\begin{equation*}
R = 
\begin{pmatrix}
1 & 2 &\cd  & 2m - 1 & 2m  \\
\nu_1 & \mu_1 & \cd  & \nu_m & \mu_m \\
\end{pmatrix}
\end{equation*}

\item Rule 7  $x_1 \cd x_m \kphi = \sum\limits_{b_p f.c.} N\ol{[ x_1 \cd x_m]}\kphi$
where $ \sum\limits_{b_p f.c.}$ denotes that any $a_p\dg$ where $p \in occ$ or $a_q$ where $q \in unocc$ is fully contracted.
\end{itemize}
\subsection{Example: Deriving Slater's first rule} 

We can use what we have learned so far to quickly derive Slater's first rule. 

\[\braket{\Phi | H | \Phi} = \braket{\Phi | \sum_{pq} h_{pq} a_p\dg a_q + \frac{1}{2} \sum_{pqrs} \braket{pq | rs} a_p\dg a_q\dg a_s a_r | \Phi} \]
One-electron term: 
\begin{align*}
 \sum_{pq} h_{pq} \braket{\Phi | a_p\dg a_q | \Phi} &=   \sum_{pq} h_{pq} \braket{\Phi | N[\fctr{}{a}{{}_p\dg}{a} a_p\dg a_q] | \Phi}) \\
 &=   \sum_{pq} h_{pq} \chi(p)\delta_{pq} \braket{\Phi | \Phi}  \\
 &=  \sum_{pq} h_{pq} \gamma_{pq} \\
 &=  \sum_{ij} h_{ij}\delta_{ij} \\
 &= \sum_i h_{ii}
\end{align*}
Two-electron term: 
\begin{align*}
\frac{1}{2} \sum_{pqrs} \braket{pq | rs} \braket{\Phi| a_p\dg a_q\dg a_s a_r | \Phi} &= \frac{1}{2} \sum_{pqrs} \braket{pq | rs}  \sum_{f.c.}\braket{\Phi| N\ol{\ol{[a_p\dg a_q\dg a_s a_r]}} | \Phi} \\
&=  \frac{1}{2} \sum_{pqrs} \braket{pq | rs} ( \braket{\Phi| N[\fctr[2ex]{}{a}{{}_p\dg a_q\dg a_s}{a} \fctr{a_p\dg }{a}{{}_q\dg}{a} a_p\dg a_q\dg a_s a_r] | \Phi} 
+ \braket{\Phi| N[ \fctr{}{a}{{}_p\dg a_q\dg}{a}  \fctr[2ex]{a_p\dg }{a}{{}_q\dg a_s}{a} a_p\dg a_q\dg a_s a_r] | \Phi} )\\
&= \frac{1}{2} \sum_{pqrs} \braket{pq | rs}  (\gamma_{pr}\gamma_{qs} - \gamma_{ps}\gamma_{qr}) \\
&\text{make assignments } p\rightarrow i \quad  q \rightarrow j \quad  r\rightarrow k \quad  s\rightarrow l \\
&= \frac{1}{2} \sum_{ijkl} \braket{ij | kl}  (\delta_{ik}\delta_{jl} - \delta_{il}\delta_{jk}) \\
&= \frac{1}{2} \sum_{ij}  \braket{ij | ij} -   \braket{ij | ji} \\
&= \frac{1}{2} \sum_{ij} \braket{ij || ij} 
\end{align*}
What if we wanted to solve for matrix elements of terms like $ \braket{\Phi_{kl}^{cd} | H |\Phi_{ij}^{ab}}$?
This means we are working with 12 operators: 
\[\braket{\Phi | \cg{a_l\dg a_d a_k\dg a_c }\cb{  a_p\dg a_q\dg a_s a_r } \co{a_a\dg a_i a_b\dg a_j } | \Phi} \]
Previously, in the particle formalism, we took advantage of the Generalized Wick's theorem to reduce the number of contractions we have to consider.
We only considered contractions between operators that were not from the same normal-ordered group. 
We see that if we could put different groups in the above term in $\Phi$-normal ordering, we can reduce the number of terms we have to consider.

\section{Excitation operators from \kphi are already in $\Phi$-normal order}
We see that the operators associated with excitations from the reference determinant are already in $\Phi$-normal order. \\ \\
For example,
\[a_a\dg a_i a_b\dg a_j = b_a\dg b_i\dg b_b\dg b_j\dg = N[b_a\dg b_i\dg b_b\dg b_j\dg] = N[ a_a\dg a_i a_b\dg a_j ]\]
\\
We see that using anticommuntation relations, we can also equivalently write: 
 \[N[b_a\dg b_i\dg b_b\dg b_j\dg] = N[b_a\dg b_b\dg b_j\dg b_i\dg  ] = N[ a_a\dg a_b\dg  a_j a_i  ]\]
 Both forms can be used to express excitations from the reference determinant. 

\section{The $\Phi$-normal ordered Hamiltonian}
We can use Wick's theorem to help put the Hamiltonian in $\Phi$-normal order.
\\ \\
One-electron operator:
\begin{align*}
\hat{h} &= \sum_{pq} h_{pq} a_p\dg a_q \\
 &= \sum_{pq}h_{pq} (N[a_p\dg a_q] + N[\fctr{}{a}{{}_p\dg}{a} a_p\dg a_q]) \\
&= \sum_{pq}h_{pq} N[a_p\dg a_q] + \sum_{pq}h_{pq} \gamma_{pq} \\
&= \sum_{pq}h_{pq} N[a_p\dg a_q] + \sum_{i}h_{ii} \\
\end{align*}
Two-electron operator:
\begin{align*}
\hat{g} &=  \frac{1}{2} \sum_{pqrs} \braket{pq | rs} a_p\dg a_q\dg a_s a_r \\
&= \cg{\frac{1}{2} \sum_{pqrs} \braket{pq | rs} N[a_p\dg a_q\dg a_s a_r] } \\ \\
&+ \cb{\frac{1}{2} \sum_{pqrs} \braket{pq | rs} \left(N[\fctr{}{a}{{}_p\dg a_q\dg}{a}  a_p\dg a_q\dg a_s a_r] + 
N[\fctr{}{a}{{}_p\dg a_q\dg a_s}{a}  a_p\dg a_q\dg a_s a_r] + 
N[\fctr{a_p\dg}{a}{{}_q\dg }{a}  a_p\dg a_q\dg a_s a_r] + 
N[\fctr{a_p\dg}{a}{{}_q\dg a_s }{a}  a_p\dg a_q\dg a_s a_r] \right)} \\
&+ \co{ \frac{1}{2} \sum_{pqrs} \braket{pq | rs} \left( 
N[\fctr[2ex]{}{a}{{}_p\dg a_q\dg a_s }{a} \fctr{a_p\dg}{a}{{}_q\dg}{a}  a_p\dg a_q\dg a_s a_r] +
N[\fctr{}{a}{{}_p\dg a_q\dg }{a}  \fctr[2ex]{a_p\dg}{a}{{}q\dg a_s}{a} a_p\dg a_q\dg a_s a_r]  
  \right) } \\  \\
&= \cg{\frac{1}{2} \sum_{pqrs} \braket{pq | rs} N[a_p\dg a_q\dg a_s a_r] } \\ 
&+ \cb{\frac{1}{2} \sum_{pqrs} \braket{pq | rs} \left(-\gamma_{ps} N[ a_q\dg a_r] + 
\gamma_{pr} N[ a_q\dg a_s ] + 
\gamma_{qs} N[ a_p\dg a_r] -
\gamma_{qr} N[ a_p\dg a_s] \right)} \\
&+ \co{ \frac{1}{2} \sum_{pqrs} \braket{pq | rs} ( \gamma_{pr}\gamma_{qs} -\gamma_{ps}\gamma_{qr}) }\\ \\
&= \cg{\frac{1}{2} \sum_{pqrs} \braket{pq | rs} N[a_p\dg a_q\dg a_s a_r] } \\
& \cb{- \frac{1}{2} \sum_{qr} \sum_i \braket{iq | ri} N[ a_q\dg a_r] +  
\frac{1}{4} \sum_{qs} \sum_i \braket{iq | is} N[ a_q\dg a_s ] + 
\frac{1}{4} \sum_{pr} \sum_i \braket{pi | ri} N[ a_p\dg a_r] -
\frac{1}{4} \sum_{ps} \sum_i \braket{pi | is} N[ a_p\dg a_s] } \\
&+ \co{ \frac{1}{2} \sum_{ij} \braket{ij || ij} }
\end{align*}
We can perform a change of variables: 
\begin{align*}
\hat{g}&= \cg{\frac{1}{2} \sum_{pqrs} \braket{pq | rs} N[a_p\dg a_q\dg a_s a_r] } \\
& \cb{- \frac{1}{2} \sum_{pq} \sum_i \braket{ip | qi} N[ a_p\dg a_q] +  
\frac{1}{2} \sum_{pq} \sum_i \braket{ip | iq} N[ a_p\dg a_q ] + 
\frac{1}{2} \sum_{pq} \sum_i \braket{pi | qi} N[ a_p\dg a_q] -
\frac{1}{2} \sum_{pq} \sum_i \braket{pi | iq} N[ a_p\dg a_q] } \\
&+ \co{ \frac{1}{2} \sum_{ij} \braket{ij || ij} } \\ \\
&= \cg{\frac{1}{2} \sum_{pqrs} \braket{pq | rs} N[a_p\dg a_q\dg a_s a_r] } \\
& \cb{+ \frac{1}{2} \sum_{pq} \sum_i \braket{pi | qi} N[ a_p\dg a_q] +  
\frac{1}{2} \sum_{pq} \sum_i \braket{pi | qi} N[ a_p\dg a_q ] - 
\frac{1}{2} \sum_{pq} \sum_i \braket{pi | iq} N[ a_p\dg a_q] -
\frac{1}{2} \sum_{pq} \sum_i \braket{pi | iq} N[ a_p\dg a_q] } \\
&+ \co{ \frac{1}{2} \sum_{ij} \braket{ij || ij} }\\ \\
&= \cg{\frac{1}{2} \sum_{pqrs} \braket{pq | rs} N[a_p\dg a_q\dg a_s a_r] } \\
& \cb{+ \sum_{pq} \sum_i ( \braket{pi | qi} - \braket{pi | iq})  N[ a_p\dg a_q]} \\
&+ \co{ \frac{1}{2} \sum_{ij} \braket{ij || ij} } \\ \\
&=  \cg{\frac{1}{2} \sum_{pqrs} \braket{pq | rs} N[a_p\dg a_q\dg a_s a_r] } +  \cb{ \sum_{pq} \sum_i \braket{pi || qi}   N[ a_p\dg a_q]} + \co{ \frac{1}{2} \sum_{ij} \braket{ij || ij} } 
\end{align*}
All together, we get: 
\[H = \sum_{pq}h_{pq} N[a_p\dg a_q] + \sum_{i}h_{ii} + \frac{1}{2} \sum_{pqrs} \braket{pq | rs} N[a_p\dg a_q\dg a_s a_r]  +   \sum_{pq} \sum_i \braket{pi || qi}   N[ a_p\dg a_q] +  \frac{1}{2} \sum_{ij} \braket{ij || ij}  \]
We see that the second and fifth terms have no operators, and
the first and fourth terms both have 2 operators.
We can group these terms together: 
\[H = \co{ \sum_{i}h_{ii} + \frac{1}{2} \sum_{ij} \braket{ij || ij}}  + \cb{\sum_{pq}h_{pq} N[a_p\dg a_q]  +   \sum_{pq} \sum_i \braket{pi || qi}   N[ a_p\dg a_q]}
  + \cg{\frac{1}{2} \sum_{pqrs} \braket{pq | rs} N[a_p\dg a_q\dg a_s a_r]}  \]
  The first two terms we see are simply $\braket{\Phi |H|\Phi}$: 
 \[H = \co{ \braket{\Phi |H|\Phi}} + \cb{\sum_{pq}h_{pq} N[a_p\dg a_q]  +   \sum_{pq} \sum_i \braket{pi || qi}   N[ a_p\dg a_q]}
  + \cg{\frac{1}{2} \sum_{pqrs} \braket{pq | rs} N[a_p\dg a_q\dg a_s a_r]}  \]
  The remaining terms now form our $\Phi$-normal ordered Hamiltonian, $\hat{H}_N$:
   \[H = \co{ \braket{\Phi |H|\Phi}} + H_N\]
   \[H_N = \cb{\sum_{pq}h_{pq} N[a_p\dg a_q]  +   \sum_{pq} \sum_i \braket{pi || qi}   N[ a_p\dg a_q]}
  + \cg{\frac{1}{2} \sum_{pqrs} \braket{pq | rs} N[a_p\dg a_q\dg a_s a_r]} \]
  The $\Phi$-normal ordered Hamiltonian can be further divided into 2 terms: 
     \[H_N = \cb{F_N} + \cg{G_N} \]
     \[  \cb{F_N = \sum_{pq}h_{pq} N[a_p\dg a_q]  +   \sum_{pq} \sum_i \braket{pi || qi}   N[ a_p\dg a_q]} \]
     \[ \cg{G_N = \frac{1}{2} \sum_{pqrs} \braket{pq | rs} N[a_p\dg a_q\dg a_s a_r]}\]
  $F_N$, called the Fock operator, is an effective one-electron operator.
  The first term tells us about state q leaving and arriving at state p. 
  The second term tells about state q leaving and arriving at state p \textit{in the presence of occupied orbitals i}. 
  It describes an effective one-body interaction that comes from a two-body operator.
  In other words, it contains information about two-body interactions in a mean-field manner. 
$G_N$ is the usual two-electron operator. 
\\ \\ 
We can re-express $F_N$ as: 
 \[  \cb{F_N = \sum_{pq} (h_{pq} + \sum_i \braket{pi || qi}  ) N[a_p\dg a_q] }  \]
  \[  \cb{F_N = \sum_{pq} f_{pq} N[a_p\dg a_q] } \]
  And re-express $H_N$ as: 
       \[H_N =  \cb{\sum_{pq} f_{pq} N[a_p\dg a_q] }  + \cg{ \frac{1}{2} \sum_{pqrs} \braket{pq | rs} N[a_p\dg a_q\dg a_s a_r] }\]
 \section{Implications of the $\Phi$-normal ordered Hamiltonian}
 We have thus arrived at a $\Phi$-normal ordered Hamiltonian:
 \[ H_N =  H-  \braket{\Phi |H|\Phi}\]
What have we done qualitatively? 
We see that the $\Phi$-normal ordered Hamiltonian is obtained by shifting the original Hamiltonian by the expectation value in the reference determinant. 
In other words, it is removing the energy contribution of the reference determinant. 
This makes sense. 
In our original mapping to the particle-hole formalism, we make the reference determinant a vacuum state. 
Thus, the $\Phi$-normal ordered Hamiltonian, which is in the particle-hole formalism does not have any interaction with reference determinant.
In the same way that in our original formalism,
\[\braket{0| H | 0} = 0\]
In the particle-hole formalism, 
\[\braket{\Phi| H_N | \Phi} = 0\]
 Mathematically, the choice of the reference determinant is arbitrary. It does not matter which determinant you use in the wavefunction expansion.
 Practically, it is more efficient to choose a determinant that dominants the wavefunction expansion (large coefficient). 
 If we choose our reference determinant \kphi to be the Hartree-Fock result, then shifting the Hamiltonian by the Hartree-Fock
 energy $E_{HF} = \braket{\Phi |H|\Phi}$ means we are really just defining a Hamiltonian that takes care of correlation effects. 
\[H = E_{HF} + H_{correlation} \]
 The $\Phi$-normal ordered Hamiltonian is thus equal to a correlation Hamiltonian $H_c$: 
 \[H_N = H_c \]
\end{document}