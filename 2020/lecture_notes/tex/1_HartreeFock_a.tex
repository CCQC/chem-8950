\documentclass[11pt]{article}
\usepackage[T1]{fontenc}
\usepackage{mathpazo}
\usepackage{graphicx}
\usepackage{amsmath}
\usepackage{caption}
\usepackage{adjustbox} % Used to constrain images to a maximum size 
\usepackage{xcolor} % Allow colors to be defined
\usepackage{enumerate} % Needed for markdown enumerations to work
\usepackage{geometry} % Used to adjust the document margins
\usepackage{amsmath} % Equations
\usepackage{amssymb} % Equations
\usepackage{textcomp} % defines textquotesingle
\usepackage{upquote} % Upright quotes for verbatim code
\title{Hartree Fock Part I}
\geometry{verbose,tmargin=1in,bmargin=1in,lmargin=1in,rmargin=1in}
\begin{document}
\maketitle

The task at hand is to apply the time-independent Schrodinger equation to molecular systems.

\[\hat{H} |\Psi \rangle = E |\Psi \rangle \]

Earlier we defined our clamped-nuclei electronic Hamiltonian, in atomic units, as the following:

\[ \hat{H} = -\sum\limits_{i} \frac{1}{2} \nabla^2_{\boldsymbol{r}_i} - \sum\limits_{i} \sum\limits_{A} \frac{Z_A}{|\boldsymbol{R}_A - \boldsymbol{r}_i|} + \sum\limits_{i < j} \frac{1}{\boldsymbol{r}_{ij}} + V_{\mathrm{nuc}} \]

where the sum over $i \textless{} j$ is just a sum over all pairs of
$i,j$ up to N that obey that inequality $(1,2), (1,3), .. (N-1,N)$.
From here on, we are going to drop the nuclear repulsion energy term
$V_{\mathrm{nuc}}$ since it just a constant factor we will add to the
electronic energy later. It is much more convenient to express the Hamiltonian
as a sum of one electron and two-electron operators

\[ \hat{H} = \sum\limits_{i} \hat{h}(i) + \sum\limits_{i < j} \hat{g}(i,j)  \]

where we express them suggestively as functions of the electronic
coordinates of electron \(i\) and \(j\). Note the sum over nuclei \(A\)
is gone; since the nuclei are fixed, we can just remove the sum, and
understand that the nuclear-electron attraction requires a sum over all
nuclei.

The remaining task is define a proper wavefunction.

    
\section{Molecular Orbital Theory}
\subsection{Spin Orbitals}
In molecular orbital theory, we build approximate electronic
wavefunctions from a set of one-electron wavefunctions $\{\phi\}$. These
one-electron wavefunctions are called \emph{molecular orbitals} (MOs),
and they depend on the spatial coordinates of a single electron.

\[ \phi _i ^ \mu = \phi_i(r^\mu) = \phi_i(x^\mu, y^\mu, z^\mu) \]

where the \(\mu\)th electron is placed in the \(i\)th MO. The total
\(N\)-electron wavefunction is built from a product of these MO's.
However, from the Pauli Principle, no more than two electrons can be
occupied in each MO, and two electrons in the same MO must have opposite
spins. A more general statement is that of the Spin Statistics Theorem,
which states that wavefunctions describing fermions (like electrons)
must change sign under the exchange of two fermions. We must build in a
notion of spin into our MOs, and ensure that our \(N\)-electron
wavefunction is antisymmetric (changes sign) when two electron's
positions are swapped. We introduce the \emph{molecular spin orbital}
(MSO) (or, just \textbf{spin orbitals})

\[\psi_i^\mu(r,s) = \phi_i(r^\mu) \omega_i(s^\mu)\]

where the 'spin coordinate' \(s\) is just some abstract, discrete,
two-level quantity. You can think of it as taking on values of \(+\) or
\(-\). It is not really used directly, but is needed to express a
rigorous inner product (and thus an integral over some coordinate). In
practice, the spin function takes on either spin up (\(\alpha\)) or spin
down (\(\beta\)) status. These spin states are orthonormal, and
the integrations are expressed as

\[\langle \alpha | \alpha \rangle =  \int \alpha^*(s) \alpha(s) ds = 1\]

\[\langle \beta | \beta \rangle = \int \beta^*(s) \beta(s) ds = 1 \]

\[\langle \beta | \alpha \rangle = \int \beta^*(s) \alpha(s) ds = 0 \]

\[\langle \alpha | \beta \rangle = \int \alpha^*(s) \beta(s) ds = 0 \]

\subsection{Slater Determinants}

By including the spin functions into our MO's, we have a sufficient
incorporation of spin. To satisfy the Spin-Statistics Theorem/Pauli
Principle, the total \(N\)-electron wavefunction must be made to be
antisymmetric (change sign) under interchange of electrons. To do this,
we write the wavefunction as an \textbf{antisymmetric product} of
molecular spin orbitals, otherwise known as a Slater Determinant. This
construct has the advantage of automatically enforcing antisymmetry; if
two electron coordinates are exchanged, we get a minus sign.

\[\Phi = \frac{1}{\sqrt{N!}} \begin{vmatrix}
\psi_1^1 & \psi_2^1 & \psi_3^1 & ...  & \psi_N^1 \\
\psi_1^2 & \psi_2^2 & \psi_3^2 & ...  & \psi_N^2 \\
\psi_1^3 & \psi_2^3 & \psi_3^3  &...    & \psi_N^N \\
\vdots & \vdots & \vdots & \ddots &  \vdots \\
\psi_1^N & \psi_2^N & \psi_3^N  &...    & \psi_N^N \\
\end{vmatrix}\]

$\Phi$ is not a matrix. This is a linear combination of products of one
electron wavefunctions (MSO's) where the superscripts denote the electron
label and the subscripts denote the MSO label. Each term in the linear
combination is some permutation electron labels (or, equivalently, MO
labels) of the product \(\psi_1^1 \psi_2^2 \dots \psi_N^N\). Odd
permutations give a minus sign, and even permutations are positive. This
can be expressed in another way, using the \textbf{antisymmetrizer}
operator \(\hat{A}\):

\[\Phi = \sqrt{N!} \hat{A} (\psi_1^1 \psi_2^2 \dots \psi_N^N)\]

where \(\hat{A} = \frac{1}{N!} \sum\limits_{P \in S_N} (-1)^{\pi} \hat{P}\)
which is summing over all permutation operators which exist for an
ordered sequenced of \(N\) objects, odd permutations negative, even
permutations positive. Colloquially, the antisymmetrizer says "Hey, you
see that product of MO's? I'm going to sum all possible permutations of
them. If a permutation needs an odd number of transpositions to get back
to the original, its an odd permutation, and that term in the linear
combination gets a minus sign. If its an even permutation, its remains
positive." A simple example of a two electron wavefunction with two MO's
is

\[\Phi = \sqrt{2!} \hat{A} (\psi_1^1 \psi_2^2) = \sqrt{2!}( \psi_1^1 \psi_2^2 - \psi_1^2 \psi_2^1)\]

Note how we exchanged the \emph{electron labels} (superscripts) in the
second term. It may seem odd at first, but it is equivalent to exchange
electron labels vs MO labels. This is useful to note for later
derivations. In the context of the above example, we could exchange the
\emph{MO labels} (subscripts), and it gives the same result

\[\sqrt{2!}( \psi_1^1 \psi_2^2 - \psi_2^1 \psi_1^2)\]

where that second term is just flipped around compared to before (the
product of the two MO's commute, as you might expect).

\subsection{Orthonormality of
orbitals}\label{orthonormality-of-orbitals}

Suppose we collect our set of spin orbitals, which make up our Slater
Determinant wavefunction, into a row vector:

\[\boldsymbol{\psi} = [\psi_1, \psi_2, \dots, \psi_N] \]

and subject this row vector to a linear transformation, by multiplying
it by some NxN nonsingular matrix

\[\psi_k' = \sum\lambda \psi_\lambda A_{\lambda k} \]

\[\boldsymbol{\psi'} = \boldsymbol{\psi} \boldsymbol{A}\]

Note we can build a Slater determinant made up of spin orbitals in
\(\boldsymbol{\psi'}\), or a Slater determinant made up of spin orbitals
in \(\boldsymbol{\psi}\).

The corresponding expression for relating the two Slater determinants is

\[\Phi' = \Phi \mathrm{Det}(\boldsymbol{A}) \]

This means our wavefunction is only altered by a scalar quantity under
linear transformations. This does not change anything about the
wavefunction or quantities obtained from it, such as the energy (to see
this, imagine scaling the wavefunction in the time independent
Schrodinger equation by a constant).

Why am I talking about this? Well, this all means that we can always
multiply our orbitals by a matrix \(\boldsymbol{A}\). \textbf{This
implies we can always choose a transformation \(\boldsymbol{A}\) such that our spin
orbitals are orthonormal}

\[\int \psi^*_i \psi_j d\tau = \delta_{ij} \]

One such way to ensure any linearly independent set of spin orbitals are
orthonormal is using the Gram Schmidt orthogonalization procedure. We
will be doing something else, as we will discuss later. The important
takeaway is \textbf{we will always assume our spin orbitals are
orthonormal from here on}. If we didn't, our equations would not be easy
to deal with.

A final important note is that \textbf{since our spin orbitals are
orthonormal, the corresponding Slater determinant is normalized}:

\[ \int \Phi^* \Phi d\tau = 1 \]

\section{The Schrodinger Equation and the First Slater-Condon Rule}

\subsection{Applying the Schrodinger Equation}

We have thus far defined a \textbf{Hamiltonian} which describes all of
the relevent energetic interactions in our system. We have also defined
an approximate form for an \textbf{\(N\)-electron wavefunction, the
Slater determinant}, which is an antisymmetric linear combination of
\textbf{products of one-electron wavefunctions, which we called spin
orbitals.} The time independent Schrodinger equation says

\[ \hat{H} |\Psi \rangle = E |\Psi \rangle \]

multiplying each side on the left by $\langle \Psi|$ gives
\[\langle \Psi |  \hat{H} |\Psi \rangle = E\langle \Psi |\Psi \rangle \]

Here our wavefunction takes the form of a Slater determinant, and the Hamiltonian is a set of operators acting on three-dimensional space
of each electron. Making this substitution converts this abstract state vector equation into an equation involving integrals over all electronic coordinates.

\[\langle \Phi |  \hat{H} |\Phi \rangle = E\langle \Phi |\Phi \rangle \]

assuming an \textbf{orthonormal} set of spin orbitals, the Slater
determinant is normalized so we just have

\[\langle \Phi |  \hat{H} |\Phi \rangle = E \]

Let's not be confused by the abstraction here. This is an
\emph{integral} with a bunch of operators and functions sandwiched
inside. We might write it in an overwhelmingly explicit way by using our
definitions of each thing,

\[E = \int \sqrt{N!} \hat{A} (\psi_1^{1*} \psi_2^{2*} \dots \psi_N^{N*})  \left[ -\sum\limits_{i} \frac{1}{2} \nabla^2_{\boldsymbol{r}_i} - \sum\limits_{i} \sum\limits_{A} \frac{Z_A}{|\boldsymbol{R}_A - \boldsymbol{r}_i|} + \sum\limits_{i < j} \frac{1}{\boldsymbol{r}_{ij}} \right] \sqrt{N!} \hat{A} (\psi_1^{1} \psi_2^{2} \dots \psi_N^{N}) d\tau_1 \dots d\tau_N \]

Yikes. See why we don't do that? We didn't even expand our spin orbitals
into their explicit form
\(\psi_1^1(\boldsymbol{r}_1,s_1) = \phi_1(\boldsymbol{r}_1) \omega_1(s_1)\), nor did we expand out the effect of the antisymmetrizer.
Note we use \(\tau\) for the integration over both spin and spatial
coordinates. Let's clean the above up a little:

\[E = \int \sqrt{N!} \hat{A} (\psi_1^{1*} \psi_2^{2*} \dots \psi_N^{N*})  \left[ \sum\limits_{i} \hat{h}(i) + \sum\limits_{i < j} \hat{g}(i,j) \right] \sqrt{N!} \hat{A} (\psi_1^{1} \psi_2^{2} \dots \psi_N^{N}) d\tau_1 \dots d\tau_N \]

Noting the following facts:

\begin{itemize}
\item An integral \(\int A + B\) is really just \(\int A + \int B\)
\item An integral $\int f(r_1) g(r_2) dr_1 dr_2 =\int f(r_1) dr_1 \cdot \int g(r_2) dr_2 $
\item The antisymmetrizer \(\hat{A}\) produces a linear combination of products of spin orbitals
\item The operators which sum over different electrons can be expanded and separated into separate integrals according to the above rules
\end{itemize}

The above energy expression is really just an enormously complex linear
combination of products of simple integrals over spin orbitals and
operators. This will become apparent in our derivation of the energy
expression, where the above integral monstrosity simplifies tremendously
to the following, which is commonly denoted as the \emph{first
Slater-Condon Rule},

\[ E =  \sum\limits_{i} \langle\psi_i^i|h(i)|\psi_i^i \rangle + \sum\limits_{i<j}  \langle \psi_i^i\psi_j^j|g(i,j)|\psi_i^i\psi_j^j \rangle - \langle \psi_i^i\psi_j^j|g(i,j)|\psi_j^i\psi_i^j \rangle \]

This gives a general expression for the electronic energy, assuming a
Slater determinant wavefunction made up of orthonormal spin orbitals and
the above defined electronic Hamiltonian. 
\subsection{Derivation of First Slater-Condon Rule}

Back to bra-ket notation, we begin with
\[E = \langle \Phi  | \sum\limits_{i} \hat{h}(i) + \sum\limits_{i < j} \hat{g}(i,j) | \Phi \rangle\]

We can separate out just the one electron part and two electron parts of
our energy expression to make things a little easier.

\subsubsection{One-electron energy}
Using our definition of Slater determinant,
\[ \langle \Phi | \sum\limits_{i} \hat{h}_i | \Phi \rangle = N! \langle \hat{A} (\psi_1^1 \psi_2^2 \dots \psi_N^N) |  \sum\limits_{i} \hat{h}(i)  | \hat{A} (\psi_1^1 \psi_2^2 \dots \psi_N^N) \rangle \]

To simplify this, we note three facts: 

1. \(\hat{A}\) is a Hermitian
operator. Like all Hermitian operators, we can move it through bra's and
ket's $\langle f \textbar{} \hat{A}\textbar{} g \rangle =
\langle \hat{A} f \textbar{} g \rangle = \langle f \textbar{} \hat{A} g
\rangle $

\[ = N! \langle \psi_1^1 \psi_2^2 \dots \psi_N^N | \hat{A}  \sum\limits_{i} \hat{h}(i)  | \hat{A} (\psi_1^1 \psi_2^2 \dots \psi_N^N) \rangle \]

2. 

\begin{enumerate}
\def\labelenumi{\arabic{enumi}.}
\setcounter{enumi}{1}
\item
  \(\hat{A}\) commutes with \(\hat{H}\), and therefore \(\hat{h}\)
  \[ = N! \langle \psi_1^1 \psi_2^2 \dots \psi_N^N |  \sum\limits_{i} \hat{h}(i)  | \hat{A}  \hat{A} (\psi_1^1 \psi_2^2 \dots \psi_N^N) \rangle \]
\item
  \(\hat{A}\) is \emph{idempotent} ; \(\hat{A}^2 = \hat{A}\), so we are
  left with just one \(\hat{A}\).
\end{enumerate}

\[ = N! \langle \psi_1^1 \psi_2^2 \dots \psi_N^N |  \sum\limits_{i} \hat{h}(i)  | \hat{A} (\psi_1^1 \psi_2^2 \dots \psi_N^N) \rangle \]

    so far so good. Now lets flip-flop \(\hat{A}\) and the one electron
operator, (note we can only do this because these operators commute) and expand the sum over \(\hat{h}\)
explicitly and see if that leads anywhere.

\[ = N! \langle \psi_1^1 \psi_2^2 \dots \psi_N^N | \hat{A} |  \hat{h}(1)\psi_1^1 \psi_2^2 \dots \psi_N^N + \psi_1^1 \hat{h}(2) \psi_2^2 \dots \psi_N^N  + \dots + \psi_1^1  \psi_2^2 \dots \hat{h}(N) \psi_N^N \rangle \]

Alright, so now in the ket we just have a sum of spin orbital products,
where the one electron operator for the \(i\)th electron is being
applied to the spin orbital which holds that electron. It has no effect
on any of the other spin orbitals. The above equation looks ugly, but it
simplifies tremendously. See that first term in the ket? The 
"$\hat{h}(1)\psi_1^1 \psi_2^2 \dots \psi_N^N$" ? We are
going to investigate just that term alone, and deduce what analogously
will occur for all \emph{other} terms in the ket. So, we just have this:
\[ \langle \psi_1^1 \psi_2^2 \dots \psi_N^N | \hat{A} |  \hat{h}(1)\psi_1^1 \psi_2^2 \dots \psi_N^N \rangle \]

Now, recall the antisymmetrizer \(\hat{A}\) is going to take what's in
that ket and expand it as an antisymmetric sum of spin orbital products,
each a different permutation with the appropriate parity (sign). It
turns out, we can antisymmetrize over the spin orbital indices
(subscripts) \textbf{OR} the electron label indices (superscripts), and
we get the same thing. We discussed this earlier, and you will convince
yourself of this in a homework problem.

Let's write out a few terms that occur when applying the
antisymmetrizer. I'm going to be permuting the spin orbital indices,
because I like that it keeps the electrons ordered from left to right in
every product of spin orbitals.

\[\langle \psi_1^1 \psi_2^2 \dots \psi_N^N | \hat{h}(1) |\psi_1^1 \psi_2^2 \dots \psi_N^N  \rangle - \langle \psi_1^1 \psi_2^2 \dots \psi_N^N | \hat{h}(1) |\psi_2^1 \psi_1^2 \dots \psi_N^N  \rangle  + ... \]

Remember each of these terms are \textbf{integrals} over
\textbf{orthonormal spin orbitals}. Thus they are multiplicatively
separable for functions over different integration coordinates. The
first term above (ignoring spin function coordinates for now, they don't
end up doing anything here) looks like a product of integrals,

\[\int \psi_1^{*}(\boldsymbol{r}_1) \hat{h}(\boldsymbol{r}_1) \psi_1(\boldsymbol{r}_1) d\boldsymbol{r}_1 \int \psi_2^{*}(\boldsymbol{r}_2)\psi_2(\boldsymbol{r}_2) d\boldsymbol{r}_2 \cdots \int \psi_N^*(\boldsymbol{r}_N)\psi_N(\boldsymbol{r}_N) d\boldsymbol{r}_N \]

But, our spin orbitals are orthonormal, so most of these just equal 1.
We are left with just the first integral
\[ \langle \psi_1^1 | \hat{h}(1) | \psi_1^1 \rangle \] Hol' up tho.
There's a bunch of other terms in our antisymmetrizer expansion above.
Let's take a close look at that second term, in its integral form,

\[\int \psi_1^{*}(\boldsymbol{r}_1) \hat{h}(\boldsymbol{r}_1) \psi_2(\boldsymbol{r}_1) d\boldsymbol{r}_1 \int \psi_2^{*}(\boldsymbol{r}_2)\psi_1(\boldsymbol{r}_2) d\boldsymbol{r}_2 \cdots \int \psi_N^*(\boldsymbol{r}_N)\psi_N(\boldsymbol{r}_N) d\boldsymbol{r}_N \]

\textbf{The first two integrals are 0} by orthonormality because the
spin orbitals are different.

It's not hard to convince yourself that \emph{every} permutation of the
ket spin orbital product is going to become 0 due to spin-orbital
orthonormality. So, we are only left with one term from the
antisymmetrizer expansion.

\[ \langle \psi_1^1 \psi_2^2 \dots \psi_N^N | \hat{A} |  \hat{h}(1)\psi_1^1 \psi_2^2 \dots \psi_N^N \rangle = \langle \psi_1^1 | \hat{h}(1) | \psi_1^1 \rangle \]

This is going to be true for \emph{every} one electron operator
\(\hat{h}(i)\) in the summation in the ket from earlier:

\[ \langle \Phi | \sum\limits_{i} \hat{h}_i | \Phi \rangle  = N! \langle \psi_1^1 \psi_2^2 \dots \psi_N^N | \hat{A} |  \hat{h}(1)\psi_1^1 \psi_2^2 \dots \psi_N^N + \psi_1^1 \hat{h}(2) \psi_2^2 \dots \psi_N^N  + \dots + \psi_1^1  \psi_2^2 \dots \hat{h}(N) \psi_N^N \rangle \]

We can conclude then, after factoring out the \(\frac{1}{N!}\) factor
built into \(\hat{A}\) and canceling the \(N!\) factor
\[ \langle \Phi | \sum\limits_{i} \hat{h}(i) | \Phi \rangle  = \sum\limits_{i} \langle \psi_i | \hat{h}(i) | \psi_i \rangle \]

This corresponds to the 'one electron energy'.

    \subsubsection{Two-electron energy}\label{two-electron-energy}

Our Hamiltonian has another operator; the two-electron operator
\(\hat{g}\)

\[E = \langle \Phi  | \sum\limits_{i} \hat{h}(i) + \sum\limits_{i < j} \hat{g}(i,j) | \Phi \rangle\]

We need to find the energy contribution of this operator as well.
\[ \langle \Phi | \sum\limits_{i < j} \hat{g}(i,j) | \Phi \rangle = N! \langle \hat{A} (\psi_1^1 \psi_2^2 \dots \psi_N^N) |  \sum\limits_{i < j} \hat{g}(i,j) | \hat{A} (\psi_1^1 \psi_2^2 \dots \psi_N^N)  \rangle  \]

We use the same facts as before (\(\hat{A}\) is Hermitian, idempotent,
and commutes with \(\hat{g}\)) to simplify:
\[  = N! \langle \psi_1^1 \psi_2^2 \dots \psi_N^N |  \sum\limits_{i < j} \hat{g}(i,j) | \hat{A} (\psi_1^1 \psi_2^2 \dots \psi_N^N)  \rangle  \]

Swapping \(\hat{A}\) and \(\hat{g}\) and expanding the sum over
\(i < j\) in the ket we obtain

\[= N! \langle \psi_1^1 \psi_2^2 \psi_3^3 \dots \psi_N^N | \hat{A} |  \hat{g}(1,2) \psi_1^1 \psi_2^2 \psi_3^3 \dots \psi_N^N + \hat{g}(1,3) \psi_1^1 \psi_2^2 \psi_3^3 \dots \psi_N^N + \dots + \hat{g}(N-1,N) \psi_1^1 \psi_2^2 \psi_3^3 \dots \psi_N^N  \rangle \]

As before, let's look at just one term in the expansion, the one
containing \(\hat{g}(1,2)\):

\[\langle \psi_1^1 \psi_2^2 \psi_3^3 \dots \psi_N^N | \hat{A} |  \hat{g}(1,2) \psi_1^1 \psi_2^2 \psi_3^3 \dots \psi_N^N  \rangle \]

The antisymmetrizer operator \(\hat{A}\) expands the ket as a sum over
all permutations of the spin orbital product. Let's write down the first
few terms, permuting the spin orbital labels (subscripts):

\[\langle \psi_1^1 \psi_2^2 \psi_3^3 \dots \psi_N^N |\hat{g}(1,2)| \psi_1^1 \psi_2^2 \psi_3^3 \dots \psi_N^N  \rangle  -  \langle \psi_1^1 \psi_2^2 \psi_3^3 \dots \psi_N^N |\hat{g}(1,2)| \psi_2^1 \psi_1^2 \psi_3^3 \dots \psi_N^N  \rangle  + \langle \psi_1^1 \psi_2^2 \psi_3^3 \dots \psi_N^N |\hat{g}(1,2)| \psi_3^1 \psi_1^2 \psi_2^3 \dots \psi_N^N  \rangle + ...  \]

The first term can be rearranged into a product of integrals which only
depend on like-integration variables

\[\langle \psi_1^1 \psi_2^2 | \hat{g}(1,2) | \psi_1^1 \psi_2^2 \rangle \langle \psi_3^3 | \psi_3^3 \rangle ... \langle \psi_N^N | \psi_N^N \rangle = \langle \psi_1^1 \psi_2^2 | \hat{g}(1,2) | \psi_1^1 \psi_2^2 \rangle  \]

Notice how the superscripts, which denote electron coordinate labels,
are always kept together in the same integral (bra-ket).

The second term is also nonzero:

\[-\langle \psi_1^1 \psi_2^2 | \hat{g}(1,2) | \psi_2^1 \psi_1^2 \rangle \langle \psi_3^3 | \psi_3^3 \rangle ... \langle \psi_N^N | \psi_N^N \rangle = -\langle \psi_1^1 \psi_2^2 | \hat{g}(1,2) | \psi_2^1 \psi_1^2 \rangle \]

The third term, and all other terms arising from various permutations of
the ket, do not survive. Let's look at an example to see why. The third
term from above is

\[\langle \psi_1^1 \psi_2^2 \psi_3^3 \dots \psi_N^N |\hat{g}(1,2)| \psi_3^1 \psi_1^2 \psi_2^3 \dots \psi_N^N  \rangle \]

Now, the spin orbitals which involve coordinates of electrons 1 and 2
are stuck with the operator \(\hat{g}(1,2)\) in the same integral. Thus,
ignoring spin functions, we have the following

\[\int \psi_1(\boldsymbol{r}_1) \psi_2(\boldsymbol{r}_2) \hat{g}(1,2) \psi_3(\boldsymbol{r}_1) \psi_1(\boldsymbol{r}_2) d\boldsymbol{r}_1d\boldsymbol{r}_2  \langle \psi_3^3 | \psi_2^3 \rangle \langle \psi_4^4 | \psi_4^4 \rangle \dots \langle \psi_N^N | \psi_N^N \rangle\]

The second integral in this product
\(\langle \psi_3^3 | \psi_2^3 \rangle\) is zero by orthonormality of the
spin orbitals. So this whole term goes to zero. Every other term
(fourth, fifth, ... N!$^{th}$) resulting from expanding the antisymmetrizer
also will go to zero. Generally, for a given operator \(\hat{g}(i,j)\) ,
a 0-overlap integral (like \(\langle \psi_3^3 | \psi_2^3 \rangle\)
above) will \textbf{always} factor out and cancel terms whose two
electron integral part includes any spin orbitals other than \(\psi_i\)
and \(\psi_j\). To put it another way, the only two terms which survive
are those whose two electron integral part, containing the operator
\(\hat{g}(i,j)\), contains either \(\psi_i^i\psi_j^j\) or
\(\psi_j^i\psi_i^j\) in the ket.

The above reasoning is not only true for \(\hat{g}(1,2)\), but for all
\(\hat{g}(i,j)\), so in our original expansion,

\[\langle \Phi | \sum\limits_{i < j} \hat{g}(i,j) | \Phi \rangle  = N! \langle \psi_1^1 \psi_2^2  \dots \psi_N^N | \hat{A} |  \hat{g}(1,2) \psi_1^1 \psi_2^2  \dots \psi_N^N + \hat{g}(1,3) \psi_1^1 \psi_2^2  \dots \psi_N^N + \dots + \hat{g}(N-1,N) \psi_1^1 \psi_2^2  \dots \psi_N^N  \rangle \]

each \(i,j\) pair with \(i < j\) will pick up exactly two terms

\[ \langle \psi_i^i \psi_j^j | \hat{g}(i,j)| \psi_i^i \psi_j^j \rangle - \langle \psi_i^i \psi_j^j | \hat{g}(i,j)| \psi_j^i \psi_i^j \rangle \]

leading us to write the total energy contribution of the two electron
operator as a sum over \(i < j\),
\[\langle \Phi | \sum\limits_{i < j} \hat{g}(i,j) | \Phi \rangle = \sum\limits_{i < j}  \langle \psi_i^i \psi_j^j | \hat{g}(i,j)| \psi_i^i \psi_j^j \rangle - \langle \psi_i^i \psi_j^j | \hat{g}(i,j)| \psi_j^i \psi_i^j \rangle \]

We have shown that the expectation value of our electronic Hamiltonian
for a Slater determinant wavefunction composed of orthonormal spin
orbitals simplifies to the following relation

\[ E =  \sum\limits_{i} \langle\psi_i^i|h(i)|\psi_i^i \rangle + \sum\limits_{i<j}  \langle \psi_i^i\psi_j^j|g(i,j)|\psi_i^i\psi_j^j \rangle - \langle \psi_i^i\psi_j^j|g(i,j)|\psi_j^i\psi_i^j \rangle \]

\section{Summary and Commentary}\label{summary-and-commentary}

Hopefully you can see that we have done nothing weird (ad hoc, arbitrary
assertions) to get to this result. We were able to come up with our
electronic Hamiltonian by logically thinking about all the physical
interactions in our system. This Hamiltonian isn't exact; we are
neglecting all relativistic effects, as well as nuclear motion
(Born-Oppenheimer approximation). Nevertheless, this Hamiltonian is
\emph{really really good}. We further reasoned our wavefunction for a
many electron system would be best approximated by a product of one
electron wavefunctions, called spin orbitals. There's a few reasons for
this. First, assuming electrons occupy local spatial functions in a
molecule is not a bad approximation; we know they do, because we obtain local spatial functions (orbitals) 
when applying the Schrodinger equation to the hydrogen atom.
The only thing that should cause us to
hesitate is that electrons can in principle move around and go to
different orbitals, especially when influenced by each other's
repulsion. We are mostly ignoring this, except the exchange term in our
energy provides a little bit of what we need (\emph{electron
correlation}) for that. Correlated methods such as coupled cluster
attempt to patch out this issue. We finally note that we are not too
worried about coming up with one electron wavefunctions; we have a
decent idea of what they should look like from the hydrogen atom
solutions.

But, a product of one electron wavefunctions is not good enough in
itself. We need an \emph{antisymmetric product}, a Slater determinant,
to account for the fundamental truth that fermionic wavefunctions must
change sign when electrons are exchanged.

So, we have a Hamiltonian and a form for a wavefunction. We don't assume
anything about the details of the wavefunction, other than that it is
normalized, and each spin orbital is orthonormal (they can \emph{always}
be made to be orthonormal). This is sufficient to plug things into the
time-independent Schrodinger equation and get an expression for the
energy.

So, are we done? Have we just solved Quantum Chemistry? We have an
equation for the energy of any molecular system, after all. Well... no.
Sure, we can get \emph{some energy} by just choosing some random
functions for our orbitals. This probably won't work out too well. We
need a systematic way to get a good energy from this expression. This is
what Hartree-Fock theory does for us.

The Hartree Fock equations are obtained by minimizing the energy of the
first Slater-Condon rule under the constraint that the orbitals are
orthonormal. This makes sense, since we depended on orbital
orthonormality over and over again in order to derive the energy
expression. The only way we can alter the result of our energy is by
changing our spin orbitals. By minimizing the energy through orbital
variations, we are implicitly believing in the Variational Principle;
any trial wavefunction we write down which depends on a set of
parameters (the functional form of our spin orbitals) will be bounded
below by the true ground state energy. So, varying our wavefunction's
parameters such that the energy is lowest will give the best Slater
determinant wavefunction and best energy. The Hartree-Fock equations
provide a method for finding this best single Slater determinant
wavefunction and energy. We will discuss this more later.

\end{document}
