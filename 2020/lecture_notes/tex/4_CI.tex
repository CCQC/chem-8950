\documentclass{article}
\usepackage{amsmath,mathtools,amssymb}
\usepackage{graphicx}
\usepackage{blkarray}
\usepackage{gensymb}
\usepackage{verbatim}
\usepackage{mathrsfs}
\usepackage{bbm}
\usepackage{braket}
\usepackage{hyperref}
\usepackage{verbatim}
\usepackage{cancel}
\usepackage[margin=1.0in]{geometry}
\newcommand{\ol}{\overline}
\newcommand{\lp}{\left(}
\newcommand{\rp}{\right)}
\newcommand{\ve}{\varepsilon}
\newcommand{\h}{\circ}
\newcommand{\p}{\bullet}
\newcommand{\Ecorr}{E_{\mathrm{corr}}}
\newcommand{\Hc}{H_{\mathrm{c}}}
\newcommand{\dg}{\ensuremath{^\dagger} }
\def\*#1{\mathbf{#1}}
\DeclarePairedDelimiter\floor{\lfloor}{\rfloor}


\title{Lecture 4: Configuration Interaction}
\date{March 30, 2020}
\begin{document}
\maketitle
\noindent
Now that you have been properly bombarded with new notations and algebra, we are ready to do some actual quantum chemistry again. 
The rest of this course will be about post-Hartree-Fock (post-HF) methods (though, our current mathematical machinery will quickly run out of steam, 
    and we will have to digress once again and learn diagrams to move any further).  
What do post-HF methods do? They account for the electronic energy contribution due to electron correlation. 
What is electron correlation? Well, colloquially, it's how much the movement of an electron is influenced by other nearby interacting electrons. 
Most quantum chemists define the correlation energy as simply the difference between the exact energy and the HF energy. 
\[E_{\mathrm{corr}}  = E_{\mathrm{exact}} - E_{\mathrm{HF}} \]
By ``exact'' energy, we mean the energy eigenvalue corresponding to our electronic Hamiltonian and the 
    exact wavefunction $\hat{H} \Psi_{\mathrm{exact}} = E_{\mathrm{exact}}  \Psi_{\mathrm{exact}} $. 
If you're inclined to be snarky, HF exchange energy is technically accounting 
    for a tiny amount of correlation by considering electrons of the same spin swapping orbitals. 
In this way, the entire exchange energy portion of the HF energy is a correlation energy. 
Nevertheless, the above definition of correlation energy still stands.

The most conceptually simple scheme for finding the correlation energy is known as configuration interaction (CI) theory.
The full configuration interaction (FCI) wavefunction is a valid form for the exact wavefunction described above within a given basis.
Thus, the corresponding FCI energy is the exact energy for that basis, and in the complete basis set (CBS) limit it is the exact electronic energy.
By ``exact'', we mean exact w.r.t. our Hamiltonian and the time-independent Schr{\"o}dinger equation.
Barring any relativistic effects, the FCI/CBS energy is the \textit{true} value of the energy for that molecular system.

\section{The CI Wavefunction} 
Recall from the first set of second quantization notes that for some molecular system with a set of spin-orbitals,
the set of all Slater determinants which may be constructed from these spin orbitals form a basis for expanding our $N$-particle wavefunction $\ket{\Psi}$:
\[ \ket{\Psi} = \sum_{p_1 < \cdots < p_N} c_p \ket{\Phi_{p_1 \cdots p_N}} \]
To put it another way, the form of the exact $N$-particle wavefunction $\ket{\Psi}$ is just a linear combination of all possible Slater determinants given a set of spin-orbitals.

The above summation is just a linear combination of every unique Slater determinant for a set a spin-orbitals. 
An equivalent way of expressing a linear combination of every unique Slater determinant is to choose some arbitrary reference determinant, 
    and then consider all possible excitations from that reference determinant. 
%We point out that each of these Slater determinants can be expressed in terms of some arbitrary reference determinant. 
Suppose we choose the Hartree-Fock wavefunction  $\ket{\Phi}$ as our reference determinant.
Now we can write the wavefunction expansion above as 
\[ \ket{\Psi} = c_0 \ket{\Phi} + \sum_{ia} c_i^a \ket{\Phi_i^a} +  \sum_{ \substack{i < j \\ a < b}} c_{ij}^{ab} \ket{\Phi_{ij}^{ab}} + \sum_{ \substack{i < j < k \\ a < b < c }} c_{ijk}^{abc} \ket{\Phi_{ijk}^{abc}} + \cdots \]
Where each $c_i^a$ can be read as ``the coefficient associated with the Slater determinant which defers from the reference determinant 
    by one excitation of an electron from spin orbital $i$ to spin orbital $a$.''
An analogous description exists for each of the other coefficients.
The above can be simplified signficantly by pulling out the excitation operators and taking advantage of the Einstein summation convention: 
    we swap the upper and lower indice labels of the $c$'s to imply summation.
However, to do this we need to unrestict our sums (theres no way to build-in $i <j$, etc. in the Einstein summation notation). 
Summing over unrestricted indices for $n$ occupied indices and $n$ virtual indices introduces a redundancy of $(n!)^2$:
%\[ \ket{\Psi} = c_0 \ket{\Phi} + c_a^i \tilde{a}_{i}^{a} \ket{\Phi} + c_{ab}^{ij} \tilde{a}_{ij}^{ab} \ket{\Phi} + c_{abc}^{ijk} \tilde{a}_{ij}^{ab} \ket{\Phi} + \cdots \]
\[ \ket{\Psi} = c_0 \ket{\Phi} + \left(\frac{1}{1!}\right)^2 \sum_{ia} c_i^a \ket{\Phi_i^a} +  \left(\frac{1}{2!}\right)^2\sum_{ \substack{ij \\ ab}} c_{ij}^{ab} \ket{\Phi_{ij}^{ab}} + \left(\frac{1}{3!}\right)^2 \sum_{ \substack{ijk \\ abc}} c_{ijk}^{abc} \ket{\Phi_{ijk}^{abc}} + \cdots \]
\[ \ket{\Psi} = \left(c_0 + \left(\frac{1}{1!}\right)^2 c_a^i \tilde{a}_{i}^{a} +  \left(\frac{1}{2!}\right)^2c_{ab}^{ij} \tilde{a}_{ij}^{ab} + \left(\frac{1}{3!}\right)^2 c_{abc}^{ijk} \tilde{a}_{ijk}^{abc} + \cdots\right) \ket{\Phi}  \]

the above is now just a sum of excitation operators being applied to the reference wavefunction. 
We can cleanly define a set of CI operators $C_n$ and arrive at the final form for the CI wavefunction. 
\[ C_1 = \left(\frac{1}{1!}\right)^2 c_a^i \tilde{a}_{i}^{a} \quad \quad C_2 = \left(\frac{1}{2!}\right)^2c_{ab}^{ij} \tilde{a}_{ij}^{ab} \quad \quad C_3 = \left(\frac{1}{3!}\right)^2 c_{abc}^{ijk} \tilde{a}_{ijk}^{abc} \quad \quad \cdots  \]
\[ \ket{\Psi} = \lp c_0 + C_1 + C_2 + C_3 + \cdots + C_n \rp \ket{\Phi} \]

\subsection{Intermediate Normalization}
It is often chosen to work with $\ket{\Psi}$ in \textit{intermediate normalized form} so that the overlap with the reference determinant is $\braket{\Phi | \Psi} = 1$.  
This way the first coefficient $c_0$ is equal to 1, so that 
\[ \ket{\Psi} = \lp 1 + C_1 + C_2 + C_3 + \cdots + C_n \rp \ket{\Phi} \]
and the overall wavefunction is not normalized:
\[ \braket{\Psi|\Psi} = 1 + \sum_{ia} (c_a^i)^2 + \sum_{ \substack{ij \\ ab}} (c_{ab}^{ij})^2 + \cdots  \]
However one can easily retrieve the truely normalized $\ket{\Psi}$ by multiplying the intermediate normalized $\ket{\Psi}$ by a constant.


\section{The CI Schr{\"o}dinger Equation and Correlation Energy}
We have a form for the wavefunction (the \textit{intermediate normalized full CI wavefunction}), as well as our $\Phi$-normal ordered Hamiltonian, 
the \textit{correlation} Hamiltonian $\Hc$ from before:
\[\Hc =  \sum_{pq} f_{pq} N[a_p\dg a_q]  + \frac{1}{4} \sum_{pqrs} \braket{pq || rs} N[a_p\dg a_q\dg a_s a_r] = f_p^q \tilde{a}_q^p + \frac{1}{4} \bar{g}_{pq}^{rs} \tilde{a}_{rs}^{pq} \] 
where $f_{p}^{q} = h_{p}^{q} + \sum_i \braket{pi||qi} $.
We are ready to derive an energy expression for the correlation energy.

\[\Hc \ket{\Psi} = \Ecorr \ket{\Psi} \]
\[\Hc (1 + C_1 + C_2 + \cdots) \ket{\Phi} = \Ecorr \ket{\Psi} \]
\[\braket{\Phi | \Hc (1 + C_1 + C_2 + \cdots) | \Phi} = \Ecorr \braket{\Phi|\Psi} \]

Here we will use the fact that  $\braket{\Phi|\Hc|\Phi} = 0 $ and $\braket{\Phi|\Psi} = 1$ to arrive at

\[\braket{\Phi | \Hc (C_1 + C_2 + \cdots) | \Phi} = \Ecorr \]
\[ \braket{\Phi | \Hc C_1 | \Phi} + \braket{\Phi | \Hc C_2 | \Phi} + \braket{\Phi | \Hc C_3 | \Phi} + \cdots = \Ecorr \]

\[ \left(\frac{1}{1!}\right)^2 \braket{\Phi | \Hc c_a^i \tilde{a}_i^a  | \Phi}
+  \left(\frac{1}{2!}\right)^2 \braket{\Phi | \Hc c_{ab}^{ij} \tilde{a}_{ij}^{ab} | \Phi}
+  \left(\frac{1}{3!}\right)^2 \braket{\Phi | \Hc c_{abc}^{ijk} \tilde{a}_{ijk}^{abc} |\Phi} + \cdots = \Ecorr
\]

\[ \left(\frac{1}{1!}\right)^2 c_a^i \braket{\Phi | \Hc | \Phi_i^a}
+  \left(\frac{1}{2!}\right)^2 c_{ab}^{ij} \braket{\Phi | \Hc | \Phi_{ij}^{ab}}
+  \left(\frac{1}{3!}\right)^2 c_{abc}^{ijk} \braket{\Phi | \Hc |\Phi_{ijk}^{abc}} + \cdots = \Ecorr
\]

\[ c_a^i f_p^q \braket{\Phi| \tilde{a}_{q}^{p} \tilde{a}_{i}^{a} |\Phi} + \frac{1}{16} c_{ab}^{ij} \bar{g}_{pq}^{rs} \braket{\Phi| \tilde{a}_{rs}^{pq} \tilde{a}_{ij}^{ab} | \Phi} + 0 + \cdots = \Ecorr \]

\[ c_a^i f_p^q \braket{\Phi| \tilde{a}_{q \p}^{p \h} \tilde{a}_{i \h}^{a \p} |\Phi} + 
   \frac{1}{16} c_{ab}^{ij} \bar{g}_{pq}^{rs} \hat{P}_{(r/s)}^{(p/q)} \braket{\Phi| \tilde{a}_{r \p s \p \p}^{p \h q \h \h} \tilde{a}_{i \h j \h \h}^{a \p b \p \p} | \Phi} = \Ecorr \]

\[ c_a^i f_p^q \gamma^{p}_{i} \eta_{q}^{a} + \frac{1}{16} c_{ab}^{ij} \bar{g}_{pq}^{rs} \hat{P}_{(r/s)}^{(p/q)} \gamma_i^p \gamma_j^q \eta_r^a \eta_s^b \]
\[ c_a^i f_p^q \gamma^{p}_{i} \eta_{q}^{a} + \frac{1}{16} c_{ab}^{ij}  \lp \bar{g}_{ij}^{ab} - \bar{g}_{ji}^{ab} - \bar{g}_{ij}^{ba} + \bar{g}_{ji}^{ba}  \rp\]
\[ f_i^a c_a^i + \frac{1}{4} \bar{g}_{ij}^{ab} c_{ab}^{ij} = \Ecorr \]

where the first term is always 0 for a Hartree-Fock reference due to Brillouin's theorem. 
Note that all triple excitations and beyond go to zero since the Wick expansion has no complete contractions 
(or equivalently, from Slater's rule $\braket{\Phi | H | \Phi_{ijk}^{abc}} = 0  \implies \braket{\Phi | \Hc | \Phi_{ijk}^{abc}} = 0 $).

We arrive at the conclusion that the FCI electronic energy for a HF reference is just
\[ \frac{1}{4} \bar{g}_{ij}^{ab} c_{ab}^{ij} = \Ecorr \]

\textbf{This does not mean only double excitation coefficients contribute to the energy}.
All of the coefficients for different excitations are implicitly coupled together; their values influence each other. 
In full-CI, this is accounted for in the matrix diagonalization.
In truncated-CI, the task is reduced to a series of coupled equations which must be solved iteratively. 
These topics will be explored in the exercises.

\section{The FCI Algorithm} 
Expressed in the most simple way possible, the form of our full-CI wavefunction from before is really
just a linear combination of base states (excited determinants) $\Phi$.
\[ \ket{\Psi} = \sum_i c_i \ket{\Phi_i} \]
The problem of finding the optimal coefficients which minimize our energy to give the exact result can be found by diagonalizing the 
matrix representation of our Hamiltonian in this basis of Slater determinants (for a review of this, see Szabo and Ostlund, 1.3.2).
\[\mathbf{HC} = \mathbf{CE}\]
where  $\mathbf{H}$ is the CI matrix, $\mathbf{C}$ is a matrix of eigenvectors, and $\mathbf{E}$ is a diagonal matrix of energy eigenvalues.
Upon solving the above equation, one obtains a set of states and corresponding energies
\[ \Psi_\alpha = \sum_i c_i^\alpha \ket{\Phi_i} \quad , \quad E_\alpha \]
the state with the lowest energy is the \textit{ground state}.

$\mathbf{H}$ in our case is the full CI matrix, which contains every possible expectation value of our Hamiltonian 
for two Slater determinants, which are either the reference or one of many excitations from that reference. 
That is, it is a matrix of every expectation value you can create out of Slater determinants in the CI wavefunction expansion from before
\[ \ket{\Psi} = c_0 \ket{\Phi} + \left(\frac{1}{1!}\right)^2 \sum_{ia} c_i^a \ket{\Phi_i^a} +  \left(\frac{1}{2!}\right)^2\sum_{ \substack{ij \\ ab}} c_{ij}^{ab} \ket{\Phi_{ij}^{ab}} + \left(\frac{1}{3!}\right)^2 \sum_{ \substack{ijk \\ abc}} c_{ijk}^{abc} \ket{\Phi_{ijk}^{abc}} + \cdots \]

Typically the full CI matrix is depicted with a rather abusive notation consisting of ``excitation blocks''. 
We write one block of the full CI matrix as, say, $\braket{\Phi_{r}^{e}|\Hc|\Phi_{ij}^{ab}}$ or $\braket{S|\Hc|D}$, and 
by this we mean ``the block of all matrix elements consisting of singly excited determinants in the bra and doubly excited determinants in the ket.'' 
If you were to have, say, 1,000 possible singly excited determinants and 20,000 possible doubly excited determinants,
this block would be 1,000 by 20,000 elements. Each of these elements would be a \textit{number} representing the value of the expectation value.
The reason I call this an abuse of notation is that typically the expression $\braket{\Phi_{r}^{e}|\Hc|\Phi_{ij}^{ab}}$ means \textit{a particular} 
expectation value between \textit{two particular excited determinants} $\Phi_{r}^{e}$ and  $\Phi_{ij}^{ab}$.
In the context of the full-CI matrix, this intuitive interpretation is abandoned in favor of the excitation blocks interpretation.
Anyways, here it is, the full CI matrix:
\[
\begin{blockarray}{ccccccc}
 & 	\ket{\Phi} & \ket{\Phi_{i}^{a}} & \ket{\Phi_{ij}^{ab}}  &\ket{\Phi_{ijk}^{abc}} & \ket{\Phi_{ijkl}^{abcd}} & \cdots \\
\begin{block}{c(cccccc)}
\bra{\Phi}               & \braket{\Phi|\Hc|\Phi} &  \mathbf{0} &  \braket{\Phi|\Hc|\Phi_{ij}^{ab}} & \mathbf{0} & \mathbf{0} & \cdots  \\
\bra{\Phi_{m}^{e}}       &   &  \braket{\Phi_{m}^{e}|\Hc|\Phi_{i}^{a}} &  \braket{\Phi_{m}^{e}|\Hc|\Phi_{ij}^{ab}} & \braket{\Phi_{m}^{e}|\Hc|\Phi_{ijk}^{abc}} & \mathbf{0} & \cdots \\
\bra{\Phi_{mn}^{ef}}     &   &   &  \braket{\Phi_{mn}^{ef}|\Hc|\Phi_{ij}^{ab}} & \braket{\Phi_{mn}^{ef}|\Hc|\Phi_{ijk}^{abc}}  & \braket{\Phi_{mn}^{ef}|\Hc|\Phi_{ijkl}^{abcd}}  & \cdots \\
\bra{\Phi_{mno}^{efg}}   &   &   &   & \braket{\Phi_{mno}^{efg}|\Hc|\Phi_{ijk}^{abc}} & \braket{\Phi_{mno}^{efg}|\Hc|\Phi_{ijkl}^{abcd}}  & \cdots \\
\bra{\Phi_{mnop}^{efgh}} &   &   &   &   &  \braket{\Phi_{mnop}^{efgh}|\Hc|\Phi_{ijkl}^{abcd}}  & \cdots \\
\vdots                   &   &   &   &   &   & \ddots \\
\end{block}
\end{blockarray}
\]

With the exception of $\braket{\Phi|\Hc|\Phi}$, each entry in the matrix above \textbf{is not a single number} it is a (probably huge) 
    submatrix of all possible matrix elements $\braket{\Psi_{r \cdots}^{e \cdots} | \Hc | \Psi_{i \cdots}^{a \cdots}  }$.
Above, we have only filled in the upper triangle since it is a symmetric matrix. 
Futhermore, some entries are all 0's since excitations between determinants differing by more than two excitations must be 0 (Slater's rules).
% TODO check if this is bullshit:
Anyways, you just diagonalize that bad boy and the lowest eigenvalue is your full-CI 
    correlation energy for the ground state. 
You don't even have to use the energy expression we derived earlier (that's only used directly for truncated CI).
You \textit{could} extract out the resulting doubles coefficients $c_{ab}^{ij}$, 
contract them with $\bar{g}_{ij}^{ab}$ and multiply by $\frac{1}{4}$ to get the correlation energy,
but there's no need.

\section{Truncated CI}
Full-CI is really expensive, and kind of overkill; do you \textit{really} need to include
septuple excitations in your wavefunction? Is this a probable physical occurance?
That 7 electrons would want to simultaneously be away from their ever-so-comfy reference state orbitals
in favor of virtual ones? Almost certainly not.
So, a reasonable approach is to truncate the level of excitations to some maximum amount.
This would reduce the number of determinants we need to consider, and reduce the cost of the method.
Furthermore, we could develop a hierarchy of methods which systematically approach the exact FCI result, 
and get close enough to it without the hefty cost of FCI. (CISD, CISDT, CISDTQ, CISDTQP, CISDTQPH, ...)

There's one small price to pay by doing this: we lose the \textit{size consistency} property of the FCI wavefunction.
More on that later.

\subsection{CID}
Our energy expression for the CI energy for a HF reference determinant, 
    regardless of excitations we choose to include in the wavefunction, is
\[ \frac{1}{4} \bar{g}_{ij}^{ab} c_{ab}^{ij} = \Ecorr \]
So, the main task is to solve for the double excitation coefficients $c_{ab}^{ij}$.
However, the double excitation coefficient values are implicitly influenced by the values of all other  
    excitation coefficients (see exercises).
In the CID approximation, this influence is neglected, and we only consider doubly excited determinants.
\[\Psi = (1 + \frac{1}{4} c_{cd}^{kl} \tilde{a}_{kl}^{cd}) \Phi \]
\[\Hc \Psi = E \Psi\]

Projecting each side of the CI doubles Schrodinger equation by $\Phi$ and $\Phi_{ij}^{ab}$ gives a system of equations 
which can be used for solving the CID energy.
Projection by $\Phi$:
\[\braket{\Phi |\Hc (1 + \frac{1}{4} c_{cd}^{kl} \tilde{a}_{kl}^{cd})| \Phi} = \Ecorr \]
\[ \frac{1}{4} \braket{\Phi |\Hc|\Phi_{kl}^{cd}} c_{cd}^{kl} = \Ecorr \]
We have already solved this equation before,
\[ \frac{1}{4} \bar{g}_{ij}^{ab} c_{ab}^{ij} = \Ecorr \]
where we have left out the $f_i^a c_a^i$ since it is zero for a HF reference.

Projection by $\Phi_{ij}^{ab}$
\[\braket{\Phi_{ij}^{ab} |\Hc (1 + \frac{1}{4} c_{cd}^{kl} \tilde{a}_{kl}^{cd})| \Phi} = \Ecorr \braket{\Phi_{ij}^{ab}| (1 + \frac{1}{4} c_{cd}^{kl} \tilde{a}_{kl}^{cd}) | \Phi} \]
\[\braket{\Phi_{ij}^{ab} |\Hc (1 + \frac{1}{4} c_{cd}^{kl} \tilde{a}_{kl}^{cd})| \Phi} 
          =  \Ecorr \frac{1}{4} c_{cd}^{kl} \braket{\Phi| \hat{P}_{(a/b)}^{(i/j)} \tilde{a}_{a \p b \p \p}^{i \h j \h \h}  \tilde{a}_{k \h l \h \h}^{c \p d \p \p} | \Phi} \]
\[\braket{\Phi_{ij}^{ab} |\Hc (1 + \frac{1}{4} c_{cd}^{kl} \tilde{a}_{kl}^{cd})| \Phi} 
          =  \Ecorr c_{ab}^{ij} \]
\[\braket{\Phi_{ij}^{ab} |\Hc | \Phi } + \frac{1}{4} \braket{\Phi_{ij}^{ab}| \Hc | \Phi_{kl}^{cd}} c_{cd}^{kl}
          =  \Ecorr c_{ab}^{ij} \]


Simplifying the above equation should yield the following expression for the
CID coefficients 
\[  \Ecorr c_{ab}^{ij} = \bar{g}_{ab}^{ij} + \hat{P}_{(a/b)} f_a^c c_{cb}^{ij} - \hat{P}^{(i/j)} f_k^i c_{ab}^{kj}
 + \frac{1}{2} \bar{g}_{ab}^{cd} c_{cd}^{ij} + \frac{1}{2} \bar{g}_{kl}^{ij} c_{ab}^{kl} +
  \hat{P}^{(i/j)}_{(a/b)} \bar{g}_{ak}^{ic} c_{bc}^{jk}
\]
This is left as an exercise. 
Note the coefficients depend on themselves, so they must be solved iteratively,
much like how our MO coefficients/density matrix needed to be iteratively solved in HF.

\section{Size Consistency}
A quantum chemistry method is said to be \textit{size consistent} if the obtained energy of non-interacting systems
is the same result as summing the individual energies of the non-interacting systems separately.
So if you have two systems $A$ and $B$, 
\[E(A,B) = E(A) + E(B) \]
Practically, if you were to do a HF computation on two atoms separated by 100 Angstroms, and then two HF computations on each atom indiviudally, 
the sum of the latter two energies would be the same as the original energy of the combined system.
So, we say that HF is \textit{size consistent}.
Truncated CI, such as CID, is not size consistent.
The CID energy of combined noninteracting systems (say, two monomers) is not the same as the sum of the individual CID energies of the two monomers.
Why? The CID wavefunctions of each of the monomers contains double excitations within the monomer.
The CID wavefunction of the combined system contains double excitations of just the combined system,
and completely ignores the possibility that both monomers are doubly excited simultaneously; this would be a quadruple excitation.
Hence, CID, and all truncated CI, have a similar manifestation of this issue.
Anyone who knows what they're doing will always include an additive correction to their CI energy (CISD, MRCISD, etc)
    to help fix the size consistency issue.
The most common of these corrections is the Davidson correction,
    which is unfortunately denoted with a ``+Q'' notation (CISD+Q, MRCISD+Q), not to be confused with quadruple excitations.




\subsection{Epilogue}
There are many layers of complexity which we keep burying deeper and deeper into the ground, so it might be good to review the story so far. 

\textbf{For a given molecular system},
\begin{itemize}
\item Our spin-orbitals are constructed from a linear combination of atomic orbitals (LCAO), which are \textit{fixed, known} one-electron functions (typically Gaussians). 
The spin-orbitals are typically optimized with an LCAO-HF procedure, such as the Roothaan-Hall or Pople-Nesbet approaches. 
\item There is a finite set of Slater determinants (linear combinations of products of spin-orbitals) one can construct given a finite set of spin-orbitals. This set of Slater determinants form a \textit{basis} for representing our $N$-electron wavefunction.
\item The CI wavefunction is a linear combination of all possible Slater determinants which may be constructed from a set of spin-orbitals. These Slater determinants
can be expressed w.r.t. excitations from a reference determinant, which is almost always the HF wavefunction. 
\item The set of CI coefficients in the CI wavefunction which minimize the energy can be found by diagonalizing the matrix representation of the Hamiltonian in this basis 
of Slater determinants. 
\end{itemize}

\textbf{Why} do the excitations from the reference determinant (which is almost always the HF wavefunction, but need not be) in the CI expansion account for electron correlation?
What is it about these ``excitations'' that fully accounts for the instantaneous interactions between electrons which causes them to move in response to the presence of one another? 
Well, we proved that the exact $N$-electron wavefunction (and thus, the FCI wavefunction) does in fact account for all electron correlation, and Hartree-Fock obviously does not.
Since the primary difference between the HF wavefunction and FCI wavefunction is that the FCI wavefunction includes excitations from the HF wavefunction, it necessarily
follows that including these excitation contributions is \textit{equivalent} to accounting for electron correlation.
But this does not answer the question.
Qualitatively, the CI expansion is considering the possibility that the electrons rearrange themselves by constructing different Slater determinants, which differ from one another in that 
the electrons are occupying different spin-orbitals. 
By mixing in all those possible determinants, or ``excited states'', you allow the flexibility for electrons to avoid each other, or ``correlate'',
    whereas before in the Hartree-Fock picture they were clamped down into one Slater determinant configuration. 
In the limit of an infinite basis, we are allowing infinite flexibility for electrons to occupy different orbitals, which is equivalent to perfectly describing electron correlation.

%For a Hartree-Fock/STO-3G computation on \textit{just the Hydrogen atom}, the Hartree-Fock wavefunction \textbf{is the full-CI wavefunction} for that basis set, 
%    because there is only \textbf{one basis function}, thus there is only \textbf{one possible Slater determinant} you can even form.
%
%What about Hartree-Fock/CBS, the \textit{complete basis set limit} for Hartree-Fock, performed on \textit{just the Hydrogen atom}?
%Well, this
%
%
%the Hartree-Fock limit, which is the Hartree-Fock energy at the complete basis set limit, 
%
%A complete basis of spin-orbitals will in turn allow a complete basis of Slater determinants.

%%%%%%%%%%%%%%%%%%%%%%%%%%%%%%%%%%%%%%%%%%%%%%%%%%%%%%%%%%%%%%%%%%%%%%%%%%%%%%%%%%%%%%%%%%%%%%%%%%%%%%%%
%%%%%%%%%%%%%%%%%%%%%%%%%%%%%%%%%%%%%%%%%%%%%%%%%%%%%%%%%%%%%%%%%%%%%%%%%%%%%%%%%%%%%%%%%%%%%%%%%%%%%%%%
%%%%%%%%%%%%%%%%%%%%%%%%%%%%%%%%%%%%%%%%%%%%%%%%%%%%%%%%%%%%%%%%%%%%%%%%%%%%%%%%%%%%%%%%%%%%%%%%%%%%%%%%
%%%%%%%%%%%%%%%%%%%%%%%%%%%%%%%%%%%%%%%%%%%%%%%%%%%%%%%%%%%%%%%%%%%%%%%%%%%%%%%%%%%%%%%%%%%%%%%%%%%%%%%%
%%%%%%%%%%%%%%%%%%%%%%%%%%%%%%%%%%%%%%%%%%%%%%%%%%%%%%%%%%%%%%%%%%%%%%%%%%%%%%%%%%%%%%%%%%%%%%%%%%%%%%%%
%%%%%%%%%%%%%%%%%%%%%%%%%%%%%%%%%%%%%%%%%%%%%%%%%%%%%%%%%%%%%%%%%%%%%%%%%%%%%%%%%%%%%%%%%%%%%%%%%%%%%%%%
\end{document}
