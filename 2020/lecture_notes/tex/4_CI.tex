\documentclass{article}
\usepackage{amsmath,mathtools,amssymb}
\usepackage{graphicx}
\usepackage{gensymb}
\usepackage{verbatim}
\usepackage{mathrsfs}
\usepackage{bbm}
\usepackage{braket}
\usepackage{hyperref}
\usepackage{verbatim}
\usepackage{cancel}
\usepackage[margin=1.0in]{geometry}
\newcommand{\ol}{\overline}
\newcommand{\lp}{\left(}
\newcommand{\rp}{\right)}
\newcommand{\ve}{\varepsilon}
\newcommand{\Ecorr}{E_{\mathrm{corr}}}
\newcommand{\Hc}{H_{\mathrm{c}}}
\newcommand{\dg}{\ensuremath{^\dagger} }
\def\*#1{\mathbf{#1}}
\DeclarePairedDelimiter\floor{\lfloor}{\rfloor}

\title{Lecture 4: Configuration Interaction}
\date{March 6, 2020}
\begin{document}
\maketitle
\noindent
Now that you have been properly bombarded with new notations and algebra, we are ready to do some actual quantum chemistry again. 
The rest of this course will be about post-Hartree-Fock (post-HF) methods (though, our current mathematical machinery will quickly run out of steam, 
    and we will have to digress once again and learn diagrams to move any further).  
What do post-HF methods do? They account for the electronic energy contribution due to electron correlation. 
What is electron correlation? Well, colloquially, it's how much the movement of an electron is influenced by other nearby interacting electrons. 
Most quantum chemists define the correlation energy as simply the difference between the exact energy and the HF energy. 
\[E_{\mathrm{corr}}  = E_{\mathrm{exact}} - E_{\mathrm{HF}} \]
By ``exact'' energy, we mean the energy eigenvalue corresponding to our electronic Hamiltonian and the 
    exact wavefunction $\hat{H} \Psi_{\mathrm{exact}} = E_{\mathrm{exact}}  \Psi_{\mathrm{exact}} $. 
If you're inclined to be snarky, HF exchange energy is technically accounting 
    for a tiny amount of correlation by considering electrons of the same spin swapping orbitals. 
In this way, the entire exchange energy portion of the HF energy is a correlation energy. 
Nevertheless, the above definition of correlation energy still stands.

The most conceptually simple scheme for finding the correlation energy is known as configuration interaction (CI) theory.
The full configuration interaction (FCI) wavefunction is a valid form for the exact wavefunction described above within a given basis.
Thus, the corresponding FCI energy is the exact energy for that basis, and in the complete basis set limit it is the exact electronic energy.
By ``exact'', we mean exact w.r.t. our Hamiltonian and the time-independent Schr{\"o}dinger equation.
Barring any relativistic effects, the FCI/CBS energy is the \textit{true} value of the energy for that molecular system.


\section{The CI Wavefunction} 
Recall from the first set of second quantization notes that for some molecular system with a set of spin-orbitals,
the set of all Slater determinants which may be constructed from these spin orbitals form a basis for expanding our $N$-particle wavefunction $\ket{\Psi}$:
\[ \ket{\Psi} = \sum_{p_1 < \cdots < p_N} c_p \ket{\Phi_{p_1 \cdots p_N}} \]
To put it another way, the form of the exact $N$-particle wavefunction $\ket{\Psi}$ is just a linear combination of all possible Slater determinants given a set of spin-orbitals.

The above summation is just a linear combination of every unique Slater determinant for a set a spin-orbitals. 
An equivalent way of expressing a linear combination of every unique Slater determinant is to choose some arbitrary reference determinant, 
    and then consider all possible excitations from that reference determinant. 
%We point out that each of these Slater determinants can be expressed in terms of some arbitrary reference determinant. 
Suppose we choose the Hartree-Fock wavefunction  $\ket{\Phi}$ as our reference determinant.
Now we can write the wavefunction expansion above as 
\[ \ket{\Psi} = c_0 \ket{\Phi} + \sum_{ia} c_i^a \ket{\Phi_i^a} +  \sum_{ \substack{i < j \\ a < b}} c_{ij}^{ab} \ket{\Phi_{ij}^{ab}} + \sum_{ \substack{i < j < k \\ a < b < c }} c_{ijk}^{abc} \ket{\Phi_{ijk}^{abc}} + \cdots \]
Where each $c_i^a$ can be read as ``the coefficient associated with the Slater determinant which defers from the reference determinant 
    by one excitation of an electron from spin orbital $i$ to spin orbital $a$.''
An analogous description exists for each of the other coefficients.
The above can be simplified signficantly by pulling out the excitation operators and taking advantage of the Einstein summation convention: 
    we swap the upper and lower indice labels of the $c$'s to imply summation.
However, to do this we need to unrestict our sums (theres no way to build-in $i <j$, etc. in the Einstein summation notation). 
Summing over unrestricted indices for $N$ occupied indices and $N$ virtual indices introduces a redundancy of $(N!)^2$:
%\[ \ket{\Psi} = c_0 \ket{\Phi} + c_a^i \tilde{a}_{i}^{a} \ket{\Phi} + c_{ab}^{ij} \tilde{a}_{ij}^{ab} \ket{\Phi} + c_{abc}^{ijk} \tilde{a}_{ij}^{ab} \ket{\Phi} + \cdots \]
\[ \ket{\Psi} = c_0 \ket{\Phi} + \left(\frac{1}{1!}\right)^2 \sum_{ia} c_i^a \ket{\Phi_i^a} +  \left(\frac{1}{2!}\right)^2\sum_{ \substack{ij \\ ab}} c_{ij}^{ab} \ket{\Phi_{ij}^{ab}} + \left(\frac{1}{3!}\right)^2 \sum_{ \substack{ijk \\ abc}} c_{ijk}^{abc} \ket{\Phi_{ijk}^{abc}} + \cdots \]
\[ \ket{\Psi} = \left(c_0 + \left(\frac{1}{1!}\right)^2 c_a^i \tilde{a}_{i}^{a} +  \left(\frac{1}{2!}\right)^2c_{ab}^{ij} \tilde{a}_{ij}^{ab} + \left(\frac{1}{3!}\right)^2 c_{abc}^{ijk} \tilde{a}_{ijk}^{abc} + \cdots\right) \ket{\Phi}  \]

the above is now just a sum of excitation operators being applied to the reference wavefunction. 
We can cleanly define a set of CI operators $C_n$ and arrive at the final form for the CI wavefunction. 
\[ C_1 = \left(\frac{1}{1!}\right)^2 c_a^i \tilde{a}_{i}^{a} \quad \quad C_2 = \left(\frac{1}{2!}\right)^2c_{ab}^{ij} \tilde{a}_{ij}^{ab} \quad \quad C_3 = \left(\frac{1}{3!}\right)^2 c_{abc}^{ijk} \tilde{a}_{ijk}^{abc} \quad \quad \cdots  \]
\[ \ket{\Psi} = \lp c_0 + C_1 + C_2 + C_3 + \cdots + C_n \rp \ket{\Phi} \]

\subsection{Intermediate Normalization}
It is often chosen to work with $\ket{\Psi}$ in \textit{intermediate normalized form} so that the overlap with the reference determinant is $\braket{\Phi | \Psi} = 1$.  
This way the first coefficient $c_0$ is equal to 1, so that 
\[ \ket{\Psi} = \lp 1 + C_1 + C_2 + C_3 + \cdots + C_n \rp \ket{\Phi} \]
and the overall wavefunction is not normalized:
\[ \braket{\Psi|\Psi} = 1 + \sum_{ia} (c_a^i)^2 + \sum_{ \substack{ij \\ ab}} (c_{ab}^{ij})^2 + \cdots  \]
However one can easily retrieve the truely normalized $\ket{\Psi}$ by multiplying the intermediate normalized $\ket{\Psi}$ by a constant.


\section{The CI Schr{\"o}dinger Equation and Correlation Energy}
We have a form for the wavefunction (the \textit{intermediate normalized full CI wavefunction}), as well as our $\Phi$-normal ordered Hamiltonian, 
the \textit{correlation} Hamiltonian $\Hc$ from before:
\[\Hc =  \sum_{pq} f_{pq} N[a_p\dg a_q]  + \frac{1}{4} \sum_{pqrs} \braket{pq || rs} N[a_p\dg a_q\dg a_s a_r] = f_p^q \tilde{a}_q^p + \frac{1}{4} \bar{g}_{pq}^{rs} \tilde{a}_{rs}^{pq} \] 
where $f_{p}^{q} = h_{p}^{q} + \sum_i \braket{pi||qi} $,
we are ready to derive an energy expression for the correlation energy.

\[\Hc \ket{\Psi} = \Ecorr \ket{\Psi} \]
\[\Hc (1 + C_1 + C_2 \cdots) \ket{\Phi} = \Ecorr \ket{\Psi} \]
\[\braket{\Phi | \Hc (1 + C_1 + C_2 + \cdots) | \Phi} = \Ecorr \braket{\Phi|\Psi} \]

Here we will use the fact that  $\braket{\Phi|\Hc|\Phi} = 0 $ and $\braket{\Phi|\Psi} = 1$ to arrive at

\[\braket{\Phi | \Hc (C_1 + C_2 + \cdots) | \Phi} = \Ecorr \]
\[ \braket{\Phi | \Hc C_1 | \Phi} + \braket{\Phi | \Hc C_2 | \Phi} + \braket{\Phi | \Hc C_3 | \Phi} + \cdots = \Ecorr \]

\[ \left(\frac{1}{1!}\right)^2 \braket{\Phi | \Hc c_a^i \tilde{a}_i^a  | \Phi}
+  \left(\frac{1}{2!}\right)^2 \braket{\Phi | \Hc c_{ab}^{ij} \tilde{a}_{ij}^{ab} | \Phi}
+  \left(\frac{1}{3!}\right)^2 \braket{\Phi | \Hc c_{abc}^{ijk} \tilde{a}_{ijk}^{abc} |\Phi} + \cdots = \Ecorr
\]

\[ \left(\frac{1}{1!}\right)^2 c_a^i \braket{\Phi | \Hc | \Phi_i^a}
+  \left(\frac{1}{2!}\right)^2 c_{ab}^{ij} \braket{\Phi | \Hc | \Phi_{ij}^{ab}}
+  \left(\frac{1}{3!}\right)^2 c_{abc}^{ijk} \braket{\Phi | \Hc |\Phi_{ijk}^{abc}} + \cdots = \Ecorr
\]

\[ f_i^a c_a^i + \frac{1}{4} \bar{g}_{ij}^{ab} c_{ab}^{ij} = \Ecorr \]

where the first term is always 0 for a Hartree-Fock reference due to Brillouin's theorem, and all triple excitations and beyond go to zero 
since the Wick expansion has no complete contractions (or equivalently, from Slater's rule $\braket{\Phi | H | \Phi_{ijk}^{abc}} = 0  \implies \braket{\Phi | \Hc | \Phi_{ijk}^{abc}} = 0 $).

We arrive at the conclusion that the FCI electronic energy is just
\[ \frac{1}{4} \bar{g}_{ij}^{ab} c_{ab}^{ij} = \Ecorr \]


% Brillouins: ground electronic states from one-particle methods arlready mply configuration interaction of the ground state configuration with singly excited ones.
% furhter inclusion redundant

% Contract first term explicitly, third and higher terms are 0 because of no complete contractions
% Equivalently,  0 because of 4th Slater rule, which is the same as saying the wick expansion has no complete contractions 


%\left(\frac{1}{1!}\right)^2 c_a^i \tilde{a}_{i}^{a} 
%\left(\frac{1}{2!}\right)^2c_{ab}^{ij} \tilde{a}_{ij}^{ab}
%\left(\frac{1}{3!}\right)^2 c_{abc}^{ijk} \tilde{a}_{ijk}^{abc}
%
%\[ \braket{\Phi | \Hc C_1 | \Phi} + \braket{\Phi | \Hc C_2 | \Phi} + \braket{\Phi | \Hc C_3 | \Phi} + \cdots \]
%
%
%\[\braket{\Phi | \Hc (C_1 + C_2 + \cdots) | \Phi} = \Ecorr \]
%


%intermediate normalization

CI is conceptually very simple: Diagonalize the matrix representation of the $N$-electron Hamiltonian in a basis of all possible Slater determinants constructable from your basis set.

There are a lot of linear combinations and basis expansions going on here. So let's set that straight right now:

\textbf{For a given molecular system},
\begin{itemize}
\item Our spin-orbitals are constructed from a linear combination of atomic orbitals (LCAO), which are \textit{fixed, known} one-electron functions (typically Gaussians). 
The spin-orbitals are typically optimized with an LCAO-HF procedure, such as the Roothaan-Hall or Pople-Nesbet approaches. 
\item There is a finite set of Slater determinants one can construct given a finite set of spin-orbitals. This set of Slater determinants form a \textit{basis} for representing our $N$-electron wavefunction.
\end{itemize}

\textbf{Why} do the excitations from the reference determinant (which is almost always the HF wavefunction, but need not be) in the CI expansion account for electron correlation?
What is it about these ``excitations'' that fully accounts for the instantaneous interactions between electrons which causes them to move in response to the presence of one another? 
Well, we proved that the exact $N$-electron wavefunction (and thus, the FCI wavefunction) does in fact account for all electron correlation, and Hartree-Fock obviously does not.
Since the primary difference between the HF wavefunction and FCI wavefunction is that the FCI wavefunction includes excitations from the HF wavefunction, it necessarily
follows that including these excitation contributions is \textit{equivalent} to accounting for electron correlation.
But this does not answer the question.
Qualitatively, the CI expansion is considering the possibility that the electrons rearrange themselves by constructing different Slater determinants, which differ from one another in that 
the electrons are occupying different spin-orbitals. 
By mixing in all those possible determinants, or ``excited states'', you allow the flexibility for electrons to avoid each other, or ``correlate'',
    whereas before in the Hartree-Fock picture they were clamped down into one Slater determinant configuration. 
In the limit of an infinite basis, we are allowing infinite flexibility for electrons to occupy different orbitals, which is equivalent to perfectly describing electron correlation.

If you're not buying the qualitative picture, consider the following philosophical-like argument:
\begin{itemize}
\item 1: The Hartree-Fock wavefunction does not account for electron correlation
\item 2: The exact $N$-electron wavefunction does account for electron correlation 
\item 2: The exact $N$-electron wavefunction can be represented as a linear combination of Slater determinant wavefunctions (this is proven in 1st set of second quantization notes)
\item 3: Each Slater determinant in the above linear combination can be expressed with respect to some arbitrary reference determinant using creation and annihilation operators
\item 4: We can choose the reference determinant in our expansion to be the Hartree-Fock wavefunction 
\item 5: Expanding all excitations of our Hartree-Fock wavefunction constructs the exact $N$-electron wavefunction 
\item 6: The exact $N$-electron wavefunction  Expanding all excitations of our Hartree-Fock wavefunction constructs the exact $N$-electron wavefunction 


\item 1: The Hartree-Fock wavefunction does not account for electron correlation
\item 2: The exact $N$-electron wavefunction fully accounts for electron correlation
\item 3: The exact $N$-electron wavefunction can be expressed as a linear combination of excitations from the Hartree-Fock reference determinant 
% includes all excitations from some reference determinant 
\item C: Excitations from the reference determinant account for electron corrleation


\item 1: The exact $N$-electron wavefunction can be represented as a linear combination of Slater determinant wavefunctions (this is proven in 1st set of second quantization notes)
\[ \ket{\Psi} = \sum_{p_1 < \cdots < p_N} c_p \ket{\Phi_{p_1 \cdots p_N}} \]
\item 2: Each Slater determinant in the above linear combination can be expressed with respect to some arbitrary reference determinant using creation and annihilation operators
\item 3: We can freely choose the reference determinant in our expansion to be the Hartree-Fock wavefunction 
\item 4: The Hartree-Fock wavefunction does not fully account for electron correlation

Conclusion: 

The exact $N$-electron wavefunction in a given basis is, well, \textit{exact}, it must perfectly describe all electron correlation effects.
Since this exact wavefunction can be expressed in terms of excitations from the Hartree-Fock reference determinant, it follows that those excitation terms are accounting for electron correlation.

\end{itemize}

  we obtain electron correlation 


\section{Interesting Conceptual Tidbits}
For a Hartree-Fock/STO-3G computation on \textit{just the Hydrogen atom}, the Hartree-Fock wavefunction \textbf{is the full-CI wavefunction} for that basis set, 
    because there is only \textbf{one basis function}, thus there is only \textbf{one possible Slater determinant} you can even form.

What about Hartree-Fock/CBS, the \textit{complete basis set limit} for Hartree-Fock, performed on \textit{just the Hydrogen atom}?
Well, this


the Hartree-Fock limit, which is the Hartree-Fock energy at the complete basis set limit, 

A complete basis of spin-orbitals will in turn allow a complete basis of Slater determinants.

%%%%%%%%%%%%%%%%%%%%%%%%%%%%%%%%%%%%%%%%%%%%%%%%%%%%%%%%%%%%%%%%%%%%%%%%%%%%%%%%%%%%%%%%%%%%%%%%%%%%%%%%
%%%%%%%%%%%%%%%%%%%%%%%%%%%%%%%%%%%%%%%%%%%%%%%%%%%%%%%%%%%%%%%%%%%%%%%%%%%%%%%%%%%%%%%%%%%%%%%%%%%%%%%%
%%%%%%%%%%%%%%%%%%%%%%%%%%%%%%%%%%%%%%%%%%%%%%%%%%%%%%%%%%%%%%%%%%%%%%%%%%%%%%%%%%%%%%%%%%%%%%%%%%%%%%%%
%%%%%%%%%%%%%%%%%%%%%%%%%%%%%%%%%%%%%%%%%%%%%%%%%%%%%%%%%%%%%%%%%%%%%%%%%%%%%%%%%%%%%%%%%%%%%%%%%%%%%%%%
%%%%%%%%%%%%%%%%%%%%%%%%%%%%%%%%%%%%%%%%%%%%%%%%%%%%%%%%%%%%%%%%%%%%%%%%%%%%%%%%%%%%%%%%%%%%%%%%%%%%%%%%
%%%%%%%%%%%%%%%%%%%%%%%%%%%%%%%%%%%%%%%%%%%%%%%%%%%%%%%%%%%%%%%%%%%%%%%%%%%%%%%%%%%%%%%%%%%%%%%%%%%%%%%%
\end{document}
