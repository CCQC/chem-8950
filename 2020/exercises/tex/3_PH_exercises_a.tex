\documentclass{article}
\usepackage{amsmath,mathtools,amssymb}
\usepackage{graphicx}
\usepackage{gensymb}
\usepackage{verbatim}
\usepackage{cancel}
\usepackage{mathrsfs}
\usepackage{bbm}
\usepackage{braket}
\usepackage{simplewick}
\usepackage{xcolor}
\definecolor{orange}{RGB}{255,127,0}
\usepackage{verbatim}
\usepackage[margin=1.0in]{geometry}
\newcommand{\ol}{\overline}
\newcommand{\fctr}{\contraction}
\newcommand{\ve}{\varepsilon}
\newcommand{\ap}{\ensuremath{a_p} }
\newcommand{\dg}{\ensuremath{^\dagger} }
\newcommand{\cd}{\ensuremath{\cdots} }
\newcommand{\apd}{\ensuremath{a_p^\dagger} }
\def\*#1{\mathbf{#1}}
\DeclarePairedDelimiter\floor{\lfloor}{\rfloor}

\title{Homework for Lecture 3.5 Particle-Hole Formalism}
\date{}
\begin{document}
\maketitle
\noindent
\begin{enumerate}
\item Explain what we mean by ``particle-hole isomorphism"
\item Prove the anticommutation relations for $b_p$ and $b_p\dg$
\item Prove to yourself that the rules listed in Section 5 of Lecture 3.5 are true
\item Prove Slater's first rule using the Particle-Hole formalism
\[\braket{\Phi|H|\Phi} = \sum_i h_{ii} + \frac{1}{2} \sum_{ij} \braket{ij || ij} \]
\item Prove Slater's second rule using the Particle-Hole formalism 
\[\braket{\Phi|H|\Phi_i^a} = h_{ia} + \sum_{j} \braket{ij || aj} \]
\end{enumerate}
\end{document}

