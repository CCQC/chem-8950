\documentclass{article}
\usepackage{amsmath,mathtools,amssymb}
\usepackage{graphicx}
\usepackage{gensymb}
\usepackage{verbatim}
\usepackage{cancel}
\usepackage{mathrsfs}
\usepackage{bbm}
\usepackage{braket}
\usepackage{xcolor}
\definecolor{orange}{RGB}{255,127,0}
\usepackage{verbatim}
\usepackage[margin=1.0in]{geometry}
\newcommand{\ol}{\overline}
\newcommand{\ve}{\varepsilon}
\newcommand{\ap}{\ensuremath{a_p} }
\newcommand{\dg}{\ensuremath{^\dagger} }
\newcommand{\cd}{\ensuremath{\cdots} }
\newcommand{\apd}{\ensuremath{a_p^\dagger} }
\def\*#1{\mathbf{#1}}
\DeclarePairedDelimiter\floor{\lfloor}{\rfloor}

\title{Homework for Lecture 3.2 }
\date{}
\begin{document}
\maketitle
\noindent
\begin{enumerate}
\item Evaluate \[\braket{\Phi_{pq} | \Phi_{rst}} \]
\item Show that the first Slater's rule using the second quantization formalism: 
\[\braket{\Phi | \hat{H} | \Phi} = h_{ii} + \frac{1}{2} \sum_{ij} \braket{ij || ij} \]
holds for $\Phi_{p_1}$ and $\Phi_{p_1 p_2}$
\item (Optional, if you need more practice) Show that the second Slater's rule using the second quantization formalism: 
\[\braket{\Phi | \hat{H} | \Phi_i^a } = h_{ia} + \frac{1}{2} \sum_{j} \braket{ij || aj} \]
where $\ket{\Phi_i^a } = a_a\dg a_i \ket{\Phi}$ holds for $\Phi_{p_1}$ and $\Phi_{p_1 p_2}$
\item Work through the derivation of the one-electron and two-electron operators in the second quantization formalism yourself and make sure you understand what is going on 
\item Show that normal vs antisymmetrized forms of the two-electron operator are equivalent. Bonus: show that the antisymmetrized form for a general k-body operator is equivalent to its normal form 
\item Show that $\hat{h}(\*{r_i}) \phi_p (\*{r_i})  = \sum_q \braket{q | \hat{h} | p } \phi_q (\*{r_i}) $
\item (Optional) Show that $\hat{g}(\*{r_i}, \*{r_j}) \phi_{p_i} (\*{r_i}) \phi_{p_j} (\*{r_j})  \sum_{pq} \braket{pq | \hat{h} | p_i p_j  }  \phi_p (\*{r_i})  \phi_q (\*{r_i})$
\item (Optional) Show that 
\[a_r a_s \prod a\dg \ket{0} = \sum\limits_{i < j} ^N  (-1)^{i + j }   \delta_{rp_i} \delta_{sp_j}  \prod{}^{p_ip_j} a\dg \ket{0} + 
\sum\limits_{i > j} ^N  (-1)^{i + j -1} \delta_{rp_i} \delta_{sp_j} \prod{}^{p_ip_j}a\dg \ket{0} \]
by applying Equation 5 twice 
\end{enumerate}
\end{document}

