\documentclass{article}
\usepackage{amsmath,mathtools,amssymb}
\usepackage{graphicx}
\usepackage{gensymb}
\usepackage{verbatim}
\usepackage{cancel}
\usepackage{mathrsfs}
\usepackage{bbm}
\usepackage{braket}
\usepackage{simplewick}
\usepackage{xcolor}
\definecolor{orange}{RGB}{255,127,0}
\usepackage{verbatim}
\usepackage[margin=1.0in]{geometry}
\newcommand{\ol}{\overline}
\newcommand{\Rz}{\mathcal{R}_{0}}
\newcommand{\ve}{\varepsilon}
\newcommand{\ap}{\ensuremath{a_p} }
\newcommand{\dg}{\ensuremath{^\dagger} }
\newcommand{\cd}{\ensuremath{\cdots} }
\newcommand{\Hc}{H_{\mathrm{c}}}
\newcommand{\mr}{\mathrm}
\newcommand{\apd}{\ensuremath{a_p^\dagger} }
\def\*#1{\mathbf{#1}}
\DeclarePairedDelimiter\floor{\lfloor}{\rfloor}

\title{Homework for Lecture 6: Coupled Cluster}
\date{}
\begin{document}
\maketitle
\noindent
\begin{enumerate}
\item Write out the Hausdorff expansion for the CCSD case. You should get 15 terms 
containing at least one of $\Hc$, $T_1$, and $T_2$. 
\item Derive an expression for the CCSD correlation energy 
\[ \braket{\Phi| (\Hc \exp(T))| \Phi}_\mathrm{C}  = E_{\mathrm{CCSD}} \]
Hint: the only components of the Hausdorff expansion in question 1 which contribute
are $\Hc T_1$,  $\Hc T_2$,  $ \frac{1}{2}\Hc T_1^2$.  

\item Pick 3 terms from the Hausdorff expansion in question 1 which did not contribute to 
the CCSD energy in question 2, and explain why in terms of excitation levels and/or because the term is disconnected.

\item Explain why the following terms vanish, either due excitation levels
and/or because the term is disconnected.
\[ \braket{\Phi_i^a | V_\mr{c} T_2^2 | \Phi} \]
\[ \braket{\Phi_{ijk}^{abc} | V_\mr{c} T_2^2 | \Phi} \]
\[ \braket{\Phi_{ijk}^{abc} | V_\mr{c} T_1 | \Phi} \]

\end{enumerate}

\end{document}

