\documentclass{article}
\usepackage{amsmath,mathtools,amssymb}
\usepackage{graphicx}
\usepackage{gensymb}
\usepackage{verbatim}
\usepackage{cancel}
\usepackage{mathrsfs}
\usepackage{bbm}
\usepackage{braket}
\usepackage{simplewick}
\usepackage{xcolor}
\definecolor{orange}{RGB}{255,127,0}
\usepackage{verbatim}
\usepackage[margin=1.0in]{geometry}
\newcommand{\ol}{\overline}
\newcommand{\hole}{\circ}
\newcommand{\fctr}{\contraction}
\newcommand{\ve}{\varepsilon}
\newcommand{\tl}{\tilde}
\newcommand{\ap}{\ensuremath{a_p} }
\newcommand{\dg}{\ensuremath{^\dagger} }
\newcommand{\cd}{\ensuremath{\cdots} }
\newcommand{\apd}{\ensuremath{a_p^\dagger} }
\def\*#1{\mathbf{#1}}
\DeclarePairedDelimiter\floor{\lfloor}{\rfloor}

\title{Homework for Lecture 3.7 KM notation}
\date{}
\begin{document}
\maketitle
\noindent
\begin{enumerate}
\item
  Show algebraically that $\hat{P}_{(p/q/r)}=\hat{P}_{(p/qr)}\hat{P}_{(q/r)}=\hat{P}_{(pq/r)}\hat{P}_{(p/q)}$ and explain why these identities follow from the definition of the index antisymmetrizers.
\item 
  Prove the following identities.
\begin{align*}
  \tl{a}^{p_1\cd p_m}_{q_1\cd q_m}
=
N[a^{p_1}_{q_1}\cd a^{p_m}_{q_m}]
&&
N[a^{p_1\cd p_m}_{q_1\cd q_m}a^{r_1\cd r_n}_{s_1\cd s_n}]
=
  N[a^{p_1}_{q_1}\cd a^{p_m}_{q_m}a^{r_1}_{s_1}\cd a^{r_n}_{s_n}]
=
  \tl{a}^{p_1\cd p_mr_1\cd r_n}_{q_1\cd q_ms_1\cd s_n}
\end{align*}
Furthermore, explain why these identities do not apply to contracted operators. 
  \item Explain why we cannot write: 
\[\hat{P}^{(p/q)}_{(r/s)} \tl{a}^p_r\tl{a}^q_s \]
\item Put the Hamiltonian in $\Phi$-Normal ordering using KM notation 
\item Prove Slater's second rule using KM notation
\[\braket{\Phi|H|\Phi_i^a} = h_{ia} + \sum_{j} \braket{ij || aj} \]
\item Prove Slater's third rule using KM notation
\[\braket{\Phi|H|\Phi_{ij}^{ab}} = \braket{ij || ab} \]
\item Show that the Wick expansion of a triple excitation operator is: 
\[
  a^{pqr}_{stu}
=
  \tl{a}^{pqr}_{stu}
+
  \hat{P}^{(p/qr)}_{(s/tu)}
  \tl{a}^{p^\hole qr}_{s^\hole tu}
+
  \hat{P}^{(p/q/r)}_{(st/u)}
  \tl{a}^{p^\hole q^{\hole\hole} r}_{s^\hole t^{\hole\hole} u}
+
  \hat{P}^{(p/q/r)}
  \tl{a}^{p^\hole q^{\hole\hole} r^{\hole\hole\hole}}
        _{s^\hole t^{\hole\hole} u^{\hole\hole\hole}}
\]
\item Practice evaluating the following matrix elements using KM notation:
\[ \braket{\Phi_j^b |\Phi_{i}^{a}} \]
\[ \braket{\Phi_j^b |H_c| \Phi_{i}^{a}} \]
\[ \braket{\Phi_{kl}^{cd}|\Phi_{ij}^{ab}} \]
\[ \braket{\Phi_{kl}^{cd}|H_c|\Phi_{ij}^{ab}} \]
The answers can be found in the last section of Andreas's ``3q-1h-kutzelnigg-mukherjee.pdf" notes.
Use it to check your results, but be sure you understand the work involved. 
\end{enumerate}
\end{document}

