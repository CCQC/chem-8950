\documentclass{article}
\usepackage{amsmath,mathtools,amssymb}
\usepackage{graphicx}
\usepackage{gensymb}
\usepackage{verbatim}
\usepackage{cancel}
\usepackage{mathrsfs}
\usepackage{bbm}
\usepackage{braket}
\usepackage{simplewick}
\usepackage{xcolor}
\definecolor{orange}{RGB}{255,127,0}
\usepackage{verbatim}
\usepackage[margin=1.0in]{geometry}
\newcommand{\ol}{\overline}
\newcommand{\ve}{\varepsilon}
\newcommand{\ap}{\ensuremath{a_p} }
\newcommand{\dg}{\ensuremath{^\dagger} }
\newcommand{\cd}{\ensuremath{\cdots} }
\newcommand{\apd}{\ensuremath{a_p^\dagger} }
\def\*#1{\mathbf{#1}}
\DeclarePairedDelimiter\floor{\lfloor}{\rfloor}

\title{Homework for Lecture 3.3 Wick's Theorem}
\date{}
\begin{document}
\maketitle
\noindent
\begin{enumerate}
\item Show the following are true: 
 \[ \bcontraction{n[}{a}{{}_p a_q}{a}
\bcontraction[2ex]{n[a_p }{a}{{}_q a_r\dg a_s }{a}
n[a_p a_q a_r\dg a_s a_t\dg a_u\dg]  = -\delta_{pr}\delta_{qt} a_u\dg a_s\] 
 \[ \bcontraction{n[}{a}{{}_p a_q}{a}
\bcontraction[2ex]{n[a_p }{a}{{}_q a_r\dg a_s }{a}
\bcontraction{n[a_p a_q a_r\dg }{a}{{}_s a_t\dg }{a}
n[a_p a_q a_r\dg a_s a_t\dg a_u\dg]  = \delta_{pr}\delta_{qt} \delta_{su}\] 
% Note here that the term is \textit{fully contracted} and reduces to a number

\item Show that
\[\bcontraction{}{a}{{}_p}{a}a_p a_q\dg = n[\bcontraction{}{a}{{}_p}{a}a_p a_q\dg]\]


\item Show that $a_p a_q\dg a_r\dg  = \delta_{pq}a_r\dg - \delta_{pr} a_q\dg +  a_q\dg a_r\dg a_p$ can be rewritten as:
\[a_p a_q\dg a_r\dg  = n[a_p a_q\dg a_r\dg ] + n[\bcontraction{}{a}{{}_p}{a} a_p a_q\dg a_r\dg ]  + n[\bcontraction{}{a}{{}_p a_q\dg}{a} a_p a_q\dg a_r\dg ] \]

\item Explicitly write out Wick's theorem for a general product of 3 operators, $x_1 x_2 x_3$. If you want more practice, try writing out Wick's theorem for a general product of 5 operators. 

\item Explicitly write out Wick's theorem for a general product of 6 operators,  $n[x_1 x_2 x_3]n[ x_4 x_5 x_6]$ (use generalized Wick's theorem)

 \item Evaluate \[\braket{\Phi_{pq} | \Phi_{rst}} \] using Wick's theorem.
 Compare the work with the work done for HW 3.2 Problem 1. 
 
  \item Evaluate \[\braket{\Phi_{pqr} | \Phi_{stu}} \] using Wick's theorem.
(Hint: see Problem 5)
 
 \item You showed in HW 3.1 Problem 4 that the overlap of a Slater determinant is a Slater determinant of overlaps for $\braket{\Phi_{pq} | \Phi_{rs}}$.
 Now, show this generally for a Slater determinant:
 \[\braket{\Phi_{p_1 \cd p_N} | \Phi_{q_1 \cd q_N}} \]
 
 \item For $\braket{\Phi_{p_1 p_2 \cd p_N} | H | \Phi_{q_1 q_2 \cd q_N} }$ which differ by one spin-orbital such that $p_1 \neq q_1, p_2 = q_2 \cd p_N = q_N$, show that 
 
 \[\braket{\Phi_{p_1 p_2 \cd p_N} | H | \Phi_{q_1 q_2 \cd q_N} } = h_{p_1 q_1}  + \sum_k \braket{p_1 p_k || q_1 p_k}  \]
 This is an example of Slater's second rule
 
 \item For $\braket{\Phi_{p_1 p_2 \cd p_N} | H | \Phi_{q_1 q_2 \cd q_N} }$ which differ by two spin-orbitals such that $p_1 \neq q_1, p_2 \neq q_2, p_3 = q_3 \cd p_N = q_N$, show that
\[\braket{\Phi_{p_1 p_2 \cd p_N} | H | \Phi_{q_1 q_2 \cd q_N} } =  \braket{p_1 p_2 || q_1 p_2}  \]
   This is an example of Slater's third rule
 \item Apply Slater's first rule to explicit write out (i.e. without summation symbols) all terms for  \[\braket{\Phi_{pqr} | H| \Phi_{stu}} \]
 (Hint: There should be six terms total)
 
\end{enumerate}
\end{document}

