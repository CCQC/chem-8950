\documentclass[11pt]{article}
\usepackage[cm]{fullpage}
%%AVC PACKAGES
\usepackage{avcgreek}
\usepackage{avcfonts}
\usepackage{avcmath}
\usepackage[numberby=section]{avcthm} % 
\usepackage{qcmacros}
\usepackage{goldstone}
%%MACROS FOR THIS DOCUMENT
\numberwithin{equation}{section}
\usepackage[
  margin=1.5cm,
  includefoot,
  footskip=30pt,
  headsep=0.2cm,headheight=1.3cm
]{geometry}
\usepackage{fancyhdr}
\pagestyle{fancy}
\fancyhf{}
\fancyhead[LE,RO]{Quiz 1, Handout 2: Second Quantization}
\fancyfoot[CE,CO]{\thepage}
\usepackage{url}

\begin{document}

\setlength{\abovedisplayskip}{3pt}
\setlength{\belowdisplayskip}{3pt}

\section{Second Quantization}


\begin{dfn}\label{slater-determinant}
\thmtitle{Slater determinant}
An \textit{Slater determinant} is a normalized antisymmetric product of spin-orbitals
\begin{align}\label{eq:slater-determinant-position-representation}
  \F_{(p_1\cd p_n)}(1,\ld,n)
=
  \tfr{1}{\sqrt{n!}}
  \sum_{\pi\in\mr{S}_n}
  \e_{\pi}
  \y_{p_{\pi(1)}}(1)\cd\y_{p_{\pi(n)}}(n)
\end{align}
where $\pi\in\mr{S}_n$ is a permutation of $1\cd n$ with signature $\e_{\pi}$.\footnote{The signature of a permutation is $(-)^{\text{\# transpositions}}$.}
\end{dfn}

\subsection{Deriving the second-quantized Hamiltonian}\label{direct-derivation-of-second-quantization}


Let $\mc{F}_n$ denote the span of $n$-electron determinants and consider the integral operator $\op{a}_p:\mc{F}_n\rightarrow \mc{F}_{n-1}$ given by
\begin{align}
  (\op{a}_p\Y)(2,\cd,n)
\equiv
  \sqrt{n}\int d(1) \y_p^*(1)\Y(1,2,\cd,n)\,.
\end{align}
This operator acts on a Slater determinant as follows.
\begin{align}
  (\op{a}_p\F_{(p_1\cd p_n)})(2,\cd,n)
=
\left\{
\ar{
  (-)^{k-1}\F_{(p_1\cd \cancel{p_k}\cd p_n)}(2,\ld,n) & p=p_k\in(p_1\cd p_n)\\[5pt]
  0 & \text{otherwise}
}
\right.
\end{align}
In words, it deletes $\y_p$ from $\F_{(p_1\cd p_n)}$ if present, otherwise killing the determinant.
The restriction to an antisymmetric space makes these operators anticommute, $\op{a}_p\op{a}_q=-\op{a}_q\op{a}_p$, which can be demonstrated as follows.
\begin{align*}
  \int d(1)d(2)\y_p^*(1)\y_q^*(2)\Y(1,2,\cd,n)
=&\
  \int d(2)d(1)\y_p^*(2)\y_q^*(1)\Y(2,1,\cd,n)
\\=&\
-
  \int d(1)d(2)\y_q^*(1)\y_p^*(2)\Y(1,2,\cd,n)
\end{align*}
These deletion operators provide the following decompositions of functions in $F_n$.
\begin{align}
  \Y(1,\cd,n)
=
  \tfr{1}{\sqrt{n}}
  \sum_p^\infty\y_p(1)\pr{\op{a}_p\Y}(2,\cd,n)
=
  \tfr{1}{\sqrt{n(n-1)}}
  \sum_{pq}^\infty
  \y_p(1)\y_q(2)(\op{a}_q\op{a}_p\Y)(3,\cd,n)
\end{align}
Therefore, matrix elements of the electronic Hamiltonian with respect to $\Y,\Y'\in \mc{F}_n$ can be expressed as
\begin{align*}
  \ip{\Y|\op{H}_e\Y'}
=
  \sum_{i=1}^n\ip{\Y|\op{h}(i)\Y'}
+
  \sum_{i<j}^n\ip{\Y|\op{g}(i,j)\Y'}
=&\
  n\ip{\Y|\op{h}(1)\Y'}
+
  \tfr{n(n-1)}{2}
  \ip{\Y|\op{g}(1,2)\Y'}
\\=&\
  \sum_{pq}^\infty
  h_{pq}\ip{\op{a}_p\Y|\op{a}_q\Y'}
+
  \tfr{1}{2}
  \sum_{pqrs}^\infty
  \ip{pq|rs}\ip{\op{a}_q\op{a}_p\Y|\op{a}_s\op{a}_r\Y'}
\end{align*}
in terms of the usual one- and two-electron integrals.
Since $\Y$ and $\Y'$ are arbitary elements of $\mc{F}_n$, this implies
\begin{align}\label{eq:second-quantized-hamiltonian}
  \left.
  \op{H}_e
  \right|_{\mc{F}_n}
=
  \sum_{pq}^\infty
  h_{pq}
  \op{a}_p\dg \op{a}_q
+
  \tfr{1}{2}
  \sum_{pqrs}^\infty
  \ip{pq|rs}
  \op{a}_p\dg\op{a}_q\dg\op{a}_s\op{a}_r
\end{align}
which is the \textit{second quantized} form of the Hamiltonian, as opposed to the \textit{first quantized} form which is not restricted to antisymmetric functions.
A defining feature of the second quantization formalism is that $\op{H}_e$ is independent of the number of electrons, because \cref{eq:second-quantized-hamiltonian} holds for all $n$.



\subsection{Formal treatment of second quantization}


\begin{dfn}
\thmtitle{Hilbert space}
If $\mc{H}$ is a one-electron Hilbert space spanned by a set of spin-orbitals $\{\y_p\}$, then $\mc{H}^{\otimes n}=\mc{H}\otimes\cd\otimes\mc{H}=\spn\{\y_{p_1}{}\otimes\cd\otimes\y_{p_n}\}$ is an \textit{$n$-electron Hilbert space}.\footnote{These basis vectors are abstract representations spin-orbital product functions, $\ip{1\otimes\cd\otimes n|\y_{p_1}\otimes\cd\otimes\y_{p_n}}=\y_{p_1}(1)\cd\y_{p_n}(n)$, which are known as \textit{Hartree products}.}
\end{dfn}



\begin{dfn}\label{fock-space}
\thmtitle{Fock space}
Let $\mc{F}_n(\mc{H})$ denote $\spn\{\F_{(p_1\cd p_n)}\}$,\footnote{
These basis vectors are Slater determinants, abstracted from position space:
$
  \F_{(p_1\cd p_n)}
=
  \fr{1}{\sqrt{n!}}
  \sum_{\pi\in\mr{S}_n}
  \y_{p_{\pi(1)}}\otimes\cd\otimes
  \y_{p_{\pi(n)}}
$.
Equation \ref{eq:slater-determinant-position-representation} corresponds to
$
  \ip{1\otimes\cd\otimes n|\F_{(p_1\cd p_n)}}
$.
}
the antisymmetric subspace of $\mc{H}^{\otimes n}$.
The fermionic \textit{Fock space} is the union of all of these spaces, $\mc{F}(\mc{H})=\mc{F}_0(\mc{H})\oplus \mc{F}_1(\mc{H})\oplus \mc{F}_2(\mc{H})\oplus\cd\oplus \mc{F}_{\infty}(\mc{H})$, which comprises all possible electronic wavefunctions.
\end{dfn}

\begin{dfn}\label{occupation-number-representation}
\thmtitle{The occupation number representation of $\mc{F}(\mc{H})$}
In the \textit{occupation number representation} of Fock space, the basis vectors are represented as lists of 1s and 0s,
$
  \kt{\bo{n}}
\equiv
  \kt{n_1,n_2,n_3,\cd,n_\infty}
$,
where $n_p=1$ when $\y_p$ is occupied and $n_p=0$ when $\y_p$ is unoccupied in the state.
One possible basis for $\mc{F}(\mc{H})$ is given by distributing 1s and 0s over the occupation vector in all possible ways.
The state in which no spin-orbitals are occupied is called the \textit{vacuum state}, denoted $\kt{\vac}$, which spans $\mc{F}_0(\mc{H})$.
\end{dfn}

\begin{dfn}\label{particle-hole-operators}
\thmtitle{Particle-hole operators}
\textit{Particle-hole operators} change the occupation numbers of one-particle states.
The \textit{annihilation operator} of $\y_p$ is a linear mapping $a_p:\mc{F}_n(\mc{H})\rightarrow \mc{F}_{n-1}(\mc{H})$ defined by
\begin{align}\label{occ-num-annihilation-operator-action}
  a_p\kt{\cd n_p\cd}
=
  (-)^{n_1+\cd+n_{p-1}}
  \kt{\cd n_p-1\cd}
\ \ \ \text{if $n_p=1$}
&&
  a_p\kt{\cd n_p\cd}
=
  0
\ \ \ \text{if $n_p=0$}
\end{align}
and the \textit{creation operator} of $\y_p$ is a linear mapping $c_p:\mc{F}_n(\mc{H})\rightarrow \mc{F}_{n+1}(\mc{H})$ defined by
\begin{align}\label{occ-num-creation-operator-action}
  c_p\kt{\cd n_p\cd}
=
  (-)^{n_1+\cd+n_{p-1}}
  \kt{\cd n_p+1\cd}
\ \ \ \text{if $n_p=0$\ }
&&
  c_p\kt{\cd n_p\cd}
=
  0
\ \ \ \text{if $n_p=1$.}
\end{align}
\end{dfn}

\begin{prop}
\thmtitle{$c_p=a_p\dg$}
\thmstatement{Creation and annihilation operators of the same state $\y_p$ are adjoints of each other.}
\thmproof{
  $\ip{n_1'n_2'\cd|a_p[n_1n_2\cd]}$ vanishes unless $n_p'=0$, $n_p=1$, and $n_q'=n_q\ \forall q\neq p$.
  Likewise for $\ip{c_p[n_1'n_2'\cd]|n_1n_2\cd}$.
  Therefore, $\ip{\Y|a_p\Y'}=\ip{c_p\Y|\Y'}$ for all $\Y,\Y'\in \mc{F}(\mc{H})$ and $c_p=a_p\dg$ by the definition of adjoint.
}
\end{prop}

\begin{prop}\label{particle-hole-operator-anticommutator}
\thmtitle{$[q,q']_+=\d_{q'q\dg}$}
\thmstatement{Particle-hole operators $q$ and $q'$ anticommute unless $q'=q\dg$, for which $[q,q\dg]_+=1$.}
\thmproof{
  Let $q$ and $q'$ be arbitrary particle-hole operators acting on $\y_p$ and $\y_{p'}$, respectively.
  First, suppose $p\neq p'$. Then
  \begin{align*}
  &
    qq'\kt{\cd n_p\cd n_{p'}\cd}
  =
    (-)^{n_p+\sum_{r=p+1}^{p'}n_r}
    \kt{\cd\ol{n_p}\cd\ol{n_{p'}}\cd}
  \,\text{, and}
  \\
  &
    q'q\kt{\cd n_p\cd n_{p'}\cd}
  =
    (-)^{\ol{n_p}+\sum_{r=p+1}^{p'}n_r}
    \kt{\cd\ol{n_p}\cd\ol{n_{p'}}\cd}
  \end{align*}
  where $\ol{n_p}$ and $\ol{n_{p'}}$ are the occupations after applying $q$ and $q'$.
  Since $n_p$ and $\ol{n_p}$ differ by one, $qq'=-q'q$.
  The second case, $p=p'$, implies $q'\in\{q,q\dg\}$.
  If $q'=q$, then $qq'=-q'q=0$.
  If $q'=q\dg$, either $n_p=1\implies(a_p\dg a_p + a_pa_p\dg)\kt{\cd n_p\cd}=(1+0)\kt{\cd n_p\cd}$ or $n_p=0\implies(a_p\dg a_p + a_pa_p\dg)\kt{\cd n_p\cd}=(0+1)\kt{\cd n_p\cd}$.
  Either way, $q'=q\dg\implies(qq' + q'q)=1$.
}
\end{prop}

\begin{rmk}
\thmtitle{Relating the determinant and occupation number representations}
When $p_1<\cd<p_n$, $\F_{(p_1\cd p_n)}$ is equivalent to the occupation vector $\kt{\bo{n}_{(p_1\cd p_n)}}$ with 1s at $p_1,\cd,p_n$.
Otherwise, this determinant is equivalent to $\e_{\pi}\kt{\bo{n}_{(p_1\cd p_n)}}$ for $\pi\in\mr{S}_n$ such that $p_{\pi(1)}<\cd<p_{\pi(n)}$.
The actions of $a_p$ and $a_p\dg$ on $\F_{(p_1\cd p_n)}$ are given by
\begin{align}\label{abstract-annihilation-operator-action}
  a_p\F_{(p_1\cd p_n)}
=
  (-)^{k-1}\F_{(p_1\cd\cancel{p_k}\cd p_n)}
  \ \text{if $p=p_k\in(p_1\cd p_n)$}
&&
  a_p\F_{(p_1\cd p_n)}
=
  0
  \ \text{if $p\notin(p_1\cd p_n)$}
\\\label{abstract-creation-operator-action}
  a_p\dg\F_{(p_1\cd p_n)}
=
  (-)^{k-1}\F_{(p_1\cd p_{k-1}pp_k\cd p_n)}
  \ \text{if $p\notin(p_1\cd p_n)$}
&&
  a_p\dg\F_{(p_1\cd p_n)}
=
  0
  \ \text{if $p\in(p_1\cd p_n)$}
\end{align}
which follows directly from \Cref{occ-num-annihilation-operator-action,occ-num-creation-operator-action} when $p_1<\cd<p_n$.
Other cases follow from the fact that any sign factors for permuting $(p_1\cd p_n)$ cancel on both sides of the equation, including the position of insertion or deletion, $p_k$, whose phase is tracked by $(-)^{k-1}$ on the right.
That is, both sides of the equation are antisymmetric to permutations of $(1\cd n)$.
Note that \Cref{abstract-annihilation-operator-action} was also derived in \Cref{direct-derivation-of-second-quantization} using the position-space representation of $a_p$.
One advantage of the determinant basis is that, unlike occupation vectors, determinants translate directly into strings of creations operators
\begin{align}
  \kt{\F_{(p_1\cd p_n)}}
=
  a_{p_1}\dg\cd a_{p_n}\dg\kt{\vac}
\end{align}
without any phase ambiguity.
Together with the second quantized form of the electronic Hamiltonian (\Cref{second-quantized-hamiltonian}), this boils much of the grunt work of electronic structure theory down to particle-hole operator algebra.
\end{rmk}

\begin{dfn}
\thmtitle{Excitation operators and excited determinants}
Operator strings of the form $a_{p_1}\dg\cd a_{p_m}\dg a_{q_m}\cd a_{q_1}$ are called \textit{excitation operators}.
For a given reference determinant $\F$, excited determinants can be constructed as
\begin{align}
  \F_{i_1\cd i_m}^{a_1\cd a_m}
=
  a_{a_1}\dg\cd a_{a_m}\dg a_{i_m}\cd a_{i_1}\F
=
  a_{a_1}\dg a_{i_1}\cd a_{a_m}\dg a_{i_m}\F
\end{align}
where $i_1,\cd,i_m$ are occupied and $a_1,\cd,a_m$ are virtual indices with respect to $\F$.
\end{dfn}




\end{document}