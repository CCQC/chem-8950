\documentclass[11pt]{article}
\usepackage[cm]{fullpage}
%%AVC PACKAGES
\usepackage{avcgreek}
\usepackage{avcfonts}
\usepackage{avcmath}
\usepackage[numberby=section]{avcthm}
\usepackage{qcmacros}
\usepackage{goldstone}
%%MACROS FOR THIS DOCUMENT
\numberwithin{equation}{section}
\usepackage[
  margin=1.5cm,
  includefoot,
  footskip=30pt,
  headsep=0.2cm,headheight=1.3cm
]{geometry}
\usepackage{fancyhdr}
\pagestyle{fancy}
\fancyhf{}
\fancyhead[LE,RO]{\textbf{Quiz 1}}
\fancyfoot[CE,CO]{\thepage}
\usepackage{url}

\begin{document}

\begin{enumerate}
\item
  Answer each of the following in one sentence, using words only.
  \begin{enumerate}
  \item
    Define canonical Hartree-Fock orbitals.
    \vspace{2cm}
  \item
    Explain why the choice of Hartree-Fock orbitals is not unique.
    \vspace{2cm}
  \end{enumerate}

\item
  Expand $a_pa_qa_s\dg a_r\dg$ as a linear combination of strings which are in normal order.
  Identify the vacuum expectation value of this operator product.
  \vspace{11cm}

\item
  Derive the Slater determinant expectation value of a two-electron operator in terms of two-electron integrals, showing your steps along the way.
  You may use second quantization methods (and your result from problem 2.) if you first expand the expectation value in terms of particle-hole operators.\footnote{%
  You may take the following expansion as given:
  \begin{align*}
    \Y(1,2,3\ld,n)
  =
    \fr{1}{\sqrt{n(n-1)}}
    \sum_{pq}^\infty
    \y_p(1)
    \y_q(2)
    (\op{a}_q\op{a}_p\Y)(3,\ld,n)
  \end{align*}
  }
  \begin{align*}
    \tfr{1}{2}
    \sum_{i\neq j}^n
    \ip{\F|\op{g}(i,j)\F}
  =
    \,\,?
  \end{align*}
\end{enumerate}

\end{document}
