%%%%%%%%%%%%%%%%%%%%%%%%%%%%%%%%%%%%%%%%%%%%%%%%%%%%%%%%%%%%%%%%%%%%%%%%%%%%%%%%%%
% This work is licensed under the Creative Commons Attribution 4.0 International %
% License. To view a copy of this license, visit                                 %
% http://creativecommons.org/licenses/by/4.0/.                                   %
%%%%%%%%%%%%%%%%%%%%%%%%%%%%%%%%%%%%%%%%%%%%%%%%%%%%%%%%%%%%%%%%%%%%%%%%%%%%%%%%%%
\documentclass[11pt]{article}
\usepackage[cm]{fullpage}
%%AVC PACKAGES
\usepackage{avcgreek}
\usepackage{avcfonts}
\usepackage{avcmath}
\usepackage[numberby=section]{avcthm}
\usepackage{qcmacros}
\usepackage{goldstone}
%%MACROS FOR THIS DOCUMENT
\numberwithin{equation}{section}
\usepackage[
  margin=1.5cm,
  includefoot,
  footskip=30pt,
  headsep=0.2cm,headheight=1.3cm
]{geometry}
\usepackage{fancyhdr}
\pagestyle{fancy}
\fancyhf{}
\fancyhead[LE,RO]{\textbf{Quiz 3}}
\fancyfoot[CE,CO]{\thepage}
\usepackage{url}

\begin{document}

\begin{enumerate}
\item
  Translate the following expression from KM notation into our original notation, using daggers to denote creation operators and lines to denote contractions.
\begin{align*}
  \gno{
    a^{p^\hole q^{\ptcl\phantom{\ptcl}} r^{\ptcl\ptcl}}
     _{s^\ptcl t^{\ptcl\ptcl} u^\hole}
  }
=
  \,\,?
\end{align*}
  Your final expression should be a $\F$-normal-ordered string of six operators with three contraction lines.

\newpage
\item
Expand the following as a linear combination of $\F$-normal-ordered operators.
\begin{align*}
  \tl{a}^p_q
  \tl{a}^{rs}_{tu}
=
  \,\,?
\end{align*}


\newpage
\item
Evaluate the following matrix element.\footnote{You may use either of the following equivalent expressions for the Hamiltonian. (I recommend the one on the right!)
\begin{align*}
  H
=
  h_p^q
  a^p_q
+
  \tfr{1}{4}
  \ol{g}_{pq}^{rs}
  a^{pq}_{rs}
&&
  H
=
  E_\mr{ref}
+
  f_p^q
  \tl{a}^p_q
+
  \tfr{1}{4}
  \ol{g}_{pq}^{rs}
  \tl{a}^{pq}_{rs}
\end{align*}
}
\begin{align*}
  \ip{\F_i^a|H - E_\mr{ref}|\F_j^b}
=
  \,\,?
\end{align*}
\end{enumerate}

\end{document}
