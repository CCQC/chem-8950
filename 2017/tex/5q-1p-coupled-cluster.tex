%%%%%%%%%%%%%%%%%%%%%%%%%%%%%%%%%%%%%%%%%%%%%%%%%%%%%%%%%%%%%%%%%%%%%%%%%%%%%%%%%%
% This work is licensed under the Creative Commons Attribution 4.0 International %
% License. To view a copy of this license, visit                                 %
% http://creativecommons.org/licenses/by/4.0/.                                   %
%%%%%%%%%%%%%%%%%%%%%%%%%%%%%%%%%%%%%%%%%%%%%%%%%%%%%%%%%%%%%%%%%%%%%%%%%%%%%%%%%%
\documentclass[11pt]{article}
\usepackage[cm]{fullpage}
%%AVC PACKAGES
\usepackage{avcgreek}
\usepackage{avcfonts}
\usepackage{avcmath}
\usepackage[numberby=section]{avcthm}
\usepackage{qcmacros}
\usepackage{goldstone}
%%MACROS FOR THIS DOCUMENT
\usepackage[
  margin=1.5cm,
  includefoot,
  footskip=30pt,
  headsep=0.2cm,headheight=1.3cm
]{geometry}
\usepackage{fancyhdr}
\pagestyle{fancy}
\fancyhf{}
\fancyhead[LE,RO]{Quiz 5, Suggested Problems 1: Coupled-Cluster Theory}
\fancyfoot[CE,CO]{\thepage}
\usepackage{url}
\usepackage{multicol}

\begin{document}

\begin{enumerate}
\item
Derive the Hausdorff expansion.\footnote{For an alternative to the direct, somewhat bullheaded proof shown in the notes look at the solution to Exercise 3.1.6 in Helgaker's big purple book, but note that you should give a proper proof by induction.\footnotemark}
\footnotetext{See \url{https://en.wikipedia.org/wiki/Mathematical_induction}.}
\begin{align}
  e^X Ye^{-X}
=
  Y
+
  [X,Y]
+
  \tfr{1}{2!}
  [X,[X,Y]]
+
  \tfr{1}{3!}
  [X,[X,[X,Y]]]
+
  \cd
\end{align}

\item
Prove the following.
\begin{align}
\label{eq:h-t-commutators}
  [H_\mr{c},T]
=
\gno{
  \ctr[0.0]{}{H}{_\mr{c}}{T}
  H_\mr{c}T
}
&&
  [[H_\mr{c},T],T]
=
\gno{
  \ctr[0.7]{}{H}{_\mr{c}T}{T}
  \ctr[0.0]{}{H}{_\mr{c}}{T}
  H_\mr{c}TT
}
&&
  [[[H_\mr{c},T],T],T]
=
\gno{
  \ctr[1.4]{}{H}{_\mr{c}TT}{T}
  \ctr[0.7]{}{H}{_\mr{c}T}{T}
  \ctr[0.0]{}{H}{_\mr{c}}{T}
  H_\mr{c}TTT
}
&&
  \cd
\end{align}

\item
Using equation~\ref{eq:h-t-commutators}, explain the following.
\begin{align}
  [\,\cdot\,,T]^n(H_\mr{c})
=
  0
\hspace{5pt}
  \text{for $n> 4$}
\end{align}

\item
Prove that the determinant basis consists of eigenfunctions of the diagonal Fock operator.
\begin{align}
\label{eq:diagonal-fock-eigenvalue-equation}
  H_0\F_{i_1\cd i_k}^{a_1\cd a_k}
=
  \mc{E}_{i_1\cd i_k}^{a_1\cd a_k}
  \F_{i_1\cd i_k}^{a_1\cd a_k}
&&
  H_0
\equiv
  f_p^p\tl{a}^p_p
&&
  \mc{E}_{i_1\cd i_k}^{a_1\cd a_k}
\equiv
  \sum_{r=1}^k
  f_{a_r}^{a_r}
-
  \sum_{r=1}^k
  f_{i_r}^{i_r}
\end{align}

\item
Use equations~\ref{eq:h-t-commutators} and~\ref{eq:diagonal-fock-eigenvalue-equation} to write the coupled-cluster amplitude equation $\ip{\F_{ij\cd}^{ab\cd}|\ol{H}_\mr{c}|\F}=0$ as follows.
\begin{align}
  t_{ab\cd}^{ij\cd}
=
  (\mc{E}_{ab\cd}^{ij\cd})^{-1}
  \ip{\F_{ij\cd}^{ab\cd}|V_\mr{c}\,\mr{exp}(T)|\F}_\mr{C}
&&
  V_\mr{c}
\equiv
  H_\mr{c}
-
  H_0
\end{align}


\item
Explain why the following terms vanish.\footnote{Hint: Use arguments about the excitation levels of their operators.}
\begin{align*}
  \tfr{1}{2}
  \ip{\F_i^a|V_\mr{c}T_2^2|\F}_\mr{C}
&&
  \ip{\F_{ijk}^{abc}|V_\mr{c}|\F}_\mr{C}
&&
  \ip{\F_{ijk}^{abc}|V_\mr{c}T_1|\F}_\mr{C}
\end{align*}

\end{enumerate}


\end{document}
