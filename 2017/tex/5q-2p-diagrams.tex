\documentclass[11pt]{article}
\usepackage[cm]{fullpage}
%%AVC PACKAGES
\usepackage{avcgreek}
\usepackage{avcfonts}
\usepackage{avcmath}
\usepackage[numberby=section]{avcthm}
\usepackage{qcmacros}
\usepackage{goldstone}
%%MACROS FOR THIS DOCUMENT
\usepackage[
  margin=1.5cm,
  includefoot,
  footskip=30pt,
  headsep=0.2cm,headheight=1.3cm
]{geometry}
\usepackage{fancyhdr}
\pagestyle{fancy}
\fancyhf{}
\fancyhead[LE,RO]{Quiz 5, Suggested Problems 2: Diagrams}
\fancyfoot[CE,CO]{\thepage}
\usepackage{url}
\usepackage{multicol}

\begin{document}

\begin{enumerate}
\item
  Interpret the following coefficient graph algebraically, denoting the bare excitation operator by $\tl{a}_{abc}^{ijk}$.
\begin{align*}
\diagram{
  \interaction{3}{c}{(0,-0.5)}{ddot}{overhang};
  \draw[->-=0.25,->-=0.75] (c1)
    to node[midway,ddot] (g1) {} ++(-0.25,+1) node[smalldot] {};
  \draw[-<-] (c1) to ++(+0.25,+1) node[smalldot] {};
  \draw[->-=0.25,->-=0.75] (c2)
    to node[midway,ddot] (g2) {} ++(-0.25,+1) node[smalldot] {};
  \draw[-<-] (c2) to ++(+0.25,+1) node[smalldot] {};
  \draw[->-] (c3) to ++(-0.25,+1) node[smalldot] {};
  \draw[-<-] (c3) to ++(+0.25,+1) node[smalldot] {};
  \draw[sawtooth] (g1) to (g2);
}
\end{align*}


\item
  Interpret the following coefficient graph algebraically, denoting the top bare excitation operator by $\tl{a}^{ij}_{ab}$ and the bottom one by $\tl{a}^{cd}_{kl}$ and their interaction tensors by ${}^1\ol{\d}_{mn}^{ef}$ and ${}^2\ol{\d}_{ef}^{mn}$.
\begin{align*}
\diagram{
  \draw (0,0) node[ddot] (f1) {} to ++(-0.5,0) node[circlex] {};
  \draw[-<-] (f1)
    to
    ++(0,+0.5)
      node[smalldot] {};
  \draw[->-] (f1)
    to
    ++(0,-0.5)
      node[smalldot] {};
  \draw[->-]
    (+0.5,-0.5)
      node[smalldot] {}
    to
    ++(0,1)
      node[smalldot] {};
  \draw[-<-]
    (+1.0,-0.5)
      node[smalldot] {}
    to
    ++(0,1)
      node[smalldot] {};
  \draw[->-]
    (+1.5,-0.5)
      node[smalldot] {}
    to
    ++(0,1)
      node[smalldot] {};
}
\end{align*}


\end{enumerate}


\newpage\noindent
\textbf{Solutions}:
\begin{enumerate}
\item
Axiom 4.1 gives
\begin{align*}
\diagram{
  \interaction{3}{c}{(0,-0.5)}{ddot}{overhang};
  \draw[->-=0.25,->-=0.75] (c1)
    to node[midway,ddot] (g1) {} ++(-0.25,+1) node[smalldot] {};
  \draw[-<-] (c1) to ++(+0.25,+1) node[smalldot] {};
  \draw[->-=0.25,->-=0.75] (c2)
    to node[midway,ddot] (g2) {} ++(-0.25,+1) node[smalldot] {};
  \draw[-<-] (c2) to ++(+0.25,+1) node[smalldot] {};
  \draw[->-] (c3) to ++(-0.25,+1) node[smalldot] {};
  \draw[-<-] (c3) to ++(+0.25,+1) node[smalldot] {};
  \draw[sawtooth] (g1) to (g2);
}
=
  \fr{1}{2!2!3!}
  \sum_{\substack{defgh\\lmn}}
\diagram{
  \interaction{3}{c}{(0,-0.5)}{ddot}{overhang};
  \draw[->-=0.25,->-=0.75] (c1)
    to
      node[pos=0.25,left=0pt] {$g$}
      node[midway,ddot] (g1) {}
      node[pos=0.75,left=1pt] {$d$}
    ++(-0.25,+1)
      node[smalldot] {};
  \draw[-<-] (c1)
    to
      node[pos=0.3,right=-1pt] {$l$}
    ++(+0.25,+1)
      node[smalldot] {};
  \draw[->-=0.25,->-=0.75] (c2)
    to
      node[pos=0.25,left=0pt] {$h$}
      node[midway,ddot] (g2) {}
      node[pos=0.75,left=0pt] {$e$}
    ++(-0.25,+1)
      node[smalldot] {};
  \draw[-<-] (c2)
    to
      node[pos=0.3,right=-1pt] {$m$}
    ++(+0.25,+1)
      node[smalldot] {};
  \draw[->-] (c3)
    to
      node[pos=0.75,left=0pt] {$f$}
    ++(-0.25,+1)
      node[smalldot] {};
  \draw[-<-] (c3)
    to
      node[pos=0.3,right=-1pt] {$n$}
    ++(+0.25,+1)
      node[smalldot] {};
  \draw[sawtooth] (g1) to (g2);
}
=
  \fr{1}{2!2!3!}
  \sum_{\substack{defgh\\lmn}}
  \ol{\d}_{lmn}^{def}
  \ol{g}_{de}^{gh}\,
  c_{ghf}^{lmn}
  \gno{
    a^{l^{\hole1}m^{\hole2}n^{\hole3}}_{d^{\ptcl1}e^{\ptcl2}f^{\ptcl3}}
    a^{d^{\ptcl1}e^{\ptcl2}}_{g^{\ptcl4}h^{\ptcl5}}
    a^{g^{\ptcl4}h^{\ptcl5}f^{\ptcl3}}_{l^{\hole1}m^{\hole2}n^{\hole3}}
  }
\end{align*}
since there are two pairs of equivalent lines on the repulsion operator, and the CI triples operator has a set of three equivalent lines.
The interaction tensor for the bare excitation operator is
$
  \ol{\d}_{lmn}^{def}
=
  \op{P}_{(a/b/c)}^{(i/j/k)}
  \d^d_a\d^e_b\d^f_c
  \d^i_l\d^j_m\d^k_n
$, so this can be simplified as
\begin{align*}
  \sum_{\substack{def\\lmn}}
  \ol{\d}_{lmn}^{def}
  T_{def}^{lmn}
=
  P_{(a/b/c)}^{(i/j/k)}
  T_{abc}^{ijk}
&&
\text{
  where
}\
  T_{def}^{lmn}
\equiv
  \fr{1}{2!2!3!}
  \sum_{gh}
  \ol{g}_{de}^{gh}\,
  c_{ghf}^{lmn}
  \gno{
    a^{l^{\hole1}m^{\hole2}n^{\hole3}}_{d^{\ptcl1}e^{\ptcl2}f^{\ptcl3}}
    a^{d^{\ptcl1}e^{\ptcl2}}_{g^{\ptcl4}h^{\ptcl5}}
    a^{g^{\ptcl4}h^{\ptcl5}f^{\ptcl3}}_{l^{\hole1}m^{\hole2}n^{\hole3}}
  }\,.
\end{align*}
Using item 3 in Remark 4.3, the contracted operator string evaluates as follows
\begin{align*}
  \gno{
    a^{l^{\hole1}m^{\hole2}n^{\hole3}}_{d^{\ptcl1}e^{\ptcl2}f^{\ptcl3}}
    a^{d^{\ptcl1}e^{\ptcl2}}_{g^{\ptcl4}h^{\ptcl5}}
    a^{g^{\ptcl4}h^{\ptcl5}f^{\ptcl3}}_{l^{\hole1}m^{\hole2}n^{\hole3}}
  }
=
  (-1)^{3+3}
=
  +1
\end{align*}
since there are three hole lines and three loops in the graph.
At this point, we have simplified our interpretation to the following
\begin{align*}
\diagram{
  \interaction{3}{c}{(0,-0.5)}{ddot}{overhang};
  \draw[->-=0.25,->-=0.75] (c1)
    to node[midway,ddot] (g1) {} ++(-0.25,+1) node[smalldot] {};
  \draw[-<-] (c1) to ++(+0.25,+1) node[smalldot] {};
  \draw[->-=0.25,->-=0.75] (c2)
    to node[midway,ddot] (g2) {} ++(-0.25,+1) node[smalldot] {};
  \draw[-<-] (c2) to ++(+0.25,+1) node[smalldot] {};
  \draw[->-] (c3) to ++(-0.25,+1) node[smalldot] {};
  \draw[-<-] (c3) to ++(+0.25,+1) node[smalldot] {};
  \draw[sawtooth] (g1) to (g2);
}
=
  \fr{1}{2!2!3!}
  \sum_{de}
  \op{P}_{(a/b/c)}^{(i/j/k)}
  \ol{g}_{ab}^{de}\,
  c_{dec}^{ijk}
\end{align*}
where I have relabeled the summation indices $g\mapsto d$, $h\mapsto e$.
Finally, using item 4 under Remark 4.3, we can cancel the degeneracy factors $2!3!$ coming from equivalent coefficient lines by replacing
$
  \op{P}^{(i/j/k)}_{(a/b/c)}
$
with
$
  \op{P}_{(ab/c)}
$.
This works because the operand
$
  \ol{g}_{ab}^{de}
  c_{dec}^{ijk}
$
is already antisymmetric with respect to $\{a,b\}$ and $\{i,j,k\}$.
\begin{align*}
\diagram{
  \interaction{3}{c}{(0,-0.5)}{ddot}{overhang};
  \draw[->-=0.25,->-=0.75] (c1)
    to node[midway,ddot] (g1) {} ++(-0.25,+1) node[smalldot] {};
  \draw[-<-] (c1) to ++(+0.25,+1) node[smalldot] {};
  \draw[->-=0.25,->-=0.75] (c2)
    to node[midway,ddot] (g2) {} ++(-0.25,+1) node[smalldot] {};
  \draw[-<-] (c2) to ++(+0.25,+1) node[smalldot] {};
  \draw[->-] (c3) to ++(-0.25,+1) node[smalldot] {};
  \draw[-<-] (c3) to ++(+0.25,+1) node[smalldot] {};
  \draw[sawtooth] (g1) to (g2);
}
=
  \fr{1}{2!}
  \sum_{de}
  \op{P}_{(ab/c)}
  \ol{g}_{ab}^{de}\,
  c_{dec}^{ijk}
\end{align*}


\item
Axiom 4.1 gives
\begin{align*}
\diagram{
  \draw (0,0) node[ddot] (f1) {} to ++(-0.5,0) node[circlex] {};
  \draw[-<-] (f1)
    to
    ++(0,+0.5)
      node[smalldot] {};
  \draw[->-] (f1)
    to
    ++(0,-0.5)
      node[smalldot] {};
  \draw[->-]
    (+0.5,-0.5)
      node[smalldot] {}
    to
    ++(0,1)
      node[smalldot] {};
  \draw[-<-]
    (+1.0,-0.5)
      node[smalldot] {}
    to
    ++(0,1)
      node[smalldot] {};
  \draw[->-]
    (+1.5,-0.5)
      node[smalldot] {}
    to
    ++(0,1)
      node[smalldot] {};
}
=
  \fr{1}{2!}
  \sum_{\substack{ef\\mno}}
\diagram{
  \draw (0,0) node[ddot] (f1) {} to ++(-0.5,0) node[circlex] {};
  \draw[-<-] (f1)
    to
      node[pos=0.5,left=0pt] {$m$}
    ++(0,+0.5)
      node[smalldot] {};
  \draw[->-] (f1)
    to
      node[pos=0.5,left=0pt] {$o$}
    ++(0,-0.5)
      node[smalldot] {};
  \draw[->-]
    (+0.5,-0.5)
      node[smalldot] {}
    to
      node[pos=0.5,left=0pt] {$e$}
    ++(0,1)
      node[smalldot] {};
  \draw[-<-]
    (+1.0,-0.5)
      node[smalldot] {}
    to
      node[pos=0.5,left=0pt] {$n$}
    ++(0,1)
      node[smalldot] {};
  \draw[->-]
    (+1.5,-0.5)
      node[smalldot] {}
    to
      node[pos=0.5,left=0pt] {$f$}
    ++(0,1)
      node[smalldot] {};
}
=
  \fr{1}{2!}
  \sum_{\substack{ef\\mno}}
  {}^1\ol{\d}_{mn}^{ef}
  f_o^m\,
  {}^2\ol{\d}_{ef}^{on}\,
  \gno{
    a^{m^{\hole1}n^{\hole2}}_{e^{\ptcl1}f^{\ptcl2}}
    a^{o^{\hole3}}_{m^{\hole1}}
    a^{e^{\ptcl1}f^{\ptcl2}}_{o^{\hole3}n^{\hole2}}
  }
\end{align*}
since the two particle lines are equivalent.
Using item 3 in Remark 4.3, the operator string evaluates to
\begin{align*}
  \gno{
    a^{m^{\hole1}n^{\hole2}}_{e^{\ptcl1}f^{\ptcl2}}
    a^{o^{\hole3}}_{m^{\hole1}}
    a^{e^{\ptcl1}f^{\ptcl2}}_{o^{\hole3}n^{\hole2}}
  }
=
  (-1)^{3+2}
=
  -1
\end{align*}
since there are three hole lines and two loops in the diagram.
Substituting in the definitions of the interaction tensors, we can simplify the result as follows.
\begin{align*}
\diagram{
  \draw (0,0) node[ddot] (f1) {} to ++(-0.5,0) node[circlex] {};
  \draw[-<-] (f1)
    to
    ++(0,+0.5)
      node[smalldot] {};
  \draw[->-] (f1)
    to
    ++(0,-0.5)
      node[smalldot] {};
  \draw[->-]
    (+0.5,-0.5)
      node[smalldot] {}
    to
    ++(0,1)
      node[smalldot] {};
  \draw[-<-]
    (+1.0,-0.5)
      node[smalldot] {}
    to
    ++(0,1)
      node[smalldot] {};
  \draw[->-]
    (+1.5,-0.5)
      node[smalldot] {}
    to
    ++(0,1)
      node[smalldot] {};
}
=
-
  \fr{1}{2!}
  \sum_{\substack{ef\\mno}}
  {}^1\ol{\d}_{mn}^{ef}
  f_o^m\,
  {}^2\ol{\d}_{ef}^{on}
=&\
-
  \fr{1}{2!}
  \sum_{\substack{ef\\mno}}
  \pr{
    \op{P}_{(a/b)}^{(i/j)}
    \d_a^e\d_b^f\d_m^i\d_n^j
  }
  f_o^m
  \pr{
    \op{P}_{(k/l)}^{(c/d)}
    \d_k^o
    \d_l^n
    \d_e^c
    \d_f^d
  }
\\=&\
-
  \fr{1}{2!}
  \op{P}_{(a/b)}^{(i/j)}
  \op{P}_{(k/l)}^{(c/d)}
  f_k^i
  \d_l^j
  \d_a^c
  \d_b^d
\end{align*}
Finally, using item 4 under Remark 4.3, we can cancel the degeneracy factor $2!$ by replacing $\op{P}^{(i/j)}_{(a/b)}$ with $\op{P}^{(i/j)}$.
\begin{align*}
\diagram{
  \draw (0,0) node[ddot] (f1) {} to ++(-0.5,0) node[circlex] {};
  \draw[-<-] (f1)
    to
    ++(0,+0.5)
      node[smalldot] {};
  \draw[->-] (f1)
    to
    ++(0,-0.5)
      node[smalldot] {};
  \draw[->-]
    (+0.5,-0.5)
      node[smalldot] {}
    to
    ++(0,1)
      node[smalldot] {};
  \draw[-<-]
    (+1.0,-0.5)
      node[smalldot] {}
    to
    ++(0,1)
      node[smalldot] {};
  \draw[->-]
    (+1.5,-0.5)
      node[smalldot] {}
    to
    ++(0,1)
      node[smalldot] {};
}
=
-
  \op{P}_{(k/l)}^{(i/j|c/d)}
  f_k^i
  \d_l^j
  \d_a^c
  \d_b^d
\end{align*}


\end{enumerate}


\end{document}
