\documentclass[11pt]{article}
\usepackage[cm]{fullpage}
%%AVC PACKAGES
\usepackage{avcgreek}
\usepackage{avcfonts}
\usepackage{avcmath}
\usepackage[numberby=section]{avcthm}
\usepackage{qcmacros}
\usepackage{goldstone}
%%MACROS FOR THIS DOCUMENT
\numberwithin{equation}{section}
\usepackage[
  margin=1.5cm,
  includefoot,
  footskip=30pt,
  headsep=0.2cm,headheight=1.3cm
]{geometry}
\usepackage{fancyhdr}
\pagestyle{fancy}
\fancyhf{}
\fancyhead[LE,RO]{Quiz 4, Suggested Problems 1: Diagrams}
\fancyfoot[CE,CO]{\thepage}
\usepackage{url}
\usepackage{multicol}

\begin{document}

\begin{enumerate}
\item
  Come up with an example to illustrate each of the following terms.
  \begin{multicols}{3}
  \begin{enumerate}
  \item
    equivalent lines
  \item
    closed graph
  \item
    open graph
  \item
    interchangeable operators
  \item
    equivalent operators
  \item
    interchangeable subgraphs
  \item
    equivalent subgraphs
  \item
    Goldstone path
  \item
    Hugenholtz path
  \item
    open cycle
  \item
    loop
  \item
    connected graph
  \item
    disconnected graph
  \item
    unlinked graph
  \item
    equivalent contractions
  \item
    energy graph
  \item
    coefficient graph
  \item
    wavefunction graph
  \end{enumerate}
  \end{multicols}

\item
  Using
$
\diagram{
  \draw (-0.5,0) node[circlex] (f) {} -- (0,0) node[ddot=white] (f1) {};
  \draw[->-] (f1) to ++(0,+0.35);
  \draw[-<-] (f1) to ++(0,-0.35);
}
\equiv
  f_p^q
  \tl{a}^p_q
$, interpret each of the following as an algebraic expression.
  \begin{multicols}{4}
  \begin{enumerate}
  \item
\diagram{
  \draw (-0.5,0) node[circlex] (h) {} -- (0,0) node[ddot] (h1) {};
  \draw[->-] (h1) to ++(0,+0.5);
  \draw[-<-] (h1) to ++(0,-0.5);
}
  \item
\diagram{
  \draw (-0.5,0) node[circlex] (h) {} -- (0,0) node[ddot] (h1) {};
  \draw[->-] (h1) to ++(-0.25,+0.5);
  \draw[-<-] (h1) to ++(+0.25,+0.5);
}
  \item
\diagram{
  \draw (-0.5,0) node[circlex] (h) {} -- (0,0) node[ddot] (h1) {};
  \draw[->-] (h1) to ++(-0.25,-0.5);
  \draw[-<-] (h1) to ++(+0.25,-0.5);
}
  \item
\diagram{
  \draw (-0.5,0) node[circlex] (h) {} -- (0,0) node[ddot] (h1) {};
  \draw[->-] (h1) to ++(0,-0.5);
  \draw[-<-] (h1) to ++(0,+0.5);
}
  \end{enumerate}
  \end{multicols}



\item
  Using
$
\diagram{
  \interaction{2}{g}{(0,0)}{ddot=white}{sawtooth};
  \draw[->-] (g1) to ++(0,+0.35);
  \draw[-<-] (g1) to ++(0,-0.35);
  \draw[->-] (g2) to ++(0,+0.35);
  \draw[-<-] (g2) to ++(0,-0.35);
}
\equiv
  \pr{
    \tfr{1}{2}
  }^2
  \ol{g}_{pq}^{rs}
  \tl{a}^{pq}_{rs}
$,
  interpret each of the following as an algebraic expression.
  \begin{multicols}{4}
  \begin{enumerate}
  \item
\diagram{
  \interaction{2}{g}{(0,0)}{ddot}{sawtooth};
  \draw[->-] (g1) to ++(0,+0.5);
  \draw[-<-] (g1) to ++(0,-0.5);
  \draw[->-] (g2) to ++(0,+0.5);
  \draw[-<-] (g2) to ++(0,-0.5);
}
  \item
\diagram{
  \interaction{2}{g}{(0,0)}{ddot}{sawtooth};
  \draw[->-] (g1) to ++(0,+0.5);
  \draw[-<-] (g1) to ++(0,-0.5);
  \draw[->-] (g2) to ++(-0.25,+0.5);
  \draw[-<-] (g2) to ++(+0.25,+0.5);
}
  \item
\diagram{
  \interaction{2}{g}{(0,0)}{ddot}{sawtooth};
  \draw[->-] (g1) to ++(0,+0.5);
  \draw[-<-] (g1) to ++(0,-0.5);
  \draw[->-] (g2) to ++(-0.25,-0.5);
  \draw[-<-] (g2) to ++(+0.25,-0.5);
}
  \item
\diagram{
  \interaction{2}{g}{(0,0)}{ddot}{sawtooth};
  \draw[->-] (g1) to ++(-0.25,+0.5);
  \draw[-<-] (g1) to ++(+0.25,+0.5);
  \draw[->-] (g2) to ++(-0.25,+0.5);
  \draw[-<-] (g2) to ++(+0.25,+0.5);
}
  \item
\diagram{
  \interaction{2}{g}{(0,0)}{ddot}{sawtooth};
  \draw[->-] (g1) to ++(-0.25,-0.5);
  \draw[-<-] (g1) to ++(+0.25,-0.5);
  \draw[->-] (g2) to ++(-0.25,-0.5);
  \draw[-<-] (g2) to ++(+0.25,-0.5);
}
  \item
\diagram{
  \interaction{2}{g}{(0,0)}{ddot}{sawtooth};
  \draw[->-] (g1) to ++(-0.25,-0.5);
  \draw[-<-] (g1) to ++(+0.25,-0.5);
  \draw[->-] (g2) to ++(-0.25,+0.5);
  \draw[-<-] (g2) to ++(+0.25,+0.5);
}
  \item
\diagram{
  \interaction{2}{g}{(0,0)}{ddot}{sawtooth};
  \draw[->-] (g1) to ++(0,-0.5);
  \draw[-<-] (g1) to ++(0,+0.5);
  \draw[->-] (g2) to ++(-0.25,+0.5);
  \draw[-<-] (g2) to ++(+0.25,+0.5);
}
  \item
\diagram{
  \interaction{2}{g}{(0,0)}{ddot}{sawtooth};
  \draw[->-] (g1) to ++(0,-0.5);
  \draw[-<-] (g1) to ++(0,+0.5);
  \draw[->-] (g2) to ++(-0.25,-0.5);
  \draw[-<-] (g2) to ++(+0.25,-0.5);
}
  \item
\diagram{
  \interaction{2}{g}{(0,0)}{ddot}{sawtooth};
  \draw[->-] (g1) to ++(0,-0.5);
  \draw[-<-] (g1) to ++(0,+0.5);
  \draw[->-] (g2) to ++(0,-0.5);
  \draw[-<-] (g2) to ++(0,+0.5);
}
  \end{enumerate}
  \end{multicols}

\item
  Interpret each of the following as an algebraic expression.
  \begin{multicols}{3}
  \begin{enumerate}
  \item
\diagram{
  \draw (-0.5,0) node[circlex] (h) {} -- (0,0) node[ddot=white] (h1) {};
  \draw[-<-] (h1) arc (0:360:-0.25);
}
  \item
\diagram{
  \interaction{2}{g}{(0,0)}{ddot=white}{sawtooth};
  \draw[->-] (g1) to ++(0,+0.5);
  \draw[-<-] (g1) to ++(0,-0.5);
  \draw[-<-] (g2) arc (0:360:-0.25);
}
  \item
\diagram{
  \interaction{2}{g}{(0,0)}{ddot=white}{sawtooth};
  \draw[->-] (g1) arc (0:360:+0.25);
  \draw[-<-] (g2) arc (0:360:-0.25);
}
  \end{enumerate}
  \end{multicols}

\item
  Diagrams for the singles, doubles, and triples operators have the form
\begin{gather*}
\hspace{-2em}
\begin{aligned}
  c_a^i
  \tl{a}^a_i
\equiv
\diagram{
  \draw[overhang] (0,-0.25) node[ddot] (t1) {};
  \draw[->-] (t1) to ++(-0.25,+0.5);
  \draw[-<-] (t1) to ++(+0.25,+0.5);
}
=
\pr{
\diagram{
  \draw[overhang] (0,0.25) node[ddot] (t1) {};
  \draw[-<-] (t1) to ++(-0.25,-0.5);
  \draw[->-] (t1) to ++(+0.25,-0.5);
}}\dg
&&
  \pr{
    \tfr{1}{2}
  }^2
  c_{ab}^{ij}
  \tl{a}_{ij}^{ab}
\equiv
\diagram{
  \interaction{2}{t}{(0,-0.25)}{ddot}{overhang};
  \draw[->-] (t1) to ++(-0.25,+0.5);
  \draw[-<-] (t1) to ++(+0.25,+0.5);
  \draw[->-] (t2) to ++(-0.25,+0.5);
  \draw[-<-] (t2) to ++(+0.25,+0.5);
}
=
\pr{
\diagram{
  \interaction{2}{t}{(0,+0.25)}{ddot}{overhang};
  \draw[-<-] (t1) to ++(-0.25,-0.5);
  \draw[->-] (t1) to ++(+0.25,-0.5);
  \draw[-<-] (t2) to ++(-0.25,-0.5);
  \draw[->-] (t2) to ++(+0.25,-0.5);
}
}\dg
&&
  \pr{
    \tfr{1}{3!}
  }^2
  c_{abc}^{ijk}
  \tl{a}_{ijk}^{abc}
\equiv
\diagram{
  \interaction{3}{t}{(0,-0.25)}{ddot}{overhang};
  \draw[->-] (t1) to ++(-0.25,+0.5);
  \draw[-<-] (t1) to ++(+0.25,+0.5);
  \draw[->-] (t2) to ++(-0.25,+0.5);
  \draw[-<-] (t2) to ++(+0.25,+0.5);
  \draw[->-] (t3) to ++(-0.25,+0.5);
  \draw[-<-] (t3) to ++(+0.25,+0.5);
}
=
\pr{
\diagram{
  \interaction{3}{t}{(0,+0.25)}{ddot}{overhang};
  \draw[-<-] (t1) to ++(-0.25,-0.5);
  \draw[->-] (t1) to ++(+0.25,-0.5);
  \draw[-<-] (t2) to ++(-0.25,-0.5);
  \draw[->-] (t2) to ++(+0.25,-0.5);
  \draw[-<-] (t3) to ++(-0.25,-0.5);
  \draw[->-] (t3) to ++(+0.25,-0.5);
}
}\dg
\end{aligned}
\end{gather*}
  where the coefficients $c_{abc\cd}^{ijk\cd}$ are antisymmetric in their upper and lower indices.\\
  Interpret each of the following as an algebraic expression.
  \begin{multicols}{4}
  \begin{enumerate}
  \item
\diagram{
  \draw[overhang] (0,0) node[ddot] (t1) {};
  \draw[->-] (t1) to ++(-0.25,+0.5);
  \draw[-<-] (t1) to ++(+0.25,+0.5);
  \draw[overhang] (1,0) node[ddot] (t2) {};
  \draw[->-] (t2) to ++(-0.25,+0.5);
  \draw[-<-] (t2) to ++(+0.25,+0.5);
  \draw[overhang] (2,0) node[ddot] (t3) {};
  \draw[->-] (t3) to ++(-0.25,+0.5);
  \draw[-<-] (t3) to ++(+0.25,+0.5);
}
  \item
\diagram{
  \interaction{2}{1t}{(0,0)}{ddot}{overhang};
  \draw[->-] (1t1) to ++(-0.25,+0.5);
  \draw[-<-] (1t1) to ++(+0.25,+0.5);
  \draw[->-] (1t2) to ++(-0.25,+0.5);
  \draw[-<-] (1t2) to ++(+0.25,+0.5);
  \interaction{2}{2t}{(2,0)}{ddot}{overhang};
  \draw[->-] (2t1) to ++(-0.25,+0.5);
  \draw[-<-] (2t1) to ++(+0.25,+0.5);
  \draw[->-] (2t2) to ++(-0.25,+0.5);
  \draw[-<-] (2t2) to ++(+0.25,+0.5);
}
  \item
\diagram{
  \interaction{2}{g}{(0,+0.5)}{ddot}{sawtooth};
  \interaction{2}{t}{(0,-0.5)}{ddot}{overhang};
  \draw[->-,bend left]  (t1) to (g1);
  \draw[-<-,bend right] (t1) to (g1);
  \draw[->-,bend left]  (t2) to (g2);
  \draw[-<-,bend right] (t2) to (g2);
}
  \item
\diagram{
  \draw[overhang] (0,+0.5) node[ddot] (c1*) {};
  \draw[overhang] (0,-0.5) node[ddot] (c1)  {};
  \draw[->-=0.25,->-=0.75,bend left ] (c1) to node[midway,ddot] (f1) {} (c1*);
  \draw[-<-,bend right] (c1) to (c1*);
  \draw (f1) to ++(-0.5,0) node[circlex] {};
}
  \item
\diagram{
  \draw[overhang] (0,+0.5) node[ddot] (c1*) {};
  \draw[overhang] (0,-0.5) node[ddot] (c1)  {};
  \draw[-<-=0.25,-<-=0.75,bend left] (c1) to node[midway,ddot] (f1) {} (c1*);
  \draw[->-,bend right] (c1) to (c1*);
  \draw (f1) to ++(-0.5,0) node[circlex] {};
}
  \item
\diagram{
  \interaction{2}{t}{(0,-0.5)}{ddot}{overhang};
  \draw[->-=0.25,->-=0.75] (t1) to node[midway,ddot] (f1) {}
    ++(-0.25,1) {};
  \draw[-<-] (t1) to ++(+0.25,1) {};
  \draw[->-] (t2) to ++(-0.25,1) {};
  \draw[-<-] (t2) to ++(+0.25,1) {};
  \draw (f1) to ++(-0.5,0) node[circlex] {};
}
  \item
\diagram{
  \interaction{2}{t}{(0,-0.5)}{ddot}{overhang};
  \draw[-<-=0.25,-<-=0.75] (t1) to node[midway,ddot] (f1) {}
    ++(-0.25,1) {};
  \draw[->-] (t1) to ++(+0.25,1) {};
  \draw[-<-] (t2) to ++(-0.25,1) {};
  \draw[->-] (t2) to ++(+0.25,1) {};
  \draw (f1) to ++(-0.5,0) node[circlex] {};
}
  \item
\diagram{
  \interaction{2}{t}{(0,-0.5)}{ddot}{overhang};
  \draw[->-=0.25,->-=0.75] (t1) to node[midway,ddot] (g1) {}
    ++(-0.25,1) {};
  \draw[-<-=0.7] (t1) to ++(+0.25,1) {};
  \draw[->-=0.25,->-=0.75] (t2) to node[midway,ddot] (g2) {}
    ++(-0.25,1) {};
  \draw[-<-=0.7] (t2) to ++(+0.25,1) {};
  \draw[sawtooth] (g1)--(g2);
}
  \item
\diagram{
  \interaction{2}{t}{(0,-0.5)}{ddot}{overhang};
  \draw[-<-=0.25,-<-=0.75] (t1) to node[midway,ddot] (g1) {}
    ++(-0.25,1) {};
  \draw[->-=0.7] (t1) to ++(+0.25,1) {};
  \draw[-<-=0.25,-<-=0.75] (t2) to node[midway,ddot] (g2) {}
    ++(-0.25,1) {};
  \draw[->-=0.7] (t2) to ++(+0.25,1) {};
  \draw[sawtooth] (g1)--(g2);
}
  \item
\diagram{
  \interaction{2}{t}{(0,-0.5)}{ddot}{overhang};
  \interaction{2}{g}{(1,+0.0)}{ddot}{sawtooth};
  \draw[->-] (t1) to ++(-0.25,1) {};
  \draw[-<-] (t1) to ++(+0.25,1) {};
  \draw[->-,bend left] (t2) to (g1);
  \draw[-<-,bend right] (t2) to (g1);
  \draw[->-] (g2) to ++(-0.25,0.5) {};
  \draw[-<-] (g2) to ++(+0.25,0.5) {};
}
  \item
\diagram{
  \interaction{2}{ta}{(0,-0.5)}{ddot}{overhang};
  \interaction{2}{tb}{(2,-0.5)}{ddot}{overhang};
  \interaction{2}{g}{(1,0)}{ddot}{sawtooth};
  \draw[->-=0.7] (ta1) to ++(-0.25,1) {};
  \draw[-<-=0.3] (ta1) to (g1);
  \draw[->-=0.7] (ta2) to ++(-0.25,1) {};
  \draw[-<-=0.3] (ta2) to (g2);
  \draw[->-=0.3] (tb1) to (g1);
  \draw[-<-=0.7] (tb1) to ++(+0.25,1) {};
  \draw[->-=0.3] (tb2) to (g2);
  \draw[-<-=0.7] (tb2) to ++(+0.25,1) {};
}
  \item
\diagram{
  \interaction{2}{ta}{(0,-0.5)}{ddot}{overhang};
  \interaction{2}{tb}{(2,-0.5)}{ddot}{overhang};
  \interaction{2}{g}{(1,0.5)}{ddot}{sawtooth};
  \draw[->-] (ta1) to ++(-0.25,1) {};
  \draw[-<-] (ta1) to ++(+0.25,1) {};
  \draw[->-,bend left]  (ta2) to (g1);
  \draw[-<-,bend right] (ta2) to (g1);
  \draw[->-,bend left]  (tb1) to (g2);
  \draw[-<-,bend right] (tb1) to (g2);
  \draw[->-] (tb2) to ++(-0.25,1) {};
  \draw[-<-] (tb2) to ++(+0.25,1) {};
}
  \item
\diagram{
  \interaction{2}{ta}{(0,-0.5)}{ddot}{overhang};
  \interaction{2}{tb}{(2,-0.5)}{ddot}{overhang};
  \draw[sawtooth] (0,0) node[ddot] (g1) {} to (1.5,0) node[ddot] (g2) {};
  \draw[->-,bend left]  (ta1) to (g1);
  \draw[-<-,bend right] (ta1) to (g1);
  \draw[->-=0.7] (ta2) to ++(-0.25,1) {};
  \draw[-<-] (ta2) to (g2);
  \draw[->-] (tb1) to (g2);
  \draw[-<-=0.7] (tb1) to ++(+0.25,1) {};
  \draw[->-] (tb2) to ++(-0.25,1) {};
  \draw[-<-] (tb2) to ++(+0.25,1) {};
}
  \item
\diagram{
  \interaction{2}{ta}{(0,-0.5)}{ddot}{overhang};
  \interaction{2}{tb}{(2,-0.5)}{ddot}{overhang};
  \draw[sawtooth] (0,0) node[ddot] (g1) {} to (1.5,0) node[ddot] (g2) {};
  \draw[->-,bend left]  (ta1) to (g1);
  \draw[-<-,bend right] (ta1) to (g1);
  \draw[-<-=0.7] (ta2) to ++(-0.25,1) {};
  \draw[->-] (ta2) to (g2);
  \draw[-<-] (tb1) to (g2);
  \draw[->-=0.7] (tb1) to ++(+0.25,1) {};
  \draw[-<-] (tb2) to ++(-0.25,1) {};
  \draw[->-] (tb2) to ++(+0.25,1) {};
}
  \item
\diagram{
  \interaction{2}{ta}{(0,-0.5)}{ddot}{overhang};
  \interaction{2}{tb}{(2,-0.5)}{ddot}{overhang};
  \interaction{2}{g}{(0,+0.5)}{ddot}{sawtooth};
  \draw[->-,bend left]  (ta1) to (g1);
  \draw[-<-,bend right] (ta1) to (g1);
  \draw[->-,bend left]  (ta2) to (g2);
  \draw[-<-,bend right] (ta2) to (g2);
  \draw[->-] (tb1) to ++(-0.25,1) {};
  \draw[-<-] (tb1) to ++(+0.25,1) {};
  \draw[->-] (tb2) to ++(-0.25,1) {};
  \draw[-<-] (tb2) to ++(+0.25,1) {};
}
  \item
\diagram{
  \interaction{3}{d}{(0,+0.5)}{ddot}{overhang};
  \interaction{3}{t}{(0,-0.5)}{ddot}{overhang};
  \draw[->-,bend left]  (t1) to (d1);
  \draw[-<-,bend right] (t1) to (d1);
  \draw[->-,bend left]  (t2) to (d2);
  \draw[-<-,bend right] (t2) to (d2);
  \draw[->-,bend left]  (t3) to (d3);
  \draw[-<-,bend right] (t3) to (d3);
}
  \end{enumerate}
  \end{multicols}
\end{enumerate}


\end{document}
