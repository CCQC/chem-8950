%%%%%%%%%%%%%%%%%%%%%%%%%%%%%%%%%%%%%%%%%%%%%%%%%%%%%%%%%%%%%%%%%%%%%%%%%%%%%%%%%%
% This work is licensed under the Creative Commons Attribution 4.0 International %
% License. To view a copy of this license, visit                                 %
% http://creativecommons.org/licenses/by/4.0/.                                   %
%%%%%%%%%%%%%%%%%%%%%%%%%%%%%%%%%%%%%%%%%%%%%%%%%%%%%%%%%%%%%%%%%%%%%%%%%%%%%%%%%%
\documentclass[11pt]{article}
\usepackage[cm]{fullpage}
%%AVC PACKAGES
\usepackage{avcgreek}
\usepackage{avcfonts}
\usepackage{avcmath}
\usepackage{avcthm}
\usepackage{qcmacros}
\usepackage{goldstone}
%%MACROS FOR THIS DOCUMENT
\usepackage[
  margin=1.5cm,
  includefoot,
  footskip=30pt,
  headsep=0.2cm,headheight=1.3cm
]{geometry}
\usepackage{fancyhdr}
\pagestyle{fancy}
\fancyhf{}
\fancyhead[LE,RO]{\textbf{Quiz 6}}
\fancyfoot[CE,CO]{\thepage}
\usepackage{url}

\begin{document}

\begin{enumerate}
\item
Derive the recursive equation for the wavefunction, starting from the $\la$-dependent Schr\"odinger equation.
\begin{align}
\label{eq:recursive-wavefunction-equation}
  \Y(\la)
=
  \F
+
  R_0
  (
    \la V_\mr{c}
  -
    E(\la)
  )
  \Y(\la)
\end{align}
Assume intermediate normalization and note that $R_0H_0=-Q$ follows** from the definition of $R_0$.\\[10pt]
\textbf{Answer:}
Operating $R_0$ on both sides of $0=\pr{H(\la)-E(\la)}\Y(\la)=\pr{H_0 + \la V_\mr{c}-E(\la)}\Y(\la)$ gives
\begin{align*}
  0
=&\
  R_0
  H_0
  \Y(\la)
+
  R_0
  (
    \la V_\mr{c}
  -
    E(\la)
  )
  \Y(\la)
\\=&\
-
  Q
  \Y(\la)
+
  R_0
  (
    \la V_\mr{c}
  -
    E(\la)
  )
  \Y(\la)
\\=&\
-
  \Y(\la)
+
  \F
+
  R_0
  (
    \la V_\mr{c}
  -
    E(\la)
  )
  \Y(\la)
\end{align*}
where the last step follows from
$
  Q
\,{=}\,
  1_n
\,{-}\,
  P
$
and intermediate normalization:
$
  \ip{\F|\Y(\la)}
\,{=}\,
  1
{\implies}
  P\Y(\la)
\,{=}\,
  \F
$.
Adding $\Y(\la)$ to both sides gives equation~\ref{eq:recursive-wavefunction-equation}.\\[10pt]
**\textbf{Extra Credit}: Define ``resolvent'' and explain why this follows from your definition.\\[10pt]
\textbf{Answer}:
The resolvent is the negative inverse of $H_0$ in the orthogonal space,
$
  R_0
=
-
\left.
  H_0^{-1}
\right|_\mr{o}
$,
which implies that
$
  R_0H_0
=
-
\left.
  1
\right|_\mr{o}
$.
Resolution of the identity in the orthogonal space gives $Q$.
\begin{align*}
\left.
  1
\right|_\mr{o}
=
  \sum_{k=1}^n
  (\tfr{1}{k!})^2
  \sum_{\substack{a_1\cd a_k\\i_1\cd i_k}}
  \kt{\F^{a_1\cd a_k}_{i_1\cd i_k}}
  \br{\F^{a_1\cd a_k}_{i_1\cd i_k}}
=
  Q
\end{align*}

\newpage
\item
Determine the first- and second-order components of $\Y$ by differentiating equation~\ref{eq:recursive-wavefunction-equation}.
You do \ul{not} need to fully evaluate and simplify your answer,\footnote{That is, your final answer may contain $R_0$'s and $V_\mr{c}$'s.} but you should eliminate all terms that vanish and explain why each one evaluates to zero.\footnote{\label{fn:pt-energy}You may take $E_\mr{c}\ord{m+1}=\ip{\F|V_\mr{c}|\Y\ord{m}}$ as given.}\\[10pt]
\textbf{Answer}:
The $\la$-dependent first and second derivatives are as follows
\begin{align*}
  \pd{\Y(\la)}{\la}
=&\
  R_0
  \pr{
    V_\mr{c}
  -
    \pd{E(\la)}{\la}
  }
  \Y(\la)
+
  R_0
  (
    \la
    V_\mr{c}
  -
    E(\la)
  )
  \pd{\Y(\la)}{\la}
\\
  \pd{^2\Y(\la)}{\la^2}
=&\
-
  R_0
  \pd{^2E(\la)}{\la^2}
  \Y(\la)
+
  2
  R_0
  \pr{
    V_\mr{c}
  -
    \pd{E(\la)}{\la}
  }
  \pd{\Y(\la)}{\la}
+
  R_0
  (
    \la
    V_\mr{c}
  -
    E(\la)
  )
  \pd{^2\Y(\la)}{\la^2}
\end{align*}
where we have used the product rule:
if $f=gh$, then $f'=g'h+ gh'$ and $f''=g''h + 2g'h' + gh''$.
Evaluating these derivatives at $\la=0$ after dividing the second equation by 2 gives the following
\begin{align*}
  \Y\ord{1}
=&\
  R_0
  V_\mr{c}
  \F
-
  E\ord{1}
  R_0
  \F
-
  R_0
  E\ord{0}
  \Y\ord{1}
\\
  \Y\ord{2}
=&\
-
  E_\mr{c}\ord{2}
  R_0
  \F
+
  R_0
  V_\mr{c}
  \Y\ord{1}
-
  E_\mr{c}\ord{1}
  R_0
  \Y\ord{1}
-
  R_0
  E_\mr{c}\ord{0}
  \Y\ord{2}
\end{align*}
since $\Y\ord{0}$ is the ground-state eigenvector of $H_0$, which is $\F$.
It follows that
$
  E_\mr{c}\ord{0}
=
  \ip{\F|H_0|\F}
=
  0
$.
Furthermore, using the relation in \cref{fn:pt-energy} we have that
$
  E_\mr{c}\ord{1}
=
  \ip{\F|V_\mr{c}|\F}
=
  0
$.
Since $R_0$ is only acts on the orthogonal space we have $R_0=R_0Q$, which implies
$
  R_0
  \F
=
  R_0
  Q
  \F
=
  0
$.
Therefore,
\begin{align*}
  \Y\ord{1}
=&\
  R_0
  V_\mr{c}
  \F
\\
  \Y\ord{2}
=&\
  R_0
  V_\mr{c}
  \Y\ord{1}
\end{align*}
are the non-vanishing contributions to $\Y\ord{1}$ and $\Y\ord{2}$.

\newpage
\item
Evaluate the following contributions to the CI doubles and quadruples coefficients.
\begin{align}
  {}\ord{1}
  c_{ab}^{ij}
=
  \ip{\F_{ij}^{ab}|
    R_0V_\mr{c}
  |\F}
&&
  {}\ord{2}
  c_{abcd}^{ijkl}
=
  \ip{\F_{ijkl}^{abcd}|
    R_0V_\mr{c}R_0V_\mr{c}
  |\F}
\end{align}
Use your answer to show that ${}\ord{2}C_4=\tfr{1}{2}{}\ord{1}C_2^2$.\\[10pt]
\textbf{Answer}:
Only the $+2$ fluctuation potential contributions can fully contract the products, so there is only one unique graph contraction for each one.
Note that the off-diagonal Fock operator has no excitation level $+2$ component, so the results are the same whether or not we assume Brillouin's theorem.
\begin{align*}
  {}\ord{1}
  c_{ab}^{ij}
=&\
\diagram[top,bottom]{
  \interaction{2}{a}{(0,+0.6)}{ddot}{dotted};
  \interaction{2}{g}{(0,-0.6)}{ddot=white}{sawtooth};
  \draw[-<-] (a1) to ++(-0.25,-0.5);
  \draw[->-] (a1) to ++(+0.25,-0.5);
  \draw[-<-] (a2) to ++(-0.25,-0.5);
  \draw[->-] (a2) to ++(+0.25,-0.5);
  \draw[->-] (g1) to ++(0,+0.5);
  \draw[-<-] (g1) to ++(0,-0.5);
  \draw[->-] (g2) to ++(0,+0.5);
  \draw[-<-] (g2) to ++(0,-0.5);
  \draw[densely dotted] (-0.3,0) to ++(1.6,0);
}
=
\diagram{
  \interaction{2}{g}{(0,-0.5)}{ddot}{sawtooth};
  \draw[->-] (g1) to ++(-0.25,1) node[smalldot] {};
  \draw[-<-] (g1) to ++(+0.25,1) node[smalldot] {};
  \draw[->-] (g2) to ++(-0.25,1) node[smalldot] {};
  \draw[-<-] (g2) to ++(+0.25,1) node[smalldot] {};
  \draw[densely dotted] (-0.3,+0.2) to ++(1.6,0);
}
=
  \fr{\ol{g}_{ab}^{ij}}{\mc{E}_{ab}^{ij}}
\\[10pt]
  {}\ord{2}
  c_{abcd}^{ijkl}
=&\
\diagram[top,bottom]{
  \interaction{4}{a}{(0,+1.2)}{ddot}{dotted};
  \interaction{2}{1g}{(1,0)}{ddot=white}{sawtooth};
  \interaction{2}{2g}{(1,-1.2)}{ddot=white}{sawtooth};
  \draw[-<-] (a1) to ++(-0.25,-0.5);
  \draw[->-] (a1) to ++(+0.25,-0.5);
  \draw[-<-] (a2) to ++(-0.25,-0.5);
  \draw[->-] (a2) to ++(+0.25,-0.5);
  \draw[-<-] (a3) to ++(-0.25,-0.5);
  \draw[->-] (a3) to ++(+0.25,-0.5);
  \draw[-<-] (a4) to ++(-0.25,-0.5);
  \draw[->-] (a4) to ++(+0.25,-0.5);
  \draw[->-] (1g1) to ++(0,+0.5);
  \draw[-<-] (1g1) to ++(0,-0.5);
  \draw[->-] (1g2) to ++(0,+0.5);
  \draw[-<-] (1g2) to ++(0,-0.5);
  \draw[->-] (2g1) to ++(0,+0.5);
  \draw[-<-] (2g1) to ++(0,-0.5);
  \draw[->-] (2g2) to ++(0,+0.5);
  \draw[-<-] (2g2) to ++(0,-0.5);
  \draw[densely dotted] (-0.3,+0.6) to ++(3.6,0);
  \draw[densely dotted] (-0.3,-0.6) to ++(3.6,0);
}
=
\diagram{
  \interaction{2}{1g}{(0,-0.5)}{ddot}{sawtooth};
  \interaction{2}{2g}{(2,+0.0)}{ddot}{sawtooth};
  \draw[->-] (1g1) to ++(-0.25,1) node[smalldot] {};
  \draw[-<-] (1g1) to ++(+0.25,1) node[smalldot] {};
  \draw[->-] (1g2) to ++(-0.25,1) node[smalldot] {};
  \draw[-<-] (1g2) to ++(+0.25,1) node[smalldot] {};
  \draw[->-] (2g1) to ++(-0.25,0.5) node[smalldot] {};
  \draw[-<-] (2g1) to ++(+0.25,0.5) node[smalldot] {};
  \draw[->-] (2g2) to ++(-0.25,0.5) node[smalldot] {};
  \draw[-<-] (2g2) to ++(+0.25,0.5) node[smalldot] {};
  \draw[densely dotted] (-0.4,+0.4) to ++(3.8,0);
  \draw[densely dotted] (-0.3,-0.2) to ++(1.6,0);
}
=
  P_{(ab/cd)}^{(ij/kl)}
  \fr{\ol{g}_{ab}^{ij}\,\ol{g}_{cd}^{kl}}{\mc{E}_{abcd}^{ijkl}\mc{E}_{ab}^{ij}}
\end{align*}
The corresponding operators are
\begin{align*}
  {}\ord{1}C_2
=&\
\diagram{
  \interaction{2}{g}{(0,-0.5)}{ddot}{sawtooth};
  \draw[->-] (g1) to ++(-0.25,1);
  \draw[-<-] (g1) to ++(+0.25,1);
  \draw[->-] (g2) to ++(-0.25,1);
  \draw[-<-] (g2) to ++(+0.25,1);
  \draw[densely dotted] (-0.3,+0.2) to ++(1.6,0);
}
=
  \fr{1}{2^2}
  \sum_{\substack{ab\\ij}}
  \fr{\ol{g}_{ab}^{ij}}{\mc{E}_{ab}^{ij}}
  \tl{a}^{ab}_{ij}
\\[10pt]
  {}\ord{2}
  C_4
=&\
=
\diagram{
  \interaction{2}{1g}{(0,-0.5)}{ddot}{sawtooth};
  \interaction{2}{2g}{(2,+0.0)}{ddot}{sawtooth};
  \draw[->-] (1g1) to ++(-0.25,1);
  \draw[-<-] (1g1) to ++(+0.25,1);
  \draw[->-] (1g2) to ++(-0.25,1);
  \draw[-<-] (1g2) to ++(+0.25,1);
  \draw[->-] (2g1) to ++(-0.25,0.5);
  \draw[-<-] (2g1) to ++(+0.25,0.5);
  \draw[->-] (2g2) to ++(-0.25,0.5);
  \draw[-<-] (2g2) to ++(+0.25,0.5);
  \draw[densely dotted] (-0.4,+0.4) to ++(3.8,0);
  \draw[densely dotted] (-0.3,-0.2) to ++(1.6,0);
}
=
  \fr{1}{2^4}
  \sum_{\substack{abcd\\ijkl}}
  \fr{\ol{g}_{ab}^{ij}\,\ol{g}_{cd}^{kl}}{\mc{E}_{abcd}^{ijkl}\mc{E}_{ab}^{ij}}
  \tl{a}^{abcd}_{ijkl}
\end{align*}
and by duplicating and reindexing the quadruples operator we get the following.
\begin{align*}
  {}\ord{2}
  C_4
=
  \fr{1}{2^4}
  \sum_{\substack{abcd\\ijkl}}
  \fr{\ol{g}_{ab}^{ij}\,\ol{g}_{cd}^{kl}}{\mc{E}_{abcd}^{ijkl}\mc{E}_{ab}^{ij}}
  \tl{a}^{abcd}_{ijkl}
=&\
\fr{1}{2}
\pr{
  \fr{1}{2^4}
  \sum_{\substack{abcd\\ijkl}}
  \fr{\ol{g}_{ab}^{ij}\,\ol{g}_{cd}^{kl}}{\mc{E}_{abcd}^{ijkl}\mc{E}_{ab}^{ij}}
  \tl{a}^{abcd}_{ijkl}
+
  \fr{1}{2^4}
  \sum_{\substack{cdab\\klij}}
  \fr{\ol{g}_{cd}^{kl}\,\ol{g}_{ab}^{ij}}{\mc{E}_{cdab}^{klij}\mc{E}_{cd}^{kl}}
  \tl{a}^{cdab}_{klij}
}
\\=&\
  \fr{1}{2}
  \cdot
  \fr{1}{2^4}
  \sum_{\substack{abcd\\ijkl}}
  \fr{
    \ol{g}_{ab}^{ij}\,
    \ol{g}_{cd}^{kl}
    (\cancel{
      \mc{E}_{cd}^{kl}
    +
      \mc{E}_{ab}^{ij}
    })
  }{
    \cancel{\mc{E}_{abcd}^{ijkl}}
    \mc{E}_{ab}^{ij}
    \mc{E}_{cd}^{kl}
  }
  \tl{a}^{abcd}_{ijkl}
\\=&\
\fr{1}{2}
\pr{
  \fr{1}{2^4}
  \sum_{\substack{ab\\ij}}
  \fr{
    \ol{g}_{ab}^{ij}
  }{
    \mc{E}_{ab}^{ij}
  }
  \tl{a}^{ab}_{ij}
}
\pr{
  \fr{1}{2^4}
  \sum_{\substack{cd\\kl}}
  \fr{
    \ol{g}_{cd}^{kl}
  }{
    \mc{E}_{cd}^{kl}
  }
  \tl{a}^{cd}_{kl}
}\\=&\
  \fr{1}{2}
  \,{}\ord{1}C_2^2
\end{align*}

\end{enumerate}

\end{document}
