\documentclass[11pt]{article}
\usepackage[cm]{fullpage}
%%AVC PACKAGES
\usepackage{avcgreek}
\usepackage{avcfonts}
\usepackage{avcmath}
\usepackage{avcthm}
\usepackage{qcmacros}
\usepackage{goldstone}
%%MACROS FOR THIS DOCUMENT
\usepackage[
  margin=1.5cm,
  includefoot,
  footskip=30pt,
  headsep=0.2cm,headheight=1.3cm
]{geometry}
\usepackage{fancyhdr}
\pagestyle{fancy}
\fancyhf{}
\fancyhead[LE,RO]{\textbf{Quiz 5}}
\fancyfoot[CE,CO]{\thepage}
\usepackage{url}

\begin{document}

\begin{enumerate}
\item
  Explain why each of the following terms vanishes.
\begin{align*}
  \text{(a)}\hspace{3pt}
  \tfr{1}{5!}
  \ip{\F_{ijklm}^{abcde}|V_\mr{c}T_1^5|\F}_\mr{C}
&&
  \text{(b)}\hspace{3pt}
  \ip{\F_{ij}^{ab}|V_\mr{c}T_2T_3|\F}_\mr{C}
&&
  \text{(c)}\hspace{3pt}
  \tfr{1}{2!}
  \ip{\F_{ijkl}^{abcd}|V_\mr{c}T_1^2|\F}_\mr{C}
&&
  \text{(d)}\hspace{3pt}
  \tfr{1}{2!}
  \ip{\F_{ijk}^{abc}|V_\mr{c}T_1^2|\F}_\mr{C}
\end{align*}

\newpage
\item
Interpret the following graph and fully simplify your answer.
\begin{align*}
\diagram{
  \interaction{3}{ta}{(0,-0.5)}{ddot}{overhang};
  \interaction{2}{tb}{(3,-0.5)}{ddot}{overhang};
  \draw[sawtooth] (1,0) node[ddot] (g1) {} to (2.5,0) node[ddot] (g2) {};
  \draw[->-] (ta1) to ++(-0.25,1) node[smalldot] {};
  \draw[-<-] (ta1) to ++(+0.25,1) node[smalldot] {};
  \draw[->-,bend left]  (ta2) to (g1);
  \draw[-<-,bend right] (ta2) to (g1);
  \draw[->-=0.7] (ta3) to ++(-0.25,1) node[smalldot] {};
  \draw[-<-] (ta3) to (g2);
  \draw[->-] (tb1) to (g2);
  \draw[-<-=0.7] (tb1) to ++(+0.25,1) node[smalldot] {};
  \draw[->-] (tb2) to ++(-0.25,1) node[smalldot] {};
  \draw[-<-] (tb2) to ++(+0.25,1) node[smalldot] {};
}
\end{align*}

\newpage
\item
Interpret the following graph and fully simplify it the ``long way.''  That is, you may use Rules 1-3 but you must start from Axiom 1 and show each step to get to your final answer.
\begin{align*}
\diagram{
  \draw[->-=0.25,->-=0.75]
    (0,-0.5)
      node[smalldot] {}
    to
      node[midway,ddot] (g1) {}
    ++(0,+1)
      node[smalldot] {};
  \draw[-<-=0.65]
    (0.5,-0.5)
      node[smalldot] {}
    to
    ++(0,+1)
      node[smalldot] {};
  \draw[->-=0.25,->-=0.75]
    (1,-0.5)
      node[smalldot] {}
    to
      node[midway,ddot] (g2) {}
    ++(0,+1)
      node[smalldot] {};
  \draw[-<-=0.65]
    (1.5,-0.5)
      node[smalldot] {}
    to
    ++(0,+1)
      node[smalldot] {};
  \draw[sawtooth] (g1) to (g2);
  \draw[->-]
    (2.0,-0.5)
      node[smalldot] {}
    to
    ++(0,+1)
      node[smalldot] {};
  \draw[-<-=0.65]
    (2.5,-0.5)
      node[smalldot] {}
    to
    ++(0,+1)
      node[smalldot] {};
}
\end{align*}

\end{enumerate}

\newpage
\noindent
\textbf{Extra Credit.}
Prove Rule~3 for a closed graph with a single bare excitation operator of the following form.
\begin{align*}
  \tl{a}^{i_1\cd i_m}_{a_1\cd a_m}
=
  (\tfr{1}{m!})^2\,
  \ol{\d}_{j_1\cd j_m}^{b_1\cd b_m}\,
  \tl{a}^{j_1\cd j_m}_{b_1\cd b_m}
&&
  \ol{\d}_{j_1\cd j_m}^{b_1\cd b_m}
\equiv
  \op{P}_{(a_1/\cd/a_m)}^{(i_1/\cd/i_m)}
  \d_{j_1}^{i_1}
  \cd
  \d_{j_m}^{i_m}
  \d_{a_1}^{b_1}
  \cd
  \d_{a_m}^{b_m}
\end{align*}

\vfill
\noindent
\hrulefill

\noindent
\textbf{Appendix.}
\vspace{10pt}

{\small

\noindent
\bmit{Axiom 1.}
The algebraic of a graph $G$ is obtained from a corresponding summand graph $\Si(G)$ as follows.
\begin{align*}
  G
=
  \fr{1}{\mr{dg}(G)}
  \sum_{\mr{labels}}\Si(G)
\end{align*}
\noindent
\bmit{Rule~1.}
  Each set of $k$ equivalent lines or equivalent subgraphs contributes a factor of $k!$ to the degeneracy.

\noindent
\bmit{Rule~2.}
  The overall sign of a closed graph is $(-)^{h+l}$, where $h$ and $l$ denote the total number of hole lines and loops.

\noindent
\bmit{Rule~3.}
  For bare excitation operators, cancel the degeneracy factors from their equivalent coefficient lines by replacing the full antisymmetrizer, $P^{(p_1/\cd/p_m)}_{(q_1/\cd/q_m)}$, with a reduced antisymmetrizer over inequivalent coefficient lines, $\op{P}^{(P_1/\cd /P_h)}_{(Q_1/\cd /Q_k)}$.\footnote{
  Here $\{p_1,\ld,p_m\}=P_1\cup\cd\cup P_h$ and $\{q_1,\ld,q_m\}=Q_1\cup\cd\cup Q_k$ are the upper and lower indices on the bare excitation operator $\tl{a}^{p_1\cd p_m}_{q_1\cd q_m}$, and the $P_i$'s and $Q_i$'s label subsets of equivalent coefficient lines.
}~\footnote{
  For equivalent lines connecting two bare excitation operators, this cancellation can only be performed once.
}
}



\end{document}