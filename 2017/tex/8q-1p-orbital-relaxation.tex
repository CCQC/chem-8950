%%%%%%%%%%%%%%%%%%%%%%%%%%%%%%%%%%%%%%%%%%%%%%%%%%%%%%%%%%%%%%%%%%%%%%%%%%%%%%%%%%
% This work is licensed under the Creative Commons Attribution 4.0 International %
% License. To view a copy of this license, visit                                 %
% http://creativecommons.org/licenses/by/4.0/.                                   %
%%%%%%%%%%%%%%%%%%%%%%%%%%%%%%%%%%%%%%%%%%%%%%%%%%%%%%%%%%%%%%%%%%%%%%%%%%%%%%%%%%
\documentclass[11pt]{article}
\usepackage[cm]{fullpage}
%%AVC PACKAGES
\usepackage{avcgreek}
\usepackage{avcfonts}
\usepackage{avcmath}
\usepackage[numberby=section]{avcthm}
\usepackage{qcmacros}
\usepackage{goldstone}
%%MACROS FOR THIS DOCUMENT
\usepackage[
  margin=1.5cm,
  includefoot,
  footskip=30pt,
  headsep=0.2cm,headheight=1.3cm
]{geometry}
\usepackage{fancyhdr}
\pagestyle{fancy}
\fancyhf{}
\fancyhead[LE,RO]{Quiz 8, Suggested Problems 1: Orbital relaxation}
\fancyfoot[CE,CO]{\thepage}
\usepackage{url}
\usepackage{multicol}

\begin{document}

\begin{enumerate}
\item
Prove the Thouless theorem.

\item
Assuming the spectral theorem for normal matrices, prove that every unitary matrix can be written as
$
  \bo{U}
=
  \mr{exp}(\bo{X} - \bo{X}\dg)
$
for some square matrix $\bo{X}$.
If the dimension of these matrices is $m$, explain why $m(m-1)/2$ of the elements in $\bo{X}$ are redundant for this parametrization.
Explain why $m$ additional elements are redundant when we require $\bo{X}$ to be real.


\item
Prove the following.
\begin{align}
  \mr{exp}(G)\,a_p\dg\,\mr{exp}(-G)
=
  \sum_q
  a_q\dg\,
  (\mr{exp}(\bo{G}))_{qp}
&&
  G
=
  \sum_{pq}
  (\bo{G})_{pq}\,
  a_p\dg a_q
\end{align}



\item
Show that the creation and annihilation operators associated with a set of spin-orbitals $\{\y_p'\}$ transformed from the original basis by $\bo{U}=\mr{exp}(\bo{X} - \bo{X}\dg)$ can be written as follows.
\begin{align}
\begin{array}{r@{\ }l}
  a_p^{\prime\,\dagger}
&=
  \mr{exp}(X - X\dg)
  a_p\dg\,
  \mr{exp}(X\dg - X)
\\
  a_p^{\prime}
&=
  \mr{exp}(X - X\dg)
  a_p\,
  \mr{exp}(X\dg - X)
\end{array}
&&
  X
\equiv
  \sum_{pq}
  (\bo{X})_{pq}\,
  a_p\dg a_q
\end{align}


\item
Prove the following.
\begin{align}
  \kt{\F'_{(p_1\cd p_n)}}
=
  \mr{exp}(X - X\dg)
  \kt{\F_{(p_1\cd p_n)}}
&&
\begin{array}{r@{\ }l}
  \kt{\F'_{(p_1\cd p_n)}}
&=
  a_{p_1}^{\prime \dagger}
\cd
  a_{p_n}^{\prime \dagger}
  \kt{\vac}
\\[5pt]
  \kt{\F_{(p_1\cd p_n)}}
&=
  a_{p_1}^{ \dagger}
\cd
  a_{p_n}^{ \dagger}
  \kt{\vac}
\end{array}
\end{align}

\item
Derive the following conditions for Brueckner/optimized orbitals
\begin{align}
\label{eq:brueckner-and-oo-explicit-condition}
  \{\y_p\}_\mr{B}\,:\,\,
  \ip{\F_i^a|\Y}
\overset{!}{=}
  0
&&
  \{\y_p\}_\mr{O}\,:\,\,
  \ip{\Y|[a_a^i, H]|\Y}
\overset{!}{=}
  0
\end{align}
from the best-overlap/best-energy criteria.  Explain why the Brueckner condition is equivalent to requiring that singles coefficients vanish, for both the coupled-cluster and configuration interaction Ans\"atze.


\item
Write down an algorithm for computing BCC orbitals.  In particular, explain how to rotate the orbitals each iteration in order to satisfy the Brueckner condition.


\item
Write down an algorithm for computing OCC orbitals.    In particular, explain how to rotate the orbitals each iteration in order to satisfy the optimization condition.

\item
Derive an explicit formula for the orbital Newton-Raphson step, using a zeroth-order approximation for the orbital Hessian.

\item
Derive density matrices for the CEPA$_0$ approximation.
Give both the diagrams and algebraic expressions.


\end{enumerate}


\end{document}
