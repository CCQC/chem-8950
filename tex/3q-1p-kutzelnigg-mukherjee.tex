\documentclass[11pt]{article}
\usepackage[cm]{fullpage}
%%AVC PACKAGES
\usepackage{avcgreek}
\usepackage{avcfonts}
\usepackage{avcmath}
\usepackage[numberby=section]{avcthm} % 
\usepackage{qcmacros}
\usepackage{goldstone}
%%MACROS FOR THIS DOCUMENT
\numberwithin{equation}{section}
\usepackage[
  margin=1.5cm,
  includefoot,
  footskip=30pt,
  headsep=0.2cm,headheight=1.3cm
]{geometry}
\usepackage{fancyhdr}
\pagestyle{fancy}
\fancyhf{}
\fancyhead[LE,RO]{Quiz 3, Suggested Problems 1: Kutzelnigg-Mukherjee}
\fancyfoot[CE,CO]{\thepage}
\usepackage{url}

\begin{document}

\begin{enumerate}
\item
  Derive the expansion of $H_e$ in terms of $\F$-normal excitations using KM notation with index antisymmetrizers.

\item
  Evaluate the following expectation values by Wick's theorem, using KM notation.
\begin{align*}
  \ip{\F|a^p_q|\F}
=&\
  \,\,?
&
  \ip{\F|a^p_qa^a_i|\F}
=&\
  \,\,?
&
  \ip{\F|a^p_qa^{ab}_{ij}|\F}
=&\
  \,\,?
&
  \ip{\F|a^p_qa^{abc}_{ijk}|\F}
=&\
  \,\,?
\\
  \ip{\F|a^{pq}_{rs}|\F}
=&\
  \,\,?
&
  \ip{\F|a^{pq}_{rs}a^a_i|\F}
=&\
  \,\,?
&
  \ip{\F|a^{pq}_{rs}a^{ab}_{ij}|\F}
=&\
  \,\,?
&
  \ip{\F|a^{pq}_{rs}a^{abc}_{ijk}|\F}
=&\
  \,\,?
\end{align*}
  Substitute your results into $\ip{\F|H_e|\F}$, $\ip{\F|H_e|\F_i^a}$, $\ip{\F|H_e|\F_{ij}^{ab}}$, and $\ip{\F|H_e|\F_{ijk}^{abc}}$ to get the Slater rules.

\item
  Show that
\begin{align*}
  \tfr{1}{m!}
  v_{p_1\cd p_m}^{q_1\cd q_m}
  a^{p_1\cd p_m}_{q_1\cd q_m}
=
  (\tfr{1}{m!})^2\,
  \ol{v}_{p_1\cd p_m}^{q_1\cd q_m}
  a^{p_1\cd p_m}_{q_1\cd q_m}
\end{align*}
  which proves that any $m$-electron operator can be represented with an antisymmetrized interaction tensor.
  This generalizes the expression for electron repulsion in terms of antisymmetrized two-electron integrals.

\item
  Prove the following identities.
\begin{align*}
  \tl{a}^{p_1\cd p_m}_{q_1\cd q_m}
=
  \gno{a^{p_1}_{q_1}\cd a^{p_m}_{q_m}}
&&
  \gno{a^{p_1\cd p_m}_{q_1\cd q_m}a^{r_1\cd r_n}_{s_1\cd s_n}}
=
  \gno{a^{p_1}_{q_1}\cd a^{p_m}_{q_m}a^{r_1}_{s_1}\cd a^{r_n}_{s_n}}
=
  \tl{a}^{p_1\cd p_mr_1\cd r_n}_{q_1\cd q_ms_1\cd s_n}
\end{align*}
  Furthermore, explain how the presence of a contraction line restricts which rearrangements are possible, and how this is remedied by the use of dot notation.

\item
  Show algebraically that $\op{P}_{(p/q/r)}=\op{P}_{(p/qr)}\op{P}_{(q/r)}=\op{P}_{(pq/r)}\op{P}_{(p/q)}$ and explain why these identities follow from the definition of the index antisymmetrizers.

\item
  Show that the following identity holds for any four-index tensor $t^{pq}_{rs}$, whether antisymmetric or not.
  \begin{align*}
    \ol{g}_{pq}^{rs}\,
    \op{P}^{(p/q)}_{(r/s)}\,
    t^{pq}_{rs}
  =
    4\,
    \ol{g}_{pq}^{rs}\,
    t^{pq}_{rs}
  \end{align*}

\item
  Show that the following identity holds for any $w^{pqr}$ if $v_{pqr}$ is antisymmetric.
  \begin{align*}
    v_{pqr}
    \op{P}^{(p/qr)}
    w^{pqr}
  =
    3\,
    v_{pqr}\,
    w^{pqr}
  \end{align*}

\item
  Derive the Wick expansion of $a^{pqr}_{stu}$ in terms of $\F$-normal excitations, using index antisymmetrizers to generate the full expansion from the unique contraction patterns.

\item
  Derive the CIS equations in KM notation.

\item
  Derive the CID equations in KM notation.
\end{enumerate}


\end{document}