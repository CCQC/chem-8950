\documentclass[11pt]{article}
\usepackage[cm]{fullpage}
%%AVC PACKAGES
\usepackage{avcgreek}
\usepackage{avcfonts}
\usepackage{avcmath}
\usepackage[numberby=section]{avcthm}
\usepackage{qcmacros}
\usepackage{goldstone}
%%MACROS FOR THIS DOCUMENT
\numberwithin{equation}{section}
\usepackage[
  margin=1.5cm,
  includefoot,
  footskip=30pt,
  headsep=0.2cm,headheight=1.3cm
]{geometry}
\usepackage{fancyhdr}
\pagestyle{fancy}
\fancyhf{}
\fancyhead[LE,RO]{\textbf{Quiz 1}}
\fancyfoot[CE,CO]{\thepage}
\usepackage{url}

\begin{document}

\begin{enumerate}
\item
  Answer each of the following in one sentence, using words only.
  \begin{enumerate}
  \item
    Define canonical Hartree-Fock orbitals.\\
    \textbf{Answer:}
    Hartree-Fock orbitals are ``canonical'' when the Lagrange multiplier matrix is diagonal.
  \item
    Explain why the choice of Hartree-Fock orbitals is not unique.\\
    \textbf{Answer:}
    The Hartree-Fock energy and the orbital overlaps are invariant to a unitary transformation, so any unitary variation of the canonical orbitals satisfies the Hartree-Fock optimization conditions.
  \end{enumerate}

\item
  Expand $a_pa_qa_s\dg a_r\dg$ as a linear combination of strings which are in normal order.
  Identify the vacuum expectation value of this operator product.\\
  \textbf{Answer:}
  \begin{align*}
    a_p a_q a_s\dg a_r\dg
  =&\
    \d_{qs} a_p a_r\dg
  -
    a_p a_s\dg a_q a_r\dg
  \\=&\
    \d_{qs} (\d_{pr} - a_r\dg a_p)
  -
    (\d_{ps} - a_s\dg a_p)(\d_{qr} - a_r\dg a_q)
  \\=&\
    \d_{qs} \d_{pr}
  -
    \d_{qs} a_r\dg a_p
  -
    \d_{ps}\d_{qr}
  +
    \d_{ps} a_r\dg a_q
  +
    \d_{qr} a_s\dg a_p
  -
    a_s\dg a_p a_r\dg a_q
  \\=&\
    \d_{qs} \d_{pr}
  -
    \d_{qs} a_r\dg a_p
  -
    \d_{ps}\d_{qr}
  +
    \d_{ps} a_r\dg a_q
  +
    \d_{qr} a_s\dg a_p
  -
    \d_{pr} a_s\dg a_q
  +
    a_s\dg a_r\dg a_p a_q
  \\
    \ip{\vac|a_p a_q a_s\dg a_r\dg|\vac}
  =&\
    \d_{qs} \d_{pr}
  -
    \d_{ps}\d_{qr}
  \end{align*}

\item
  Derive the Slater determinant expectation value of a two-electron operator in terms of two-electron integrals, showing your steps along the way.
  You may use second quantization methods (and your result from problem 2.) if you first expand the expectation value in terms of particle-hole operators.\footnote{
  You may take the following expansion as given:
  \begin{align*}
    \Y(1,2,3\ld,n)
  =
    \fr{1}{\sqrt{n(n-1)}}
    \sum_{pq}^\infty
    \y_p(1)
    \y_q(2)
    (\op{a}_q\op{a}_p\Y)(3,\ld,n)
  \end{align*}
  }
  \begin{align*}
    \tfr{1}{2}
    \sum_{i\neq j}^n
    \ip{\F|\op{g}(i,j)\F}
  =
    \,\,?
  \end{align*}
  \textbf{Answer:}
  \begin{align*}
    \fr{1}{2}
    \sum_{i\neq j}^n
    \ip{\F|\op{g}(i,j)\F}
  =&\
    \fr{n^2 - n}{2}
    \ip{\F|\op{g}(1,2)\F}
  &&
    \begin{array}{l}
      \text{Changing integration variables}\\
      \text{and using antisymmetry of $\F$}
    \end{array}
  \\=&\
    \fr{\cancel{n^2 - n}}{2}
    \fr{1}{\cancel{n(n-1)}}
    \sum_{pqrs}^\infty
    \ip{pq|rs}
    \ip{a_qa_p\F|a_sa_r\F}
  &&
    \begin{array}{l}
      \text{Using expansion from footnote}\\
      \text{in bra and ket}
    \end{array}
  \\=&\
    \fr{1}{2}
    \sum_{pqrs}^\infty
    \ip{pq|rs}
    \ip{\F|a_p\dg a_q\dg a_sa_r\F}
  &&
    \begin{array}{l}
      \text{Definition of adjoint}
    \end{array}
  \\=&\
    \fr{1}{2}
    \sum_{ijkl}^n
    \ip{ij|kl}
    \ip{\widetilde{\vac}|b_ib_jb_l\dg b_k\dg|\widetilde{\vac}}
  &&
    \begin{array}{l}
      \text{Particle-hole isomorphism}
    \end{array}
  \\=&\
    \fr{1}{2}
    \sum_{ijkl}^n
    \ip{ij|kl}
    \pr{
      \d_{ik}\d_{jl}
    -
      \d_{il}\d_{jk}
    }
  &&
    \begin{array}{l}
      \text{Result from problem 2.}
    \end{array}
  \\=&\
    \fr{1}{2}
    \sum_{ij}^n
    \ip{ij||ij}
  \end{align*}
  On the fourth line, we omit all other contributions to the quasiparticle expansion because either they don't have a balanced number of quasiparticle creation and annihilation operators or because they are in $\F$-normal order.
\end{enumerate}

\end{document}
