\documentclass[11pt]{article}
\usepackage[cm]{fullpage}
%%AVC PACKAGES
\usepackage{avcgreek}
\usepackage{avcfonts}
\usepackage{avcmath}
\usepackage[numberby=section]{avcthm}
\usepackage{qcmacros}
\usepackage{goldstone}
%%MACROS FOR THIS DOCUMENT
\numberwithin{equation}{section}
\usepackage[
  margin=1.5cm,
  includefoot,
  footskip=30pt,
  headsep=0.2cm,headheight=1.3cm
]{geometry}
\usepackage{fancyhdr}
\pagestyle{fancy}
\fancyhf{}
\fancyhead[LE,RO]{Quiz 7, Handout 1: Perturbative analysis}
\fancyfoot[CE,CO]{\thepage}
\usepackage{url}
\makeatother
\newcommand{\resolventline}[2][1]{
  \tikz[overlay]{
      \draw[thick,flexdotted] (0,-1ex) to ++(0,#1*4.5ex) node[above,inner sep=1pt] {#2};
  }
}

\newcommand{\fwkt}[1]{\kt{\makebox[0.8em][c]{\ensuremath{#1}}}}
\newcommand{\fwbr}[1]{\br{\makebox[0.8em][c]{\ensuremath{#1}}}}
\newcommand{\bord}[1]{\ensuremath{^{\scriptscriptstyle[#1]}}}

\begin{document}

\setlength{\abovedisplayskip}{3pt}
\setlength{\belowdisplayskip}{3pt}



\setcounter{section}{6}
\section{Perturbative analysis}

\begin{rmk}
\thmtitle{Perturbative analysis of the configuration interaction (CI) equations}
The CI eigenvalue equation can be expressed as
$
  \bo{H}_\mr{c}\,
  \bo{c}
=
  E_\mr{c}\,
  \bo{c}
$
in terms of the reference-shifted Hamiltonian matrix
$
  (\bo{H}_\mr{c})_{\si\ta}
\equiv
  \ip{\F_\si|H_\mr{c}|\F_\ta}
$.\footnote{By ``reference-shifted Hamiltonian matrix'' we mean that 
$
  \bo{H}_\mr{c}
$
equals
$
  \bo{H}_e - E_\mr{ref}\bo{1}
$,
which has elements
$
  \ip{\F_\si|H_e|\F_\ta}
-
  E_\mr{ref}
  \d_{\si\ta}
$.
}
Separating this matrix into model-Hamiltonian and fluctuation-potential contributions gives an equivalent matrix equation
\begin{align}
  (
  -
    \bo{H}_0
  +
    E_\mr{c}
    \bo{1}
  )\,
  \bo{c}
=
  \bo{V}_\mr{c}\,
  \bo{c}
&&
\begin{array}{l@{\ }l}
  (\bo{H}_0)_{\si\ta}
&\equiv
  \ip{\F_\si|H_0|\F_\ta}
\\
  (\bo{V}_\mr{c})_{\si\ta}
&\equiv
  \ip{\F_\si|V_\mr{c}|\F_\ta}
\end{array}
\end{align}
which provides a good starting point for a perturbative analysis of CI and for comparison to the coupled-cluster equations.
The matrix elements of the model Hamiltonian are given by
$
  \mc{E}_\si
  \d_{\si\ta}
$,
so the matrix on the left is diagonal with eigenvalues of the form
$
  \mc{E}_{a_1\cd a_k}^{i_1\cd i_k}
+
  E_\mr{c}
$.
The first five rows of this equation can be written in terms of CI operators as follows\footnote{The $C_0$ operator simply is simply $c_0$ times the identity.  Under intermediate normalization $C_0=1$.}
\begin{align}
\label{eq:ci-reference-equation}
  \underset{(0^+)}{\vphantom{(}
  c_0
  }
  \underset{(2^+)}{\vphantom{(}
    E_\mr{c}
  }
=&\
  \ip{\F|
  \underset{(1)}{\vphantom{(}
    V_\mr{c}
  }
    (
    \underset{(1^+)}{\vphantom{(}
      C_1
    }
    +
    \underset{(1^+)}{\vphantom{(}
      C_2
    }
    )
  |\F}
\\
\label{eq:ci-singles-equation}
  \underset{(1^+)}{\vphantom{(}
  c_a^i
  }
  (\hspace{1pt}
  \underset{(0)}{\vphantom{(}
    \mc{E}_a^i
  }
  +
  \underset{(2^+)}{\vphantom{(}
    E_\mr{c}
  }
  \hspace{-2pt}
  )
=&\
  \ip{\F_i^a|
  \underset{(1)}{\vphantom{(}
    V_\mr{c}
  }
    (
    \underset{(0^+)}{\vphantom{(}
      C_0
    }
    +
    \underset{(1^+)}{\vphantom{(}
      C_1
    }
    +
    \underset{(1^+)}{\vphantom{(}
      C_2
    }
    +
    \underset{(2^+)}{\vphantom{(}
      C_3
    }
    )
  |\F}
\\
\label{eq:ci-doubles-equation}
  \underset{(1^+)}{\vphantom{(}
  c_{ab}^{ij}
  }
  (\hspace{1pt}
  \underset{(0)}{\vphantom{(}
    \mc{E}_{ab}^{ij}
  }
  +
  \underset{(2^+)}{\vphantom{(}
    E_\mr{c}
  }
  \hspace{-2pt}
  )
=&\
  \ip{\F_{ij}^{ab}|
  \underset{(1)}{\vphantom{(}
    V_\mr{c}
  }
    (
    \underset{(0^+)}{\vphantom{(}
      C_0
    }
    +
    \underset{(1^+)}{\vphantom{(}
      C_1
    }
    +
    \underset{(1^+)}{\vphantom{(}
      C_2
    }
    +
    \underset{(2^+)}{\vphantom{(}
      C_3
    }
    +
    \underset{(3^+)}{\vphantom{(}
      C_4
    }
    )
  |\F}
\\
  \underset{(2^+)}{\vphantom{(}
  c_{abc}^{ijk}
  }
  (\hspace{1pt}
  \underset{(0)}{\vphantom{(}
    \mc{E}_{abc}^{ijk}
  }
  +
  \underset{(2^+)}{\vphantom{(}
    E_\mr{c}
  }
  \hspace{-2pt}
  )
=&\
  \ip{\F_{ijk}^{abc}|
  \underset{(1)}{\vphantom{(}
    V_\mr{c}
  }
    (
    \underset{(1^+)}{\vphantom{(}
      C_1
    }
    +
    \underset{(1^+)}{\vphantom{(}
      C_2
    }
    +
    \underset{(2^+)}{\vphantom{(}
      C_3
    }
    +
    \underset{(2^+)}{\vphantom{(}
      C_4
    }
    +
    \underset{(3^+)}{\vphantom{(}
      C_5
    }
    )
  |\F}
\\
  \underset{(2^+)}{\vphantom{(}
  c_{abcd}^{ijkl}
  }
  (\hspace{1pt}
  \underset{(0)}{\vphantom{(}
    \mc{E}_{abcd}^{ijkl}
  }
  +
  \underset{(2^+)}{\vphantom{(}
    E_\mr{c}
  }
  \hspace{-2pt}
  )
=&\
  \ip{\F_{ijkl}^{abcd}|
  \underset{(1)}{\vphantom{(}
    V_\mr{c}
  }
    (
    \underset{(1^+)}{\vphantom{(}
      C_2
    }
    +
    \underset{(2^+)}{\vphantom{(}
      C_3
    }
    +
    \underset{(2^+)}{\vphantom{(}
      C_4
    }
    +
    \underset{(3^+)}{\vphantom{(}
      C_5
    }
    +
    \underset{(3^+)}{\vphantom{(}
      C_6
    }
    )
  |\F}
\end{align}
where the numbers in parentheses denote orders in perturbation theory.
The notation
$
  (p^+)
$
denotes that a term involves contributions of order $p$ and higher.
The orders of the CI operators follow from the fact that each order in perturbation theory increases the maximum excitation level of the wavefunction by $+2$, starting from $\Y\ord{1}$ which contains up to doubles, which implies that the lowest-order contributions to $C_k$ appear at order $\ceil{\frac{k}{2}}$.
If Brillouin's theorem holds, the $C_0$ term in equation~\ref{eq:ci-singles-equation} vanishes and the order of $C_1$ is $2^+$.
In general, the coefficient equation for doubles look as follows.
\begin{align}
  \underset{(\ceil{k/2}^+)}{\vphantom{(}
  c_{a_1\cd a_k}^{i_1\cd i_k}
  }
  (\hspace{1pt}
  \underset{(0)}{\vphantom{(}
    \mc{E}_{a_1\cd a_k}^{i_1\cd i_k}
  }
  +
  \underset{(2^+)}{\vphantom{(}
    E_\mr{c}
  }
  \hspace{-2pt}
  )
=&\
  \ip{\F_{i_1\cd i_k}^{a_1\cd a_k}|
  \underset{(1)}{\vphantom{(}
    V_\mr{c}
  }
    (
    \underset{(\ceil{k/2}^+-1)}{\vphantom{(}
      C_{k-2}
    }
    +
    \underset{(\ceil{k/2-1/2}^+)}{\vphantom{(}
      C_{k-1}
    }
    +
    \underset{(\ceil{k/2}^+-1)}{\vphantom{(}
      C_k
    }
    +
    \underset{(\ceil{k/2+1/2}^+)}{\vphantom{(}
      C_{k+1}
    }
    +
    \underset{(\ceil{k/2}^++\,1)}{\vphantom{(}
      C_{k+2}
    }
    )
  |\F}
&&
  k
\geq
  2
\end{align}
\end{rmk}

\begin{dfn}
\thmtitle{Complete orders in perturbation theory}
We say that approximations to a quantity $X$ are \textit{complete to $p\eth$-order in perturbation theory} when they contain all contributions to $X\ord{1},\ld,X\ord{p}$.
The quantity may also contain higher-order contributions in perturbation theory, but the  polynomial dependence of its error is $\mc{O}(V_\mr{c}^{p+1})$ or better.
\end{dfn}

\begin{dfn}
\thmtitle{The Davidson correction}
Truncating the CI wave operator at doubles yields
\begin{align}
  c_0
    E_\mr{c}
=
  \ip{\F|
    V_\mr{c}
    (
      C_1
    +
      C_2
    )
  |\F}
&&
  c_a^i
  (
    \mc{E}_a^i
  +
    E_\mr{c}
  )
=
  \ip{\F_i^a|
    V_\mr{c}
    (
      C_1
    +
      C_2
    )
  |\F}
&&
  c_{ab}^{ij}
  (
    \mc{E}_{ab}^{ij}
  +
    E_\mr{c}
  )
=
  \ip{\F_{ij}^{ab}|
    V_\mr{c}
    (
      C_0
    +
      C_1
    +
      C_2
    )
  |\F}
\end{align}
which are
the \textit{CI singles and doubles (CISD) equations}.
Comparison to equations~\ref{eq:ci-reference-equation}--\ref{eq:ci-doubles-equation} the CISD singles and doubles are complete to second order in perturbation theory and the CISD correlation energy is complete to third order.

\end{dfn}



\begin{align}
-
  \bo{H}_0\,
  \bo{t}
=
  \ip{\bm{\F}|
    V_\mr{c}\,
    \mr{exp}(T(\bo{t}))
  |\F}_\mr{C}
&&
  \bm{\F}
=
  \pma[l]{
    \F
  \\[-3pt]
    \bo{s}_1
  \\
    \bo{s}_2
  \\
    \vd
  \\
    \bo{s}_n
  }
\end{align}

\begin{rmk}
From PT, we know that the lowest order contributions to $T_1$, $T_2$, and $T_3$ (or $C_1$ and $C_2$) occur at first and second order in perturbation theory.
\begin{align}
  t_a^i
  \mc{E}_a^i
=&\
  \ip{\F_i^a|
  \underset{(1)}{\vphantom{(}
    V_\mr{c}
  }
    (
    \underset{(0)}{\vphantom{(}
      1
    }
    +
    \underset{(1^{+})}{\vphantom{(}
      T_2
    }
    +
    \underset{(2^{+})}{\vphantom{(}
      T_1
    }
    +
      T_1T_2
    +
      \tfr{1}{2}
      T_1^2
    +
      \tfr{1}{3!}
      T_1^3
    +
      T_3
    )
  |\F}_\mr{C}
\\
  t_{ab}^{ij}
  \mc{E}_{ab}^{ij}
=&\
  \ip{\F_{ij}^{ab}|
    V_\mr{c}
    (
      1
    +
      T_2
    +
      \tfr{1}{2}
      T_2^2
    +
      T_1
    +
      T_1T_2
    +
      \tfr{1}{2}
      T_1^2
    +
      \tfr{1}{2}
      T_1^2T_2
    +
      \tfr{1}{3!}
      T_1^3
    +
      \tfr{1}{4!}
      T_1^4
    +
      T_3
    +
      T_1T_3
    +
      T_4
    )
  |\F}_\mr{C}
\\
  t_{abc}^{ijk}
  \mc{E}_{abc}^{ijk}
=&\
  \ip{\F_{ijk}^{abc}|
    V_\mr{c}
    (
      T_2
    +
      T_3
    +
      \tfr{1}{2}
      T_2^2
    +
      T_1T_2
    +
      T_2T_3
    +
      T_1T_3
    +
      \tfr{1}{2}
      T_1^2T_2
    +
      \tfr{1}{2}
      T_1T_2^2
    +
      \tfr{1}{2}
      T_1^2T_3
    +
      \tfr{1}{3!}
      T_1^3T_2
    )
  |\F}_\mr{C}
\end{align}
\end{rmk}

Now, just to be confusing, redefine some shit.
\begin{align}
  \bo{1}_\mr{i}
\equiv
\pma[l]{
  \bo{1} & \bo{0} \\
  \bo{0} & \bo{0}
}
&&
  \bo{1}_\mr{e}
\equiv
\pma[l]{
  \bo{0} & \bo{0} \\
  \bo{0} & \bo{1}
}
&&
  \bo{H}_\mr{xy}
\equiv
  \bo{1}_\mr{x}\,
  \bo{H}\,
  \bo{1}_\mr{y}
&&
  \bo{c}_\mr{x}
\equiv
  \bo{1}_\mr{x}\,
  \bo{c}
\end{align}

\begin{align}
  \bo{H}
=
  \bo{H}_\mr{ii}
+
  \bo{H}_\mr{ie}
+
  \bo{H}_\mr{ei}
+
  \bo{H}_\mr{ee}
\end{align}

\begin{align}
  \bo{R}_\mr{ee}
\equiv
\left.
  (
    E
  -
    \bo{H}
  )^{-1}
\right|_\mr{e}
&&
  \bo{R}_\mr{ee}
  (
    E
  -
    \bo{H}
  )
  \bo{1}_\mr{e}
=
  \bo{1}_\mr{e}
&&
\begin{array}{r@{\ }l}
  \bo{R}_\mr{ee}\,
  (
    E
  -
    \bo{H}
  )
&=
-
  \bo{R}_\mr{ee}\,
  \bo{H}_\mr{ei}
+
  \bo{1}_\mr{e}
\\
  (
    E
  -
    \bo{H}
  )\,
  \bo{R}_\mr{ee}
&=
-
  \bo{H}_\mr{ie}\,
  \bo{R}_\mr{ee}
+
  \bo{1}_\mr{e}
\end{array}
\end{align}
Operating the upper equation on $\bo{c}$ gives zero due to the Schr\"odinger equation, which implies
$
  \bo{c}_\mr{e}
=
  \bo{R}_\mr{ee}
  \bo{H}_\mr{ei}
  \bo{c}_\mr{i}
$.
Projecting the Schr\"odinger equation by $\bo{1}_\mr{i}$ and substituting in this result leads to the following.
\begin{align}
  (
    \bo{H}_\mr{ii}
  +
    \bo{V}_\mr{ii}
  )
  \bo{c}_\mr{i}
=
  E
  \bo{c}_\mr{i}
&&
  \bo{V}_\mr{ii}
\equiv
  \bo{H}_\mr{ie}
  \bo{R}_\mr{ee}
  \bo{H}_\mr{ei}
\end{align}

\begin{align}
  E
=
  \fr{
    \bo{c}_\mr{i}\dg
    (
      \bo{H}_\mr{ii}
    +
      \bo{V}_\mr{ii}
    )
    \bo{c}_\mr{i}
  }{
    \bo{c}_\mr{i}\cdot
    \bo{c}_\mr{i}
  }
\end{align}


\begin{align*}
  E
=
  \ip{\F|
  (
    1
  +
    \La
  )
  \ol{H}
  |\F}
+
  \ip{\F|\La\ol{H}|\bo{e}}
  \ip{\bo{e}|E - \ol{H}|\bo{e}}^{-1}
  \ip{\bo{e}|\ol{H}|\F}
\end{align*}

\begin{align*}
  \d E
=
  \ip{\F|\La\ol{H}|\bo{e}}
  \ip{\bo{e}|E - \ol{H}|\bo{e}}^{-1}
  \ip{\bo{e}|\ol{H}|\F}
\approx
  \ip{\F|\La\ol{H}\ord{1}|\bo{e}}
  \ip{\bo{e}|E\ord{0} - \ol{H}\ord{0}|\bo{e}}^{-1}
  \ip{\bo{e}|\ol{H}\ord{m}|\F}
\end{align*}


\begin{align}
  \ol{H}
=
  E_\mr{ref}
+
  H_0
+
  (
    H_0
    T
  +
    V_\mr{c}\,
    \mr{exp}(T)
  )_\mr{C}
\end{align}

\begin{align}
  \ol{H}\ord{0}
=
  E_\mr{ref}
+
  H_0
&&
  \ol{H}\ord{1}
=
  (H_0T\ord{1})_\mr{C}
+
  V_\mr{c}
&&
  \ol{H}\ord{2}
=
  (
    H_0
    T\ord{2}
  +
    V_\mr{c}\,
    T\ord{1}
  )_\mr{C}
\end{align}

\begin{align}
\end{align}

\newpage
\begin{align}
  \ol{H}
  \mc{R}_k
  \kt{\F}
=
  E_k
  \mc{R}_k
  \kt{\F}
&&
  \br{\F}
  \mc{L}_k
  \ol{H}
=
  \br{\F}
  \mc{L}_k
  \w_k
&&
  \fwbr{\F}
    \mc{L}_k
    \mc{R}_l
  {\F}
=
  \d_{kl}
&&
  \ol{H}
=
  \mr{exp}(-T)
  H\,
  \mr{exp}(T)
=
  E_\mr{ref}
+
  \ol{H}_\mr{c}
\end{align}
\textbf{Note to self:}
Go back and change $E_0$ to $E_\mr{ref}$ throughout to avoid problems like this ($E_0$ should refer to the ground state energy).\\
\textbf{Note to self:}
Go back and change $H_e$ to $H$ for simplicity.
\begin{align}
  \mc{R}_0
=
  1
&&
  \mc{L}_0
=
  1
+
  \La
&&
  \La
=
  \sum_{h=1}^n
  \La_h
&&
  \La_h
\equiv
  (\tfr{1}{h!})^2\,
  \la_{i_1\cd i_h}^{a_1\cd a_h}
  \tl{a}^{i_1\cd i_h}_{a_1\cd a_h}
\end{align}
\textbf{Note to self:}
Go back and change $k$ to $h$ for summations over excitation levels.
\begin{align*}
  E_\mr{c}
\equiv
  \fwbr{\F}\mc{L}_0\ol{H}\mc{R}_0{\F}
-
  E_\mr{ref}
=
  \fwbr{\F}(1+\La)\ol{H}_\mr{c}\fwkt{\F}
\end{align*}
Structure of matrix
\begin{align*}
  \ol{\bo{H}}_\mr{c}
=
\pma{
  E_\mr{c} & \fwbr{\F}\ol{H}_\mr{c}\fwkt{\bo{i}} & \fwbr{\F}\ol{H}_\mr{c}\fwkt{\bo{e}} \\
  \bo{0} & \fwbr{\bo{i}}\ol{H}_\mr{c}\fwkt{\bo{i}} & \fwbr{\bo{i}}\ol{H}_\mr{c}\fwkt{\bo{e}} \\
  \fwbr{\bo{e}}\ol{H}_\mr{c}\fwkt{\F} & \fwbr{\bo{e}}\ol{H}_\mr{c}\fwkt{\bo{i}} & \fwbr{\bo{e}}\ol{H}_\mr{c}\fwkt{\bo{e}} \\
}
&&
  \bo{i}
\equiv
\pma{
  \bo{s}_1 & \cd & \bo{s}_m
}
&&
  \bo{e}
\equiv
\pma{
  \bo{s}_{m-1} & \cd & \bo{s}_n
}
&&
  \bo{s}_h
\equiv
\pma{
  \cd
\
  \F_{i_1\cd i_h}^{a_1\hspace{-1pt}\cd a_h}
\
  \cd
}
\end{align*}


\begin{thm}
\thmtitle{Faà di Bruno's formula}
\begin{align}
  \pd{^n}{x_1\cd \pt x_n}
  f(g(\bm{x}))
=
  \sum_{k=1}^n
  \sum_{(\bm{x}_1,\ld,\bm{x}_k)}^{\mc{P}_k(\bm{x})}
  f\ord{k}(g(\bm{x}))
  \prod_{i=1}^k
  \pd{
    ^{|\bm{x}_i|}
    g(\bm{x})
  }{
    x_{i,1}
  \cd
    \pt
    x_{i,|\bm{x}_i|}
  }
\end{align}
\end{thm}


\end{document}