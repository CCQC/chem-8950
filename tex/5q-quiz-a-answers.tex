\documentclass[11pt]{article}
\usepackage[cm]{fullpage}
%%AVC PACKAGES
\usepackage{avcgreek}
\usepackage{avcfonts}
\usepackage{avcmath}
\usepackage{avcthm}
\usepackage{qcmacros}
\usepackage{goldstone}
%%MACROS FOR THIS DOCUMENT
\usepackage[
  margin=1.5cm,
  includefoot,
  footskip=30pt,
  headsep=0.2cm,headheight=1.3cm
]{geometry}
\usepackage{fancyhdr}
\pagestyle{fancy}
\fancyhf{}
\fancyhead[LE,RO]{\textbf{Quiz 5}}
\fancyfoot[CE,CO]{\thepage}
\usepackage{url}

\begin{document}

\begin{enumerate}
\item
  Explain why each of the following terms vanishes.
\begin{align*}
  \text{(a)}\hspace{3pt}
  \tfr{1}{5!}
  \ip{\F_{ijklm}^{abcde}|V_\mr{c}T_1^5|\F}_\mr{C}
&&
  \text{(b)}\hspace{3pt}
  \ip{\F_{ij}^{ab}|V_\mr{c}T_2T_3|\F}_\mr{C}
&&
  \text{(c)}\hspace{3pt}
  \tfr{1}{2!}
  \ip{\F_{ijkl}^{abcd}|V_\mr{c}T_1^2|\F}_\mr{C}
&&
  \text{(d)}\hspace{3pt}
  \tfr{1}{2!}
  \ip{\F_{ijk}^{abc}|V_\mr{c}T_1^2|\F}_\mr{C}
\end{align*}
\vspace{10pt}
\textbf{Answer}:
\begin{enumerate}
\item
  Because $V_\mr{c}$ has at most four lines available for contraction and there are five $T$-operators, there is no way to satisfy the connectedness requirement.
\item
  Because the net excitation level of the $T$-operators and the bare excitation operator is $2+3-2=+3$, and $V_\mr{c}$ has no contribution with excitation level $-3$.
\item
  The net excitaton level of the $T$-operators and the bare excitation operator is $1+1-4=-2$, so we need the $+2$ component of $V_\mr{c}$ to balance the product.
  This diagram has no quasiparticle annihilation lines and therefore cannot contract with the $T$ operators to satisfy the connectedness requirement.
\begin{align*}
\left(
\diagram{
  \interaction{2}{g}{(0,0.1)}{ddot}{sawtooth};
  \draw[->-] (g1) to ++(-0.25,0.5);
  \draw[-<-] (g1) to ++(+0.25,0.5);
  \draw[->-] (g2) to ++(-0.25,0.5);
  \draw[-<-] (g2) to ++(+0.25,0.5);
  \interaction{1}{1t}{(0.1,-0.6)}{ddot}{overhang};
  \interaction{1}{2t}{(0.9,-0.6)}{ddot}{overhang};
  \draw[->-] (1t1) to ++(-0.25,0.5);
  \draw[-<-] (1t1) to ++(+0.25,0.5);
  \draw[->-] (2t1) to ++(-0.25,0.5);
  \draw[-<-] (2t1) to ++(+0.25,0.5);
}
\right)_\mr{C}
=
  0
\end{align*}
\item
  The net excitation level of the $T$-operators and the bare excitation operator is $1+1-3=-1$, so we need a $+1$ component from $V_\mr{c}$ to balance the product.
  Of the three diagrams in $V_\mr{c}$ with excitation level $+1$, one of them has no quasiparticle annihilation lines and the other two have only one.
  Therefore, only one of the $T$ operators can be connected to $V_\mr{c}$ and there is no way to satisfy the connectedness requirement for the other one.
\begin{align*}
\left(
\diagram{
  \draw
    (0,0.2)
      node[circlep] (f) {}
    to
    ++(0.5,0)
      node[ddot] (f1) {};
  \draw[->-] (f1) to ++(-0.25,0.5);
  \draw[-<-] (f1) to ++(+0.25,0.5);
  \interaction{1}{1t}{(0.1,-0.6)}{ddot}{overhang};
  \interaction{1}{2t}{(0.9,-0.6)}{ddot}{overhang};
  \draw[->-] (1t1) to ++(-0.25,0.5);
  \draw[-<-] (1t1) to ++(+0.25,0.5);
  \draw[->-] (2t1) to ++(-0.25,0.5);
  \draw[-<-] (2t1) to ++(+0.25,0.5);
}
\right)_\mr{C}
=
  0
&&
\left(
\diagram{
  \interaction{2}{g}{(0,0.4)}{ddot}{sawtooth};
  \draw[->-] (g1) to ++(-0.25,0.5);
  \draw[-<-] (g1) to ++(+0.25,0.5);
  \draw[->-] (g2) to ++(0,+0.5);
  \draw[-<-] (g2) to ++(0,-0.5);
  \interaction{1}{1t}{(0.1,-0.6)}{ddot}{overhang};
  \interaction{1}{2t}{(0.9,-0.6)}{ddot}{overhang};
  \draw[->-] (1t1) to ++(-0.25,0.5);
  \draw[-<-] (1t1) to ++(+0.25,0.5);
  \draw[->-] (2t1) to ++(-0.25,0.5);
  \draw[-<-] (2t1) to ++(+0.25,0.5);
}
\right)_\mr{C}
=
  0
&&
\left(
\diagram{
  \interaction{2}{g}{(0,0.4)}{ddot}{sawtooth};
  \draw[->-] (g1) to ++(-0.25,0.5);
  \draw[-<-] (g1) to ++(+0.25,0.5);
  \draw[-<-] (g2) to ++(0,+0.5);
  \draw[->-] (g2) to ++(0,-0.5);
  \interaction{1}{1t}{(0.1,-0.6)}{ddot}{overhang};
  \interaction{1}{2t}{(0.9,-0.6)}{ddot}{overhang};
  \draw[->-] (1t1) to ++(-0.25,0.5);
  \draw[-<-] (1t1) to ++(+0.25,0.5);
  \draw[->-] (2t1) to ++(-0.25,0.5);
  \draw[-<-] (2t1) to ++(+0.25,0.5);
}
\right)_\mr{C}
=
  0
\end{align*}
\end{enumerate}

\newpage
\item
Interpret the following graph and fully simplify your answer.
\begin{align*}
\diagram{
  \interaction{2}{ta}{(0,-0.5)}{ddot}{overhang};
  \interaction{2}{tb}{(2,-0.5)}{ddot}{overhang};
  \draw[sawtooth] (0,0) node[ddot] (g1) {} to (1.5,0) node[ddot] (g2) {};
  \draw[->-,bend left]  (ta1) to (g1);
  \draw[-<-,bend right] (ta1) to (g1);
  \draw[->-=0.7] (ta2) to ++(-0.25,1) node[smalldot] {};
  \draw[-<-] (ta2) to (g2);
  \draw[->-] (tb1) to (g2);
  \draw[-<-=0.7] (tb1) to ++(+0.25,1) node[smalldot] {};
  \draw[->-] (tb2) to ++(-0.25,1) node[smalldot] {};
  \draw[-<-] (tb2) to ++(+0.25,1) node[smalldot] {};
}
\end{align*}
\textbf{Answer}:
Denote the bare excitation operator at the top by $\tl{a}_{ab}^{ij}$.
\begin{align*}
\diagram{
  \interaction{2}{ta}{(0,-0.5)}{ddot}{overhang};
  \interaction{2}{tb}{(2,-0.5)}{ddot}{overhang};
  \draw[sawtooth] (0,0) node[ddot] (g1) {} to (1.5,0) node[ddot] (g2) {};
  \draw[->-,bend left]  (ta1) to (g1);
  \draw[-<-,bend right] (ta1) to (g1);
  \draw[->-=0.7] (ta2) to ++(-0.25,1) node[smalldot] {};
  \draw[-<-] (ta2) to (g2);
  \draw[->-] (tb1) to (g2);
  \draw[-<-=0.7] (tb1) to ++(+0.25,1) node[smalldot] {};
  \draw[->-] (tb2) to ++(-0.25,1) node[smalldot] {};
  \draw[-<-] (tb2) to ++(+0.25,1) node[smalldot] {};
}
=
  (-)^{4+3}\,
  \tfr{1}{2!2!}\,
  \op{P}_{(a/b)}^{(i/j)}\,
  \ol{g}_{kl}^{cd}
  t_{ca}^{kl}
  t_{db}^{ij}
=
-
  \tfr{1}{2!}\,
  \op{P}_{(a/b)}
  \ol{g}_{kl}^{cd}
  t_{ca}^{kl}
  t_{db}^{ij}
\end{align*}
(Implicit summation over $k,l,c,d$.)


\newpage
\item
Interpret the following graph and fully simplify it the ``long way.''  That is, you may use Rules 1-3 but you must start from Axiom 1 and show each step to get to your final answer.
\begin{align*}
\diagram{
  \draw[->-=0.25,->-=0.75]
    (0,-0.5)
      node[smalldot] {}
    to
      node[midway,ddot] (g1) {}
    ++(0,+1)
      node[smalldot] {};
  \draw[-<-=0.65]
    (0.5,-0.5)
      node[smalldot] {}
    to
    ++(0,+1)
      node[smalldot] {};
  \draw[->-=0.25,->-=0.75]
    (1,-0.5)
      node[smalldot] {}
    to
      node[midway,ddot] (g2) {}
    ++(0,+1)
      node[smalldot] {};
  \draw[-<-=0.65]
    (1.5,-0.5)
      node[smalldot] {}
    to
    ++(0,+1)
      node[smalldot] {};
  \draw[sawtooth] (g1) to (g2);
}
\end{align*}
\textbf{Answer}:
Denote the top bare excitation operator by $\tl{a}_{ab}^{ij}$ and the bottom one by $\tl{a}_{kl}^{cd}$.
Axiom 1 gives
\begin{align*}
\diagram{
  \draw[->-=0.25,->-=0.75]
    (0,-0.5)
      node[smalldot] {}
    to
      node[midway,ddot] (g1) {}
    ++(0,+1)
      node[smalldot] {};
  \draw[-<-=0.65]
    (0.5,-0.5)
      node[smalldot] {}
    to
    ++(0,+1)
      node[smalldot] {};
  \draw[->-=0.25,->-=0.75]
    (1,-0.5)
      node[smalldot] {}
    to
      node[midway,ddot] (g2) {}
    ++(0,+1)
      node[smalldot] {};
  \draw[-<-=0.65]
    (1.5,-0.5)
      node[smalldot] {}
    to
    ++(0,+1)
      node[smalldot] {};
  \draw[sawtooth] (g1) to (g2);
}
=
\fr{1}{2!2!2!}
\sum_{\substack{efgh\\mn}}
\diagram{
  \draw[->-=0.25,->-=0.75]
    (0,-0.5)
      node[smalldot] {}
    to
      node[pos=0.25,left] {$g$}
      node[midway,ddot] (g1) {}
      node[pos=0.75,left] {$e$}
    ++(0,+1)
      node[smalldot] {};
  \draw[-<-=0.65]
    (0.5,-0.5)
      node[smalldot] {}
    to
      node[pos=0.75,left=-2pt] {$m$}
    ++(0,+1)
      node[smalldot] {};
  \draw[->-=0.25,->-=0.75]
    (1,-0.5)
      node[smalldot] {}
    to
      node[pos=0.25,left] {$h$}
      node[midway,ddot] (g2) {}
      node[pos=0.75,left] {$f$}
    ++(0,+1)
      node[smalldot] {};
  \draw[-<-=0.65]
    (1.5,-0.5)
      node[smalldot] {}
    to
      node[pos=0.75,left=-2pt] {$n$}
    ++(0,+1)
      node[smalldot] {};
  \draw[sawtooth] (g1) to (g2);
}
=
  \fr{1}{2!2!2!}
  \sum_{\substack{efgh\\mn}}
  \op{P}_{(a/b)}^{(i/j)}
  \d_m^i
  \d_n^j
  \d_a^e
  \d_b^f
  \,
  \ol{g}_{ef}^{gh}
  \,
  \op{P}_{(k/l)}^{(c/d)}
  \d_k^m
  \d_l^n
  \d_g^c
  \d_h^d
  \,
  \gno{
    \tl{a}^{m^{\hole1}n^{\hole2}}_{e^{\ptcl1}f^{\ptcl2}}
    \tl{a}^{e^{\ptcl1}f^{\ptcl2}}_{g^{\ptcl3}h^{\ptcl4}}
    \tl{a}^{g^{\ptcl3}h^{\ptcl4}}_{m^{\hole1}n^{\hole2}}
  }
\end{align*}
where I have used Rule 1 to determine the degeneracy factor: three pairs of equivalent lines contribute $2!$ each, and there are no equivalent subgraphs.
Using Rule 2, the operator string evaluates as follows
\begin{align*}
  \gno{
    \tl{a}^{m^{\hole1}n^{\hole2}}_{e^{\ptcl1}f^{\ptcl2}}
    \tl{a}^{e^{\ptcl1}f^{\ptcl2}}_{g^{\ptcl3}h^{\ptcl4}}
    \tl{a}^{g^{\ptcl3}h^{\ptcl4}}_{m^{\hole1}n^{\hole2}}
  }
=
  (-1)^{2+2}
=
  +1
\end{align*}
since there are two hole lines and two loops.
Summing over the line labels gives the following.
\begin{align*}
\diagram{
  \draw[->-=0.25,->-=0.75]
    (0,-0.5)
      node[smalldot] {}
    to
      node[midway,ddot] (g1) {}
    ++(0,+1)
      node[smalldot] {};
  \draw[-<-=0.65]
    (0.5,-0.5)
      node[smalldot] {}
    to
    ++(0,+1)
      node[smalldot] {};
  \draw[->-=0.25,->-=0.75]
    (1,-0.5)
      node[smalldot] {}
    to
      node[midway,ddot] (g2) {}
    ++(0,+1)
      node[smalldot] {};
  \draw[-<-=0.65]
    (1.5,-0.5)
      node[smalldot] {}
    to
    ++(0,+1)
      node[smalldot] {};
  \draw[sawtooth] (g1) to (g2);
}
=
  \fr{1}{2!2!2!}
  \sum_{\substack{efgh\\mn}}
  \op{P}_{(a/b)}^{(i/j)}
  \d_m^i
  \d_n^j
  \d_a^e
  \d_b^f
  \,
  \ol{g}_{ef}^{gh}
  \,
  \op{P}_{(k/l)}^{(c/d)}
  \d_k^m
  \d_l^n
  \d_g^c
  \d_h^d
=
  \fr{1}{2!2!2!}
  \op{P}_{(a/b)}^{(i/j)}
  \op{P}_{(k/l)}^{(c/d)}
  \ol{g}_{ab}^{cd}
  \d_k^i
  \d_l^j
\end{align*}
Using Rule 3, we can cancel the degeneracy factors for equivalent lines connected to the top bare excitation operator against $\op{P}_{(a/b)}^{(i/j)}$.
\begin{align*}
\diagram{
  \draw[->-=0.25,->-=0.75]
    (0,-0.5)
      node[smalldot] {}
    to
      node[midway,ddot] (g1) {}
    ++(0,+1)
      node[smalldot] {};
  \draw[-<-=0.65]
    (0.5,-0.5)
      node[smalldot] {}
    to
    ++(0,+1)
      node[smalldot] {};
  \draw[->-=0.25,->-=0.75]
    (1,-0.5)
      node[smalldot] {}
    to
      node[midway,ddot] (g2) {}
    ++(0,+1)
      node[smalldot] {};
  \draw[-<-=0.65]
    (1.5,-0.5)
      node[smalldot] {}
    to
    ++(0,+1)
      node[smalldot] {};
  \draw[sawtooth] (g1) to (g2);
}
=
  \fr{1}{\cancel{2!}\cancel{2!}2!}
  \cancel{\op{P}_{(a/b)}^{(i/j)}}
  \op{P}_{(k/l)}^{(c/d)}
  \ol{g}_{ab}^{cd}
  \d_k^i
  \d_l^j
=
  \fr{1}{2!}
  \op{P}_{(k/l)}^{(c/d)}
  \ol{g}_{ab}^{cd}
  \d_k^i
  \d_l^j
\end{align*}
Applying Rule 3 to the lower operator, we can cancel the permutation over $c,d$ but not the one over $k,l$, since the degeneracy factor for the hole lines was already canceled in the previous step.
\begin{align*}
\diagram{
  \draw[->-=0.25,->-=0.75]
    (0,-0.5)
      node[smalldot] {}
    to
      node[midway,ddot] (g1) {}
    ++(0,+1)
      node[smalldot] {};
  \draw[-<-=0.65]
    (0.5,-0.5)
      node[smalldot] {}
    to
    ++(0,+1)
      node[smalldot] {};
  \draw[->-=0.25,->-=0.75]
    (1,-0.5)
      node[smalldot] {}
    to
      node[midway,ddot] (g2) {}
    ++(0,+1)
      node[smalldot] {};
  \draw[-<-=0.65]
    (1.5,-0.5)
      node[smalldot] {}
    to
    ++(0,+1)
      node[smalldot] {};
  \draw[sawtooth] (g1) to (g2);
}
=
  \cancel{
    \fr{1}{2!}
  }
  \op{P}_{(k/l)}^{\cancel{(c/d)}}
  \ol{g}_{ab}^{cd}
  \d_k^i
  \d_l^j
=
  \op{P}_{(k/l)}
  \ol{g}_{ab}^{cd}
  \d_k^i
  \d_l^j
\end{align*}


\end{enumerate}

\newpage
\noindent
\textbf{Extra Credit.}
Prove Rule~3 for a closed graph with a single bare excitation operator of the following form.
\begin{align*}
  \tl{a}^{i_1\cd i_m}_{a_1\cd a_m}
=
  (\tfr{1}{m!})^2\,
  \ol{\d}_{j_1\cd j_m}^{b_1\cd b_m}\,
  \tl{a}^{j_1\cd j_m}_{b_1\cd b_m}
&&
  \ol{\d}_{j_1\cd j_m}^{b_1\cd b_m}
\equiv
  \op{P}_{(a_1/\cd/a_m)}^{(i_1/\cd/i_m)}
  \d_{j_1}^{i_1}
  \cd
  \d_{j_m}^{i_m}
  \d_{a_1}^{b_1}
  \cd
  \d_{a_m}^{b_m}
\end{align*}
\textbf{Answer}:
Using Axiom~1, a closed graph containing this bare excitation operator will have the form
\begin{align*}
  \fr{1}{
    |I_1|!\cd|I_k|!
    |A_1|!\cd|A_h|!
  }
  \sum_{\substack{b_1\cd b_m\\j_1\cd j_m}}
  \ol{\d}_{j_1\cd j_m}^{b_1\cd b_m}\,
  T_{b_1\cd b_m}^{j_1\cd j_m}
=
  \fr{1}{
    |I_1|!\cd|I_k|!
    |A_1|!\cd|A_h|!
  }
  \op{P}_{(a_1/\cd/a_m)}^{(i_1/\cd/i_m)}
  T_{a_1\cd a_m}^{i_1\cd i_m}
\end{align*}
where $I_1\cup\cd\cup I_k=\{i_1,\ld, i_m\}$ and $A_1\cup\cd\cup A_h=\{a_1,\ld,a_m\}$ partition the indices of the bare excitation operator into subsets that fall on equivalent coefficient lines.
Any remaining degeneracy factors, interaction tensors, or contracted operators are contained in
$
  T_{a_1\cd a_m}^{i_1\cd i_m}
$.
According to the definition of equivalent lines, then, the indices in a given subset $I_p$ or $A_q$ must occur on a single interaction tensor in $T_{a_1\cd a_m}^{i_1\cd i_m}$, which is therefore already antisymmetric with respect to these indices.
Denoting the indices in $I_p$ by $\{i_{p,1},\ld,i_{p,|I_p|}\}$, this enables the following cancellation.
\begin{align*}
  \op{P}_{(a_1/\cd/a_m)}^{(i_1/\cd/i_m)}
  T_{a_1\cd a_m}^{i_1\cd i_m}
=
  \op{P}_{(a_1/\cd/a_m)}^{(i_1/\cd/I_p/\cd/i_m)}
  \op{P}^{(i_{p,1},\ld,i_{p,|I_p|})}
  T_{a_1\cd a_m}^{i_1\cd i_m}
=
  |I_p|!
  \op{P}_{(a_1/\cd/a_m)}^{(i_1/\cd/I_p/\cd/i_m)}
  T_{a_1\cd a_m}^{i_1\cd i_m}
\end{align*}
Repeating this procedure for the remaining subsets completes the proof.
\begin{align*}
  \fr{1}{
    |I_1|!\cd|I_k|!
    |A_1|!\cd|A_h|!
  }
  \op{P}_{(a_1/\cd/a_m)}^{(i_1/\cd/i_m)}
  T_{a_1\cd a_m}^{i_1\cd i_m}
=
  \fr{1}{
    \cancel{|I_1|!}\cd\cancel{|I_k|!}
    \cancel{|A_1|!}\cd\cancel{|A_h|!}
  }
  \op{P}_{(A_1/\cd/A_h)}^{(I_1/\cd/I_k)}
  T_{a_1\cd a_m}^{i_1\cd i_m}
=
  \op{P}_{(A_1/\cd/A_h)}^{(I_1/\cd/I_k)}
  T_{a_1\cd a_m}^{i_1\cd i_m}
\end{align*}



\vfill
\noindent
\hrulefill

\noindent
\textbf{Appendix.}
\vspace{10pt}

{\small

\noindent
\bmit{Axiom 1.}
The algebraic of a graph $G$ is obtained from a corresponding summand graph $\Si(G)$ as follows.
\begin{align*}
  G
=
  \fr{1}{\mr{dg}(G)}
  \sum_{\mr{labels}}\Si(G)
\end{align*}
\noindent
\bmit{Rule~1.}
  Each set of $k$ equivalent lines or equivalent subgraphs contributes a factor of $k!$ to the degeneracy.

\noindent
\bmit{Rule~2.}
  The overall sign of a closed graph is $(-)^{h+l}$, where $h$ and $l$ denote the total number of hole lines and loops.

\noindent
\bmit{Rule~3.}
  For bare excitation operators, cancel the degeneracy factors from their equivalent coefficient lines by replacing the full antisymmetrizer, $P^{(p_1/\cd/p_m)}_{(q_1/\cd/q_m)}$, with a reduced antisymmetrizer over inequivalent coefficient lines, $\op{P}^{(P_1/\cd /P_h)}_{(Q_1/\cd /Q_k)}$.\footnote{
  Here $\{p_1,\ld,p_m\}=P_1\cup\cd\cup P_h$ and $\{q_1,\ld,q_m\}=Q_1\cup\cd\cup Q_k$ are the upper and lower indices on the bare excitation operator $\tl{a}^{p_1\cd p_m}_{q_1\cd q_m}$, and the $P_i$'s and $Q_i$'s label subsets of equivalent coefficient lines.
}~\footnote{
  For equivalent lines connecting two bare excitation operators, this cancellation can only be performed once.
}
}



\end{document}
