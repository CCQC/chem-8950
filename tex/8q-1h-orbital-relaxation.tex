\documentclass[11pt]{article}
\usepackage[cm]{fullpage}
%%AVC PACKAGES
\usepackage{avcgreek}
\usepackage{avcfonts}
\usepackage{avcmath}
\usepackage[numberby=section,skip=9pt plus 2pt minus 7pt]{avcthm}
\usepackage{qcmacros}
\usepackage{goldstone}
%%MACROS FOR THIS DOCUMENT
\numberwithin{equation}{section}
\usepackage[
  margin=1.5cm,
  includefoot,
  footskip=30pt,
  headsep=0.2cm,headheight=1.3cm
]{geometry}
\usepackage{fancyhdr}
\pagestyle{fancy}
\fancyhf{}
\fancyhead[LE,RO]{Quiz 8, Handout 1: Orbital relaxation}
\fancyfoot[CE,CO]{\thepage}
\usepackage{url}
\makeatother
\newcommand{\resolventline}[2][1]{
  \tikz[overlay]{
      \draw[thick,flexdotted] (0,-1ex) to ++(0,#1*4.5ex) node[above,inner sep=1pt] {#2};
  }
}
\usepackage{accents}
\newcommand{\oc}[1]{\ensuremath{\accentset{\circ}{#1}}}

\begin{document}

\setlength{\abovedisplayskip}{5pt}
\setlength{\belowdisplayskip}{5pt}


\setcounter{section}{7}
\section{Orbital relaxation}


\begin{rmk}
\thmtitle{Orbital relaxation}
According to the Thouless theorem (\cref{appendix:thouless}), the effect of the singles CC operator is to transform the orbitals of the reference determinant into a new set $\{\widetilde{\y}_i\}$ by ``mixing in'' some of the virtual orbitals.
\begin{align}
  \Y_\mr{CC}
=
  \mr{exp}(T_2 + T_3 + \cd)
  \widetilde{\F}
&&
  \widetilde{\F}
\equiv
  \mr{exp}(T_1)
  \F
=
  \tfr{1}{\sqrt{n!}}\,
  \mr{det}(\widetilde{\y}_1\cd \widetilde{\y}_n)
&&
  \widetilde{\y}_i
=
  \y_i
+
  \sum_a
  \y_a
  t_a^i
\end{align}
This can be thought of as ``relaxing'' the orbitals in the presence of electron correlation.
The size of this \textit{orbital relaxation effect} can be monitored as the root mean square difference from the reference orbitals, which is known as the \textit{$\mc{T}_1$ diagnostic}.
\begin{align}
  \mc{T}_1
\equiv
  \sqrt{
  \fr{1}{n}
  \sum_{i=1}^n
  \|\widetilde{\y}_i - \y_i\|^2
  }
=
  \fr{\|\bo{t}_1\|}{\sqrt{n}}
\end{align}
Significant orbital relaxation generally indicates that the reference determinant forms a poor approximation to the wavefunction, which can lead to large errors for low-order truncated methods like CCSD or CCSD(T).
In closed-shell systems, significant orbital relaxation is usually associated with an inherent \textit{multireference character}, which means that no single determinant dominates the wavefunction for any choice of orbitals.
Empirically, $\mc{T}_1\geq 0.02$ is considered large for closed-shell species.
In open-shell systems, mean-field methods like Hartree-Fock theory are often deficient even for non-multireference systems.
In this case, orbital relaxation effects can generally be cured by choosing a new determinant which is optimized in the presence of dynamical\footnote{As opposed to mean-field.} electron correlation.
\end{rmk}

\begin{rmk}
\thmtitle{Brueckner and orbital-optimized methods}
The two most common ways of defining an ideal reference determinant for the correlated wavefunction are the \textit{best overlap criterion} and the \textit{best energy criterion}.\footnote{See \url{https://en.wikipedia.org/wiki/Arg_max} for the notation used here.}
\begin{align}
\label{eq:brueckner-and-oo-general-condition}
  \{\y_p\}_{\mr{B}}
=
  \underset{\{\y_p\}}{\arg\max}\,
  \|\ip{\F|\Y}\|^2
&&
  \{\y_p\}_{\mr{O}}
=
  \underset{\{\y_p\}}{\arg\min}\,
  \ip{\Y|H|\Y}
\end{align}
The \textit{best overlap} or \textit{Brueckner orbitals} yield a reference determinant, $\F_\mr{B}$, that has maximum overlap with the wavefunction.
The \textit{best energy} or \textit{optimized orbitals} yield the lowest energy expectation value within a given Ansatz.
Note that these two conditions are in general mutually exclusive, so one must choose one or the other.
\end{rmk}

\begin{dfn}
\thmtitle{Orbital invariance}
An Ansatz is usually termed \textit{orbital invariant} if it preserves the rotational invariances of the reference function.
For single-reference methods, this means that the total energy is unchanged by orbital rotations within the occupied and virtual spaces.
In this context, the occupied and virtual blocks of $\bo{X}$ in equation~\ref{eq:operator-transformation} are redundant in the sense that they produce transformations which do not change the energy.
\end{dfn}

\begin{rmk}
The non-redundant orbital rotations (\cref{appendix:orbital-rotations}) for an orbital-invariant Ansatz can be parametrized as\footnote{Note that we are using Einstein summation.}
\begin{align}
  \Th(\bo{x})
=
  U(\bo{x})
  \Th
&&
  U(\bo{x})
=
  \mr{exp}(X - X\dg)
&&
  X
=
  x_a^i
  a^a_i
\end{align}
where $\Th$ is an arbitrary Fock state.
Brueckner orbitals and optimized orbitals can then be determined from the following.\footnote{%
  Note that $a^i_a\kt{\F}=0$, which eliminates the second term in the overlap derivative.
}
\begin{align}
  \left.
  \pd{}{x_a^{i*}}
  \|\ip{\F(\bo{x})|\Y}\|^2
  \right|_{\bo{x}=\bo{0}}
=
  \ip{\F|\,a^i_a|\Y}
  \ip{\Y|\F}
&&
  \left.
  \pd{}{x_a^{i*}}
  \ip{\Y(\bo{x})|H|\Y(\bo{x})}
  \right|_{\bo{x}=\bo{0}}
=
  \ip{\Y|a^i_a H|\Y}
-
  \ip{\Y|H a^i_a|\Y}
\end{align}
In the overlap derivative, we have allowed the orbitals of the determinant to vary while holding the wavefunction constant.
In the energy derivative, we have allowed the wavefunction to vary while holding the orbitals of the Hamiltonian constant.\footnote{
  In each case we could have made the opposite choice, but note that we do not want to transform \textit{everything}.
  Transforming everything corresponds to a Fock space isomorphism rather than a transformation of orbitals used to construct the wavefunction.
  By definition, an isomorphism would leave these matrix elements unchanged:
  $
    \ip{\F(\bo{x})|\Y(\bo{x})}
  =
    \ip{\F|\Y}
  $
  and
  $
    \ip{\Y(\bo{x})|H(\bo{x})|\Y(\bo{x})}
  =
    \ip{\Y|H|\Y}
  $.
}
The best overlap and best energy orbitals make these derivatives vanish, which leads to a new set of conditions
\begin{align}
  \{\y_p\}_\mr{B}\,:\,\,
  \ip{\F_i^a|\Y}
\overset{!}{=}
  0
&&
  \{\y_p\}_\mr{O}\,:\,\,
  \ip{\Y|[a_a^i, H]|\Y}
\overset{!}{=}
  0
\end{align}
which are equivalent to equation~\ref{eq:brueckner-and-oo-general-condition}.
\end{rmk}

\newpage
\begin{rmk}
\thmtitle{Brueckner algorithm}
\end{rmk}

\begin{rmk}
\thmtitle{Orbital optimization algorithm}
\begin{align}
  E(\bo{x})
-
  E
\approx
  \bo{x}\dg
  \bo{w}
+
  \tfr{1}{2}\,
  \bo{x}\dg
  \bo{A}
  \bo{x}
&&
  (\bo{w})_{ia}
\equiv
  \left.
  \pd{E}{x_a^{i*}}
  \right|_{\bo{x}=\bo{0}}
&&
  (\bo{A})_{ia,jb}
\equiv
  \left.
  \pd{^2 E}{x_a^{i*}\pt x_b^j}
  \right|_{\bo{x}=\bo{0}}
\end{align}
\begin{align}
  \bo{x}
=
-
  \bo{A}^{-1}\,
  \bo{w}
&&
  (\bo{w})_{ia}
=
  \ip{\Y|
    [a^i_a, H]
  |\Y}
&&
  (\bo{A})_{ia,jb}
=
  \ip{\Y|
    [[a^i_a, H], a^b_j]
  |\Y}
\end{align}

\begin{enumerate}
\item
$
  t_{a_1\cd a_k}^{i_a\cd a_k}
=
  (\mc{E}_{a_1\cd a_k}^{i_1\cd a_k})^{-1}\,
  \ip{\F_{i_1\cd i_k}^{a_1\cd a_k}|
    V_\mr{c}\,
    \mr{exp}(T)
  |\F}_\mr{C}
$

\item
$
  \la_{i_1\cd i_k}^{a_1\cd a_k}
=
  (\mc{E}_{a_1\cd a_k}^{i_1\cd a_k})^{-1}\,
  \ip{\F|
    (1 + \La)
    V_\mr{c}\,
    \mr{exp}(T)
  |\F_{i_1\cd i_k}^{a_1\cd a_k}}_\mr{C}
$

\item
build
$
  \bo{A}^{-1}
$
and
$
  \bo{w}
$

\item
$
  x_a^i
=
  (-\bo{A}^{-1}\,\bo{w})_{ia}
$

\item
$
  \bo{U}
=
  \mr{exp}(\bo{X} - \bo{X}\dg)
$
where
$
  \bo{X}_\mr{vo}
=
  [x_a^i]
$

\item
$
  \bo{C}
\mapsto
  \bo{C}\bo{U}
$

\item
Recompute integrals in the new MO basis

\item
Check $\|\bo{w}\|$, $\bo{t}$, and $\bm{\la}$ for convergence
\end{enumerate}
\end{rmk}

\begin{rmk}
\end{rmk}


\begin{dfn}
\thmtitle{Density matrices}
\begin{align}
  H
=
  h_p^q
  a^p_q
+
  \tfr{1}{4}
  \ol{g}_{pq}^{rs}
  a^{pq}_{rs}
\end{align}
\begin{align}
  E
=
  h_p^q
  \g^p_q
+
  \tfr{1}{4}
  \ol{g}_{pq}^{rs}
  \g^{pq}_{rs}
&&
  \g^p_q
=
  \ip{\Y|a^p_q|\Y}
&&
  \g^{pq}_{rs}
=
  \ip{\Y|a^{pq}_{rs}|\Y}
\end{align}
\end{dfn}

\begin{rmk}
\begin{align}
  [a_a^i, a^p_q]
=&\
  \no{
    a^i_{a^\ptcl}
    a^{p^\ptcl}_q
  }
-
  \no{
    a^p_{q^\ptcl}
    a^{i^\ptcl}_a
  }
=
  \d^p_a\,
  a^i_q
-
  \d^q_i\,
  a^p_a
\\
  [a_a^i, a^{pq}_{rs}]
=&\
  P^{(p/q)}
  \no{
    a^i_{a^\ptcl}
    a^{p^\ptcl q}_{r^{\phantom\ptcl}s}
  }
-
  P^{(r/s)}
  \no{
    a^{p^{\phantom\ptcl}q}_{r^\ptcl s}
    a^{i^\ptcl}_a
  }
=
  P^{(p/q)}
  \d^p_a\,
  a^{iq}_{rs}
-
  P_{(r/s)}
  \d^i_r
  a^{pq}_{as}
\end{align}
\begin{align}
  (\bo{w})_{ia}
=
  \ip{\Y|[a_a^i, H]|\Y}
=
  (
    \bo{F}
  -
    \bo{F}\dg
  )_a^i
&&
  (\bo{F})_p^q
\equiv
  h_p^r
  \g^q_r
+
  \tfr{1}{2}
  \ol{g}_{pr}^{st}
  \g^{qr}_{st}
\end{align}
\begin{align}
  (\bo{A})_{ia,jb}
=
  \ip{\Y|[[a^i_a, H], a^b_j]|\Y}
\approx
  \ip{\F|[[a^i_a, H_0], a^b_j]|\F}
=
  \ip{\F^a_i|H_0|\F^b_j}
=
  \mc{E}_j^b
  \d^i_j
  \d^b_a
\implies
  (-\bo{A}^{-1})_{ia,jb}
=
  (\mc{E}_b^j)^{-1}\,
  \d^i_j
  \d^b_a
\end{align}
\begin{align}
  (-\bo{A}^{-1}\,\bo{w})_{ia}
=
-
  \sum_{jb}
  (\bo{A}^{-1})_{ia,jb}
  (\bo{w})_{jb}
=
-
  \sum_{jb}
  (\mc{E}_b^j)^{-1}
  \d^i_j
  \d^b_a
  (\bo{F} - \bo{F}\dg)_b^j
=
  \fr{(\bo{F} - \bo{F}\dg)_a^i}{\mc{E}_a^i}
\end{align}
\end{rmk}



\begin{rmk}
\begin{align}
  a^p_q
=
  \tl{a}^p_q
+
  \oc{\g}^p_q
&&
  a^{pq}_{rs}
=
  \tl{a}^{pq}_{rs}
+
  P^{(p/q)}_{(r/s)}
  \tl{a}^p_r\,
  \oc{\g}^r_s
+
  P_{(r/s)}
  \oc{\g}^p_r
  \oc{\g}^q_s
\end{align}
\begin{align}
  \g^p_q
=
  \tl{\g}^p_q
+
  \oc{\g}^p_q
&&
  \g^{pq}_{rs}
=
  \tl{\g}^{pq}_{rs}
+
  P^{(p/q)}_{(r/s)}
  \tl{\g}^p_r\,
  \oc{\g}^r_s
+
  P_{(r/s)}
  \oc{\g}^p_r
  \oc{\g}^q_s
\end{align}

\begin{align}
  \tl{\g}^p_q
=
  \ip{\F|(1 + \La)\,\tl{a}^p_q\,\mr{exp}(T)|\F}_\mr{C}
=
\diagram{
  \interaction{2}{g}{(0,0)}{ddot}{sawtooth};
}
\end{align}
\end{rmk}

\newpage
\appendix

\section{The Thouless theorem}
\label{appendix:thouless}

\begin{ntt}
\label{ntt:orbital-transformation}
Let $\bm{\y}=[\y_p]$ be a row vector of orthonormal spin-orbitals, composed of occupied and virtual blocks, $\bm{\y}=[\bm{\y}_\mr{o}\ \bm{\y}_\mr{v}]$, with respect to a reference determinant, $\F$.
Other spin-orbital bases relate to this one via
$
  \kt{\bm{\y}'}
=
  \kt{\bm{\y}}\,
  \bo{U}
$,
which is a unitary transformation if the primed orbitals are orthonormal.
Let $\F'$ be the \textit{transformed reference determinant}, constructed from the first $n$ orbitals in $\bm{\y}'$.
Then the occupied and virtual orbitals of the transformed space are given by
\begin{align}
\label{eq:block-transformation}
  \kt{\bm{\y}'_\mr{o}}
=
  \kt{\bm{\y}_\mr{o}}\,
  \bo{U}_\mr{oo}
+
  \kt{\bm{\y}_\mr{v}}\,
  \bo{U}_\mr{vo}
&&
  \kt{\bm{\y}'_\mr{v}}
=
  \kt{\bm{\y}_\mr{o}}\,
  \bo{U}_\mr{ov}
+
  \kt{\bm{\y}_\mr{v}}\,
  \bo{U}_\mr{vv}
\end{align}
in terms of the occupied and virtual blocks of $\bo{U}$.
This kind of transformation is sometimes called an \textit{orbital rotation}.
\end{ntt}


\begin{thm}
\label{thm:thouless}
\thmtitle{The Thouless theorem}
\begin{enumerate}
\item
\label{item:thouless-part-1}
\thmstatement{
  The function
  $e^{T_1}\F$
  is an intermediately normalized determinant
  $
    \tfr{1}{\sqrt{n!}}
    \mr{det}(\widetilde{\y}_1\cd\widetilde{\y}_n)
  $
  with orbitals
  $
    \widetilde{\y}_i
  =
    \y_i
  +
    \sum_a
    \y_a
    t_a^i
  $.
}
\thmproof{
  Intermediate normalization follows from
  $
    \ip{\F|e^{T_1}\F}
  =
    1
  $.
  This function has the form of a determinant
\begin{align*}
  e^{T_1}
  \kt{\F}
=
  e^{\sum_{a}t_a^1 a_1^a + \cd + \sum_a t_a^n a_n^a}
  a_1\dg
  \cd
  a_n\dg
  \kt{\vac}
=
  \widetilde{a}_1\dg
  \cd
  \widetilde{a}_n\dg
  \kt{\vac}
=
  \kt{\widetilde{\F}}
&&
  \widetilde{a}_i\dg
\equiv
  \mr{exp}(\ts{\sum_{a}t_a^i a_i^a})\,
  a_i\dg
\end{align*}
  since $\sum_a t_a^ia_i^a$ commutes with all creation operators except $a_i\dg$.
  The transformed orbitals are given by
\begin{align*}
  \kt{\widetilde{\y}_i}
=
  \widetilde{a}_i\dg
  \kt{\vac}
=
  \mr{exp}(\ts{\sum_{a}t_a^i a_a\dg a_i})\,
  a_i\dg
  \kt{\vac}
=
  (\ts{
    1
  +
    \sum_a
    t_a^i
    a_a\dg
    a_i
  })\,
    a_i\dg
  \kt{\vac}
=
  \kt{\y_i}
+
  \ts{\sum_a}
  t_a^i
  \kt{\y_a}
\end{align*}
 using $a_i^2=0$ and $a_ia_i\dg\kt{\vac}=\kt{\vac}$.
}


\item
\thmstatement{
  Any intermediately normalized determinant
  $
    \widetilde{\F}
  =
    \tfr{1}{\sqrt{n!}}
    \mr{det}(\widetilde{\y}_1\cd\widetilde{\y}_n)
  $
  can be written as $e^{T_1}\,\F$.
}
\thmproof{
  Intermediate normalization implies that $\widetilde{\F}$ has non-zero overlap with the reference determinant.
  Therefore, $\widetilde{\F}$ can be written as
  $\F'/\ip{\F|\F'}$
  where $\F'$ is a Slater determinant.
  The normalization factor is given by
\begin{align*}
\ts{
  \ip{\F|\F'}
=
  \tfr{1}{n!}
  \sum_{\pi,\si}^{\mr{S}_n}
  \e_\pi
  \e_\si
  \ip{\y_{\pi(1)}|\y'_{\si(1)}}
  \cd
  \ip{\y_{\pi(n)}|\y'_{\si(n)}}
=
  \sum_{\si}^{\mr{S}_n}
  \e_\si
  \ip{\y_{1}|\y'_{\si(1)}}
  \cd
  \ip{\y_{n}|\y'_{\si(n)}}
=
  \mr{det}(\bo{U}_{\mr{oo}})\,\,.
}
\end{align*}
  Therefore,
  $
    \widetilde{\F}
  =
    \F'/
    \mr{det}(\bo{U}_\mr{oo})
  =
    \F'\,
    \mr{det}(\bo{U}_\mr{oo}^{-1})
  $
  and the rows of $\widetilde{\F}$ are given by the following vector\footnote{
    The second equality follows from expanding $\kt{\bm{\y}_\mr{o}'}$ according to eq~\ref{eq:block-transformation}.
  }
\begin{align*}
  \kt{\bm{\widetilde{\y}}_\mr{o}}
=
  \kt{\bm{\y}'_\mr{o}}\,
  \bo{U}_\mr{oo}^{-1}
=
  \kt{\bm{\y}_\mr{o}}\,
+
  \kt{\bm{\y}_\mr{v}}\,
  \bo{U}_\mr{vo}
  \bo{U}_\mr{oo}^{-1}
\end{align*}
  with elements
  $
    \widetilde{\y}_i
  =
    \y_i
  +
    \sum_a
    \y_a\,
    (\bo{U}_\mr{vo}\bo{U}_\mr{oo}^{-1})_{ai}
  $.
  Referring back to part one,
  $
    \widetilde{\F}
  =
    e^{T_1}\F
  $
  with
  $
    t_a^i
  =
    (\bo{U}_\mr{vo}\bo{U}_\mr{oo}^{-1})_{ai}
  $.
}
\end{enumerate}
\end{thm}




\newpage
\section{Orbital rotations}
\label{appendix:orbital-rotations}


\begin{dfn}
\label{dfn:normal-matrix}
\thmtitle{Normal matrix}
A square matrix satisfying $\bo{N}\dg\bo{N}=\bo{N}\bo{N}\dg$ is termed \textit{normal}.
Several important kinds of matrices meet this criterion:
\textit{Hermitian matrices}, $\bo{H}\dg=\bo{H}$;
\textit{anti-Hermitian matrices}, $\bo{A}\dg=-\bo{A}$;
and
\textit{unitary matrices}, $\bo{U}\dg=\bo{U}^{-1}$.
Note that Hermitian and anti-Hermitian matrices can always be written as $\bo{X}+\bo{X}\dg$ and $\bo{X}-\bo{X}\dg$.
\end{dfn}

\begin{rmk}
\label{rmk:spectral-theorem}
The spectral theorem\footnote{See \url{https://en.wikipedia.org/wiki/Spectral_theorem}} for normal matrices says that $\bo{N}=\bo{V}\widetilde{\bo{N}}\bo{V}\dg$ where $\bo{V}$ is unitary and $\widetilde{\bo{N}}$ is diagonal.
A direct corollary\footnote{Since there exists a basis in which $\bo{N}$ is diagonal, statements about $\bo{N}$ translate into statements about its eigenvalues.} is that the eigenvalues of Hermitian, anti-Hermitian, and unitary matrices can be written as follows.
\begin{align}
  h^*
=
  h
\implies
  h
=
  \f
&&
  a^*
=
-
  a
\implies
  a
=
  i\f
&&
  u^*
=
  u^{-1}
\implies
  u
=
  e^{i\f}
&&
  \f
\in
  \mb{R}
\end{align}
In words, Hermitian eigenvalues are real, anti-Hermitian eigenvalues are pure imaginary, and unitary eigenvalues lie on the unit circle.
Note that unitary eigenvalues have the form $u=\mr{exp}(a)$ where $a$ is an anti-Hermitian eigenvalue.
This implies that any unitary matrix $\bo{U}$ can be written as $\mr{exp}(\bo{A})$, where $\bo{A}$ is anti-Hermitian.
\end{rmk}

\begin{rmk}
\label{rmk:spin-orbital-transformation-law}
According to \cref{dfn:normal-matrix} and \cref{rmk:spectral-theorem}, unitary transformations of the spin-orbitals can be parametrized as
\begin{align}
\label{eq:spin-orbital-transformation}
  \y_p'
=
  \sum_q
  \y_q
  (\mr{exp}(\bo{X} - \bo{X}\dg))_{qp}
\end{align}
in terms a square matrix $\bo{X}$.
The anti-Hermitian form of this parametrization leads to redundancies.
In particular, notice that $\bo{X}=[z\,\d_{pq}]$ generates the same transformation as $\bo{X}\dg=[-z^*\,\d_{qp}]$.
These redundancies are eliminated by setting the upper or lower triangle of $\bo{X}$ to zero.
The creation operators for these orbitals are given by
$
  a_p^{\prime\,\dagger}
=
  \sum_q
  a_q\dg
  (\mr{exp}(\bo{X} - \bo{X}\dg))_{qp}
$.
\end{rmk}


\begin{prop}
\label{prop:creation-operator-similarity-transform}
\thmstatement{
The identity\ \
$\ds{
  \mr{exp}(G)\,a_p\dg\,\mr{exp}(-G)
=
  \sum_q
  a_q\dg\,
  (\mr{exp}(\bo{G}))_{qp}
}$\
holds for any
$
  G
=
  \sum_{pq}
  (\bo{G})_{pq}\,
  a_p\dg a_q
$.
}\vspace{3pt}
\thmproof{
  This follows from
  $
    [G,\cdot\,]^m(a_p\dg)
  =
    \sum_q
    a_q\dg
    (\bo{G}^m)_{qp}
  $,
  which we will prove by induction.
  For $m=0$ the statement is trivially true.
  If we assume it holds for $m$, then the following shows that it also holds for $m+1$,\footnote{
  The second equality here follows from expanding $G$ and using
$
  [a_r\dg a_s, a_q\dg]
=
  \no{
    a_r\dg
    \ctr{}{a}{_s}{}
    a_s a_q\dg
  }
=
  a_r\dg\,
  \d_{sq}
$.
}
\begin{align*}
\ts{
  [G,\cdot\,]^{m+1}(a_p\dg)
=
  \sum_q
  [G, a_q\dg]\,
  (\bo{G}^m)_{qp}
=
  \sum_{qr}
  a_r\dg
  (\bo{G})_{rq}
  (\bo{G}^m)_{qp}
=
  \sum_r
  a_r\dg
  (\bo{G}^{m+1})_{rp}
}
\end{align*}
  which completes the induction.
  Substituting this result into the Hausdorff expansion of
$
  \mr{exp}(G)\,a_p\dg\,\mr{exp}(-G)
$
and  recognizing the Taylor expansion of $\mr{exp}(\bo{G})$ completes the proof.
}
\end{prop}


\begin{rmk}
Given \cref{rmk:spin-orbital-transformation-law} and \cref{prop:creation-operator-similarity-transform}, the transformation of particle-hole operators can be expressed as
\begin{align}
\label{eq:operator-transformation}
\begin{array}{r@{\ }l}
  a_p^{\prime\,\dagger}
&=
  \mr{exp}(X - X\dg)
  a_p\dg\,
  \mr{exp}(X\dg - X)
\\[4pt]
  a_p^{\prime}
&=
  \mr{exp}(X - X\dg)
  a_p\,
  \mr{exp}(X\dg - X)
\end{array}
&&
  X
=
  \sum_{p>q}
  (\bo{X})_{pq}
  a_p\dg a_q
\end{align}
where the annihilation operator transformation is simply the adjoint of the one for creation operators.
\end{rmk}


\begin{rmk}
If $\Th'$ is obtained by replacing all of the orbitals in the basis expansion of $\Th\in\mc{F}$ with primed orbitals, then
\begin{align}
  \Th'
=
  \mr{exp}(X - X\dg)
  \Th
\end{align}
which follows from substituting equation~\ref{eq:operator-transformation} into
$
  a_{p_1}^{\prime\,\dagger}
  \cd
  a_{p_n}^{\prime\,\dagger}
  \kt{\vac}
$
to prove that
$
  \kt{\F'_{(p_1\cd p_n)}}
=
  \mr{exp}(X - X\dg)
  \kt{\F_{(p_1\cd p_n)}}
$\footnote{
  Note that
$
  \mr{exp}(X\dg - X)
=
  \pr{\mr{exp}(X - X\dg)}^{-1}
$
and
$
  \mr{exp}(X\dg - X)
  \kt{\vac}
=
  0
$.
}
for any basis state.
\end{rmk}


\end{document}