\documentclass[11pt]{article}
\usepackage[cm]{fullpage}
%%AVC PACKAGES
\usepackage{avcgreek}
\usepackage{avcfonts}
\usepackage{avcmath}
\usepackage[numberby=section]{avcthm} % 
\usepackage{qcmacros}
\usepackage{goldstone}
%%MACROS FOR THIS DOCUMENT
\numberwithin{equation}{section}
\usepackage[
  margin=1.5cm,
  includefoot,
  footskip=30pt,
  headsep=0.2cm,headheight=1.3cm
]{geometry}
\usepackage{fancyhdr}
\pagestyle{fancy}
\fancyhf{}
\fancyhead[LE,RO]{\textbf{Quiz 2}}
\fancyfoot[CE,CO]{\thepage}
\usepackage{url}

\begin{document}

\begin{enumerate}
\item
  In solving the canonical Hartree-Fock equations, it is common to use eigenfunctions of the core Hamiltonian as starting guesses for the canonical molecular orbitals.
  This eigenvalue equation looks as follows
  \begin{align*}
    \op{h}
    \f_{p}
  =
    \ev_{p}
    \f_{p}
  \end{align*}
  where $\f_p$ is a spatial function.
  Project this equation by $\x_\mu$ and then expand $\f_p$ as a linear combination of atomic orbitals, $\f_p=\sum_\nu \x_\nu\, C_{\nu p}$, in order to arrive at the following matrix equation for the core guess.
  \begin{align*}
    \bo{H}\bo{C}
  =
    \bo{S}\bo{C}\bm{\ev}
  &&
    (\bo{H})_{\mu\nu}
  =
    \ip{\x_\mu|\op{h}|\x_\nu}
  &&
    (\bo{S})_{\mu\nu}
  =
    \ip{\x_\mu|\x_\nu}
  &&
    (\bo{C})_{\mu p}
  =
    C_{\mu p}
  &&
    (\bm{\ev})_{pq}
  =
    \ev_p\d_{pq}
  \end{align*}
  \textbf{For extra credit}, you may also briefly answer the following question in words: For an open-shell system, why do we end up with different spatial orbitals for different spins, even though the initial core-guess orbitals are the same for each spin?
  
  

\newpage
\item
  Derive the expansion of $H_e$ in terms of $\F$-normal-ordered excitations.


\newpage
\item
  Derive the second Slater rule using Wick's theorem, and explain why this matrix element evaluates to zero for canonical Hartree-Fock orbitals.
  \begin{align*}
    \ip{\F|H_e|\F_i^a}
  =
    \,\,?
  \end{align*}
\end{enumerate}


\end{document}