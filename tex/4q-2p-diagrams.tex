\documentclass[11pt]{article}
\usepackage[cm]{fullpage}
%%AVC PACKAGES
\usepackage{avcgreek}
\usepackage{avcfonts}
\usepackage{avcmath}
\usepackage[numberby=section]{avcthm}
\usepackage{qcmacros}
\usepackage{goldstone}
%%MACROS FOR THIS DOCUMENT
\usepackage[
  margin=1.5cm,
  includefoot,
  footskip=30pt,
  headsep=0.2cm,headheight=1.3cm
]{geometry}
\usepackage{fancyhdr}
\pagestyle{fancy}
\fancyhf{}
\fancyhead[LE,RO]{Quiz 4, Suggested Problems 2: Diagrams}
\fancyfoot[CE,CO]{\thepage}
\usepackage{url}
\usepackage{multicol}

\begin{document}

\begin{enumerate}
\item
  Prove the following identity.
\begin{align}
  \tl{a}^{p_1\cd p_m}_{q_1\cd q_m}
=
  (\tfr{1}{m!})^2\,
  \ol{\d}_{p_1'\cd p_m'}^{q_1'\cd q_m'}\,
  \tl{a}^{p_1'\cd p_m'}_{q_1'\cd q_m'}
&&
  \ol{\d}^{q_1'\cd q_m'}_{p_1'\cd p_m'}
\equiv
  \op{P}^{(p_1/\cd /p_m)}
        _{(q_1/\cd /q_m)}
  \d^{p_1}_{p_1'}\cd \d^{p_m}_{p_m'}
  \d^{q_1'}_{q_1}\cd \d^{q_m'}_{q_m}
\end{align}

\item
  Derive each of the following rules in your own words.
{\small
\begin{enumerate}
\item
  Each set of $k$ equivalent internal lines\footnote{If rule \ref{item:rule-for-coefficient-lines} is used, this excludes coefficient lines.  Otherwise, replace rule \ref{item:rule-for-coefficient-lines} with the full antisymmetrizer.} or equivalent subgraphs contributes a factor of $k!$ to the degeneracy.
\item
  Each open cycle contributes $(-)^{h_i}a^p_q$ to the normal-ordered product, where and $p$ and $q$ label the free ends.
\item
  Each closed loop contributes $(-)^{h_i+1}$ to the overall sign, where $h_i$ is the number of hole contractions.
\item
  The overall sign of a closed graph is $(-)^{h+l}$, where $h$ and $l$ denote the total number of hole lines and loops.
\item\label{item:rule-for-coefficient-lines}
  For each bare excitation operator in a coefficient graph, the coefficient lines contribute an antisymmetrizer
  $\op{P}^{(P_1/\cd /P_h)}_{(Q_1/\cd /Q_k)}$
  where the $P_i$'s and $Q_i$'s label subsets of equivalent creation and annihilation lines, respectively.
\end{enumerate}}

\item
  Assuming Lemma 4.1 holds true, prove Wick's Theorem for Graphs (WTG) in your own words.

\item
  Expand the full electronic Hamiltonian $H$ in terms of $\F$-normal-ordered operators using WTG.

\item
  Using KM notation, split the Einstein summations in $H_\mr{c}$ into summations over occupied and virtual indices.
  After combining like terms in your expansion, translate each one into a graph.

\item
  Derive the CIS Hamiltonian matrix as a coefficient graph using WTG.

\item
  Derive the CID energy and coefficient equations using WTG.

\item\label{item:ccd}
  Derive the CCD energy and coefficient (``amplitude'') equations using WTG.

\item
  Compare the CID and CCD equations, and explain the cancellation of unlinked terms in the CCD amplitude equation.

\item\label{item:ccd-final}
  Show that the CCD amplitude equation can be written as
\begin{align}
\label{eq:ccd-amplitude-equations}
  t_{ab}^{ij}
=
  (\mc{E}_{ab}^{ij})^{-1}
  \ip{\F_{ij}^{ab}|
    (H_\mr{c} - f_p^p\tl{a}^p_p)
    \mr{exp}(T_2)
  |\F}_{\mr{L}}
&&
  \mc{E}_{ab}^{ij}
\equiv
  f_i^i
+
  f_j^j
-
  f_a^a
-
  f_b^b
\end{align}
  for a general, possibly non-canonical, reference determinant.
  Write this equation in graphical form using resolvent lines.

\item
  If $\F$ is an excellent approximation to the wavefunction the coupled-cluster amplitudes will be very small, leading to $T_2\approx 0\implies\mr{exp}(T_2)\approx 1$.
  This is the so-called \textit{first-order approximation} of equation~\ref{eq:ccd-amplitude-equations}.
  Write down graphical and algebraic expressions for these \textit{first-order amplitudes}, ${}\ord{1}t_{ab}^{ij}$, assuming $\F$ is a canonical Hartree-Fock reference function.
  Substitute these amplitudes into the CCD energy expression to derive the \textit{second-order energy}, $E_c\ord{2}$.
  This is the MP2 energy expression.\footnote{MP2 stands for ``second-order M\o ller-Plesset perturbation theory''.}

\item
  If the $T_2$ amplitudes are small but not negligible, we can improve upon the first-order approximation using $\mr{exp}(T_2)\approx 1 + T_2$, which is known as linearized CCD or ``CEPA$_0$''.\footnote{CEPA stands for ``coupled electron-pair approximation''.}
  Write down the CEPA$_0$ amplitude equation in graphical form.

\begin{samepage}
\item
  Whereas the MP2 energy can be determined in a single step, the CEPA$_0$ and CCD equations require iterative solution.
  Typically, this is achieved by starting from a guess of $T_2\approx 0$ and repeatedly substituting the amplitudes into the right-hand side of equation~\ref{eq:ccd-amplitude-equations} until self-consistency is reached.\footnote{The numbers in square brackets refer to the iteration.}
\begin{align}
  {}^{[n+1]}t_{ab}^{ij}
=
  (\mc{E}_{ab}^{ij})^{-1}
  \ip{\F_{ij}^{ab}|
    (H_\mr{c} - f_p^p\tl{a}^p_p)
    \mr{exp}({}^{[n]}T_2)
  |\F}_{\mr{L}}
&&
  {}^{[0]}t_{ab}^{ij}
\equiv
  0
\end{align}
  Write down graphical and algebraic expressions for the first- and second-iteration CEPA$_0$ amplitudes, as well as the corresponding energies.
  This procedure can be carried out indefinitely, which shows that the exact CEPA$_0$ and CCD amplitudes contain contributions to infinite order in perturbation theory.
\end{samepage}
\end{enumerate}


\end{document}
