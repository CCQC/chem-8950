\documentclass[11pt]{article}
\usepackage[cm]{fullpage}
%%AVC PACKAGES
\usepackage{avcgreek}
\usepackage{avcfonts}
\usepackage{avcmath}
\usepackage[numberby=section]{avcthm}
\usepackage{qcmacros}
\usepackage{goldstone}
%%MACROS FOR THIS DOCUMENT
\numberwithin{equation}{section}
\usepackage[
  margin=1.5cm,
  includefoot,
  footskip=30pt,
  headsep=0.2cm,headheight=1.3cm
]{geometry}
\usepackage{fancyhdr}
\pagestyle{fancy}
\fancyhf{}
\fancyhead[LE,RO]{Quiz 3, Handout 1: Kutzelnigg-Mukherjee}
\fancyfoot[CE,CO]{\thepage}
\usepackage{url}

\begin{document}

\setlength{\abovedisplayskip}{3pt}
\setlength{\belowdisplayskip}{3pt}

\setcounter{section}{2}
\section{Kutzelnigg-Mukherjee tensor notation}

\begin{ntt}\label{ntt:kutzelnigg-mukherjee-notation}
\thmtitle{Kutzelnigg-Mukherjee tensor notation}
In \textit{Kutzelnigg-Mukherjee (KM) tensor notation}, creation operators are denoted using superscripts, $a^p\equiv a_p\dg$.
Consequently, the one-particle and one-hole density matrices are written as
$
  \g^p_q
\equiv
  \ip{\F|a^pa_q|\F}
$
and
$
  \h_p^q
\equiv
  \ip{\F|a_p a^q|\F}
$,
respectively.
The one- and two-electron integrals are also written with upper and lower indices: lower indices denote spin-orbitals in the bra and upper ones refer to spin-orbitals in the ket.
\begin{align}
  h_p^q
\equiv
  \ip{\y_p|\op{h}|\y_q}
&&
  g_{pq}^{rs}
\equiv
  \ip{pq|rs}
&&
  \ol{g}_{pq}^{rs}
\equiv
  \ip{pq||rs}
\end{align}
Vacuum-normal-ordered excitations are given the compact notation
$
  a^{p_1\cd p_m}_{q_1\cd q_m}
\equiv
  a^{p_1}\cd a^{p_m}a_{q_m}\cd a_{q_1}
$
and $\F$-normal-ordered excitations are written as
$
  \tl{a}^{p_1\cd p_m}_{q_1\cd q_m}
\equiv
  \gno{a^{p_1}\cd a^{p_m}a_{q_m}\cd a_{q_1}}
$.
Using upper and lower indices enables one to employ the \textit{Einstein summation convention}, in which any index that appears twice in a product, once as a lower index and once as an upper one, is implicitly summed over.
As an example, consider the electronic Hamiltonian as expressed in KM notation.
\begin{align}
  H_e
=
  h_p^q
  a^p_q
+
  \tfr{1}{4}
  \ol{g}_{pq}^{rs}
  a^{pq}_{rs}
=
  E_\mr{ref}
+
  H_\mr{c}
&&
  H_{\mr{c}}
=
  f_p^q
  \tl{a}^p_q
+
  \tfr{1}{4}
  \ol{g}_{pq}^{rs}
  \tl{a}^{pq}_{rs}
&&
  E_\mr{ref}
\equiv
  h_p^q
  \g^p_q
+
  \tfr{1}{2}
  \ol{g}_{pr}^{qs}
  \g^p_q
  \g^r_s
&&
  f_p^q
\equiv
  h_p^q
+
  \ol{g}_{pr}^{qs}
  \g_s^r
\end{align}
Here, $E_\mr{ref}$ is the Hartree-Fock reference energy, $f_p^q$ denotes a matrix element of the Fock operator, and $H_\mr{c}$ denotes the correlation component of the Hamiltonian.
More generally, if $\op{v}$ is an \textit{$m$-electron operator}, i.e. an operator that acts on $m$ electronic coordinates, its second quantized form is expressed in KM notation as
\begin{align}\label{eq:interaction-tensor}
  \left.
  \op{v}\,
  \right|_{\mc{F}(\mc{H})}
=&\
  \tfr{1}{m!}
  v_{p_1\cd p_m}^{q_1\cd q_m}
  a^{p_1\cd p_m}_{q_1\cd q_m}
&
  v_{p_1\cd p_m}^{q_1\cd q_m}
\equiv
  \int
  d(1\cd m)\,
  \y_{p_1}^*(1)\cd \y_{p_m}^*(m)
  \op{v}(1,\ld,m)
  \y_{q_1}(1)\cd \y_{q_m}(m)
&
\\
\intertext{
where $v_{p_1\cd p_m}^{q_1\cd q_m}$ is the \textit{interaction tensor} of $\op{v}$.
Equivalently, $\op{v}$ can also be expressed as
}
\label{eq:antisymmetrized-interaction-tensor}
  \left.
  \op{v}
  \right|_{\mc{F}(\mc{H})}
=&\
  (\tfr{1}{m!})^2\,
  \ol{v}_{p_1\cd p_m}^{q_1\cd q_m}
  a^{p_1\cd p_m}_{q_1\cd q_m}
&
  \ol{v}_{p_1\cd p_m}^{q_1\cd q_m}
\equiv
  \sum_\pi^{\mr{S}_m}
  \e_\pi
  v_{p_1\hphantom{_{\pi()}}\cd p_m}^{q_{\pi(1)}\cd q_{\pi(m)}}
\hspace{3cm}&
\end{align}
where $\ol{v}_{p_1\cd p_m}^{q_1\cd q_m}$ is an \textit{antisymmetrized interaction tensor}.
Ordinary interaction tensors are symmetric under simultaneous permutation of upper and lower indices, which is equivalent to changing integration variables in equation \ref{eq:interaction-tensor}.
Antisymmetrized interaction tensors allow for independent permutations of upper and lower indices, with a sign factor corresponding to the parity of the permutation.
The same permutational symmetries are shared by the excitation operators
\begin{align}\label{eq:excitation-permutational-symmetries-1}
  \tl{a}^{p_1\cd p_m}_{q_1\cd q_m}
=
  \e_{\pi}
  \tl{a}^{p_{\pi(1)}\cd p_{\pi(m)}}_{q_1\hphantom{_{\pi()}}\cd q_m}
=
  \e_{\pi}
  \tl{a}^{p_1\hphantom{_{\pi()}}\cd p_m}_{q_{\pi(1)}\cd q_{\pi(m)}}
=
  \tl{a}^{p_{\pi(1)}\cd p_{\pi(m)}}_{q_{\pi(1)}\cd q_{\pi(m)}}
&&
  \text{for all $\pi\in\mr{S}_m$}
\end{align}
since creation operators anticommute with each other, as do annihilation operators.
Note also the following rearrangements
\begin{align}\label{eq:excitation-permutational-symmetries-2}
  \tl{a}^{p_1\cd p_m}_{q_1\cd q_m}
=
  \gno{a^{p_1}_{q_1}\cd a^{p_m}_{q_m}}
&&
  \gno{a^{p_1\cd p_m}_{q_1\cd q_m}a^{r_1\cd r_n}_{s_1\cd s_n}}
=
  \gno{a^{p_1}_{q_1}\cd a^{p_m}_{q_m}a^{r_1}_{s_1}\cd a^{r_n}_{s_n}}
=
  \tl{a}^{p_1\cd p_mr_1\cd r_n}_{q_1\cd q_ms_1\cd s_n}
\end{align}
which follow from the fact that the normal-ordering mapping is antisymmetric with respect to its operator string.
\end{ntt}


\begin{ntt}\label{ntt:dot-notation}
\thmtitle{Dot notation for contractions}
To make the notation more flexible, we here augment the traditional KM notation with the following definitions of \textit{particle $\ptcl$ contractions} and \textit{hole $\hole$ contractions}.\footnote{The dot notation is borrowed from physics: \url{https://en.wikipedia.org/wiki/Wick's_theorem#Definition_of_contraction}}
\begin{align}
&&
  a_{p^\ptcl}a^{q^\ptcl}
\equiv
  \ctr{}{a}{_p}{a} a_pa^q
&&
  a^{q^\ptcl}a_{p^\ptcl}
\equiv
-
  \ctr{}{a}{_p}{a} a_pa^q
&&
  a^{q^\hole}a_{p^\hole}
\equiv
  \ctr{}{a}{^q}{a} a^qa_p
&&
  a_{p^\hole}a^{q^\hole}
\equiv
-
  \ctr{}{a}{^q}{a} a^qa_p
\end{align}
Note that $a^{p^\ptcl}_{q^\ptcl}=-\h^p_q$ and $a^{p^\hole}_{q^\hole}=\g_q^p$.
In $\vac$-normal ordering, the hole contractions vanish and the particle contractions become Kronecker deltas.
For multiply contracted strings, different contractions will be distinguished with repeated dots, $a_{p^{\ptcl\ptcl}}a^{p^{\ptcl\ptcl}}$, or dots with numbers, $a_{p^{\ptcl2}}a^{p^{\ptcl2}}$.
This notation allows for normal ordered strings with contractions to keep all of the permutational symmetries shown in equations \ref{eq:excitation-permutational-symmetries-1} and \ref{eq:excitation-permutational-symmetries-2}.
\end{ntt}

\begin{ex}\label{ex:km-notation-wick-expansions}
Using notations \ref{ntt:kutzelnigg-mukherjee-notation} and \ref{ntt:dot-notation}, the Wick expansions for vacuum-normal single and double excitations in terms of $\F$-normal ones look as follows.
The signs are determined by pairing up contracted indices using \cref{eq:excitation-permutational-symmetries-1,eq:excitation-permutational-symmetries-2}.
\begin{align*}
  a^p_q
=&\
  \tl{a}^p_q
+
  \tl{a}^{p^\hole}_{q^\hole}
=
  \tl{a}^p_q
+
  \g^p_q
\\
  a^{pq}_{rs}
=&\
  \tl{a}^{pq}_{rs}
+
  \tl{a}^{p^\hole q}_{r^\hole s}
+
  \tl{a}^{p^\hole q}_{r\ s^\hole}
+
  \tl{a}^{p\ q^\hole}_{r^\hole s}
+
  \tl{a}^{p q^\hole}_{r s^\hole}
+
  \tl{a}^{p^\hole q^{\hole\hole}}_{r^\hole s^{\hole\hole}}
+
  \tl{a}^{p^{\hole\hphantom{\hole}} q^{\hole\hole}}_{r^{\hole\hole} s^\hole}
\\=&\
  \tl{a}^{pq}_{rs}
+
  \g^p_r
  \tl{a}^{q}_{s}
-
  \g^p_s
  \tl{a}^{q}_{r}
-
  \g^q_r
  \tl{a}^{p}_{s}
+
  \g^q_s
  \tl{a}^{p}_{r}
+
  \g^p_r
  \g^q_s
-
  \g^p_s
  \g^q_r
\end{align*}
\end{ex}

\newpage
\begin{dfn}\label{dfn:riffle-shuffles}
\thmtitle{Riffle shuffle permutations}
Let $\mr{S}_R$ be the \textit{symmetric group on $R\equiv(r_1,\ld,r_k)$}, comprising all $k!$ permutations of this tuple.
For each integer composition\footnote{See \url{https://en.wikipedia.org/wiki/Composition_(combinatorics)}} $(k_1,\ld,k_m)$ of $k$, the \textit{$(k_1,\ld,k_m)$-shuffles~of~$R$} partition the tuple into blocks $R_1\cup\cd\cup R_m$ of the form $R_i\equiv(r_{h_i+1},\ld,r_{h_i+k_i})$ with $h_i\equiv\sum_{j=1}^{i-1}k_j$ and interleave them in all possible ways.
The cardinality of this subset $\mr{S}_R^{(k_1,\ld,k_m)}\subseteq \mr{S}_R$ is given by the following multinomial coefficient
\begin{align}
  |\mr{S}_R^{(k_1,\ld,k_m)}|
\equiv
  \fr{k!}{k_1!\cd k_m!}
\end{align}
since for each shuffle there are $k_1!\cd k_m!$ permutations in $\mr{S}_R$ derived by permuting the elements within each block.
These are termed \textit{riffle shuffle permutations},\footnote{See \url{https://en.wikipedia.org/wiki/Riffle_shuffle_permutation}} by analogy with the process of cutting and interleaving a deck of cards.
\end{dfn}

\begin{dfn}
\thmtitle{Index antisymmetrizers}
Let $\op{P}_{(r_1/\cd/r_k)}$ be the \textit{full antisymmetrizer} for the indices in $R$.\footnote{Note that we are borrowing some definitions from \cref{dfn:riffle-shuffles}, such as $R$ and $R_i$.}\,\footnote{For example, we can express the position-space Slater determinant $\F_{(p_1\cd p_n)}(1,\ld,n)$ as $\fr{1}{\sqrt{n!}}\op{P}_{(p_1/\cd/p_n)}\y_{p_1}(1)\cd \y_{p_n}(n)$.}
More generally, let $\op{P}_{(R_1/\cd/R_m)}$ be a \textit{reduced antisymmetrizer}, which antisymmetrizes a term in $R$ that it is already antisymmetric in each block, $R_i$.
Reduced antisymmetrization is achieved by summing over riffle shuffles.
\begin{align}
  \op{P}_{(R_1/\cd/R_m)}
\equiv
  \sum_\pi^{\mr{S}_R^{(k_1,\ld,k_m)}}
  \e_\pi
  \op{\pi}
&&
  \op{\pi}\,
  t_{r_1\cd r_k}
\equiv
  t_{\pi(r_1)\cd \pi(r_k)}
\end{align}
To antisymmetrize multiple disjoint sets of indices, we use the compact notation
$
  \op{P}^{(X_1|\cd|X_l)}_{(Y_1|\cd|Y_m)}
\equiv
  \op{P}^{(X_1)}\cd \op{P}^{(X_l)}
  \op{P}_{(Y_1)}\cd \op{P}_{(Y_m)}
$,
where each $X_i$ or $Y_i$ stands for a reduced antisymmetrizer argument $R_1/\cd/R_m$ for a different set of indices.
\end{dfn}

\begin{ex}
Permutation operators $\op{\pi}$ can be written as products of transpositions $(pq)t_{pq}\equiv t_{qp}$.
In this notation,
\begin{align*}
  \op{P}_{(p/q)}
=
  1
-
  (pq)
&&
  \op{P}_{(p/q/r)}
=
  1
-
  (pq)
-
  (pr)
-
  (qr)
+
  (pq)(qr)
+
  (pr)(qr)
&&
  \op{P}_{(p/qr)}
=
  1
-
  (pq)
+
  (pr)(qr)
\end{align*}
are the unique antisymmetrizers for two and three indices.
Note that $\op{P}_{(p/qr)}\op{P}_{(q/r)}=\op{P}_{(p/q/r)}$, which is an immediate consequence of the definition of the reduced antisymmetrizer.
\end{ex}

\begin{ex}
Using index antisymmetrizers, the second Wick expansion in \cref{ex:km-notation-wick-expansions} can be expressed even more compactly.
\begin{align*}
  a^{pq}_{rs}
=&\
  \tl{a}^{pq}_{rs}
+
  \op{P}^{(p/q)}_{(r/s)}
  \tl{a}^{p^\hole q}_{r^\hole s}
+
  \op{P}_{(r/s)}
  \tl{a}^{p^\hole q^{\hole\hole}}_{r^\hole s^{\hole\hole}}
=
  \tl{a}^{pq}_{rs}
+
  \op{P}^{(p/q)}_{(r/s)}
  \g^p_r
  \tl{a}^{pq}_{rs}
+
  \op{P}_{(r/s)}
  \g^p_r
  \g^q_s
\end{align*}
Note that permuting $p$ and $q$ in the second term, for example, is equivalent to ``moving the contraction dot'' from one to the other, since the excitation operators are antisymmetric in their upper and lower indices:
$
  (pq)
  \tl{a}^{p^\hole q}_{r^\hole s}
=
-
  \tl{a}^{q^\hole p}_{r^\hole s}
=
  \tl{a}^{p\,\, q^\hole}_{r^\hole s}
$.
In general, this allows us to reduce Wick expansions to a sum over unique contraction ``patterns'', with the remaining contractions generated from these by index antisymmetrizers.
The next two examples show how this works in practice.
\end{ex}

\begin{ex}
Reduced antisymmetrizers come into play in the Wick expansion of a triple excitation operator.
\begin{align}
  a^{pqr}_{stu}
=
  \tl{a}^{pqr}_{stu}
+
  \op{P}^{(p/qr)}_{(s/tu)}
  \tl{a}^{p^\hole qr}_{s^\hole tu}
+
  \op{P}^{(p/q/r)}_{(st/u)}
  \tl{a}^{p^\hole q^{\hole\hole} r}_{s^\hole t^{\hole\hole} u}
+
  \op{P}^{(p/q/r)}
  \tl{a}^{p^\hole q^{\hole\hole} r^{\hole\hole\hole}}
        _{s^\hole t^{\hole\hole} u^{\hole\hole\hole}}
\end{align}
The permutation factors are chosen as follows.
In the second term, permutations of $qr$ an $tu$ are omitted because they do not produce unique contractions.
In the third term, we can include permutations of either $pq$ or $st$ but not both.
Otherwise, we would double-count terms like
$
  \tl{a}^{p^\hole q^{\hole\hole} r}_{s^\hole t^{\hole\hole} u}
$
by also including
$
  \tl{a}^{p^{\hole\hole} q^\hole r}_{s^{\hole\hole} t^\hole u}
$,
which represents the same contraction.
The same reasoning applies to the last term, where we can either antisymmetrize $pqr$ or $stu$.
\end{ex}



\begin{ex}
\thmtitle{Derivation of CIS equations in KM notation}
A programmable expression for the CI singles Hamiltonian matrix elements, $\ip{\F_i^a|H_c|\F_j^b}$,  can be derived in KM notation as follows.
\begin{align*}
&
  \ip{\F|\tl{a}_a^i\tl{a}_q^p\tl{a}_j^b|\F}
=
  \gno{
    a_{a^\ptcl}^{i^\hole}
    a_{q^{\ptcl\ptcl}}^{p^\ptcl}
    a_{j^\hole}^{b^{\ptcl\ptcl}}
  }
+
  \gno{
    a_{a^\ptcl}^{i^{\hole}}
    a_{q^{\hole}}^{p^{\hole\hole}}
    a_{j^{\hole\hole}}^{b^{\ptcl}}
  }
=
  \g^i_j\h_a^p\h_q^b
-
  \g_q^i\g_j^p\h_a^b
\\
&
  \ip{\F|\tl{a}_a^i\tl{a}^{pq}_{rs}\tl{a}_j^b|\F}
=
  \op{P}^{(p/q)}_{(r/s)}
  \gno{
    a^{i^\hole}_{a^\ptcl}
    a^{p^\ptcl q^{\hole\hole}}_{r^\hole s^{\ptcl\ptcl}}
    a^{b^{\ptcl\ptcl}}_{j^{\hole\hole}}
  }
=
  \op{P}^{(p/q)}_{(r/s)}
  \g^i_r\h^p_a\g^q_j\h^b_s
\\
&
\implies
  \ip{\F_i^a|H_\mr{c}|\F_j^b}
=
  f_p^q\pr{
    \g^i_j\h_a^p\h_q^b
  -
    \g_q^i\g_j^p\h_a^b
  }
+
  \tfr{1}{4}
  \ol{g}_{pq}^{rs}\pr{
    \op{P}^{(p/q)}_{(r/s)}
    \g^i_r\h^p_a\g^q_j\h^b_s
  }
=
  f_a^b\g_j^i
-
  f_j^i\h_a^b
+
  \ol{g}^{ib}_{aj}
\end{align*}
which simplifies to
$\ip{\F_i^a|H_\mr{c}|\F_j^b}=f_a^b\d_j^i-f_j^i\d_a^b+\ol{g}^{ib}_{aj}$.
Here, we have used the fact that $\tl{a}_i^a=a_i^a$ and $\tl{a}_a^i=a_a^i$.
\end{ex}

\begin{ex}
\begin{samepage}
\thmtitle{Derivation of CID equations in KM notation}
Projecting the CI doubles Schr\"odinger equation, $H_\mr{c}\Y=E_\mr{c}\Y$ where $\Y=(1+\tfr{1}{4}c_{cd}^{kl}\tl{a}_{kl}^{cd})\F$, by $\F$ and $\F_{ij}^{ab}$
gives a system of linear equations
\begin{align*}
  E_\mr{c}
=&\
  \ip{\F|H_\mr{c}(1+\tfr{1}{4}c_{cd}^{kl}\tl{a}_{kl}^{cd})|\F}
&&
  \implies
&
  E_\mr{c}
=&\
  \tfrac{1}{4}
  \ip{\F|H_c|\F_{kl}^{cd}}c_{cd}^{kl}
\\
  E_\mr{c}c_{ab}^{ij}
=&\
  \ip{\F_{ij}^{ab}|H_\mr{c}(1+\tfr{1}{4}c_{cd}^{kl}\tl{a}_{kl}^{cd})|\F}
&&
  \implies
&
  E_\mr{c}c_{ab}^{ij}
=&\
  \ip{\F_{ij}^{ab}|H_\mr{c}|\F}
+
  \tfrac{1}{4}
  \ip{\F_{ij}^{ab}|H_\mr{c}|\F_{kl}^{cd}}c_{cd}^{kl}
\end{align*}
which can be simplified into programmable expressions as follows.\\[5pt]
$
\begin{array}{rl}
  \ip{\F|\tl{a}^{pq}_{rs}\tl{a}_{kl}^{cd}|\F}
=&
  \op{P}^{(p/q)}_{(r/s)}
  \g_k^p\g_l^q\h_r^c\h_s^d
\\
  \ip{\F|\tl{a}_{ab}^{ij}\tl{a}_q^p\tl{a}_{kl}^{cd}|\F}
=&
  \op{P}^{(c/d)}
        _{(a/b|k/l)}
  \gno{
    a_{a^{\ptcl1}b^{\ptcl3}}
     ^{i^{\hole1}j^{\hole2}}
    a_{q^{\ptcl2}}
     ^{p^{\ptcl1}}
    a_{k^{\hole1}l^{\hole2}}
     ^{c^{\ptcl2}d^{\ptcl3}}
  }
+
  \op{P}^{(i/j|c/d)}
        _{(k/l)}
  \gno{
    a_{a^{\ptcl1}b^{\ptcl2}}
     ^{i^{\hole1}j^{\hole3}}
    a_{q^{\hole1}}
     ^{p^{\hole2}}
    a_{k^{\hole2}l^{\hole3}}
     ^{c^{\ptcl1}d^{\ptcl2}}
  }
\\=&
  \op{P}^{(c/d)}
        _{(a/b|k/l)}
  \h_a^p\h_q^c\h_b^d\g_k^i\g_l^j
-
  \op{P}^{(i/j|c/d)}
        _{(k/l)}
  \g_q^i\g_k^p\g_l^j\h_a^c\h_b^d
\\
  \ip{\F|\tl{a}_{ab}^{ij}\tl{a}_{rs}^{pq}\tl{a}_{kl}^{cd}|\F}
=&
  \op{P}^{(c/d)}
        _{(a/b|k/l)}
  \gno{
    a_{a^{\ptcl1}b^{\ptcl2}}
     ^{i^{\hole1}j^{\hole2}}
    a_{r^{\ptcl3}s^{\ptcl4}}
     ^{p^{\ptcl1}q^{\ptcl2}}
    a_{k^{\hole1}l^{\hole2}}
     ^{c^{\ptcl3}d^{\ptcl4}}
  }
+
  \op{P}^{(i/j|c/d)}
        _{(k/l)}
  \gno{
    a_{a^{\ptcl1}b^{\ptcl2}}
     ^{i^{\hole1}j^{\hole2}}
    a_{r^{\hole1}s^{\hole2}}
     ^{p^{\hole3}q^{\hole4}}
    a_{k^{\hole3}l^{\hole4}}
     ^{c^{\ptcl1}d^{\ptcl2}}
  }
+
  \op{P}^{(p/q|i/j|c/d)}
        _{(r/s|k/l|a/b)}
  \gno{
    a_{a^{\ptcl1}b^{\ptcl3}}
     ^{i^{\hole1}j^{\hole3}}
    a_{r^{\hole1}s^{\ptcl2}}
     ^{p^{\hole2}q^{\ptcl1}}
    a_{k^{\hole2}l^{\hole3}}
     ^{c^{\ptcl2}d^{\ptcl3}}
  }
\\=&
  \op{P}^{(c/d)}
        _{(a/b|k/l)}
  \h_a^p\h_b^q\h_r^c\h_s^d
  \g_k^i\g_l^j
+
  \op{P}^{(i/j|c/d)}
        _{(k/l)}
  \g_r^i\g_s^j\g_k^p\g_l^q
  \h_a^c\h_b^d
-
  \op{P}^{(p/q|i/j|c/d)}
        _{(r/s|k/l|a/b)}
  \g_r^i\g_k^p\g_l^j
  \h_a^q\h_s^c\h_b^d
\end{array}$
\begin{align*}
&
  \implies
&&
  E_\mr{c}
=
  \tfr{1}{4}
  \ol{g}_{kl}^{cd}c_{cd}^{kl}
&&
  E_\mr{c}c_{ab}^{ij}
=
  \ol{g}_{ab}^{ij}
+
  \op{P}_{(a/b)}
  f_a^c
  c_{cb}^{ij}
-
  \op{P}^{(i/j)}
  f_k^i
  c_{ab}^{kj}
+
  \tfrac{1}{2}
  \ol{g}_{ab}^{cd}
  c_{cd}^{ij}
+
  \tfrac{1}{2}
  \ol{g}_{kl}^{ij}
  c_{ab}^{kl}
+
  \op{P}^{(i/j)}
        _{(a/b)}
  \ol{g}_{ak}^{ic}
  c_{bc}^{jk}
\end{align*}
\end{samepage}
\end{ex}



\end{document}
