\documentclass[11pt]{article}
\usepackage[cm]{fullpage}
%%AVC PACKAGES
\usepackage{avcgreek}
\usepackage{avcfonts}
\usepackage{avcmath}
\usepackage[numberby=section]{avcthm}
\usepackage{qcmacros}
\usepackage{goldstone}
%%MACROS FOR THIS DOCUMENT
\numberwithin{equation}{section}
\usepackage[
  margin=1.5cm,
  includefoot,
  footskip=30pt,
  headsep=0.2cm,headheight=1.3cm
]{geometry}
\usepackage{fancyhdr}
\pagestyle{fancy}
\fancyhf{}
\fancyhead[LE,RO]{\textbf{Quiz 2}}
\fancyfoot[CE,CO]{\thepage}
\usepackage{url}

\begin{document}

\begin{enumerate}
\item
  In solving the canonical Hartree-Fock equations, it is common to use eigenfunctions of the core Hamiltonian as starting guesses for the canonical molecular orbitals.
  This eigenvalue equation looks as follows
  \begin{align*}
    \op{h}
    \f_{p}
  =
    \ev_{p}
    \f_{p}
  \end{align*}
  where $\f_p$ is a spatial function.
  Project this equation by $\x_\mu$ and then expand $\f_p$ as a linear combination of atomic orbitals, $\f_p=\sum_\nu \x_\nu\, C_{\nu p}$, in order to arrive at the following matrix equation for the core guess.
  \begin{align*}
    \bo{H}\bo{C}
  =
    \bo{S}\bo{C}\bm{\ev}
  &&
    (\bo{H})_{\mu\nu}
  =
    \ip{\x_\mu|\op{h}|\x_\nu}
  &&
    (\bo{S})_{\mu\nu}
  =
    \ip{\x_\mu|\x_\nu}
  &&
    (\bo{C})_{\mu p}
  =
    C_{\mu p}
  &&
    (\bm{\ev})_{pq}
  =
    \ev_p\d_{pq}
  \end{align*}
  For extra credit, you may also briefly answer the following question in words: For an open-shell system, why do we end up with different spatial orbitals for different spins, even though the initial core-guess orbitals are the same for each spin?\\[1cm]
  \textbf{Answer}:
  \begin{align*}
  \left.
  \begin{array}{r@{\ }l}
    \ip{\x_\mu|\op{h}|\f_p}
  &=
  \ds{
    \sum_\nu
    \ip{\x_\mu|\op{h}|\x_\nu}
    C_{\nu p}
  }
  =
    (\bo{H}\bo{C})_{\mu p}
  \\
    \ip{\x_\mu|\f_p}
    \ev_p
  &=
  \ds{
    \sum_\nu
    \ip{\x_\mu|\x_\nu}
    C_{\nu p}
    \ev_p
  }
  =
    (\bo{S}\bo{C}\bm{\ev})_{\mu p}
  \end{array}
  \right\}
  \implies
    \bo{H}\bo{C}
  =
    \bo{S}\bo{C}\bm{\ev}
  \end{align*}
  Since there are different numbers of occupied $\a$ and $\b$ orbitals, the $\a$ and $\b$ density matrices will differ even for matching MO coefficients.
  This results in different exchange fields for $\op{f}_\a$ and $\op{f}_\b$, breaking the orbital degeneracy.


\newpage
\item
  Derive the expansion of $H_e$ in terms of $\F$-normal-ordered excitations.\\[1cm]
  \textbf{Answer}:
  \begin{align*}
  &
    \sum_{pq}
    h_{pq}
    a_p\dg a_q
  +
    \tfr{1}{4}
    \sum_{pqrs}
    \ip{pq||rs}
    a_p\dg a_q\dg a_s a_r
  \\&=
    \sum_{pq}
    h_{pq}
    (
      \gno{
        a_p\dg a_q
      }
    +
      \gno{
        \ctr[0.5]{}{a}{_p\dg}{a}
        a_p\dg a_q
      }
    )
  \\
  &\ +
    \tfr{1}{4}
    \sum_{pqrs}
    \ip{pq||rs}
    \left(
      \gno{
        a_p\dg a_q\dg a_s a_r
      }
    +
      \gno{
        \ctr[0.5]{}{a}{_p\dg a_q\dg}{a}
        a_p\dg a_q\dg a_s a_r
      }
    +
      \gno{
        \ctr[0.5]{}{a}{_p\dg a_q\dg a_s}{a}
        a_p\dg a_q\dg a_s a_r
      }
    +
      \gno{
        \ctr[0.5]{a_p\dg}{a}{_q\dg}{a}
        a_p\dg a_q\dg a_s a_r
      }
    +
      \gno{
        \ctr[0.5]{a_p\dg}{a}{_q\dg a_s}{a}
        a_p\dg a_q\dg a_s a_r
      }
    +
      \gno{
        \ctr[1.5]{a_p\dg }{a}{_q\dg a_s}{a}
        \ctr[0.5]{}{a}{_p\dg a_q\dg}{a}
        a_p\dg a_q\dg a_s a_r
      }
    +
      \gno{
        \ctr[1.5]{}{a}{_p\dg a_q\dg a_s}{a}
        \ctr[0.5]{a_p\dg }{a}{_q\dg}{a}
        a_p\dg a_q\dg a_s a_r
      }
    \right)
  \\&=
    \sum_{pq}
    h_{pq}
    \g_{pq}
  +
    \tfr{1}{4}
    \sum_{pqrs}
    \ip{pq||rs}
    \g_{pr}\g_{qs}
  -
    \tfr{1}{4}
    \sum_{pqrs}
    \ip{pq||rs}
    \g_{ps}\g_{qr}
  +
    \tfr{1}{4}
    \sum_{pqrs}
    \ip{pq||rs}
    \gno{a_p\dg a_q\dg a_s a_r}
  +
    \sum_{pq}
    h_{pq}
    \gno{a_p\dg a_q}
  \\
  &\ -
    \tfr{1}{4}
    \sum_{pqrs}
    \ip{pq||rs}
    \g_{ps}
    \gno{a_q\dg a_r}
  +
    \tfr{1}{4}
    \sum_{pqrs}
    \ip{pq||rs}
    \g_{pr}
    \gno{a_q\dg a_s}
  +
    \tfr{1}{4}
    \sum_{pqrs}
    \ip{pq||rs}
    \g_{qs}
    \gno{a_p\dg a_r}
  -
    \tfr{1}{4}
    \sum_{pqrs}
    \ip{pq||rs}
    \g_{qr}
    \gno{a_p\dg a_s}
  \\&=
  \underset{\ds{E_\mr{ref}}}{\underbrace{
    \sum_{pq}
    h_{pq}\g_{pq}
  +
    \tfr{1}{2}
    \sum_{pqrs}
    \ip{pq||rs}
    \g_{pr}\g_{qs}
  }}
  +
    \tfr{1}{4}
    \sum_{pqrs}
    \ip{pq||rs}
    \gno{a_p\dg a_q\dg a_s a_r}
  +
    \sum_{pq}
    \underset{\ds{f_{pq}}}{\underbrace{
    (
      h_{pq}
    +
      \sum_{rs}
      \ip{pr||qs}
      \g_{rs}
    )
    }}
    \gno{a_p\dg a_q}
  \end{align*}
  In the final step, we have relabeled summation indices and made use of the antisymmetry of $\ip{pq||rs}$ in order to combine the second and third terms as well as the last four terms.

\newpage
\item
  Derive the second Slater rule using Wick's theorem, and explain why this matrix element evaluates to zero for canonical Hartree-Fock orbitals.
  \begin{align*}
    \ip{\F|H_e|\F_i^a}
  =
    \,\,?
  \end{align*}
  \\
  \textbf{Answer}:
  Note that $a_a\dg a_i=\gno{a_a\dg a_i}$, since this pair is already in $\F$-normal order.
  \begin{align*}
    \ip{\F|H_e\gno{a_a\dg a_i}|\F}
  =&\
    E_\mr{ref}
    \ip{\F|\gno{a_a\dg a_i}|\F}
  +
    \sum_{pq}
    f_{pq}
    \ip{\F|\gno{a_p\dg a_q}\gno{a_a\dg a_i}|\F}
  +
    \tfr{1}{4}
    \sum_{pq}
    \ip{pq||rs}
    \ip{\F|\gno{a_p\dg a_q\dg a_s a_r}\gno{a_a\dg a_i}|\F}
  \\=&\
    \sum_{pq}
    f_{pq}
    \gno{
      \ctr[1]{}{a}{_p\dg a_q a_a\dg}{a}
      \ctr[0]{a_p\dg}{a}{_q}{a}
      a_p\dg a_q a_a\dg a_i
    }
  \\=&\
    \sum_{pq}
    f_{pq}
    \g_{pi}
    \h_{qa}
  \\=&\
    f_{ia}
  \end{align*}
  On the right-hand side of the first line, term one vanishes because $a_a\dg a_i$ is in $\F$-normal order and term three vanishes because there are no complete cross-contractions between two strings with different numbers of operators.\\
  Projecting the canonical Hartree-Fock equation by some $\y_q$ leads to
  \begin{align*}
    f_{pq}
  =
    \ip{\y_q|\op{f}|\y_p}
  =
    \ev_p\ip{\y_q|\y_p}
  =
    \ev_p
    \d_{qp}
  \end{align*}
  due to spin-orbital orthonormality, showing that the Fock matrix is diagonal and $f_{ia}=0$.
\end{enumerate}


\end{document}
