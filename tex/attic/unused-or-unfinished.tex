\documentclass[11pt]{article}
\usepackage[cm]{fullpage}
%%AVC PACKAGES
\usepackage{avcgreek}
\usepackage{avcfonts}
\usepackage{avcmath}
\usepackage[numberby=section,skip=9pt plus 2pt minus 7pt]{avcthm}
\usepackage{qcmacros}
\usepackage{goldstone}
%%MACROS FOR THIS DOCUMENT
\numberwithin{equation}{section}
\usepackage[
  margin=1.5cm,
  includefoot,
  footskip=30pt,
  headsep=0.2cm,headheight=1.3cm
]{geometry}
\usepackage{fancyhdr}
\pagestyle{fancy}
\fancyhf{}
\fancyhead[LE,RO]{Quiz 7, Handout 1: Perturbative analysis}
\fancyfoot[CE,CO]{\thepage}
\usepackage{url}
\makeatother
\newcommand{\resolventline}[2][1]{
  \tikz[overlay]{
      \draw[thick,flexdotted] (0,-1ex) to ++(0,#1*4.5ex) node[above,inner sep=1pt] {#2};
  }
}

\begin{document}


\appendix
\section{Fa\`a di Bruno's formula}

\begin{thm}
\thmtitle{Fa\`a di Bruno's formula}
\begin{align}
  \pd{^n}{x_1\cd \pt x_n}
  f(g(\bm{x}))
=
  \sum_{k=1}^n
  \sum_{(\bm{x}_1,\ld,\bm{x}_k)}^{\mc{P}_k(\bm{x})}
  f\ord{k}(g(\bm{x}))
  \prod_{i=1}^k
  \pd{
    ^{|\bm{x}_i|}
    g(\bm{x})
  }{
    x_{i,1}
  \cd
    \pt
    x_{i,|\bm{x}_i|}
  }
\end{align}
\end{thm}



\section{Frantz-Mills factorization theorem}

\begin{dfn}
\label{dfn:resolvent-graph}
\thmtitle{Resolvent graph}
A \textit{resolvent graph} represents a product of operators and resolvents.
Here we restrict this term to graphs where each resolvent line spans the width of the diagram, partitioning its operators into distinct \textit{levels}.
Levels are numbered starting from the bottom with zero indexing, so an operator lies in the $k\eth$ level if there are $k$ resolvent lines below it.
Formally, an \textit{$m$-level resolvent graph} is denoted by $G(\rh,m)\equiv(G,\rh,m)$ and associates each operator $o$ in $G$ with a specific level $\rh(o)=\rh_o$ in $\mb{Z}_m=\{0,1,\ld,m-1\}$ through the \textit{level map} $\rh$.\,\footnote{
  Note that an $m$-level resolvent graph contains $m-1$ resolvents.
}
This identifies each line $l$ in $G$ as crossing the resolvents
$
  r
\in
  \{1,\ld,m-1\}
$
satisfying
$
  \mr{min}(\rh_{h(l)},\rh_{t(l)})
<
  r
\leq
  \mr{max}(\rh_{h(l)},\rh_{t(l)})
$.
\end{dfn}


\begin{dfn}
\thmtitle{Substitution}
Let $G[H/o\,]$ be the graph formed by \textit{substituting} the $m$-electron operator $o$ in $G$ with a connected graph $H$ containing $m$ open cycles.
This operation is well-defined as long as there is only one way to embed $H$ in $G$.
Denoting this substituted graph by $K$ we can also perform the inverse operation, $K[o/H]$, to arrive back at $G$.
\end{dfn}

\begin{dfn}

\end{dfn}

\begin{rmk}
A general graph containing resolvent lines can be recursively defined as a resolvent graph composed of resolvent graph modules \cref{dfn:resolvent-graph} 

collapsing the subgraphs that do qualify as resolvent graphs into intermediate operators.
\end{rmk}

\begin{dfn}
\thmtitle{Module}
A \textit{module} of a graph is a subgraph $G[M]$ such that for each pair $m,m'\in M$, every $o\in M\setminus O$ is either adjacent to both or adjacent to neither.\footnote{See \url{https://en.wikipedia.org/wiki/Modular_decomposition}}
\end{dfn}

\begin{dfn}
\thmtitle{Module}
The \textit{subtitution} 
\end{dfn}

\begin{dfn}
\thmtitle{Substitution}
If $o$ is an $m$-electron operator in $G$ and $H$ is a connected subgraph containing $m$ open cycles, let $G[H/o\,]$ be the \textit{substitution} of $o$ with $H$.
\end{dfn}


\begin{dfn}
\thmtitle{Composition of modules}
A \textit{module} of a graph is a connected subgraph.

\end{dfn}

\begin{rmk}
A product of graphs $G=(L,O,h,t)$ and $G'=(L',O',h',t')$ is itself a graph, which is formally given by
\begin{align}
  GG'
=
  (L\cup L', O\cup O', h\oplus h', t\oplus t')
\end{align}
where $h\oplus h'$ acts as $h$ on lines from $L$ and as $h'$ on lines from $L'$.
The combined tail function is defined similarly.
\end{rmk}

\begin{ntt}
Given
$
  G(\rh,m)
$,
let
$
  G_k(\rh_k,k+1)
$
denote that we have collapsed the subgraph formed from $k$ and up into an intermediate operator
$
  G[O_{\geq k}](\rh-k,m-k)
$
into an intermediate operator
$
  o_{\geq k}
$
in level
$
  \rh_k(o_{\geq k})
=
  k
$
and with
$
  \rh_k(o)
=
  \rh(o)
$
for all
$
  o
\in
  O
  \setminus
  O_{\geq k}
$.
\end{ntt}

\begin{dfn}
\label{dfn:resolvent-graph}
\thmtitle{Zipper graph}
A \textit{$(k,k')$-joined zipper graph} of $G(\rh,m)$ and $G'(\rh',m')$ 
\begin{align}
  G_k
  G_{k'}'
  (
    \rh_\pi^{k,k'},\
    k+k'
  )
&&
  \rh
=
\left\{
\begin{array}{r@{\ }l}
\end{array}
\right.
&&
  \pi
\in
  \mr{S}_{\mb{Z}_{k+k'}}^{(k,k')}
\end{align}
\end{dfn}

\begin{dfn}
\label{dfn:insertion-graph}
\thmtitle{Insertion graph}
\end{dfn}

\begin{dfn}\label{dfn:resolvent-graph}
\thmtitle{Resolvent graph}
A \textit{resolvent graph} represents a normal-ordered product of operators and resolvents.
Graphs with disconnected parts that don't share any resolvent lines are considered products of separate resolvent graphs.
Vertical spaces between resolvent lines in a resolvent graph are termed \textit{levels}, which are numbered from bottom with zero indexing.
Therefore, an operator lies in the $k\eth$ level if there are $k$ resolvent lines below it.
Formally, then, an \textit{$m$-level resolvent graph} $G(\rh,m)\equiv(G,\rh,m)$ associates each operator $o$ in $G$ with a specific level $\rh(o)=\rh_o$ in $\mb{Z}_m=\{0,1,\ld,m-1\}$ through the \textit{level map} $\rh$.\,\footnote{
  Note that an $m$-level resolvent graph contains $m-1$ resolvents.
}
Therefore, each line $l$ in $G$ crosses resolvents
$
  \mr{min}(\rh_{h(l)},\rh_{t(l)}) + 1
$
through
$
  \mr{max}(\rh_{h(l)},\rh_{t(l)})
$.
\end{dfn}



\begin{dfn}
According to definition~\ref{dfn:resolvent-graph}, however, a product of resolvent graphs is not a resolvent graph.
There are several \ld\\
The \textit{combination graphs} of $G(\rh,m)$ and $G'(\rh',m')$ have the form
$GG'(\rh_{\pi,\si}^{k,k'},m+m'-1)$\\
\begin{align*}
\begin{array}{r@{\ }l}
  \rh_{\pi,\si}^{k,k'}(o)
=&
\left\{
\begin{array}{ll}
  \pi(\rh(o))      & \rh(o)<k \\
  k + k'           & \rh(o)=k \\
  \makebox[2.8cm][l]{$\si(\rh(o) + k')$} & \rh(o)>k \\
\end{array}
\right.
\\[20pt]
  \rh_{\pi,\si}^{k,k'}(o')
=&
\left\{
\begin{array}{ll}
  \pi(\rh'(o') + k)   & \rh(o')<k' \\
  k + k'              & \rh(o')=k' \\
  \makebox[2.8cm][l]{$\si(\rh'(o') + m-1)$} & \rh(o')>k' \\
\end{array}
\right.
\end{array}
\hspace{20pt}
\begin{array}{r@{\ }l}
  o\,
\in&
  O
\\[5pt]
  o'
\in&
  O'
\\[5pt]
  \pi\,
\in&
  \mr{L}_{k,k'}
\\[5pt]
  \si\,
\in&
  \mr{U}_{k,k'}^{m,m'}
\end{array}
\end{align*}
where
$
  \mr{L}_{k,k'}
\equiv
  \mr{S}^{(k,k')}_{\mb{Z}_{k+k'}}
$
and
$
  \mr{U}_{k,k'}^{m,m'}
\equiv
  \mr{S}^{(m-k-1,m'-k'-1)}_{\mb{Z}_{m'+m'-k-k'-2} + k + k' + 1}
$
are interleavings of the levels below and above level $k$ and $k'$ in the respective graphs.
\end{dfn}

\begin{thm}
\thmtitle{Frantz-Mills Factorization Theorem}
\begin{align}
  G(\rh,m)
  G'(\rh',m')
=
  \sum_\pi^{\mr{L}_{k,k'}}
  GG'(\rh_{\pi,\si}^{k,k'},m+m'-1)
&&
\begin{array}{r@{\ }l@{\,}l}
  0\leq& k &<m \\[2pt]
  0\leq& k'&<m'
\end{array}
\hspace{10pt}
  \si\in\mr{U}_{k,k'}^{m,m'}
\end{align}
\end{thm}

\begin{dfn}
\thmtitle{Insertion graph}
\end{dfn}

%\begin{dfn}
%\thmtitle{Insertion graph}
%The graphs encountered in perturbation theory contain operators separated by resolvents.
%Two disconnected parts are termed \textit{separate} if no resolvent line crosses both.
%Vertical spaces between resolvent lines are termed \textit{levels}, which we number from %top to bottom for each separate part.
%A pair of resolvents with no intervening operators encloses an \textit{empty level}.
%Inserted brackets in the bracketing expansion produce separate and unlinked \textit{insertion graphs}.
%The empty level in the \textit{remainder} created by the insertion is the \textit{level of the insertion}.
%\end{dfn}



\end{document}