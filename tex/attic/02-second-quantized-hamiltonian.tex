\documentclass[11pt]{article}
\usepackage[cm]{fullpage}
%%AVC PACKAGES
\usepackage{avcgreek}
\usepackage{avcfonts}
\usepackage{avcmath}
\usepackage[numberby=section]{avcthm}
\usepackage{qcmacros}
\usepackage{goldstone}
%%MACROS FOR THIS DOCUMENT
\numberwithin{equation}{section}
\usepackage[
  margin=1.5cm,
  includefoot,
  footskip=30pt,
  headsep=0.2cm,headheight=1.3cm
]{geometry}
\usepackage{fancyhdr}
\pagestyle{fancy}
\fancyhf{}
\fancyhead[LE,RO]{The second quantized Hamiltonian}
\fancyfoot[CE,CO]{\thepage}
\usepackage{url}

\begin{document}

\setlength{\abovedisplayskip}{8pt}
\setlength{\belowdisplayskip}{8pt}

\setcounter{section}{1}
\section{The second quantized Hamiltonian}

The space of $n$-electron Slater determinants, $\mc{F}_n$, spans all $n$-particle wavefunctions that obey Fermion exchange symmetry.
Consider an operator $\op{a}_p$, which maps $\mc{F}_n$ into $\mc{F}_{n-1}$ through the following weighted integral.\footnote{Implicitly, this is a definite integral over all values of $(1')\equiv(\bo{r}_{1'},s_{1'})$.}
\begin{align}
  (\op{a}_p\Y)(2,\cd,n)
\equiv
  \sqrt{n}\int d(1')\, \y_p^*(1')\Y(1',2,\cd,n)
\end{align}
Using the properties of determinants, one can show that $\op{a}_p$ deletes $\y_p$ from any $\F_{(p_1\cd p_n)}$ with $p_k=p$
\begin{align}
  (\op{a}_p\F_{(p_1\cd p_n)})(2,\cd,n)
=
\tfr{1}{\sqrt{(n-1)!}}
\left|\ar{
  0 & \cd & 1 & \cd & 0\\
  \y_{p_1}(2) & \cd &\y_{p_k}(2)&\cd&\y_{p_n}(2)\\
  \vd  & \dd    &\vd    &\dd&\vd    \\
  \y_{p_1}(n) & \cd &\y_{p_k}(n)&\cd&\y_{p_n}(n)}\right|
=
  (-)^{k-1}\F_{(p_1\cd \cancel{p_k}\cd p_n)}(2,\ld,n)
\end{align}
and kills determinants not containing $\y_p$.
This is known as an \textit{annihilation operator}.
The following identity\,\footnote{%
\label{fn:swap-integration-variables}%
  This follows from swapping integration variables $1'\leftrightarrow2'$ and using
  $
    \Y(1',2',\cd)
  =
  -
    \Y(2',1',\cd)
  $.
}
\begin{align}
\label{eq:spinorb-integrals-anticommute}
  \int d(1')d(2')\y_p^*(1')\y_q^*(2')\Y(1',2',\cd)
=
-
  \int d(1')d(2')\y_q^*(1')\y_p^*(2')\Y(1',2',\cd)
&&
  \Y\in\mc{F}_n
\end{align}
implies that annihilation operators for different spin-orbitals anticommute.
\begin{align}
  \op{a}_p
  \op{a}_q
=
-
  \op{a}_q
  \op{a}_p
\end{align}
By resolution of the identity along a single electronic coordinate, the annihilation operators can be used to generate the following decomposition of an antisymmetric function.
\begin{align}
\label{eq:annihilation-operator-resolution-of-the-identity}
  \Y(1,\cd,n)
=
  \sum_{p=1}^\infty
  \y_p(1)
  \int
  d(1')
  \y_p^*(1')
  \Y(1',\cd,n)
=
  \tfr{1}{\sqrt{n}}
  \sum_p^\infty\y_p(1)\pr{\op{a}_p\Y}(2,\cd,n)
\end{align}
The next paragraph shows that the electronic Hamiltonian itself can be expressed in terms of these operators.

Matrix elements of the Hamiltonian with respect to antisymmetric functions can be expressed as
\begin{align}
  \ip{\Y|\op{H}|\Y'}
=
  \sum_{i=1}^n\ip{\Y|\op{h}(i)|\Y'}
+
  \sum_{i<j}^n\ip{\Y|\op{g}(i,j)|\Y'}
=
  n\ip{\Y|\op{h}(1)|\Y'}
+
  \tfr{n(n-1)}{2}
  \ip{\Y|\op{g}(1,2)|\Y'}
\end{align}
by the algebraic maneuver of \cref{fn:swap-integration-variables}.
Repeated use of equation~\ref{eq:annihilation-operator-resolution-of-the-identity} leads to
\begin{align*}
  \ip{\Y|\op{H}|\Y'}
=
  \sum_{pq}^\infty
  \ip{\y_p|\op{h}|\y_q}\,\ip{\op{a}_p\Y|\op{a}_q\Y'}
+
  \tfr{1}{2}
  \sum_{pqrs}^\infty
  \ip{\y_p\y_q|\y_r\y_s}\,\ip{\op{a}_q\op{a}_p\Y|\op{a}_s\op{a}_r\Y'}
\end{align*}
and since $\Y$ and $\Y'$ are arbitrary, this allows us to back out a new expression for $\op{H}$ acting on $\mc{F}_n$.
\begin{align}\label{eq:second-quantized-hamiltonian}
  \left.
  \op{H}
  \right|_{\mc{F}_n}
=
  \sum_{pq}^\infty
  \ip{\y_p|\op{h}|\y_q}\,
  \op{a}_p\dg \op{a}_q
+
  \tfr{1}{2}
  \sum_{pqrs}^\infty
  \ip{\y_p\y_q|\y_r\y_s}\,
  \op{a}_p\dg\op{a}_q\dg\op{a}_s\op{a}_r
\end{align}
This is the \textit{second quantized} form of the Hamiltonian, as opposed to the \textit{first quantized} form which is not restricted to $\mc{F}_n$.
A defining feature of the second quantized formalism is that $\op{H}$ has the same form for any number of electrons, $n$, allowing us to extend its domain to include states with different numbers of particles,
$
  \mc{F}_0
\cup
  \mc{F}_1
\cup
  \mc{F}_2
\cup
  \cd
\cup
  \mc{F}_\infty
$,
which is not possible for the first-quantized form of the Hamiltonian.




\end{document}
