\documentclass[11pt]{article}
\usepackage[cm]{fullpage}
%%AVC PACKAGES
\usepackage{avcgreek}
\usepackage{avcfonts}
\usepackage{avcmath}
\usepackage[numberby=section,skip=9pt plus 2pt minus 5pt]{avcthm}
\usepackage{qcmacros}
\usepackage{goldstone}
%%MACROS FOR THIS DOCUMENT
\numberwithin{equation}{section}
\usepackage[
  margin=1.5cm,
  includefoot,
  footskip=30pt,
  headsep=0.2cm,headheight=1.3cm
]{geometry}
\usepackage{fancyhdr}
\pagestyle{fancy}
\fancyhf{}
\fancyhead[LE,RO]{Quiz 6, Handout 1: Perturbation Theory}
\fancyfoot[CE,CO]{\thepage}
\usepackage{url}
\makeatother
\newcommand{\resolventline}[2][1]{
  \tikz[overlay]{
      \draw[thick,flexdotted] (0,-1ex) to ++(0,#1*4.5ex) node[above,inner sep=1pt] {#2};
  }
}

\begin{document}


\setcounter{section}{5}
\section{Perturbation theory}

\begin{dfn}
\thmtitle{Model Hamiltonian}
The electronic Hamiltonian\footnote{For the sake of brevity I will here refer to $H_\mr{c}$ as ``the electronic Hamiltonian''.  We could also use $H_e=E_0+H_\mr{c}$, which will simply shift some of the equations by a constant.} can be expressed as the sum of a \textit{zeroth order} or \textit{``model''~Hamiltonian} $H_0$ and a \textit{perturbation} $V_\mr{c}$, known as the \textit{fluctuation potential}.
For well-behaved electronic systems, a common choice for the model Hamiltonian is the diagonal part of the Fock operator.
\begin{align}
\label{eq:diagonal-fock-model-hamiltonian}
  H_0
\equiv
  f_p^p
  \tl{a}^p_p
&&
  V_\mr{c}
\equiv
  f_p^q
  (
    1
  -
    \d_p^q
  )
  \tl{a}^p_q
+
  \tfr{1}{4}
  \ol{g}_{pq}^{rs}
  \tl{a}^{pq}_{rs}
\end{align}
This choice of $H_0$ brings the advantage that its eigenbasis is the standard basis of determinants.
\begin{align}
\label{eq:model-problem}
  H_0
  \F
=
  0\,
  \F
&&
  H_0
  \F_{i_1\cd i_k}^{a_1\cd a_k}
=
  \mc{E}_{i_1\cd i_k}^{a_1\cd a_k}
  \F_{i_1\cd i_k}^{a_1\cd a_k}
&&
  \mc{E}_{q_1\cd q_k}^{p_1\cd p_k}
\equiv
  \sum_{r=1}^k
  f_{p_r}^{p_r}
-
  \sum_{r=1}^k
  f_{q_r}^{q_r}
\end{align}
In general the model Hamiltonian is chosen to make the matrix representation of $H_\mr{c}$ in the model eigenbasis diagonally dominant.\footnote{See \url{https://en.wikipedia.org/wiki/Diagonally_dominant_matrix}.}
Our choice of $H_0$ is appropriate for \textit{weakly correlated systems}, where the reference determinant can be chosen to satisfy $\ip{\F|\Y}\gg \ip{\F_{i_1\cd i_k}^{a_1\cd a_k}|\Y}$ for all substituted determinants.
In this context it is convenient to employ intermediate normalization for the wavefunction, which will be assumed from here on out.
\end{dfn}


\begin{dfn}
\thmtitle{Perturbation theory}
\textit{Perturbation theory} analyzes the polynomial order with which the wavefunction and its observables depend on the fluctuation potential.
For this purpose, we define a continuous series of Hamiltonians
$
  H(\la)
\equiv
  H_0
+
  \la
  V_\mr{c}
$
parametrized by a \textit{strength parameter} $\la$ that smoothly toggles between the model Hamiltonian at $\la=0$ to the exact one at $\la=1$.
The \textit{$m\eth$-order contribution} to a quantity $X$ is then defined as the $m\eth$ coefficient in its Taylor series about $\la=0$, which is denoted by $X\ord{m}$.
In particular, $\Y$ and $E_\mr{c}$ can be expanded as follows.
\begin{align}
\label{eq:series-schrodinger-equation}
  \Y
=
  \sum_{m=0}^\infty
  \Y_\mr{c}\ord{m}
&&
  E_\mr{c}
=
  \sum_{n=0}^\infty
  E_\mr{c}\ord{m}
&&
  \Y\ord{m}
\equiv
  \fr{1}{m!}
  \left.
    \pd{^m\Y(\la)}{\la^m}
  \right|_{\la=0}
&&
  E_\mr{c}\ord{m}
\equiv
  \fr{1}{m!}
  \left.
    \pd{^mE(\la)}{\la^m}
  \right|_{\la=0}
&&
  H(\la)
  \Y(\la)
=
  E(\la)
  \Y(\la)
\end{align}
The order(s) at which a term contributes to the wavefunction or energy provides one measure of its relative importance.
\end{dfn}

\begin{rmk}
Projecting the Schr\"odinger equation by $\F$ and using eq~\ref{eq:model-problem}, along with intermediate normalization, implies
\begin{align}
  E_\mr{c}
=
  \ip{\F|V_\mr{c}|\Y}
\hspace{20pt}
\implies
\hspace{20pt}
  E_\mr{c}\ord{m+1}
=
  \ip{\F|V_\mr{c}|\Y\ord{m}}
\end{align}
where the equation on the right follows from generalizing the energy expression to
$
  E(\la)
=
  \ip{\F|\la V_\mr{c}|\Y(\la)}
$.
In words, this says that the $m\eth$-order wavefunction contribution determines the $(m+1)\eth$-order energy contribution.
This immediately identifies the first-order energy as
$
  E_\mr{c}\ord{1}
=
  \ip{\F|V_\mr{c}|\F}
=
  0
$,
since $V_\mr{c}$ consists of $\F$-normal-ordered operators.
\end{rmk}

\begin{dfn}
\thmtitle{Model space projection operator}
The projection onto the reference determinant, $P=\kt{\F}\br{\F}$, is termed the \textit{model space projection operator}.
Its complement is the \textit{orthogonal space projection operator}.
\begin{align}
\label{eq:orthogonal-space-projection-operator}
  Q
\equiv
  1
-
  P
=
  \sum_k
  \pr{
    \tfr{1}{k!}
  }^2
  \sum_{\substack{a_1\cd a_k\\i_1\cd i_k}}
  \kt{\F_{i_1\cd i_k}^{a_1\cd a_k}}
  \br{\F_{i_1\cd i_k}^{a_1\cd a_k}}
\end{align}
Note that $P$ and $Q$ satisfy the following relationships, which are characteristic of complementary projection operators.
\begin{align}
  P
+
  Q
=
  1
&&
  P^2
=
  P
&&
  Q^2
=
  Q
&&
  PQ
=
  QP
=
  0
\end{align}
Due to intermediate normalization, the model space projection operator also satisfies
$
  P\Y
=
  \F
$
and
$
  Q\Y
=
  \Y
-
  \F
$.
\end{dfn}

\begin{samepage}
\begin{dfn}
\thmtitle{Resolvent}
The \textit{resolvent},
$
  R_0
\equiv
  (-H_0)^{-1}Q
$, is the negative\footnote{The annoying sign factor is required for consistency with $R(\zeta)\equiv(\zeta-H_0)^{-1}Q$, which is a more general definition of the resolvent.} inverse of $H_0$ in the orthogonal space.
Expanding the orthogonal space projection operator according to equation~\ref{eq:orthogonal-space-projection-operator} yields the spectral decomposition for $R_0$.\footnote{
Note that
$
  (-H_0)^{-1}\F_{i_1\cd i_k}^{a_1\cd a_k}
=
  (-\mc{E}_{i_1\cd i_k}^{a_1\cd a_k})^{-1}\F_{i_1\cd i_k}^{a_1\cd a_k}
$
and
$
  -\mc{E}_{i_1\cd i_k}^{a_1\cd a_k}
=
  \mc{E}_{a_1\cd a_k}^{i_1\cd i_k}
$
}
\begin{align}
\label{eq:resolvent-spectral-decomposition}
  R_0
=
  \sum_k
  \pr{\tfr{1}{k!}}^2
  \sum_{\substack{a_1\cd a_k\\i_1\cd i_k}}
  \fr{
    \kt{\F_{i_1\cd i_k}^{a_1\cd a_k}}
    \br{\F_{i_1\cd i_k}^{a_1\cd a_k}}
  }{
    \mc{E}_{a_1\cd a_k}^{i_1\cd i_k}
  }
\end{align}
Note that the resolvent satisfies $R_0P=0$ and $R_0Q=R_0$.
Restriction to the orthogonal space is necessary because $H_0$ is singular in the model space, which means that $H_0^{-1}$ does not exist there.
\end{dfn}
\end{samepage}

\begin{samepage}
\begin{rmk}
\thmtitle{A recursive solution to the Schr\"odinger equation}
Operating $R_0$ on the Schr\"odinger equation gives\footnote{This follows from $R_0H_\mr{c}\Y=R_0(H_0 + V_\mr{c})\Y=-Q\Y + R_0\Y$ along with $Q\Y=\Y-\F$.}
\begin{align}
\label{eq:recursive-series}
  \Y
=
  \F
+
  R_0
  (
    V_\mr{c}
  -
    E_\mr{c}
  )
  \Y
\end{align}
which provides a recursive equation for $\Y$.
Applying the same manipulation to equation \ref{eq:series-schrodinger-equation} gives
\begin{align}
\label{eq:lambda-dependent-recursive-series}
  \Y(\la)
=
  \F
+
  R_0
  (
    \la V_\mr{c}
  -
    E(\la)
  )
  \Y(\la)
\end{align}
which can be used solve for wavefunction contributions order by order.
\end{rmk}
\end{samepage}

\begin{ex}
The first two derivatives of equation~\ref{eq:lambda-dependent-recursive-series} are given by
\begin{align*}
  \pd{\Y(\la)}{\la}
=&
  R_0
  \pr{
    V_\mr{c}
  -
    \pd{E(\la)}{\la}
  }
  \Y(\la)
+
  R_0
  (
    \la V_\mr{c}
  -
    E(\la)
  )
  \pd{\Y(\la)}{\la}
\\
  \pd{^2\Y(\la)}{\la^2}
=&
-
  R_0
  \pd{^2E(\la)}{\la^2}
  \Y(\la)
+
  R_0
  \pr{
    V_\mr{c}
  -
    \pd{E(\la)}{\la}
  }
  \pd{\Y(\la)}{\la}
+
  R_0
  (
    \la V_\mr{c}
  -
    E(\la)
  )
  \pd{^2\Y(\la)}{\la^2}
\end{align*}
which can be used to determine first- and second-order components of the wavefunction.
\begin{align*}
  \Y\ord{1}
=
  \left.
  \pd{\Y(\la)}{\la}
  \right|_{\la=0}
=
  R_0
  V_\mr{c}
  \F
&&
  \Y\ord{2}
=
  \left.
  \pd{^2\Y(\la)}{\la^2}
  \right|_{\la=0}
=
  R_0
  V_\mr{c}
  \Y\ord{1}
=
  R_0
  V_\mr{c}
  R_0
  V_\mr{c}
  \F
\end{align*}
Here we have used $E_\mr{c}\ord{0}=E_\mr{c}\ord{1}=0$ and $R_0\F=0$ to simplify the result.
\end{ex}

\begin{ex}
\label{ex:first-order-wavefunction-expansion-unsimplified}
Expanding the resolvent according to \cref{eq:resolvent-spectral-decomposition} allows us to expand $\Y\ord{1}$ in the determinant basis.
\begin{align}
\label{eq:first-order-wavefunction-expansion-unsimplified}
  \Y\ord{1}
=
  R_0
  V_\mr{c}
  \F
=
  \sum_{\substack{a\\i}}
  \F_i^a
  \fr{\ip{\F_i^a|V_\mr{c}|\F}}{\mc{E}_a^i}
+
  (\tfr{1}{2!})^2
  \sum_{\substack{ab\\ij}}
  \F_{ij}^{ab}
  \fr{\ip{\F_{ij}^{ab}|V_\mr{c}|\F}}{\mc{E}_{ab}^{ij}}
\end{align}
The expansion truncates at double excitations because the maximum excitation level of $V_\mr{c}$ is $+2$.
\end{ex}

\begin{ex}
The numerators in example~\ref{ex:first-order-wavefunction-expansion-unsimplified} are easily evaluated using Slater's rules, which leads to the following.
\begin{align*}
\label{eq:first-order-wavefunction-expansion}
  \Y\ord{1}
=
  \sum_{\substack{a\\i}}
  \F_i^a
  \fr{f_a^i}{\mc{E}_a^i}
+
  (\tfr{1}{2!})^2
  \sum_{\substack{ab\\ij}}
  \F_{ij}^{ab}\,
  \fr{\ol{g}_{ab}^{ij}}{\mc{E}_{ab}^{ij}}
\hspace{20pt}
\implies
\hspace{20pt}
  E_\mr{c}\ord{2}
=
  \ip{\F|V_\mr{c}|\Y\ord{1}}
=
  \sum_{\substack{a\\i}}
  \fr{f_i^af_a^i}{\mc{E}_a^i}
+
  (\tfr{1}{2!})^2
  \sum_{\substack{ab\\ij}}
  \fr{\ol{g}_{ij}^{ab}\,\ol{g}_{ab}^{ij}}{\mc{E}_{ab}^{ij}}
\end{align*}
Note that the singles contribution vanishes for canonical Hartree-Fock references, since $f_a^i=0$.
These extra terms are required for non-canonical orbitals, such as those obtained from restricted open-shell Hartree-Fock (ROHF) theory.
\end{ex}

\begin{ex}
Diagrammatically, $\Y\ord{1}$ and $E_\mr{c}\ord{2}$ can be expressed as follows.
\begin{align}
  \Y\ord{1}
=
\diagram[bottom]{
  \draw
    (0,-0.5)
      node[circlex] {}
    to
    ++(0.5,0)
      node[ddot] (f1) {};
  \draw[->-]
    (f1)
    to
      ++(-0.25,1);
  \draw[-<-]
    (f1)
    to
      ++(+0.25,1);
  \draw[thick,flexdotted] (0.2,+0.25) to ++(0.6,0);
}
+
\diagram[bottom]{
  \interaction{2}{g}{(0,-0.5)}{ddot}{sawtooth};
  \draw[->-]
    (g1)
    to
      ++(-0.25,1);
  \draw[-<-]
    (g1)
    to
      ++(+0.25,1);
  \draw[->-]
    (g2)
    to
      ++(-0.25,1);
  \draw[-<-]
    (g2)
    to
      ++(+0.25,1);
  \draw[opacity=0] (0.5,-0.5) circle (0.125cm);
  \draw[thick,flexdotted] (-0.3,+0.25) to ++(1.6,0);
}
&&
  E_\mr{c}\ord{2}
=
\diagram{
  %top
  \draw
    (0,+0.5)
      node[circlex] {}
    to
    ++(0.5,0)
      node[ddot] (1f1) {};
  %bottom
  \draw
    (0,-0.5)
      node[circlex] {}
    to
    ++(0.5,0)
      node[ddot] (2f1) {};
  \draw[->-=0.4,bend left ] (2f1) to (1f1);
  \draw[-<-=0.6,bend right] (2f1) to (1f1);
  \draw[thick,flexdotted] (0.2,0) to ++(0.6,0);
}
+
\diagram{
  %top
  \interaction{2}{1g}{(0,+0.5)}{ddot}{sawtooth};
  %bottom
  \interaction{2}{2g}{(0,-0.5)}{ddot}{sawtooth};
  %lines
  \draw[->-=0.4,bend left ] (2g1) to (1g1);
  \draw[-<-=0.6,bend right] (2g1) to (1g1);
  \draw[->-=0.4,bend left ] (2g2) to (1g2);
  \draw[-<-=0.6,bend right] (2g2) to (1g2);
  \draw[thick,flexdotted] (-0.3,0) to ++(1.6,0);
}
\end{align}
\end{ex}

\begin{rmk}
Notice that a fully contracted operator of the form
$
  \sum_i
  v_i\,
  a_i
  \kt{\F}
  \br{\F}
  a_i\dg
$
can be simplified as
\begin{align*}
  \ts{\sum_i}
  v_i\,
  \ctr[0.5]{}{a}{_p\dg}{a}
  a_p\dg
  a_i
  \kt{\F}
  \br{\F}
  \ctr[0.5]{}{a}{_i\dg}{a}
  a_i\dg
  a_q
=
  v_p\,
  \ctr[0.5]{}{a}{_p\dg}{a}
  a_p\dg a_q
  \kt{\F}\br{\F}
\end{align*}
where $v_p$ is required to have an occupied index by the hole contraction.
A similar argument can be made for completely contracted operators of the form
$
  \sum_a
  v_a\,
  a_a\dg
  \kt{\F}
  \br{\F}
  a_a
$
or in general
$
  \sum_{\substack{abc\cd\\ijk\cd}}
  v_{abc\cd}^{ijk\cd}\,
  a_{abc\cd}^{ijk\cd}
  \kt{\F}
  \br{\F}
  a^{abc\cd}_{ijk\cd}
$, which leads to\footnote{The dotted lines are resolvent lines.}
\begin{align}
  \ip{\F|
    X
    R_0
    X_1
    R_0
    X_2
    \cd
    R_0
    X_n
  |\F}
=
\gno{\ol{\ol{
  X
  \hspace{2pt}\resolventline[0.85]{}\hspace{2pt}
  X_1
  \hspace{2pt}\resolventline[0.85]{}\hspace{2pt}
  X_2
  \cd
  \hspace{2pt}\resolventline[0.85]{}\hspace{2pt}
  X_n
}}}
\end{align}
for a general series of excitation operators $X, X_1,\ld,X_n$.
This makes a convenient corollary to Wick's theorem for operator products with resolvents.
\end{rmk}

\begin{ex}
\label{ex:second-order-wavefunction-expansion-unsimplified}
Expanding the resolvent in $\Y\ord{2}=R_0V_\mr{c}\Y\ord{1}$ gives an expansion that includes up to quadruples
\begin{align*}
  \Y\ord{2}
{=}
  \sum_{\substack{a\\i}}
  \F_i^a
  \fr{\ip{\F_i^a|V_\mr{c}|\Y\ord{1}}}{\mc{E}_a^i}
{+}
  (\tfr{1}{2!})^2
  \sum_{\substack{ab\\ij}}
  \F_{ij}^{ab}
  \fr{\ip{\F_{ij}^{ab}|V_\mr{c}|\Y\ord{1}}}{\mc{E}_{ab}^{ij}}
{+}
  (\tfr{1}{3!})^2
  \sum_{\substack{abc\\ijk}}
  \F_{ijk}^{abc}
  \fr{\ip{\F_{ijk}^{abc}|V_\mr{c}|\Y\ord{1}}}{\mc{E}_{abc}^{ijk}}
{+}
  (\tfr{1}{4!})^2
  \sum_{\substack{abcd\\ijkl}}
  \F_{ijkl}^{abcd}
  \fr{\ip{\F_{ijkl}^{abcd}|V_\mr{c}|\Y\ord{1}}}{\mc{E}_{abcd}^{ijkl}}
\end{align*}
since the two-electron part of $\Y\ord{1}$ has an excitation level of $+2$ and the excitation level of $V_\mr{c}$ ranges from $-2$ to $+2$.
\end{ex}

\begin{ex}
The expansion for $\Y\ord{2}$ can be evaluated graphically.
Assuming Brillouin's theorem for simplicity,
\begin{align}
\nonumber
  \Y\ord{2}
=&\
\diagram[bottom]{
  \interaction{2}{1g}{(0,-0.5)}{ddot}{sawtooth};
  \node[ddot] (2g2) at (1,0) {};
  \draw[-<-] (1g1) to ++(-0.25,+1);
  \draw[->-=0.25,->-=0.75] (1g1) to node[midway,ddot] (2g1) {} ++(+0.25,+1);
  \draw[sawtooth] (2g1) to (2g2);
  \draw[->-,bend left =45] (1g2) to (2g2);
  \draw[-<-,bend right=45] (1g2) to (2g2);
  \draw[thick,flexdotted] (-0.3,-0.27) to ++(1.6,0);
  \draw[thick,flexdotted] (-0.3,+0.35) to ++(1.6,0);
  \draw[opacity=0] (0.5,-0.5) circle (0.125cm);
}
+
\diagram[bottom]{
  \interaction{2}{1g}{(0,-0.5)}{ddot}{sawtooth};
  \node[ddot] (2g2) at (1,0) {};
  \draw[->-] (1g1) to ++(-0.25,+1);
  \draw[-<-=0.25,-<-=0.75] (1g1) to node[midway,ddot] (2g1) {} ++(+0.25,+1);
  \draw[sawtooth] (2g1) to (2g2);
  \draw[->-,bend left =45] (1g2) to (2g2);
  \draw[-<-,bend right=45] (1g2) to (2g2);
  \draw[thick,flexdotted] (-0.3,-0.27) to ++(1.6,0);
  \draw[thick,flexdotted] (-0.3,+0.35) to ++(1.6,0);
  \draw[opacity=0] (0.5,-0.5) circle (0.125cm);
}
+
\diagram[bottom]{
  \interaction{2}{t}{(0,-0.5)}{ddot}{sawtooth};
  \draw[->-=0.25,->-=0.75] (t1) to node[midway,ddot] (g1) {}
    ++(-0.25,1);
  \draw[-<-=0.7] (t1) to ++(+0.25,1);
  \draw[->-=0.25,->-=0.75] (t2) to node[midway,ddot] (g2) {}
    ++(-0.25,1);
  \draw[-<-=0.7] (t2) to ++(+0.25,1);
  \draw[sawtooth] (g1)--(g2);
  \draw[thick,flexdotted] (-0.3,-0.27) to ++(1.6,0);
  \draw[thick,flexdotted] (-0.4,+0.35) to ++(1.8,0);
  \draw[opacity=0] (0.5,-0.5) circle (0.125cm);
}
+
\diagram[bottom]{
  \interaction{2}{t}{(0,-0.5)}{ddot}{sawtooth};
  \draw[-<-=0.25,-<-=0.75] (t1) to node[midway,ddot] (g1) {}
    ++(-0.25,1);
  \draw[->-=0.7] (t1) to ++(+0.25,1);
  \draw[-<-=0.25,-<-=0.75] (t2) to node[midway,ddot] (g2) {}
    ++(-0.25,1);
  \draw[->-=0.7] (t2) to ++(+0.25,1);
  \draw[sawtooth] (g1)--(g2);
  \draw[thick,flexdotted] (-0.3,-0.27) to ++(1.6,0);
  \draw[thick,flexdotted] (-0.4,+0.35) to ++(1.8,0);
  \draw[opacity=0] (0.5,-0.5) circle (0.125cm);
}
+
\diagram[bottom]{
  \interaction{2}{t}{(0,-0.5)}{ddot}{sawtooth};
  \interaction{2}{g}{(1,+0.0)}{ddot}{sawtooth};
  \draw[->-] (t1) to ++(-0.25,1);
  \draw[-<-] (t1) to ++(+0.25,1);
  \draw[->-,bend left] (t2) to (g1);
  \draw[-<-,bend right] (t2) to (g1);
  \draw[->-] (g2) to ++(-0.25,0.5);
  \draw[-<-] (g2) to ++(+0.25,0.5);
  \draw[thick,flexdotted] (-0.3,-0.27) to ++(1.6,0);
  \draw[thick,flexdotted] (-0.4,+0.35) to ++(2.8,0);
  \draw[opacity=0] (0.5,-0.5) circle (0.125cm);
}
\\&\
\label{eq:second-order-wavefunction-graphical}
+
\diagram[bottom]{
  \interaction{2}{1g}{(0,-0.5)}{ddot}{sawtooth};
  \interaction{2}{2g}{(1.125,0)}{ddot}{sawtooth};
  \draw[-<-] (1g1) to ++(-0.25,1);
  \draw[->-] (1g1) to ++(+0.25,1);
  \draw[-<-] (1g2) to ++(-0.25,1);
  \draw[->-=0.25,->-=0.75] (1g2) to ++(+0.25,1);
  \draw[-<-] (2g2) to ++(-0.25,0.5);
  \draw[->-] (2g2) to ++(+0.25,0.5);
  \draw[thick,flexdotted] (-0.4,-0.27) to ++(2.9,0);
  \draw[thick,flexdotted] (-0.4,+0.35) to ++(2.9,0);
  \draw[opacity=0] (0.5,-0.5) circle (0.125cm);
}
+
\diagram[bottom]{
  \interaction{2}{1g}{(0,-0.5)}{ddot}{sawtooth};
  \interaction{2}{2g}{(1.125,0)}{ddot}{sawtooth};
  \draw[->-] (1g1) to ++(-0.25,1);
  \draw[-<-] (1g1) to ++(+0.25,1);
  \draw[->-] (1g2) to ++(-0.25,1);
  \draw[-<-=0.25,-<-=0.75] (1g2) to ++(+0.25,1);
  \draw[->-] (2g2) to ++(-0.25,0.5);
  \draw[-<-] (2g2) to ++(+0.25,0.5);
  \draw[thick,flexdotted] (-0.4,-0.27) to ++(2.9,0);
  \draw[thick,flexdotted] (-0.4,+0.35) to ++(2.9,0);
  \draw[opacity=0] (0.5,-0.5) circle (0.125cm);
}
+
\diagram[bottom]{
  \interaction{2}{1g}{(0,-0.5)}{ddot}{sawtooth};
  \interaction{2}{2g}{(2,0)}{ddot}{sawtooth};
  \draw[-<-] (1g1) to ++(-0.25,1);
  \draw[->-] (1g1) to ++(+0.25,1);
  \draw[-<-] (1g2) to ++(-0.25,1);
  \draw[->-] (1g2) to ++(+0.25,1);
  \draw[-<-] (2g1) to ++(-0.25,0.5);
  \draw[->-] (2g1) to ++(+0.25,0.5);
  \draw[-<-] (2g2) to ++(-0.25,0.5);
  \draw[->-] (2g2) to ++(+0.25,0.5);
  \draw[thick,flexdotted] (-0.3,-0.27) to ++(1.6,0);
  \draw[thick,flexdotted] (-0.4,+0.35) to ++(3.9,0);
  \draw[opacity=0] (0.5,-0.5) circle (0.125cm);
}
\\=&\
  \tfr{1}{2}
  \sum_{\substack{abc\\ij}}
  \F_i^a\,
  \fr{
    \ol{g}_{aj}^{bc}
    \ol{g}_{bc}^{ij}
  }{
    \mc{E}_a^i
    \mc{E}_{bc}^{ij}
  }
+
  \tfr{1}{2}
  \sum_{\substack{ab\\ijk}}
  \F_i^a
  \fr{
    \ol{g}_{jk}^{ib}
    \ol{g}_{ab}^{jk}
  }{
    \mc{E}_a^i
    \mc{E}_{ab}^{jk}
  }
+
  \tfr{1}{2^3}
  \sum_{\substack{abcd\\ij}}
  \F_{ij}^{ab}
  \fr{
    \ol{g}_{ab}^{cd}
    \ol{g}_{cd}^{ij}
  }{
    \mc{E}_{ab}^{ij}
    \mc{E}_{cd}^{ij}
  }
+
  \tfr{1}{2^3}
  \sum_{\substack{ab\\ijkl}}
  \F_{ij}^{ab}
  \fr{
    \ol{g}_{kl}^{ij}
    \ol{g}_{ab}^{kl}
  }{
    \mc{E}_{ab}^{ij}
    \mc{E}_{ab}^{kl}
  }
+
  \sum_{\substack{abc\\ijk}}
  \F_{ij}^{ab}
  \fr{
    \ol{g}_{ac}^{ik}
    \ol{g}_{kb}^{cj}
  }{
    \mc{E}_{ab}^{ij}
    \mc{E}_{ac}^{ik}
  }
\nonumber
\\&\
+
  \tfr{1}{2^2}
  \sum_{\substack{abcd\\ijk}}
  \F_{ijk}^{abc}
  \fr{
    \ol{g}_{ad}^{ij}
    \ol{g}_{bc}^{dk}
  }{
    \mc{E}_{abc}^{ijk}
    \mc{E}_{ad}^{ij}
  }
-
  \tfr{1}{2^2}
  \sum_{\substack{abc\\ijkl}}
  \fr{
    \ol{g}_{ab}^{il}
    \ol{g}_{lc}^{jk}
  }{
    \mc{E}_{abc}^{ijk}
    \mc{E}_{ab}^{il}
  }
+
  \tfr{1}{2^4}
  \sum_{\substack{abcd\\ijkl}}
  \F_{ijkl}^{abcd}
  \fr{
    \ol{g}_{ab}^{ij}
    \ol{g}_{cd}^{kl}
  }{
    \mc{E}_{abcd}^{ijkl}
    \mc{E}_{ab}^{ij}
  }
\end{align}
where the operators in the final diagram do not form an equivalent pair because they pass through different resolvent lines.
This can be used to determine the third-order correlation energy
\begin{align}
  E_\mr{c}\ord{3}
=
  \ip{\F|V_\mr{c}|\Y\ord{2}}
=
\diagram{
  %top
  \interaction{2}{1g}{(0,+0.5)}{ddot}{sawtooth};
  %bottom
  \interaction{2}{2g}{(0,-0.5)}{ddot}{sawtooth};
  \draw[->-=0.25,->-=0.75, bend left]
    (2g1)
    to
      node[midway,ddot] (g1) {}
    (1g1);
  \draw[-<-=0.65,bend right] (2g1) to (1g1);
  \draw[->-=0.25,->-=0.75, bend left]
    (2g2)
    to
      node[midway,ddot] (g2) {}
    (1g2);
  \draw[-<-=0.65,bend right] (2g2) to (1g2);
  \draw[sawtooth] (g1)--(g2);
  \draw[thick,flexdotted] (-0.3,-0.3) to ++(1.6,0);
  \draw[thick,flexdotted] (-0.3,+0.3) to ++(1.6,0);
  \draw[opacity=0] (0.5,-0.5) circle (0.125cm);
}
{+}
\diagram{
  %top
  \interaction{2}{1g}{(0,+0.5)}{ddot}{sawtooth};
  %bottom
  \interaction{2}{2g}{(0,-0.5)}{ddot}{sawtooth};
  \draw[-<-=0.25,-<-=0.75, bend left]
    (2g1)
    to
      node[midway,ddot] (g1) {}
    (1g1);
  \draw[->-=0.65,bend right] (2g1) to (1g1);
  \draw[-<-=0.25,-<-=0.75, bend left]
    (2g2)
    to
      node[midway,ddot] (g2) {}
    (1g2);
  \draw[->-=0.65,bend right] (2g2) to (1g2);
  \draw[sawtooth] (g1)--(g2);
  \draw[thick,flexdotted] (-0.3,-0.3) to ++(1.6,0);
  \draw[thick,flexdotted] (-0.3,+0.3) to ++(1.6,0);
  \draw[opacity=0] (0.5,-0.5) circle (0.125cm);
}
{+}
\diagram{
  %top
  \draw[sawtooth]
    (0,+0.5)
      node[ddot] (1g1) {}
    to
    ++(2,0)
      node[ddot] (1g2) {};
  %middle
  \interaction{2}{g}{(1,+0.0)}{ddot}{sawtooth};
  %bottom
  \interaction{2}{2g}{(0,-0.5)}{ddot}{sawtooth};
  \draw[->-,bend left ] (2g1) to (1g1);
  \draw[-<-,bend right] (2g1) to (1g1);
  \draw[->-,bend left ] (2g2) to (g1);
  \draw[-<-,bend right] (2g2) to (g1);
  \draw[->-,bend left ] (g2) to (1g2);
  \draw[-<-,bend right] (g2) to (1g2);
  \draw[thick,flexdotted] (-0.3,-0.3) to ++(1.6,0);
  \draw[thick,flexdotted] (-0.3,+0.3) to ++(2.6,0);
  \draw[opacity=0] (0.5,-0.5) circle (0.125cm);
}
=
  \tfr{1}{2^3}
  \sum_{\substack{abcd\\ij}}
  \fr{
    \ol{g}_{ij}^{ab}
    \ol{g}_{ab}^{cd}
    \ol{g}_{cd}^{ij}
  }{
    \mc{E}_{ab}^{ij}
    \mc{E}_{cd}^{ij}
  }
+
  \tfr{1}{2^3}
  \sum_{\substack{ab\\ijkl}}
  \fr{
    \ol{g}_{ij}^{ab}
    \ol{g}_{kl}^{ij}
    \ol{g}_{ab}^{kl}
  }{
    \mc{E}_{ab}^{ij}
    \mc{E}_{ab}^{kl}
  }
+
  \sum_{\substack{abc\\ijk}}
  \fr{
    \ol{g}_{ij}^{ab}
    \ol{g}_{ac}^{ik}
    \ol{g}_{kb}^{cj}
  }{
    \mc{E}_{ab}^{ij}
    \mc{E}_{ac}^{ik}
  }
\end{align}
which also equals $\ip{\Y\ord{1}|V_\mr{c}|\Y\ord{1}}$.
This is an example of the \textit{Wigner $(2n+1)$ rule}, which says that
$
  E_\mr{c}\ord{2n+1}
=
  \ip{\Y\ord{n}|V_\mr{c}|\Y\ord{n}}
$.
Note that $E_\mr{c}\ord{m+1}$ always only depends on the doubles contribution to $\Y\ord{m}$, but that the doubles coefficients themselves may involve triples, quadruples and higher contributions from wavefunction components of order less than $m$.
\end{ex}

\begin{ex}
Using
${}\ord{m}c_{ab\cd}^{ij\cd}=\ip{\F_{ij\cd}^{ab\cd}|\Y\ord{m}}$, the second order CI coefficients can be determined from eq~\ref{eq:second-order-wavefunction-graphical} by contracting a bare excitation operator with the top of each diagram.
Interpreting these graphs gives the following.
\begin{align*}
  {}\ord{2}c_a^i
&=
  \tfr{1}{2}
  \sum_{\substack{bc\\j}}
  \fr{
    \ol{g}_{aj}^{bc}
    \ol{g}_{bc}^{ij}
  }{
    \mc{E}_a^i
    \mc{E}_{bc}^{ij}
  }
+
  \tfr{1}{2}
  \sum_{\substack{b\\jk}}
  \fr{
    \ol{g}_{jk}^{ib}
    \ol{g}_{ab}^{jk}
  }{
    \mc{E}_a^i
    \mc{E}_{ab}^{jk}
  }
\\
  {}\ord{2}c_{ab}^{ij}
&=
  \tfr{1}{2}
  \sum_{\substack{cd}}
  \fr{
    \ol{g}_{ab}^{cd}
    \ol{g}_{cd}^{ij}
  }{
    \mc{E}_{ab}^{ij}
    \mc{E}_{cd}^{ij}
  }
+
  \tfr{1}{2}
  \sum_{\substack{kl}}
  \fr{
    \ol{g}_{kl}^{ij}
    \ol{g}_{ab}^{kl}
  }{
    \mc{E}_{ab}^{ij}
    \mc{E}_{ab}^{kl}
  }
+
  \op{P}_{(a/b)}^{(i/j)}
  \sum_{\substack{c\\k}}
  \fr{
    \ol{g}_{ac}^{ik}
    \ol{g}_{kb}^{cj}
  }{
    \mc{E}_{ab}^{ij}
    \mc{E}_{ac}^{ik}
  }
\\
  {}\ord{2}c_{abc}^{ijk}
&=
  \op{P}_{(a/bc)}^{(ij/k)}
  \sum_{\substack{d}}
  \F_{ijk}^{abc}
  \fr{
    \ol{g}_{ad}^{ij}
    \ol{g}_{bc}^{dk}
  }{
    \mc{E}_{abc}^{ijk}
    \mc{E}_{ad}^{ij}
  }
-
  \op{P}_{(a/bc)}^{(i/jk)}
  \sum_{\substack{l}}
  \fr{
    \ol{g}_{ab}^{il}
    \ol{g}_{lc}^{jk}
  }{
    \mc{E}_{abc}^{ijk}
    \mc{E}_{ab}^{il}
  }
\\
  {}\ord{2}c_{abcd}^{ijkl}
&=
  \op{P}_{(ab/cd)}^{(ij/kl)}
  \fr{
    \ol{g}_{ab}^{ij}
    \ol{g}_{cd}^{kl}
  }{
    \mc{E}_{abcd}^{ijkl}
    \mc{E}_{ab}^{ij}
  }
\end{align*}
Note that the second order quadruples coefficient is disconnected.
Prop.~\ref{prop:second-order-c4} shows that the second-order quadruples operator is actually a simple product of first-order doubles operators.
This fact was an early motivation for coupled-pair many-electron theory,\footnote{This is the original name for coupled-cluster doubles.} since it justifies approximating
$
  \Y_\mr{CIDQ}
=
  (1+C_2+C_4)\F
$
by
$
  \Y_\mr{CPMET}
=
  (1 + C_2 + \tfr{1}{2}C_2^2)\F
$.
\end{ex}

\begin{prop}
\label{prop:second-order-c4}
\thmstatement{
$
  {}\ord{2} C_4
=
  \tfr{1}{2}
  {}\ord{1} C_2^2
$
}
\thmproof{
This follows from rearranging the resolvent denominator.
\begin{align*}
  \fr{1}{\mc{E}_{abcd}^{ijkl}\mc{E}_{ab}^{ij}}
+
  \fr{1}{\mc{E}_{abcd}^{ijkl}\mc{E}_{cd}^{kl}}
=
  \fr{
    \mc{E}_{cd}^{kl} + \mc{E}_{ab}^{ij}
  }{
    \mc{E}_{abcd}^{ijkl}
    \mc{E}_{ab}^{ij}
    \mc{E}_{cd}^{kl}
  }
=
  \fr{
    1
  }{
    \mc{E}_{ab}^{ij}
    \mc{E}_{cd}^{kl}
  }
\implies
  {}\ord{2}C_4
=
  \pr{\tfr{1}{2}}^4
  \sum_{\substack{abcd\\ijkl}}
  \tl{a}^{ijkl}_{abcd}
  \fr{
    \ol{g}_{ab}^{ij}
    \ol{g}_{cd}^{kl}
  }{
    \mc{E}_{abcd}^{ijkl}
    \mc{E}_{ab}^{ij}
  }
=
  \tfr{1}{2}\cdot
  \pr{\tfr{1}{2}}^4
  \sum_{\substack{abcd\\ijkl}}
  \tl{a}^{ijkl}_{abcd}
  \fr{
    \ol{g}_{ab}^{ij}
    \ol{g}_{cd}^{kl}
  }{
    \mc{E}_{ab}^{ij}
    \mc{E}_{cd}^{kl}
  }
=
  \tfr{1}{2}
  {}\ord{1}C_2^2
\end{align*}
}
\end{prop}

\end{document}