\documentclass[11pt]{article}
\usepackage[cm]{fullpage}
%%AVC PACKAGES
\usepackage{avcgreek}
\usepackage{avcfonts}
\usepackage{avcmath}
\usepackage[numberby=section,skip=9pt plus 2pt minus 5pt]{avcthm}
\usepackage{qcmacros}
\usepackage{goldstone}
%%MACROS FOR THIS DOCUMENT
\numberwithin{equation}{section}
\usepackage[
  margin=1.5cm,
  includefoot,
  footskip=30pt,
  headsep=0.2cm,headheight=1.3cm
]{geometry}
\usepackage{fancyhdr}
\pagestyle{fancy}
\fancyhf{}
\fancyhead[LE,RO]{Quiz 6, Handout 1: Perturbation Theory}
\fancyfoot[CE,CO]{\thepage}
\usepackage{url}
\makeatother
\newcommand{\resolventline}[2][1]{
  \tikz[overlay]{
      \draw[thick,flexdotted] (0,-1ex) to ++(0,#1*4.5ex) node[above,inner sep=1pt] {#2};
  }
}

\begin{document}


\setcounter{section}{5}
\section{Perturbation theory}

\begin{dfn}
\thmtitle{Model Hamiltonian}
The electronic Hamiltonian\footnote{For the sake of brevity I will here refer to $H_\mr{c}$ as ``the electronic Hamiltonian''.  We could also use $H_e=E_0+H_\mr{c}$, which will simply shift some of the equations by a constant.} can be expressed as the sum of a \textit{zeroth order} or \textit{``model''~Hamiltonian} $H_0$ and a \textit{perturbation} $V_\mr{c}$, known as the \textit{fluctuation potential}.
For well-behaved electronic systems, a common choice for the model Hamiltonian is the diagonal part of the Fock operator.
\begin{align}
\label{eq:diagonal-fock-model-hamiltonian}
  H_0
\equiv
  f_p^p
  \tl{a}^p_p
&&
  V_\mr{c}
\equiv
  f_p^q
  (
    1
  -
    \d_p^q
  )
  \tl{a}^p_q
+
  \tfr{1}{4}
  \ol{g}_{pq}^{rs}
  \tl{a}^{pq}_{rs}
\end{align}
This choice of $H_0$ brings the advantage that its eigenbasis is the standard basis of determinants.
\begin{align}
\label{eq:model-problem}
  H_0
  \F
=
  0\,
  \F
&&
  H_0
  \F_{i_1\cd i_k}^{a_1\cd a_k}
=
  \mc{E}_{i_1\cd i_k}^{a_1\cd a_k}
  \F_{i_1\cd i_k}^{a_1\cd a_k}
&&
  \mc{E}_{q_1\cd q_k}^{p_1\cd p_k}
\equiv
  \sum_{r=1}^k
  f_{p_r}^{p_r}
-
  \sum_{r=1}^k
  f_{q_r}^{q_r}
\end{align}
In general the model Hamiltonian is chosen to make the matrix representation of $H_\mr{c}$ in the model eigenbasis diagonally dominant.\footnote{See \url{https://en.wikipedia.org/wiki/Diagonally_dominant_matrix}.}
Our choice of $H_0$ is appropriate for \textit{weakly correlated systems}, where the reference determinant can be chosen to satisfy $\ip{\F|\Y}\gg \ip{\F_{i_1\cd i_k}^{a_1\cd a_k}|\Y}$ for all substituted determinants.
In this context it is convenient to employ intermediate normalization for the wavefunction, which will be assumed from here on out.
\end{dfn}


\begin{dfn}
\thmtitle{Perturbation theory}
\textit{Perturbation theory} analyzes the polynomial order with which the wavefunction and its observables depend on the fluctuation potential.
For this purpose, we define a continuous series of Hamiltonians
$
  H(\la)
\equiv
  H_0
+
  \la
  V_\mr{c}
$
parametrized by a \textit{strength parameter} $\la$ that smoothly toggles between the model Hamiltonian at $\la=0$ to the exact one at $\la=1$.
The \textit{$m\eth$-order contribution} to a quantity $X$ is then defined as the $m\eth$ coefficient in its Taylor series about $\la=0$, denoted $X\ord{m}$.
In particular, the wavefunction and correlation energy can be expanded as follows.
\begin{align}
\label{eq:series-schrodinger-equation}
  \Y
=
  \sum_{m=0}^\infty
  \Y\ord{m}
&&
  E_\mr{c}
=
  \sum_{m=0}^\infty
  E_\mr{c}\ord{m}
&&
  \Y\ord{m}
\equiv
  \fr{1}{m!}
  \left.
    \pd{^m\Y(\la)}{\la^m}
  \right|_{\la=0}
&&
  E_\mr{c}\ord{m}
\equiv
  \fr{1}{m!}
  \left.
    \pd{^mE(\la)}{\la^m}
  \right|_{\la=0}
&&
  H(\la)
  \Y(\la)
=
  E(\la)
  \Y(\la)
\end{align}
The order(s) at which a term contributes to the wavefunction or energy provides one measure of its relative importance.
\end{dfn}

\begin{rmk}
Projecting the Schr\"odinger equation by $\F$ and using eq~\ref{eq:model-problem}, along with intermediate normalization, implies
\begin{align}
  E_\mr{c}
=
  \ip{\F|V_\mr{c}|\Y}
\hspace{20pt}
\implies
\hspace{20pt}
  E_\mr{c}\ord{m+1}
=
  \ip{\F|V_\mr{c}|\Y\ord{m}}
\end{align}
where the equation on the right follows from generalizing the energy expression to
$
  E(\la)
=
  \ip{\F|\la V_\mr{c}|\Y(\la)}
$.
In words, this says that the $m\eth$-order wavefunction contribution determines the $(m+1)\eth$-order energy contribution.
This immediately identifies the first-order energy as
$
  E_\mr{c}\ord{1}
=
  \ip{\F|V_\mr{c}|\F}
=
  0
$,
since $V_\mr{c}$ consists of $\F$-normal-ordered operators.
\end{rmk}

\begin{dfn}
\thmtitle{Model space projection operator}
The projection onto the reference determinant, $P=\kt{\F}\br{\F}$, is termed the \textit{model space projection operator}.
Its complement is the \textit{orthogonal space projection operator}.\footnote{$1_n\equiv 1|_{\mc{F}_n}$ is the identity on $\mc{F}_n$, which is equivalent to a projection onto this subspace.  For our purposes, this is the identity.}
\begin{align}
\label{eq:orthogonal-space-projection-operator}
  Q
\equiv
  1_n
-
  P
=
  \sum_k
  \pr{
    \tfr{1}{k!}
  }^2
  \sum_{\substack{a_1\cd a_k\\i_1\cd i_k}}
  \kt{\F_{i_1\cd i_k}^{a_1\cd a_k}}
  \br{\F_{i_1\cd i_k}^{a_1\cd a_k}}
\end{align}
Note that $P$ and $Q$ satisfy the following relationships, which are characteristic of complementary projection operators.
\begin{align}
  P
+
  Q
=
  1_n
&&
  P^2
=
  P
&&
  Q^2
=
  Q
&&
  PQ
=
  QP
=
  0
\end{align}
Due to intermediate normalization, we also have that
$
  P\Y
=
  \F
$
and
$
  Q\Y
=
  \Y
-
  \F
$.
\end{dfn}

\begin{samepage}
\begin{dfn}
\thmtitle{Resolvent}
The \textit{resolvent},
$
  R_0
\equiv
  (-H_0)^{-1}Q
$, is the negative\footnote{The annoying sign factor is required for consistency with $R(\zeta)\equiv(\zeta-H_0)^{-1}Q$, which is a more general definition of the resolvent.} inverse of $H_0$ in the orthogonal space.
\begin{align}
\label{eq:resolvent-spectral-decomposition}
  R_0
  \F
=
  0
  \F
&&
  R_0
  \F_{i_1\cd i_k}^{a_1\cd a_k}
=
  (\mc{E}_{a_1\cd a_k}^{i_1\cd i_k})^{-1}
  \F_{i_1\cd i_k}^{a_1\cd a_k}
&&
  R_0
=
  \sum_k
  \pr{\tfr{1}{k!}}^2
  \sum_{\substack{a_1\cd a_k\\i_1\cd i_k}}
  \fr{
    \kt{\F_{i_1\cd i_k}^{a_1\cd a_k}}
    \br{\F_{i_1\cd i_k}^{a_1\cd a_k}}
  }{
    \mc{E}_{a_1\cd a_k}^{i_1\cd i_k}
  }
\end{align}
The equation on the right is the spectral decomposition of the resolvent.\footnote{This follows from the eigenvalue equations, but you can derive it explicitly by substituting equation~\ref{eq:orthogonal-space-projection-operator} into $R_0=(-H_0)^{-1}Q$.}
Restriction to the orthogonal space is necessary because $H_0$ is singular in the model space, which means that $H_0^{-1}$ does not exist there.
\end{dfn}
\end{samepage}

\begin{samepage}
\begin{rmk}
\thmtitle{A recursive solution to the Schr\"odinger equation}
Operating $R_0$ on $H(\la)\Y(\la)=E(\la)\Y(\la)$ gives\footnote{This follows from $R_0H_0\Y=-Q\Y=-\Y+\F$.}
\begin{align}
\label{eq:lambda-dependent-recursive-series}
  \Y(\la)
=
  \F
+
  R_0
  (
    \la V_\mr{c}
  -
    E(\la)
  )
  \Y(\la)
\end{align}
which provides a recursive equation for $\Y(\la)$ that can be used to solve for wavefunction contributions order by order.
\end{rmk}
\end{samepage}

\begin{ex}
The first two derivatives of equation~\ref{eq:lambda-dependent-recursive-series} are given by
\begin{align*}
  \pd{\Y(\la)}{\la}
=&
  R_0
  \pr{
    V_\mr{c}
  -
    \pd{E(\la)}{\la}
  }
  \Y(\la)
+
  R_0
  (
    \la V_\mr{c}
  -
    E(\la)
  )
  \pd{\Y(\la)}{\la}
\\
  \pd{^2\Y(\la)}{\la^2}
=&
-
  R_0
  \pd{^2E(\la)}{\la^2}
  \Y(\la)
+
  2
  R_0
  \pr{
    V_\mr{c}
  -
    \pd{E(\la)}{\la}
  }
  \pd{\Y(\la)}{\la}
+
  R_0
  (
    \la V_\mr{c}
  -
    E(\la)
  )
  \pd{^2\Y(\la)}{\la^2}
\end{align*}
which can be used to determine the first- and second-order wavefunction contributions.
\begin{align}
\label{eq:first-and-second-order-principal-terms}
  \Y\ord{1}
=
  \left.
  \pd{\Y(\la)}{\la}
  \right|_{\la=0}
=
  R_0
  V_\mr{c}
  \F
&&
  \Y\ord{2}
=
  \left.
  \fr{1}{2}
  \pd{^2\Y(\la)}{\la^2}
  \right|_{\la=0}
=
  R_0
  V_\mr{c}
  \Y\ord{1}
=
  R_0
  V_\mr{c}
  R_0
  V_\mr{c}
  \F
\end{align}
Here we have used $E_\mr{c}\ord{0}=E_\mr{c}\ord{1}=0$ and $R_0\F=0$ to simplify the result.
\end{ex}

\begin{ex}
\label{ex:first-order-wavefunction-expansion-unsimplified}
Plugging in the spectral decomposition for $R_0$ allows us to expand $\Y\ord{1}$ in the determinant basis.
\begin{align}
\label{eq:first-order-wavefunction-expansion-unsimplified}
  \Y\ord{1}
=
  R_0
  V_\mr{c}
  \F
=
  \sum_{\substack{a\\i}}
  \F_i^a
  \fr{\ip{\F_i^a|V_\mr{c}|\F}}{\mc{E}_a^i}
+
  (\tfr{1}{2!})^2
  \sum_{\substack{ab\\ij}}
  \F_{ij}^{ab}
  \fr{\ip{\F_{ij}^{ab}|V_\mr{c}|\F}}{\mc{E}_{ab}^{ij}}
\end{align}
The expansion truncates at double excitations because the maximum excitation level of $V_\mr{c}$ is $+2$.
\end{ex}

\begin{ex}
The numerators in example~\ref{ex:first-order-wavefunction-expansion-unsimplified} are easily evaluated using Slater's rules, which leads to the following.
\begin{align*}
\label{eq:first-order-wavefunction-expansion}
  \Y\ord{1}
=
  \sum_{\substack{a\\i}}
  \F_i^a
  \fr{f_a^i}{\mc{E}_a^i}
+
  (\tfr{1}{2!})^2
  \sum_{\substack{ab\\ij}}
  \F_{ij}^{ab}\,
  \fr{\ol{g}_{ab}^{ij}}{\mc{E}_{ab}^{ij}}
\hspace{20pt}
\implies
\hspace{20pt}
  E_\mr{c}\ord{2}
=
  \ip{\F|V_\mr{c}|\Y\ord{1}}
=
  \sum_{\substack{a\\i}}
  \fr{f_i^af_a^i}{\mc{E}_a^i}
+
  (\tfr{1}{2!})^2
  \sum_{\substack{ab\\ij}}
  \fr{\ol{g}_{ij}^{ab}\,\ol{g}_{ab}^{ij}}{\mc{E}_{ab}^{ij}}
\end{align*}
Note that the singles contribution vanishes for canonical Hartree-Fock references, since $f_a^i=0$.
These extra terms are required for non-canonical orbitals, such as those obtained from restricted open-shell Hartree-Fock (ROHF) theory.
\end{ex}

\begin{dfn}\label{dfn:resolvent-line}
\thmtitle{Resolvent line}
We can generalize our previous definition of the \textit{resolvent line} as follows
\begin{align}
  \,\resolventline[0.7]{}\,
  Y
\equiv
  \sum_k
  \pr{\tfr{1}{k!}}^2
  \sum_{\substack{a_1\cd a_k\\i_1\cd i_k}}
  \fr{y_{a_1\cd a_k}^{i_1\cd i_k}}{\mc{E}_{a_1\cd a_k}^{i_1\cd i_k}}
  \tl{a}^{a_1\cd a_k}_{i_1\cd i_k}
&&
  Y
=
  Y_{n\rightarrow n}
+
  Y_{n\not\rightarrow n}
&&
  Y_{n\rightarrow n}
=
  y_0
+
  \sum_k
  \pr{\tfr{1}{k!}}^2
  \sum_{\substack{p_1\cd p_k\\q_1\cd q_k}}
  y_{p_1\cd p_k}^{q_1\cd q_k}
  \tl{a}^{p_1\cd p_k}_{q_1\cd q_k}
\end{align}
where $Y$ is an arbitrary operator.
The last equation is the Wick expansion of $Y_{n\rightarrow n}$, which denotes the purely particle-number-conserving part\footnote{The component that maps $\mc{F}_n\rightarrow\mc{F}_n$ for all $n$, which can always be written as a linear combination of excitation operators.} of $Y$.
This definition immediately implies
$
  \,\resolventline[0.7]{}\,\,
  \kt{\Y}
=
  R_0
  \kt{\Y}
$
for all $\Y$.\footnote{Since any $\kt{\Y}$ can be written as $Y\kt{\F}$, this follows from applying eq~\ref{eq:resolvent-spectral-decomposition} to each term in the Wick expansion of $Y$ in $R_0Y\kt{\F}$.}
Other expressions are defined by giving resolvent lines priority in the order of operations, with maximum priority given to the rightmost resolvent.
\begin{align}
    Y_1
  \,\resolventline[0.7]{}\,
    Y_2
  \cd
  \,\resolventline[0.7]{}\,
    Y_n
\equiv
    Y_1
  \pr{
  \,\resolventline[0.7]{}\,
    Y_2
  \pr{
  \cd
  \pr{
  \,\resolventline[0.7]{}\,
    Y_n
    \vphantom{Y_Y^Y}
  }\cd}}
&&
\gno{\ol{
    Y_1
  \,\resolventline[0.8]{}\,
    Y_2
  \cd
  \,\resolventline[0.8]{}\,
    Y_n
}}
\equiv
\gno{\ol{
  \,\,
    Y_1
  \pr{
  \,\resolventline[0.7]{}\,
    Y_2
  \pr{
  \cd
  \pr{
  \,\resolventline[0.7]{}\,
    Y_n
    \vphantom{Y_Y^Y}
  }\cd}}
}}
\end{align}
This definition also specifies the interpretation rule for a graphs with resolvent lines, which are formally defined below.
\end{dfn}

\begin{cor}\label{cor:wicks-theorem-for-pt}
\thmtitle{Wick's theorem for perturbation theory}
\thmstatement{
$
  YR_0Y_1\cd R_0 Y_m
  \kt{\F}
=
  \pr{
    \gno{
      Y
      \,\resolventline[0.8]{}\,
      Y_1
      \cd
      \,\resolventline[0.8]{}\,
      Y_m
    }
  +
    \gno{\ol{
      Y
      \,\resolventline[0.8]{}\,
      Y_1
      \cd
      \,\resolventline[0.8]{}\,
      Y_m
    }}
  }
  \kt{\F}
$%
}%
\vspace{5pt}
\thmproof{
  This follows directly from Wick's theorem and definition~\ref{dfn:resolvent-line}.
}
\end{cor}

\begin{dfn}
\thmtitle{Resolvent graph}
A \textit{resolvent graph} represents a normal-ordered product of operators and resolvents.
Graphs with disconnected parts that don't share any resolvent lines are considered products of separate resolvent graphs.
Vertical spaces between resolvent lines in a resolvent graph are termed \textit{levels}, which are numbered from top to bottom with zero indexing.
Therefore, an operator lies in the $k\eth$ level if there are $k$ resolvent lines above it.
Formally, then, an \textit{$m$-resolvent graph} $G(\rh,m)\equiv(G,\rh,m)$ associates each operator $o$ in $G$ with a specific level $\rh(o)=\rh_o$ in $\{0,1,\ld,m\}$ through the \textit{level map} $\rh$.
Therefore, each line $l$ in $G$ crosses resolvents
$
  \mr{min}(\rh_{h(l)},\rh_{t(l)}) + 1
$
through
$
  \mr{max}(\rh_{h(l)},\rh_{t(l)})
$.
\end{dfn}

\begin{ex}
In diagram notation, $\Y\ord{1}$ and $E_\mr{c}\ord{2}$ can be expressed as follows.
\begin{align}
  \Y\ord{1}
=
\diagram[bottom]{
  \draw
    (0,-0.5)
      node[circlex] {}
    to
    ++(0.5,0)
      node[ddot] (f1) {};
  \draw[->-]
    (f1)
    to
      ++(-0.25,1);
  \draw[-<-]
    (f1)
    to
      ++(+0.25,1);
  \draw[thick,flexdotted] (0.2,+0.25) to ++(0.6,0);
}
+
\diagram[bottom]{
  \interaction{2}{g}{(0,-0.5)}{ddot}{sawtooth};
  \draw[->-]
    (g1)
    to
      ++(-0.25,1);
  \draw[-<-]
    (g1)
    to
      ++(+0.25,1);
  \draw[->-]
    (g2)
    to
      ++(-0.25,1);
  \draw[-<-]
    (g2)
    to
      ++(+0.25,1);
  \draw[opacity=0] (0.5,-0.5) circle (0.125cm);
  \draw[thick,flexdotted] (-0.3,+0.25) to ++(1.6,0);
}
&&
  E_\mr{c}\ord{2}
=
\diagram{
  %top
  \draw
    (0,+0.5)
      node[circlex] {}
    to
    ++(0.5,0)
      node[ddot] (1f1) {};
  %bottom
  \draw
    (0,-0.5)
      node[circlex] {}
    to
    ++(0.5,0)
      node[ddot] (2f1) {};
  \draw[->-=0.4,bend left ] (2f1) to (1f1);
  \draw[-<-=0.6,bend right] (2f1) to (1f1);
  \draw[thick,flexdotted] (0.2,0) to ++(0.6,0);
}
+
\diagram{
  %top
  \interaction{2}{1g}{(0,+0.5)}{ddot}{sawtooth};
  %bottom
  \interaction{2}{2g}{(0,-0.5)}{ddot}{sawtooth};
  %lines
  \draw[->-=0.4,bend left ] (2g1) to (1g1);
  \draw[-<-=0.6,bend right] (2g1) to (1g1);
  \draw[->-=0.4,bend left ] (2g2) to (1g2);
  \draw[-<-=0.6,bend right] (2g2) to (1g2);
  \draw[thick,flexdotted] (-0.3,0) to ++(1.6,0);
}
\end{align}
\end{ex}

\begin{ex}
The expansion for $\Y\ord{2}$ can be evaluated using \cref{cor:wicks-theorem-for-pt}.
Assuming Brillouin's theorem for simplicity,
\begin{align}
\nonumber
  \Y\ord{2}
=
  R_0
  V_\mr{c}
  R_0
  V_\mr{c}
  \kt{\F}
=&\
\nonumber
\diagram[bottom]{
  \interaction{2}{1g}{(0,-0.5)}{ddot}{sawtooth};
  \node[ddot] (2g2) at (1,0) {};
  \draw[-<-] (1g1) to ++(-0.25,+1);
  \draw[->-=0.25,->-=0.75] (1g1) to node[midway,ddot] (2g1) {} ++(+0.25,+1);
  \draw[sawtooth] (2g1) to (2g2);
  \draw[->-,bend left =45] (1g2) to (2g2);
  \draw[-<-,bend right=45] (1g2) to (2g2);
  \draw[thick,flexdotted] (-0.3,-0.27) to ++(1.6,0);
  \draw[thick,flexdotted] (-0.3,+0.35) to ++(0.7,0);
  \draw[opacity=0] (0.5,-0.5) circle (0.125cm);
}
+
\diagram[bottom]{
  \interaction{2}{1g}{(0,-0.5)}{ddot}{sawtooth};
  \node[ddot] (2g2) at (1,0) {};
  \draw[->-] (1g1) to ++(-0.25,+1);
  \draw[-<-=0.25,-<-=0.75] (1g1) to node[midway,ddot] (2g1) {} ++(+0.25,+1);
  \draw[sawtooth] (2g1) to (2g2);
  \draw[->-,bend left =45] (1g2) to (2g2);
  \draw[-<-,bend right=45] (1g2) to (2g2);
  \draw[thick,flexdotted] (-0.3,-0.27) to ++(1.6,0);
  \draw[thick,flexdotted] (-0.3,+0.35) to ++(0.7,0);
  \draw[opacity=0] (0.5,-0.5) circle (0.125cm);
}
+
\diagram[bottom]{
  \interaction{2}{t}{(0,-0.5)}{ddot}{sawtooth};
  \draw[->-=0.25,->-=0.75] (t1) to node[midway,ddot] (g1) {}
    ++(-0.25,1);
  \draw[-<-=0.7] (t1) to ++(+0.25,1);
  \draw[->-=0.25,->-=0.75] (t2) to node[midway,ddot] (g2) {}
    ++(-0.25,1);
  \draw[-<-=0.7] (t2) to ++(+0.25,1);
  \draw[sawtooth] (g1)--(g2);
  \draw[thick,flexdotted] (-0.3,-0.27) to ++(1.6,0);
  \draw[thick,flexdotted] (-0.4,+0.35) to ++(1.8,0);
  \draw[opacity=0] (0.5,-0.5) circle (0.125cm);
}
+
\diagram[bottom]{
  \interaction{2}{t}{(0,-0.5)}{ddot}{sawtooth};
  \draw[-<-=0.25,-<-=0.75] (t1) to node[midway,ddot] (g1) {}
    ++(-0.25,1);
  \draw[->-=0.7] (t1) to ++(+0.25,1);
  \draw[-<-=0.25,-<-=0.75] (t2) to node[midway,ddot] (g2) {}
    ++(-0.25,1);
  \draw[->-=0.7] (t2) to ++(+0.25,1);
  \draw[sawtooth] (g1)--(g2);
  \draw[thick,flexdotted] (-0.3,-0.27) to ++(1.6,0);
  \draw[thick,flexdotted] (-0.4,+0.35) to ++(1.8,0);
  \draw[opacity=0] (0.5,-0.5) circle (0.125cm);
}
+
\diagram[bottom]{
  \interaction{2}{t}{(0,-0.5)}{ddot}{sawtooth};
  \interaction{2}{g}{(1,+0.0)}{ddot}{sawtooth};
  \draw[->-] (t1) to ++(-0.25,1);
  \draw[-<-] (t1) to ++(+0.25,1);
  \draw[->-,bend left] (t2) to (g1);
  \draw[-<-,bend right] (t2) to (g1);
  \draw[->-] (g2) to ++(-0.25,0.5);
  \draw[-<-] (g2) to ++(+0.25,0.5);
  \draw[thick,flexdotted] (-0.3,-0.27) to ++(1.6,0);
  \draw[thick,flexdotted] (-0.4,+0.35) to ++(2.8,0);
  \draw[opacity=0] (0.5,-0.5) circle (0.125cm);
}
\\&\
\label{eq:second-order-wavefunction-graphical}
+
\diagram[bottom]{
  \interaction{2}{1g}{(0,-0.5)}{ddot}{sawtooth};
  \interaction{2}{2g}{(1.125,0)}{ddot}{sawtooth};
  \draw[-<-] (1g1) to ++(-0.25,1);
  \draw[->-] (1g1) to ++(+0.25,1);
  \draw[-<-] (1g2) to ++(-0.25,1);
  \draw[->-=0.25,->-=0.75] (1g2) to ++(+0.25,1);
  \draw[-<-] (2g2) to ++(-0.25,0.5);
  \draw[->-] (2g2) to ++(+0.25,0.5);
  \draw[thick,flexdotted] (-0.4,-0.27) to ++(2.9,0);
  \draw[thick,flexdotted] (-0.4,+0.35) to ++(2.9,0);
  \draw[opacity=0] (0.5,-0.5) circle (0.125cm);
}
+
\diagram[bottom]{
  \interaction{2}{1g}{(0,-0.5)}{ddot}{sawtooth};
  \interaction{2}{2g}{(1.125,0)}{ddot}{sawtooth};
  \draw[->-] (1g1) to ++(-0.25,1);
  \draw[-<-] (1g1) to ++(+0.25,1);
  \draw[->-] (1g2) to ++(-0.25,1);
  \draw[-<-=0.25,-<-=0.75] (1g2) to ++(+0.25,1);
  \draw[->-] (2g2) to ++(-0.25,0.5);
  \draw[-<-] (2g2) to ++(+0.25,0.5);
  \draw[thick,flexdotted] (-0.4,-0.27) to ++(2.9,0);
  \draw[thick,flexdotted] (-0.4,+0.35) to ++(2.9,0);
  \draw[opacity=0] (0.5,-0.5) circle (0.125cm);
}
+
\diagram[bottom]{
  \interaction{2}{1g}{(0,-0.5)}{ddot}{sawtooth};
  \interaction{2}{2g}{(2,0)}{ddot}{sawtooth};
  \draw[-<-] (1g1) to ++(-0.25,1);
  \draw[->-] (1g1) to ++(+0.25,1);
  \draw[-<-] (1g2) to ++(-0.25,1);
  \draw[->-] (1g2) to ++(+0.25,1);
  \draw[-<-] (2g1) to ++(-0.25,0.5);
  \draw[->-] (2g1) to ++(+0.25,0.5);
  \draw[-<-] (2g2) to ++(-0.25,0.5);
  \draw[->-] (2g2) to ++(+0.25,0.5);
  \draw[thick,flexdotted] (-0.3,-0.27) to ++(1.6,0);
  \draw[thick,flexdotted] (-0.4,+0.35) to ++(3.9,0);
  \draw[opacity=0] (0.5,-0.5) circle (0.125cm);
}
\\
=&\
  \tfr{1}{2}
  \sum_{\substack{abc\\ij}}
  \F_i^a\,
  \fr{
    \ol{g}_{aj}^{bc}
    \ol{g}_{bc}^{ij}
  }{
    \mc{E}_a^i
    \mc{E}_{bc}^{ij}
  }
+
  \tfr{1}{2}
  \sum_{\substack{ab\\ijk}}
  \F_i^a
  \fr{
    \ol{g}_{jk}^{ib}
    \ol{g}_{ab}^{jk}
  }{
    \mc{E}_a^i
    \mc{E}_{ab}^{jk}
  }
+
  \tfr{1}{2^3}
  \sum_{\substack{abcd\\ij}}
  \F_{ij}^{ab}
  \fr{
    \ol{g}_{ab}^{cd}
    \ol{g}_{cd}^{ij}
  }{
    \mc{E}_{ab}^{ij}
    \mc{E}_{cd}^{ij}
  }
+
  \tfr{1}{2^3}
  \sum_{\substack{ab\\ijkl}}
  \F_{ij}^{ab}
  \fr{
    \ol{g}_{kl}^{ij}
    \ol{g}_{ab}^{kl}
  }{
    \mc{E}_{ab}^{ij}
    \mc{E}_{ab}^{kl}
  }
\nonumber
\\&\
+
  \sum_{\substack{abc\\ijk}}
  \F_{ij}^{ab}
  \fr{
    \ol{g}_{ac}^{ik}
    \ol{g}_{kb}^{cj}
  }{
    \mc{E}_{ab}^{ij}
    \mc{E}_{ac}^{ik}
  }
+
  \tfr{1}{2^2}
  \sum_{\substack{abcd\\ijk}}
  \F_{ijk}^{abc}
  \fr{
    \ol{g}_{ad}^{ij}
    \ol{g}_{bc}^{dk}
  }{
    \mc{E}_{abc}^{ijk}
    \mc{E}_{ad}^{ij}
  }
-
  \tfr{1}{2^2}
  \sum_{\substack{abc\\ijkl}}
  \F_{ijk}^{abc}
  \fr{
    \ol{g}_{ab}^{il}
    \ol{g}_{lc}^{jk}
  }{
    \mc{E}_{abc}^{ijk}
    \mc{E}_{ab}^{il}
  }
+
  \tfr{1}{2^4}
  \sum_{\substack{abcd\\ijkl}}
  \F_{ijkl}^{abcd}
  \fr{
    \ol{g}_{ab}^{ij}
    \ol{g}_{cd}^{kl}
  }{
    \mc{E}_{abcd}^{ijkl}
    \mc{E}_{ab}^{ij}
  }
\end{align}
where the operators in the final diagram do not form an equivalent pair because they pass through different resolvent lines.
The third-order contribution to the correlation energy can be evaluated as the complete contractions of $V_\mr{c}R_0V_\mr{c}R_0V_\mr{c}$
\begin{align}
  E_\mr{c}\ord{3}
\,{=}\,
\diagram{
  %top
  \interaction{2}{1g}{(0,+0.5)}{ddot}{sawtooth};
  %bottom
  \interaction{2}{2g}{(0,-0.5)}{ddot}{sawtooth};
  \draw[->-=0.25,->-=0.75, bend left]
    (2g1)
    to
      node[midway,ddot] (g1) {}
    (1g1);
  \draw[-<-=0.65,bend right] (2g1) to (1g1);
  \draw[->-=0.25,->-=0.75, bend left]
    (2g2)
    to
      node[midway,ddot] (g2) {}
    (1g2);
  \draw[-<-=0.65,bend right] (2g2) to (1g2);
  \draw[sawtooth] (g1)--(g2);
  \draw[thick,flexdotted] (-0.3,-0.3) to ++(1.6,0);
  \draw[thick,flexdotted] (-0.3,+0.3) to ++(1.6,0);
  \draw[opacity=0] (0.5,-0.5) circle (0.125cm);
}
{+}
\diagram{
  %top
  \interaction{2}{1g}{(0,+0.5)}{ddot}{sawtooth};
  %bottom
  \interaction{2}{2g}{(0,-0.5)}{ddot}{sawtooth};
  \draw[-<-=0.25,-<-=0.75, bend left]
    (2g1)
    to
      node[midway,ddot] (g1) {}
    (1g1);
  \draw[->-=0.65,bend right] (2g1) to (1g1);
  \draw[-<-=0.25,-<-=0.75, bend left]
    (2g2)
    to
      node[midway,ddot] (g2) {}
    (1g2);
  \draw[->-=0.65,bend right] (2g2) to (1g2);
  \draw[sawtooth] (g1)--(g2);
  \draw[thick,flexdotted] (-0.3,-0.3) to ++(1.6,0);
  \draw[thick,flexdotted] (-0.3,+0.3) to ++(1.6,0);
  \draw[opacity=0] (0.5,-0.5) circle (0.125cm);
}
{+}
\diagram{
  %top
  \draw[sawtooth]
    (0,+0.5)
      node[ddot] (1g1) {}
    to
    ++(2,0)
      node[ddot] (1g2) {};
  %middle
  \interaction{2}{g}{(1,+0.0)}{ddot}{sawtooth};
  %bottom
  \interaction{2}{2g}{(0,-0.5)}{ddot}{sawtooth};
  \draw[->-,bend left ] (2g1) to (1g1);
  \draw[-<-,bend right] (2g1) to (1g1);
  \draw[->-,bend left ] (2g2) to (g1);
  \draw[-<-,bend right] (2g2) to (g1);
  \draw[->-,bend left ] (g2) to (1g2);
  \draw[-<-,bend right] (g2) to (1g2);
  \draw[thick,flexdotted] (-0.3,-0.3) to ++(1.6,0);
  \draw[thick,flexdotted] (-0.3,+0.3) to ++(2.6,0);
  \draw[opacity=0] (0.5,-0.5) circle (0.125cm);
}
{=}
  \tfr{1}{2^3}
  \sum_{\substack{abcd\\ij}}
  \fr{
    \ol{g}_{ij}^{ab}
    \ol{g}_{ab}^{cd}
    \ol{g}_{cd}^{ij}
  }{
    \mc{E}_{ab}^{ij}
    \mc{E}_{cd}^{ij}
  }
{+}
  \tfr{1}{2^3}
  \sum_{\substack{ab\\ijkl}}
  \fr{
    \ol{g}_{ij}^{ab}
    \ol{g}_{kl}^{ij}
    \ol{g}_{ab}^{kl}
  }{
    \mc{E}_{ab}^{ij}
    \mc{E}_{ab}^{kl}
  }
{+}
  \sum_{\substack{abc\\ijk}}
  \fr{
    \ol{g}_{ij}^{ab}
    \ol{g}_{ac}^{ik}
    \ol{g}_{kb}^{cj}
  }{
    \mc{E}_{ab}^{ij}
    \mc{E}_{ac}^{ik}
  }
\end{align}
which is equivalent to contracting the doubles contributions to $\Y\ord{2}$ with $\tfr{1}{4}\ol{g}_{ij}^{ab}\tl{a}^{ij}_{ab}$.
Note that $E_\mr{c}\ord{m+1}$ always only depends on the doubles contribution to $\Y\ord{m}$, but that the doubles coefficients themselves may involve triples, quadruples and higher contributions from wavefunction components of order less than $m$.
\end{ex}

\begin{ex}
Using
${}\ord{m}c_{ab\cd}^{ij\cd}=\ip{\F_{ij\cd}^{ab\cd}|\Y\ord{m}}$, the second order CI coefficients can be determined from eq~\ref{eq:second-order-wavefunction-graphical} by contracting a bare excitation operator with the top of each diagram.
Interpreting these graphs gives the following.
\begin{align*}
  {}\ord{2}c_a^i
&=
  \tfr{1}{2}
  \sum_{\substack{bc\\j}}
  \fr{
    \ol{g}_{aj}^{bc}
    \ol{g}_{bc}^{ij}
  }{
    \mc{E}_a^i
    \mc{E}_{bc}^{ij}
  }
+
  \tfr{1}{2}
  \sum_{\substack{b\\jk}}
  \fr{
    \ol{g}_{jk}^{ib}
    \ol{g}_{ab}^{jk}
  }{
    \mc{E}_a^i
    \mc{E}_{ab}^{jk}
  }
\\
  {}\ord{2}c_{ab}^{ij}
&=
  \tfr{1}{2}
  \sum_{\substack{cd}}
  \fr{
    \ol{g}_{ab}^{cd}
    \ol{g}_{cd}^{ij}
  }{
    \mc{E}_{ab}^{ij}
    \mc{E}_{cd}^{ij}
  }
+
  \tfr{1}{2}
  \sum_{\substack{kl}}
  \fr{
    \ol{g}_{kl}^{ij}
    \ol{g}_{ab}^{kl}
  }{
    \mc{E}_{ab}^{ij}
    \mc{E}_{ab}^{kl}
  }
+
  \op{P}_{(a/b)}^{(i/j)}
  \sum_{\substack{c\\k}}
  \fr{
    \ol{g}_{ac}^{ik}
    \ol{g}_{kb}^{cj}
  }{
    \mc{E}_{ab}^{ij}
    \mc{E}_{ac}^{ik}
  }
\\
  {}\ord{2}c_{abc}^{ijk}
&=
  \op{P}_{(a/bc)}^{(ij/k)}
  \sum_{\substack{d}}
  \F_{ijk}^{abc}
  \fr{
    \ol{g}_{ad}^{ij}
    \ol{g}_{bc}^{dk}
  }{
    \mc{E}_{abc}^{ijk}
    \mc{E}_{ad}^{ij}
  }
-
  \op{P}_{(ab/c)}^{(i/jk)}
  \sum_{\substack{l}}
  \fr{
    \ol{g}_{ab}^{il}
    \ol{g}_{lc}^{jk}
  }{
    \mc{E}_{abc}^{ijk}
    \mc{E}_{ab}^{il}
  }
\\
  {}\ord{2}c_{abcd}^{ijkl}
&=
  \op{P}_{(ab/cd)}^{(ij/kl)}
  \fr{
    \ol{g}_{ab}^{ij}
    \ol{g}_{cd}^{kl}
  }{
    \mc{E}_{abcd}^{ijkl}
    \mc{E}_{ab}^{ij}
  }
\end{align*}
Note that the second order quadruples coefficient is disconnected.
Prop.~\ref{prop:second-order-c4} shows that the second-order quadruples operator is actually a simple product of first-order doubles operators.
This fact was an early motivation for coupled-pair many-electron theory,\footnote{This is the original name for coupled-cluster doubles.} since it justifies approximating
$
  \Y_\mr{CIDQ}
=
  (1+C_2+C_4)\F
$
by
$
  \Y_\mr{CPMET}
=
  (1 + C_2 + \tfr{1}{2}C_2^2)\F
$.
\end{ex}

\begin{prop}
\label{prop:second-order-c4}
\thmstatement{
$
  {}\ord{2} C_4
=
  \tfr{1}{2}
  {}\ord{1} C_2^2
$
}
\thmproof{
This follows from rearranging the resolvent denominator.
\begin{align*}
  \fr{1}{\mc{E}_{abcd}^{ijkl}\mc{E}_{ab}^{ij}}
+
  \fr{1}{\mc{E}_{abcd}^{ijkl}\mc{E}_{cd}^{kl}}
=
  \fr{
    \mc{E}_{cd}^{kl} + \mc{E}_{ab}^{ij}
  }{
    \mc{E}_{abcd}^{ijkl}
    \mc{E}_{ab}^{ij}
    \mc{E}_{cd}^{kl}
  }
=
  \fr{
    1
  }{
    \mc{E}_{ab}^{ij}
    \mc{E}_{cd}^{kl}
  }
\implies
  {}\ord{2}C_4
=
  \pr{\tfr{1}{2}}^4
  \sum_{\substack{abcd\\ijkl}}
  \tl{a}^{ijkl}_{abcd}
  \fr{
    \ol{g}_{ab}^{ij}
    \ol{g}_{cd}^{kl}
  }{
    \mc{E}_{abcd}^{ijkl}
    \mc{E}_{ab}^{ij}
  }
=
  \tfr{1}{2}\cdot
  \pr{\tfr{1}{2}}^4
  \sum_{\substack{abcd\\ijkl}}
  \tl{a}^{ijkl}_{abcd}
  \fr{
    \ol{g}_{ab}^{ij}
    \ol{g}_{cd}^{kl}
  }{
    \mc{E}_{ab}^{ij}
    \mc{E}_{cd}^{kl}
  }
=
  \tfr{1}{2}
  {}\ord{1}C_2^2
\end{align*}
}
\end{prop}

\begin{lem}\label{lem:energy-substitution}
\thmtitle{The Energy Substitution Lemma}
\thmstatement{
  $\Y\ord{m}$ equals the sum of a ``principal term''
  $(R_0V_\mr{c})^m\F$
  plus all possible substitutions of adjacent factors $(R_0V_\mr{c})^{r_i}$ in the principal term by $R_0E_\mr{c}\ord{r_i}$.
  Each term in the sum is weighted by a sign factor $(-)^k$, where $k$ is the number of substitutions.
}
\thmproof{
  See \cref{app:linked-diagram-theorem}.
}
\end{lem}

\begin{ex}
Lemma~\ref{lem:energy-substitution} is consistent with equation~\ref{eq:first-and-second-order-principal-terms} because substitution of the rightmost factors in the principal term leaves a resolvent acting on the reference determinant and because the first-order energy contribution equals zero.
The first non-trivial examples of the energy substitution lemma begin at third order.
\begin{align}
\label{eq:energy-substitution-psi-3}
  \Y\ord{3}
=&\
  R_0V_\mr{c}R_0V_\mr{c}R_0V_\mr{c}\F
-
  R_0E_\mr{c}\ord{2}R_0V_\mr{c}\F
\\
\label{eq:energy-substitution-psi-4}
  \Y\ord{4}
=&\
  R_0V_\mr{c}R_0V_\mr{c}R_0V_\mr{c}R_0V_\mr{c}\F
-
  R_0E_\mr{c}\ord{2}R_0V_\mr{c}R_0V_\mr{c}\F
-
  R_0V_\mr{c}R_0E_\mr{c}\ord{2}R_0V_\mr{c}\F
-
  R_0E_\mr{c}\ord{3}R_0V_\mr{c}\F
\\
\nonumber
  \Y\ord{5}
=&\
  R_0V_\mr{c}R_0V_\mr{c}R_0V_\mr{c}R_0V_\mr{c}R_0V_\mr{c}\F
-
  R_0E_\mr{c}\ord{2}R_0V_\mr{c}R_0V_\mr{c}R_0V_\mr{c}\F
-
  R_0V_\mr{c}R_0E_\mr{c}\ord{2}R_0V_\mr{c}R_0V_\mr{c}\F
\\&\
\nonumber
-
  R_0V_\mr{c}R_0V_\mr{c}R_0E_\mr{c}\ord{2}R_0V_\mr{c}\F
+
  R_0E_\mr{c}\ord{2}R_0E_\mr{c}\ord{2}R_0V_\mr{c}\F
-
  R_0E_\mr{c}\ord{3}R_0V_\mr{c}R_0V_\mr{c}\F
\\&\
\label{eq:energy-substitution-psi-5}
-
  R_0V_\mr{c}R_0E_\mr{c}\ord{3}R_0V_\mr{c}\F
-
  R_0E_\mr{c}\ord{4}R_0V_\mr{c}\F
\end{align}
\end{ex}

\begin{thm}\label{thm:bracketing-theorem}
\thmtitle{The Bracketing Theorem}
\thmstatement{
  $\Y\ord{m}$ equals the principal term plus all possible insertions of nested brackets into the principal term.
  Each term in the sum is weighted by $(-)^k$ where $k$ is the total number of brackets.\footnote{
  The ``brackets'' here are reference expectation values: $\ip{W}\equiv\ip{\F|W|\F}$.
}
}
\thmproof{
  See \cref{app:linked-diagram-theorem}.
}
\end{thm}


\begin{ex}
Equations \ref{eq:energy-substitution-psi-3} and \ref{eq:energy-substitution-psi-4} are clearly consistent with \cref{thm:bracketing-theorem}, since $E_\mr{c}\ord{2}{=}\,\ip{V_\mr{c}R_0V_\mr{c}}$ and $E_\mr{c}\ord{3}{=}\,\ip{V_\mr{c}R_0V_\mr{c}R_0V_\mr{c}}$.
\begin{align}
\label{eq:bracketing-psi-3}
  \Y\ord{3}
=&\
  R_0V_\mr{c}R_0V_\mr{c}R_0V_\mr{c}\F
-
  R_0\ip{V_\mr{c}R_0V_\mr{c}}R_0V_\mr{c}\F
\\
  \Y\ord{4}
=&\
  R_0V_\mr{c}R_0V_\mr{c}R_0V_\mr{c}R_0V_\mr{c}\F
-
  R_0\ip{V_\mr{c}R_0V_\mr{c}}R_0V_\mr{c}R_0V_\mr{c}\F
-
  R_0V_\mr{c}R_0\ip{V_\mr{c}R_0V_\mr{c}}R_0V_\mr{c}\F
-
  R_0\ip{V_\mr{c}R_0V_\mr{c}R_0V_\mr{c}}R_0V_\mr{c}\F
\\
\intertext{
The first non-vanishing terms with nested brackets appear at fifth-order
}
\nonumber
  \Y\ord{5}
=&\
  R_0V_\mr{c}R_0V_\mr{c}R_0V_\mr{c}R_0V_\mr{c}R_0V_\mr{c}\F
-
  R_0\ip{V_\mr{c}R_0V_\mr{c}}R_0V_\mr{c}R_0V_\mr{c}R_0V_\mr{c}\F
-
  R_0V_\mr{c}R_0\ip{V_\mr{c}R_0V_\mr{c}}R_0V_\mr{c}R_0V_\mr{c}\F
\\&\
\nonumber
-
  R_0V_\mr{c}R_0V_\mr{c}R_0\ip{V_\mr{c}R_0V_\mr{c}}R_0V_\mr{c}\F
+
  R_0\ip{V_\mr{c}R_0V_\mr{c}}R_0\ip{V_\mr{c}R_0V_\mr{c}}R_0V_\mr{c}\F
-
  R_0\ip{V_\mr{c}R_0V_\mr{c}R_0V_\mr{c}}R_0V_\mr{c}R_0V_\mr{c}\F
\\&\
-
  R_0V_\mr{c}R_0\ip{V_\mr{c}R_0V_\mr{c}R_0V_\mr{c}}R_0V_\mr{c}\F
-
  R_0\ip{V_\mr{c}R_0V_\mr{c}R_0V_\mr{c}R_0V_\mr{c}}R_0V_\mr{c}\F
+
  R_0\ip{V_\mr{c}R_0\ip{V_\mr{c}R_0V_\mr{c}}R_0V_\mr{c}}R_0V_\mr{c}\F
\end{align}
which follows from substituting equation~\ref{eq:bracketing-psi-3} into $E\ord{4}=\ip{\F|V_\mr{c}|\Y\ord{3}}$ in the energy substitution expansion of $\Y\ord{5}$.
\end{ex}

\begin{dfn}

$
  GG'
=
  (
    L\cup L',
    O\cup O',
    h\oplus h',
    t\oplus t'
  )
$
where
$
  (h\oplus h')(l)
\equiv
\left\{
\begin{array}{ll}
  h(l) & l\in L\\
  h'(l) & l\in L'
\end{array}
\right.
$\\
\textit{combination graphs}\\
The \textit{combination graphs} of $G(\rh,m)$ and $G'(\rh',m')$ have the form
$GG'(\rh_\si,m+m')$\\
\begin{align*}
  \rh_\pi(o)
=
\left\{
\begin{array}{ll}
  \pi(\rh(o))      & o\in O\\
  \pi(\rh'(o) + m) & o\in O'
\end{array}
\right.
&&
  \pi
\in
\end{align*}
\end{dfn}

\begin{dfn}
\thmtitle{Insertion graph}
\end{dfn}

%\begin{dfn}
%\thmtitle{Insertion graph}
%The graphs encountered in perturbation theory contain operators separated by resolvents.
%Two disconnected parts are termed \textit{separate} if no resolvent line crosses both.
%Vertical spaces between resolvent lines are termed \textit{levels}, which we number from %top to bottom for each separate part.
%A pair of resolvents with no intervening operators encloses an \textit{empty level}.
%Inserted brackets in the bracketing expansion produce separate and unlinked \textit{insertion graphs}.
%The empty level in the \textit{remainder} created by the insertion is the \textit{level of the insertion}.
%\end{dfn}

\begin{ex}
Assuming Brillouin's theorem, the simplest non-vanishing term with an inserted bracket appears in $\Y\ord{3}$.
\begin{align*}
  R_0
  \ip{V_\mr{c}R_0V_\mr{c}}
  R_0
  V_\mr{c}
  \F
=
\diagram{
  \interaction{2}{1g}{(0,-0.5)}{ddot}{sawtooth};
  \draw[->-] (1g1) to ++(-0.25,1);
  \draw[-<-] (1g1) to ++(+0.25,1);
  \draw[->-] (1g2) to ++(-0.25,1);
  \draw[-<-] (1g2) to ++(+0.25,1);
  \interaction{2}{2g}{(2,-0.5)}{ddot}{sawtooth};
  \interaction{2}{3g}{(2,+0.5)}{ddot}{sawtooth};
  \draw[->-=0.4,bend left ] (2g1) to (3g1);
  \draw[-<-=0.6,bend right] (2g1) to (3g1);
  \draw[->-=0.4,bend left ] (2g2) to (3g2);
  \draw[-<-=0.6,bend right] (2g2) to (3g2);
  \draw[thick,flexdotted] (-0.3,-0.25) to ++(1.6,0);
  \draw[thick,flexdotted] (-0.4,+0.35) to ++(1.8,0);
  \draw[opacity=0] (0.5,-0.5) circle (0.125cm);
  \draw[thick,flexdotted] (1.7,0) to ++(1.6,0);
  \padborder{5pt};
  \draw[double, thick] (current bounding box.south west)--(current bounding box.south east);
  \node at (0.5,-1) {remainder};
  \node at (2.5,-1) {insertion};
  \node[inner sep=0pt] at (-1.5,0) {
    \begin{tabular}{c}
    level of the\\insertion
    \end{tabular}
  };
  \draw[->] (-2.5,0) to (-0.3,0);
  \node[inner sep=0pt] at (+4,+0.4) {$0\eth$\,level};
  \node[inner sep=0pt] at (+4,-0.3) {$1\rst$\,level};
}
\end{align*}
\end{ex}

%\begin{thm}
%\thmtitle{The Frantz-Mills Factorization Theorem}
%\thmstatement{
%A graph with an insertion equals the sum over all corresponding graphs in which the top of the inserted graph .
%}
%\end{thm}

%\begin{ex}
%$R_0\ip{V_\mr{c}R_0V_\mr{c}}R_0V_\mr{c}\F$
%\end{ex}

%\begin{thm}
%\thmtitle{The Linked Diagram Theorem}
%\thmstatement{
%An in
%}
%\end{thm}




\newpage
\appendix
\section{Proof of the Linked-Diagram Theorem}\label{app:linked-diagram-theorem}

\begin{ntt}\label{ntt:operator-combinations}
Let
``$Y^m$ choose $Z^k$'', denoted ${}^mC_k(Y:Z)$,
refer to a sum over the $m$ choose $k$ permutations of $Y^{m-k}Z^k$,\,\footnote{For example,
$
  {}^4C_2(Y:Z)
=
  Y^2Z^2
+
  YZYZ
+
  YZ^2Y
+
  ZY^2Z
+
  ZYZY
+
  Z^2Y^2
$.
}
where $Y$ and $Z$ are operators that may or may not commute.\,\footnote{
  If they do commute, then ${}^mC_k(Y:Z)={n\choose k}Y^{m-k}Z^k$.
}
This defines a generalization of the binomial theorem.
\begin{align}
\label{eq:generalized-binomial-theorem}
  (
    Y
  +
    Z
  )^m
=
  \sum_{k=0}^m
  {}^mC_k(Y:Z)
\end{align}
Furthermore, let
$
  {}^mC(Y:Z_1,\ld,Z_k)
$
be a sum over permutations of
$
  Y^{m-k}
  Z_1\cd Z_k
$ that preserve the ordering of the $Z_i$'s.\,\footnote{
  For example,
$
  {}^4C(Y:Z_1,Z_2)
=
  Y^2Z_1Z_2
+
  YZ_1YZ_2
+
  YZ_1Z_2Y
+
  Z_1Y^2Z_2
+
  Z_1YZ_2Y
+
  Z_1Z_2Y^2
$.
}
When all of the $Z_i$'s equal $Z$, we can write
$
  {}^mC(Y:Z_1,\ld,Z_k)
=
  {}^mC_k(Y:Z)
$.
\end{ntt}

\begin{prop}
\label{prop:wavefunction-infinite-recursion}
\thmstatement{
$\ds{
  \Y(\la)
=
  \sum_{m=0}^\infty
  \pr{
    R_0
    (\la V_\mr{c} - E(\la))
  }^m
  \F
}$
}
\thmproof{
  This follows by infinite recursion of equation~\ref{eq:lambda-dependent-recursive-series} with the assumption
  $\ds{
  \lim_{m\rightarrow\infty}
  \pr{
    R_0
    (\la V_\mr{c} - E(\la))
  }^m
  \Y(\la)
  =
    0
  }$.
}
\end{prop}

\begin{dfn}\label{dfn:integer-compositions}
\thmtitle{Integer compositions}
The \textit{compositions} of an integer $m$ are the ways of writing $m$ as a sum of positive integers.
The full set of integer compositions of $m$ is given by
$
  \mc{C}(m)
=
  \mc{C}_1(m)
  \cup
  \mc{C}_2(m)
  \cup
  \cd
  \cup
  \mc{C}_m(m)
$
where 
$
  \mc{C}_k(m)
=
  \{
    (r_1,\ld,r_k)\in\mb{N}_0^k
  \,|\,
    r_1+\cd+r_k
  =
    m
  \}
$
are the integer compositions of $m$ into $k$ parts.
\end{dfn}

\begin{lem}
\label{lem:energy-substitution-proof}
\thmtitle{The Energy Substitution Lemma}
\thmstatement{
  $\Y\ord{m}$ equals the sum of a ``principal term''
  $(R_0V_\mr{c})^m\F$
  plus all possible substitutions of adjacent factors $(R_0V_\mr{c})^{r_i}$ in the principal term by $R_0E_\mr{c}\ord{r_i}$.
  Each term in the sum is weighted by a sign factor $(-)^k$, where $k$ is the number of substitutions.
}\vspace{5pt}
\thmproof{
Using equation~\ref{eq:generalized-binomial-theorem} and a double sum identity\footnote{
  Reverse double-sum reduction:
  $\ds{
    \sum_{m=0}^\infty
    \sum_{k=0}^m
    t_{m-k,k}
  =
    \sum_{k'=0}^\infty
    \sum_{k=0}^\infty
    t_{k',k}
  }$.
  See
  \url{http://functions.wolfram.com/GeneralIdentities/12/}.
} in the infinite recursion formula for $\Y(\la)$ gives the following.
{\footnotesize
\begin{align*}
  \Y(\la)
=
  \sum_{m=0}^\infty
  (
    R_0
    (
      \la V_\mr{c}
    -
      E(\la)
    )
  )^m
  \F
=
  \sum_{m=0}^\infty
  \sum_{k=0}^m
  \la^{m-k}
  (-)^k\,\,
  {}^mC_k
  (
    R_0V_\mr{c}:
    R_0E(\la)
  )
  \F
=
  \sum_{k'=0}^\infty
  \sum_{k=0}^\infty
  \la^{k'}
  (-)^{k}\,\,
  {}^{k'+k}\hspace{-1pt}C_k
  (
    R_0V_\mr{c}:
    R_0E(\la)
  )
  \F
\end{align*}}%
The $k'=0$ term has no operators separating $\F$ from the resolvent and vanishes.
Taylor expansion of the energies gives
{\footnotesize
\begin{align*}
  \Y(\la)
=&\
  \sum_{k=0}^\infty
  \sum_{k'=1}^\infty
  {\sum_{p_1=1}^\infty}
  \cd
  {\sum_{p_k=1}^\infty}
  \la^{k' + p_1 + \cd + p_k}
  (-)^{k}\,\,
  {}^{k'+k}\hspace{-1pt}C
  (
    R_0V_\mr{c}:
    R_0E_\mr{c}\ord{p_1},\ld,
    R_0E_\mr{c}\ord{p_k}
  )
  \F
\\
=&\
  \sum_{m=1}^\infty
  \sum_{k=0}^{m-1}
  \sum_{(r_1,\ld,r_{k+1})}^{\mc{C}_{k+1}(m)}
  \la^m
  (-)^k\,\,
  {}^{k+r_1}\hspace{-1pt}C
  (
    R_0V_\mr{c}:
    R_0E_\mr{c}\ord{r_2},\ld,
    R_0E_\mr{c}\ord{r_{k+1}}
  )
  \F
\end{align*}}%
where we have grouped powers of $\la$ using a multi-sum reduction.
Writing the inner sums as a sum over $\mc{C}(m)$ we find
\begin{align}
  \Y\ord{m}
=
  \left.
  \fr{1}{m!}
  \pd{^m\Y(\la)}{\la^m}
  \right|_{\la=0}
=
  \sum_{(r_1,\ld,r_{k+1})}^{\mc{C}(m)}
  (-)^k\,\,
  {}^{k+r_1}\hspace{-1pt}C
  (
    R_0V_\mr{c}:
    R_0E_\mr{c}\ord{r_2},\ld,
    R_0E_\mr{c}\ord{r_{k+1}}
  )
  \F
\end{align}
which, given \cref{ntt:operator-combinations} and definition~\ref{dfn:integer-compositions}, is an algebraic statement of the proposition, completing the proof.
}
\end{lem}

\begin{thm}
\thmtitle{The Bracketing Theorem}
\thmstatement{
  $\Y\ord{m}$ equals the principal term plus all possible insertions of nested brackets into the principal term.
  Each term in the sum is weighted by $(-)^k$ where $k$ is the total number of brackets.
}\vspace{5pt}
\thmproof{
  The proposition holds for $m=1$ because 
  $\Y\ord{1}=R_0V_\mr{c}\F$ and there are no possible bracketings.
  Assume it holds for $m-1$.
  Then by the energy substitution lemma it also holds for $m$ because $E_\mr{c}\ord{r_i}$
  equals
  $\ip{\F|V_\mr{c}|\Y\ord{r_i}}$
  which, by our inductive assumption, equals
  $\ip{V_\mr{c}(R_0V_\mr{c})^{r_i}}$
  plus all nested bracketings weighted by appropriate sign factors.
}
\end{thm}


\end{document}