\documentclass[11pt]{article}
\usepackage[cm]{fullpage}
%%AVC PACKAGES
\usepackage{avcgreek}
\usepackage{avcfonts}
\usepackage{avcmath}
\usepackage[numberby=section,skip=9pt plus 2pt minus 7pt]{avcthm}
\usepackage{qcmacros}
\usepackage{goldstone}
%%MACROS FOR THIS DOCUMENT
\numberwithin{equation}{section}
\usepackage[
  margin=1.5cm,
  includefoot,
  footskip=30pt,
  headsep=0.2cm,headheight=1.3cm
]{geometry}
\usepackage{fancyhdr}
\pagestyle{fancy}
\fancyhf{}
\fancyhead[LE,RO]{Quiz 9, Handout 1: Response theory}
\fancyfoot[CE,CO]{\thepage}
\usepackage{url}
\makeatother
\newcommand{\resolventline}[2][1]{
  \tikz[overlay]{
      \draw[thick,flexdotted] (0,-1ex) to ++(0,#1*4.5ex) node[above,inner sep=1pt] {#2};
  }
}
\usepackage{accents}
\newcommand{\oc}[1]{\ensuremath{\accentset{\circ}{#1}}}
\newcommand{\wtl}[1]{\ensuremath{\widetilde{#1}}}
\usepackage{multicol}

\begin{document}

\setlength{\abovedisplayskip}{5pt}
\setlength{\belowdisplayskip}{5pt}


\setcounter{section}{8}
\section{Response theory}

\begin{rmk}
In the presence of a time-varying field, a molecule's electronic wavefunction is no longer simply an eigenfunction of the Hamiltonian.
Instead, its electronic structure is described by the \textit{time-dependent Schr\"odinger equation}
\begin{align}
\label{eq:schrodinger-equation}
  H(t)
  \Y(t)
=
  i
  \pd{\Y(t)}{t}
&&
  H(t)
=
  H
+
  V(t)
\end{align}
where $H$ is the usual electronic Hamiltonian and $V(t)$ is an \textit{interaction Hamiltonian} describing the energetic influence of the field.
A general series solution to equation~\ref{eq:schrodinger-equation}, known as the \textit{Dyson series}, is derived in \cref{appendix:dyson-series}.
The interaction Hamiltonian can be expressed as a sum over one-electron operators $V_\b$, representing the electronic degrees of freedom which couple to the external field, scaled by \textit{time-envelopes} $f_\b(t)$ which control the strength of the applied field over time.
\begin{align}
\label{eq:interaction-time-envelopes}
  V(t)
=
\ts{
  \sum_\b
  V_\b
  f_\b(t)
}
\end{align}
One of the most important examples is the Hamiltonian of a dipole in an electric field, which is discussed in \cref{ex:dipole-approximation} below.
The zeroth order solutions of equation~\ref{eq:schrodinger-equation} are termed \textit{stationary states}, which have the following form.\footnotemark
\footnotetext{
When $\bm{f}=\bo{0}$, the Hamiltonian loses its time-dependence and we can write $\left.\Y(t)\right|_{\bm{f}=\bo{0}}=\f(t)\Y$ where $\f(t)$ is independent of the electronic coordinates.
Substituting this into eq~\ref{eq:schrodinger-equation} and rearranging gives
$
  H\Y/\Y
=
  i\dot{\f}(t)/\f(t)
$, which equals a constant $E$ since each side depends in different variables.
Therefore, $H\Y=E\Y$ and $i\dot{\f}(t)=E\f(t)$.  Integrating the latter gives $\f(t)=e^{-iEt}$.
}
\begin{align}
  \left.
  \Y(t)
  \right|_{\bm{f}=\bo{0}}
=
  e^{-iE_kt}
  \Y_k
&&
  H\Y_k
=
  E_k\Y_k
\end{align}
As a boundary condition we assume that $V(t)$ vanishes in the past, where $\Y(t)$ is initially in the ground stationary state.
\begin{align}
\label{eq:boundary-conditions}
  \lim_{t\rightarrow-\infty}
  f_\b(t)
=
  0
&&
  \lim_{t\rightarrow-\infty}
  e^{+iHt}
  \Y(t)
=
  \Y_0
\end{align}
This limiting behavior can be enforced by introducing a complex shift in the frequency domain of $f_\b(t)$'s Fourier expansion.\footnotemark
\footnotetext{
  This is a slightly unusual convention for the Fourier transform.
  A useful mnemonic for checking these is
  $
    \int_{-\infty}^\infty
    dk\,
    e^{ikx}
  =
    2\pi\,
    \d(x)
  $.
}
\begin{align}
\label{eq:frequency-envelopes}
  f_\b(t)
=
  \int_{-\infty}^\infty
  d\w\,
  f_\b(\w_\ev)
  e^{-i\w_\ev t}
&&
  f_\b(\w_\ev)
\equiv
  (2\pi)^{-1}
  \int_{-\infty}^\infty
  dt\,
  f_\b(t)
  e^{+i\w_\ev t}
&&
  \w_\ev
\equiv
  \w
+
  i\ev
&&
  \ev
=
  |\ev|
\end{align}
This has the effect of scaling the time envelope by a damping factor $e^{\ev t}$.
For sufficiently small $\ev$, this scaled envelope will match the original one to arbitrary precision in an arbitrarily wide window about the time origin.
The fact that the interaction Hamiltonian and the coupling operators $\{V_\b\}$ are Hermitian implies the following identities.
\begin{align}
  f_\b^*(t)
=
  f_\b(t)
&&
  f_\b^*(\w_{\ev})
=
  f_\b(-\w_{-\ev})
\end{align}
\end{rmk}


\begin{ex}
\label{ex:dipole-approximation}
The dominant coupling of an electronic system to an external electric or magnetic field is mediated through its dipoles, leading to \textit{the dipole approximation}.
Quantizing the classical formulae for these interaction energies gives
\begin{align}
\begin{array}{r@{\ }l@{\ }l@{\hspace{2cm}}r@{\ }l@{\hspace{2cm}}r@{\ }l}
  V_{\bo{E}}(t)
\approx&
-\,
  \bm{\mu}\cdot\bo{E}(t)
&=
-
  \sum_\b
  \mu_\b
  \mc{E}_\b(t)
&
  \bm{\mu}
&=
  \sum_{pq}
  \ip{\y_p|\op{\bm{\mu}}\,|\y_q}
  a_p\dg a_q
&
  \op{\bm{\mu}}
&=
-
  \op{\bo{r}}
\\
  V_{\bo{B}}(t)
\approx&
-
  \bm{m}\cdot\bo{B}(t)
&=
-
  \sum_\b
  m_\b
  \mc{B}_\b(t)
&
  \bm{m}
&=
  \sum_{pq}
  \ip{\y_p|\op{\bm{m}}|\y_q}
  a_p\dg a_q
&
  \op{\bm{m}}
&=
-
  \tfr{1}{2}
  (
    \op{\bm{l}}
  +
    2\,
    \op{\bo{s}}
  )
\end{array}
\end{align}
where $\op{\bm{\mu}}$ and $\op{\bm{m}}$ are the first-quantized electric and magnetic dipole operators.\footnote{
More generally, these expressions are
$
  \op{\bm{\mu}}
=
  q_e\,\op{\bo{r}}
$,
where $q_e=-e$ is the charge of an electron,
and
$
  \op{\bm{m}}
=
  \mu_\mr{B}
  (
    g_l\,
    \op{\bm{l}}
  +
    g_{\mr{s}}\,
    \op{\bo{s}}
  )
$
where $\mu_{\mr{B}}=\tfr{1}{2}\cdot\tfr{e\hbar}{m_e}$ is the Bohr magneton and $g_l=-1$, $g_\mr{s}=-2$ are the spin and orbital \textit{$g$-factors}.
Note that the exact $g_\mr{s}$ actually deviates very slightly from $2$ due to effects arising in quantum field theory.
The orbital angular momentum operator is given by $\op{\bm{l}}=\op{\bo{r}}\times\op{\bo{p}}$ and $\op{\bo{s}}$ is the intrinsic spin angular momentum operator.
}
The leading terms neglected by the dipole approximation are quadratic in the field amplitudes.
These weaker interactions are mediated through the higher moments (quadrupole, octupole, etc.) of the charge and current distributions and may become important in symmetric molecules where certain dipole interactions are ``symmetry forbidden''.
\end{ex}





\begin{dfn}
\thmtitle{Quasi-energy}
\begin{align}
  \Y(t)
=
  e^{-i\th(t)}
  \bar{\Y}(t)
&&
  \left.
  \th(t)
  \right|_{\bm{f}=\bo{0}}
=
  E_0
  t
&&
  \lim_{t\rightarrow-\infty}
  \bar{\Y}(t)
=
  \Y_0
\end{align}
\begin{align}
  (
    H(t)
  -
    i
    \tpd{}{t}
  )
  \bar{\Y}(t)
=
  \dot{\th}(t)
  \bar{\Y}(t)
\end{align}
\begin{align}
  \dot{\th}(t)
=
  \int_0^t
  dt'
  \ip{\bar{\Y}(t')|
    H(t')
  -
    i\tpd{}{t'}
  |\bar{\Y}(t')}
\end{align}
\begin{align}
  \br{\d\bar{\Y}(t)}
    H(t)
  -
    i
    \tpd{}{t}
  \kt{\bar{\Y}(t)}
=
  \dot{\th}(t)
  \ip{\d\bar{\Y}(t)|\bar{\Y}(t)}
\end{align}
\begin{align}
  \ip{\d\bar{\Y}(t)|\bar{\Y}(t)}
+
  \ip{\bar{\Y}(t)|\d\bar{\Y}(t)}
=
  0
\end{align}
\begin{align}
  \d
  \ip{\bar{\Y}(t)|
    H(t)
  -
    i
    \tpd{}{t}
  |\bar{\Y}(t)}
+
  i
  \tpd{}{t}
  \ip{\bar{\Y}(t)|\d\bar{\Y}(t)}
=
  0
\end{align}
\end{dfn}


\newpage
\appendix

\section{Dyson series}
\label{appendix:dyson-series}

\begin{dfn}
\thmtitle{Time-evolution operator}
If we know the wavefunction at a particular time $t_0$, we can express the wavefunction at any other time as a unitary transformation of this initial state, $\Y(t)=U(t,t_0)\Y(t_0)$.
This unitary transformation is called the \textit{time-evolution operator}.
\end{dfn}


\begin{dfn}
\thmtitle{Interaction picture}
The \textit{interaction picture} results from to the following similarity transformation.
\begin{align}
  \tl{\Th}(t)
\equiv
  e^{+iHt}
  \Th(t)
&&
  \tl{W}(t)
\equiv
  e^{+iHt}
  W(t)
  e^{-iHt}
\end{align}
Expanding the Schr\"odinger equation in the interaction picture yields the the \textit{Schwinger-Tomonaga equation}.
\begin{align}
\label{eq:schwinger-tomonaga}
  \tl{V}(t)
  \tl{\Y}(t)
=
  i
  \pd{\tl{\Y}(t)}{t}
\end{align}
Multiplying both sides by $-i$ and integrating from $t_0$ to $t$ yields a recursive equation for the time-evolution operator
\begin{align}
\label{eq:integrated-schwinger-tomonaga-equation}
  \tl{\Y}(t)
-
  \tl{\Y}(t_0)
=
-
  i
  \int_{t_0}^t
  dt'\,
  \tl{V}(t')
  \tl{\Y}(t')
&&
\implies
&&
  \tl{U}(t,t_0)
=
  1
-
  i
  \int_{t_0}^t
  dt'
  \tl{V}(t')\,
  \tl{U}(t',t_0)
\end{align}
and infinite recursion of this identity leads to the following expansion.
\begin{align}
\label{eq:time-evolution-infinite-recursion}
  \tl{U}(t,t_0)
=
  \sum_{n=0}^\infty
  (-i)^n
  \int_{t_0}^t
  dt_1
  \int_{t_0}^{t_1}
  dt_2
  \cd
  \int_{t_0}^{t_{n-1}}
  dt_n
  \,
  \tl{V}(t_1)
  \cd
  \tl{V}(t_n)
\end{align}
\end{dfn}

\begin{dfn}
\label{dfn:time-ordering}
\thmtitle{Time-ordering}
Let 
$
  \tl{q}_1(t_1)\cd \tl{q}_n(t_n)
$
be a string of particle-hole operators in the interaction picture.\footnotemark
\footnotetext{
As in
$
  \tl{q}(t)
\equiv
  e^{+iHt}
  q
  e^{-iHt}
$
for some
$q\in\{a_p\}\cup\{a_p\dg\}$.
}
The \textit{time-ordering map} takes this string into
$
  \mc{T}\{
  \tl{q}_1(t_1)\cd \tl{q}_n(t_n)
  \}
\equiv
  \e_\pi \,\tl{q}_{\pi(1)}(t_{\pi(1)})\cd \tl{q}_{\pi(n)}(t_{\pi(n)})
$,
where $\pi\in\mr{S}_n$ is a permutation that puts the time arguments in chronological order, $t_{\pi(1)}>\cd>t_{\pi(n)}$.
\end{dfn}

\begin{ntt}
Let us define the following notation for multivariate integrals by analogy with multi-index summations.\footnotemark
\footnotetext{
Compare these integrals to
$
  \sum_{i_1i_2i_3\cd}^{\{n_0,\ld,n\}}
$
and
$
  \sum_{i_1>i_2>i_3\cd}^{\{n_0,\ld,n\}}
$.
The subscript defines the summation variables, along with any conditions restricting their values, and the superscript indicates the allowed range of values for each variable.
}
\begin{align}
  \int_{t_1t_2t_3\ld}^{[t_0,t]}
  dt_1dt_2dt_3\cd
\equiv
  \int_{t_0}^t
  dt_1
  \int_{t_0}^t
  dt_2
  \int_{t_0}^t
  dt_3
  \cd
&&
  \int_{t_1>t_2>t_3>\cd}^{[t_0,t]}
  dt_1dt_2dt_3\cd
\equiv
  \int_{t_0}^t
  dt_1
  \int_{t_0}^{t_1}
  dt_2
  \int_{t_0}^{t_2}
  dt_3
  \cd
\end{align}
This notation should elucidate the following identity, 
which breaks an unrestricted integral into all possible chronologies.\footnotemark
\footnote{
  The corresponding summation identity would be
$
  \sum_{i_1\neq i_2\neq i_3\neq\cd}^{\{n_0,n\}}
=
  \sum_{\pi}^{\mr{S}_n}
  \sum_{i_{\pi(1)}> i_{\pi(2)}> i_{\pi(3)}>\cd}^{\{n_0,n\}}
$.
  The unrestricted integral is equivalent to an integral over $t_1\neq t_2\neq t_3\neq\cd$ because individual integrand values have ``measure zero'':
$
  \int_{t_j}^{t_j}
  dt_i
=
  0
$.
}
\begin{align}
\label{eq:integral-identity}
  \int_{t_1\cd t_n}^{[t_0,t]}
  dt_1\cd t_n\,
  f(t_1\cd t_n)
=
  \sum_\pi^{\mr{S}_n}
  \int_{t_{\pi(1)}>\ld>t_{\pi(n)}}^{[t_0,t]}
  dt_1\cd t_n\,
  f(t_1\cd t_n)
\end{align}
\end{ntt}


\begin{prop}
\thmtitle{The Dyson series}
\thmstatement{
If $\tl{V}(t)$ is particle-number consering, then
$
  \tl{U}(t,t_0)
=
  \mc{T}\{
    e^{
    -
      i
      \int_{t_0}^t
      dt'\,
      \tl{V}(t')
    }
  \}
$.
}\vspace{5pt}
\thmproof{
   Expanding the time-ordered exponential in a Taylor series and applying equation~\ref{eq:integral-identity} gives the following
\begin{align}
  \sum_{n=0}^\infty
  \fr{(-i)^n}{n!}
  \int_{t_1\cd t_n}^{[t_0,t]}
  dt_1\cd dt_n\,
  \mc{T}\{
    \tl{V}(t_1)
    \cd
    \tl{V}(t_n)
  \}
=
  \sum_{n=0}^\infty
  \fr{(-i)^n}{n!}
  \sum_\pi^{\mr{S}_n}
  \int_{t_{\pi(1)}>\ld>t_{\pi(n)}}^{[t_0,t]}
  dt_1\cd dt_n\,
  \mc{T}\{
    \tl{V}(t_1)
    \cd
    \tl{V}(t_n)
  \}
\end{align}
which simplifies to equation~\ref{eq:time-evolution-infinite-recursion} because all $n!$ terms in the sum over chronologies are equal by \cref{dfn:time-ordering}.
}
\end{prop}

\begin{rmk}
Assuming the boundary conditions of eq~\ref{eq:boundary-conditions}, the Dyson series for the wavefunction is
\begin{align}
\label{eq:wavefunction-dyson-series}
  \tl{\Y}(t)
=
  \lim_{t_0\rightarrow-\infty}
  \tl{U}(t,t_0)
  \Y(t_0)
=
  \sum_{n=0}^\infty
  \fr{(-i)^n}{n!}
  \int_{\mb{R}^n}
  dt_1\cd dt_n\,
  \th(t-t_1)
  \cd
  \th(t-t_n)\,
  \mc{T}\{
    \tl{V}(t_1)
    \cd
    \tl{V}(t_n)
  \}
  \Y_0
\end{align}
where
$
  \th(x)
=
  \int_{-\infty}^x
  dx'
  \d(x')
$
is the Heaviside step function, which here enforces the upper limits of integration.
\end{rmk}



\newpage
\section{Response functions}
\label{appendix:response-functions}


\begin{dfn}
\thmtitle{Response functions}
Any quantity $X(t)$ which depends on the time-envelopes $\{f_\b(t)\}$ can be expanded in a Taylor series.
The expansion coefficients in this series are called the \textit{response functions} of $X(t)$.
\begin{align}
\label{eq:general-perturbation-expansion}
  X(t)
=
  \sum_{n=0}^\infty
  \fr{1}{n!}
  \sum_{\b_1,\ld,\b_n}
  \int_{\mb{R}^n}
  dt_1\cd t_n\,
  f_{\b_1}(t_1)
  \cd
  f_{\b_n}(t_n)\,
  X^{\b_1\cd \b_n}_{t;t_1\cd\,t_n}
&&
  X^{\b_1\cd \b_n}_{t;t_1\cd\,t_n}
\equiv
  \left.
  \fd{^n
    X(t)
  }{
    f_{\b_1}(t_1)
    \cd
    df_{\b_n}(t_n)
  }
  \right|_{\bm{f}=\bo{0}}
\end{align}
\end{dfn}

\begin{ex}
Substituting equation~\ref{eq:interaction-time-envelopes} into equation~\ref{eq:wavefunction-dyson-series} and comparing the result to equation~\ref{eq:general-perturbation-expansion} implies the following.
\begin{align}
\label{eq:general-wavefunction-response}
  \tl{\Y}^{\b_1\cd \b_n}_{t; t_1\cd t_n}
=
  (-i)^n
  \th(t-t_1)
  \cd
  \th(t-t_n)\,
  \mc{T}\{
    \tl{V}_{\b_1}(t_1)
  \cd
    \tl{V}_{\b_n}(t_n)
  \}
  \Y_0
\end{align}
Defining $\ta_i\equiv t_i-t$, we find that wavefunction responses transform as follows when we move the time origin to $t$.\footnotemark
\footnotetext{
  This follows from $\th(t-t_i)=\th(0-\ta_i)$ and
  $
    \tl{V}_{\b_i}(\ta_i)
  =
    e^{-iHt}\tl{V}_{\b_i}(t_i)e^{+iHt}
  \implies
    \tl{V}_{\b_1}(\ta_1)
    \cd
    \tl{V}_{\b_n}(\ta_n)
  =
    e^{-iHt}
    \tl{V}_{\b_1}(t_1)
    \cd
    \tl{V}_{\b_n}(t_n)
    e^{+iHt}
  $.
}
\begin{align}
\label{eq:wavefunction-time-shift}
  \tl{\Y}^{\b_1\cd \b_n}_{0; \ta_1\cd \ta_n}
=
  e^{-i(H - E_0)t}\,
  \tl{\Y}^{\b_1\cd \b_n}_{t; t_1\cd t_n}
\end{align}
\end{ex}


\begin{dfn}
\thmtitle{Property response functions}
Response functions for the expectation value of an observable property $W$ are usually denoted with the following double-brackets notation.
\begin{align}
  \iip{\tl{W}(t); \tl{V}_{\b_1}(t_1),\ld,\tl{V}_{\b_n}(t_n)}
\equiv
  \left.
  \fd{^n
    \ip{\Y(t)|W|\Y(t)}
  }{
    f_{\b_1}(t_1)
    \cd
    df_{\b_n}(t_n)
  }
  \right|_{\bm{f}=\bo{0}}
\end{align}%
In some contexts, these \textit{property response functions} are known as \textit{retarded propagators} or \textit{retarded Green's functions}.

\end{dfn}

\begin{ex}
Substituting the response-function expansion of the wavefunction into $\ip{\Y(t)|W|\Y(t)}=\ip{\tl{\Y}(t)|\tl{W}(t)|\tl{\Y}(t)}$ and grouping powers of $\bm{f}$ gives the following expression for property response functions.
\begin{align}
\label{eq:general-linear-response}
  \iip{\tl{W}(t); \tl{V}_{\b_1}(t_1),\ld,\tl{V}_{\b_n}(t_n)}
=
  \sum_{p=0}^n
  \fr{1}{p!(n-p)!}
  \sum_\pi^{\mr{S}_n}
  \ip{
    \tl{\Y}_{t;t_{\pi(1)}\cd t_{\pi(p)}}
           ^{\b_{\pi(1)}\cd\b_{\pi(p)}}
  |
    \tl{W}(t)
  |
    \tl{\Y}_{t;t_{\pi(p+1)}\cd t_{\pi(n)}}
           ^{\b_{\pi(p+1)}\cd \b_{\pi(n)}}
  }
\end{align}
Using equation~\ref{eq:wavefunction-time-shift} and $\tl{W}(t)=e^{-iHt}\tl{W}(0)e^{+iHt}$, we can show that the property responses are invariant to time translation.
\begin{align}
  \iip{\tl{W}(0); \tl{V}_{\b_1}(\ta_1),\ld,\tl{V}_{\b_n}(\ta_n)}
=
  \iip{\tl{W}(t); \tl{V}_{\b_1}(t_1),\ld,\tl{V}_{\b_n}(t_n)}
\end{align}
\end{ex}

\begin{prop}
\label{prop:linear-response-commutator-expression}
\thmtitle{Linear property response function}
\thmstatement{
$\ds{
  \iip{\tl{W}(t);\tl{V}_\b(t')}
=
-
  i
  \th(t-t')
  \ip{\Y_0|[\tl{W}(t), \tl{V}_\b(t')]|\Y_0}
}$
}\vspace{6pt}
\thmproof{
This follows from equations~\ref{eq:general-wavefunction-response} and \ref{eq:general-linear-response} with $n=1$.
}
\end{prop}

\begin{samepage}
\begin{cor}
\thmstatement{
Defining $\w_k\equiv E_k-E_0$ and $\ta\equiv t'-t$, the linear property reponse can be expressed as follows.
\begin{align*}
  \iip{\tl{W}(t);\tl{V}_\b(t')}
=
-
  i
  \th(-\ta)
  \sum_{k=0}^\infty
  (
    e^{+i\w_k \ta}
    \ip{\Y_0|W|\Y_k}
    \ip{\Y_k|V_\b|\Y_0}
  -
    e^{-i\w_k \ta}
    \ip{\Y_0|V_\b|\Y_k}
    \ip{\Y_k|W|\Y_0}
  )
\end{align*}
}\thmproof{
Expanding the interaction-picture operators of \cref{prop:linear-response-commutator-expression} in the Schr\"odinger picture yields the following
\begin{align}
  \iip{\tl{W}(t);\tl{V}_\b(t')}
=&\
-
  i
  \th(t-t')
  (
    \ip{\Y_0|We^{-i(H-E_0)(t-t')}V_\b |\Y_0}
  -
    \ip{\Y_0|V_\b e^{-i(H-E_0)(t'-t)}W|\Y_0}
  )
\end{align}
since $H\Y_0=E_0\Y_0$.
The proposition follows from a spectral resolution of $e^{\mp(H-E_0)(t-t')}$ in each term.
}
\end{cor}
\end{samepage}

\begin{dfn}
\thmtitle{Response functions (frequency domain)}
The frequency-domain response functions of $X(t)$ at $t=0$ are defined as $\ev$-shifted Fourier transforms of the time-domain response functions with respect to $\ta_1,\ld,\ta_n$.
\begin{align}
  X^{\b_1\cd \b_n}_{0;\ta_1\cd \ta_n}
=
  (2\pi)^{-n}
  \int_{\mb{R}^n}
  d\w_1\cd d\w_n\,
  X^{\b_1\cd \b_n}_{\w_{\ev,1}\cd\w_{\ev,n}}
  e^{+i\sum_j\w_{\ev,j}\ta_j}
&&
  X^{\b_1\cd \b_n}_{\w_{\ev,1}\cd\w_{\ev,n}}
\equiv
  \int_{\mb{R}^n}
  d\ta_1\cd d\ta_n\,
  X^{\b_1\cd \b_n}_{0;\ta_1\cd \ta_n}
  e^{-i\sum_j\w_{\ev,j}\ta_j}
\end{align}
From equations~\ref{eq:frequency-envelopes} and~\ref{eq:general-perturbation-expansion}, we find that these are coefficients in the frequency-envelope Taylor expansion of $X(0)$.
\begin{align}
  X(0)
=
  \sum_{n=0}^\infty
  \fr{1}{n!}
  \sum_{\b_1\cd\b_n}
  \int_{\mb{R}^n}
  d\w_1\cd d\w_n\,
  f_{\b_1}(\w_{\ev,1})\cd
  f_{\b_n}(\w_{\ev,n})\,
  X^{\b_1\cd \b_n}_{\w_{\ev,1}\cd \w_{\ev,n}}
&&
  X^{\b_1\cd \b_n}_{\w_{\ev,1}\cd \w_{\ev,n}}
=
  \left.
  \fd{^n\,X(0)}{f_{\b_1}(\w_{\ev,1})\cd df_{\b_n}(\w_{\ev,n})}
  \right|_{\bm{f}=\bo{0}}
\end{align}
Property response functions in the frequency domain are denoted by $\iip{W; V_{\b_1},\ld,V_{\b_n}}_{\w_{\ev,1}\cd\w_{\ev,n}}$, which can be written as a Fourier transform of
$
  \iip{\tl{W}(t); \tl{V}_{\b_1}(t_1),\ld,\tl{V}_{\b_n}(t_n)}
$
itself due to its translational invariance.
\Cref{prop:property-response-frequency-expansion} shows that these frequency-domain functions can be used to expand $\ip{\Y(t)|W|\Y(t)}$ away from the time origin.
\end{dfn}


\begin{prop}
\label{prop:property-response-frequency-expansion}
\thmstatement{
The expectation value of an observable $W$ at time $t$ is given by the following.
\begin{align*}
  \ip{\Y(t)|W|\Y(t)}
=
  \sum_{n=0}^\infty
  \fr{1}{n!}
  \sum_{\b_1,\ld,\b_n}
  \int_{\mb{R}^n}
  d\w_1\cd d\w_n\,
  f_{\b_1}(\w_{\ev,1})
  \cd
  f_{\b_n}(\w_{\ev,n})\,
  \iip{W; V_{\b_1},\ld,V_{\b_n}}_{\w_{\ev,1}\cd\w_{\ev,n}}\,
  e^{-i\sum_j\w_{\ev,j}t}
\end{align*}
}%
\thmproof{
This follows from substituting equation~\ref{eq:frequency-envelopes} into the time-envelope expansion and inserting $e^{-i\sum_j\w_{\ev,j}t} e^{+i\sum_j\w_{\ev,j}t}$.
}
\end{prop}


\begin{rmk}
\begin{align*}
\end{align*}
\end{rmk}

\begin{align}
  \iip{\tl{W}(t);\tl{V}_\b(t')}
=&\
  \sum_{k=0}^\infty
  (
    g_k^+(\ta)
    \ip{\Y_0|W|\Y_k}
    \ip{\Y_k|V_\b|\Y_0}
  -
    g_k^-(\ta)
    \ip{\Y_0|V_\b|\Y_k}
    \ip{\Y_k|W|\Y_0}
  )
&&
  g_k^{\pm}(\ta)
\equiv
-
  i
  \th(-\ta)
    e^{\pm i\w_k \ta}
\end{align}
\begin{align}
  g_k^\pm(\w_\ev)
=&\
  \int_{-\infty}^\infty
  d\ta\,
  g_k^\pm(\ta)
  e^{-i\w_\ev \ta}
=
-
  i
  \int_{-\infty}^0
  d\ta\,
  e^{-i(\w_\ev \mp \w_k) \ta}
=
  \fr{
    1
  }{
    \w_\ev \mp \w_k
  }
\\
  g_k^{\pm}(\ta)
=&\
  \fr{1}{2\pi}
  \int_{-\infty}^{\infty}
  d\w\,
  g_k^\pm(\w_\ev)
  e^{+i\w_\ev \ta}
=
  \fr{1}{2\pi}
  \int_{-\infty}^{\infty}
  d\w\,
  \fr{
    e^{+i\w_\ev \ta}
  }{
    \w_\ev \mp \w_k
  }
\end{align}

\newpage
\section{Complex analysis}

\begin{dfn}
\thmtitle{Continuity}
A complex-valued function $f$ is said to be \textit{continuous} at $z\in\mb{C}$ if for any positive real number $\ev$ we can choose a radius $\d>0$ such that all complex values $z'$ within $\d$ of $z$ satisfy $|f(z')-f(z)|<\ev$.
That is, we can always choose a circle small enough that all function values lie within some threshold.
\end{dfn}

\begin{dfn}
\label{dfn:holomorphic-function}
\thmtitle{Holomorphic function}
The function $f(z)$ is \textit{differentiable} at $z$ if the following limit exists and
has the same value with $h$ approaching from any direction in the complex plane.
\begin{align}
  \pd{f(z)}{z}
\equiv
  \lim_{h\rightarrow0}
  \fr{f(z+h)-f(z)}{h}
\end{align}
A \textit{holomorphic function} is a complex-valued function which is differentiable everywhere on $\mb{C}$.
\end{dfn}

\begin{dfn}
\thmtitle{Wirtinger derivatives}
Denoting the real and imaginary components of $z$ by $x$ and $y$ we find
\begin{align}
  z
=
  x
+
  iy
&&
\implies
&&
  dz
=
  dx
+
  idy
,\ \ 
  dz^*
=
  dx
-
  idy
&&
\implies
&&
  dx
=
  \tfr{1}{2}
  \pr{
    dz
  +
    dz^*
  }
,\ \ 
  dy
=
  \tfr{1}{2i}
  \pr{
    dz
  -
    dz^*
  }
\end{align}
by adding and subtracting differentials.
Comparing these to the total derivative expansion for each variable, we find
\begin{align}
  \pd{x}{z}
=
  \fr{1}{2}
&&
  \pd{x}{z^*}
=
  \fr{1}{2}
&&
  \pd{y}{z}
=
  \fr{1}{2i}
&&
  \pd{y}{z^*}
=
-
  \fr{1}{2i}
\end{align}
which can lead to the following formulas for derivatives with respect to $z$ and $z^*$, known as \textit{Wirtinger derivatives}.
\begin{align}
  \pd{}{z}
=
  \pd{x}{z}
  \pd{}{x}
+
  \pd{y}{z}
  \pd{}{y}
=
  \fr{1}{2}
  \pr{
    \pd{}{x}
  -
    i
    \pd{}{y}
  }
&&
  \pd{}{z^*}
=
  \pd{x}{z^*}
  \pd{}{x}
+
  \pd{y}{z^*}
  \pd{}{y}
=
  \fr{1}{2}
  \pr{
    \pd{}{x}
  +
    i
    \pd{}{y}
  }
\end{align}
These can be used to show that $\pd{z^*}{z}=\pd{z}{z^*}=0$, confirming that $z$ and $z^*$ are independent variables.
\end{dfn}

\begin{prop}
\thmstatement{
  The function $f$ is differentiable at $z$ if and only if
  $
    \dpd{f(z)}{z^*}
  =
    0
  $.
}
\thmproof{
  Let $z=x+iy$ and assume the derivatives with respect to $x$ and $y$ exist.
  Then we can express $f(z+h)-f(h)$ as a bivariate Taylor expansion in $\mr{Re}(h)$ and $\mr{Im}(h)$, whose linear term is given by the following.
\begin{align*}
  \pd{f(z)}{x}
  \mr{Re}(h)
+
  \pd{f(z)}{y}
  \mr{Im}(h)
=
  \pd{f(z)}{x}
  \fr{
    h + h^*
  }{
    2
  }
+
  \pd{f(z)}{y}
  \fr{
    h - h^*
  }{
    2i
  }
\end{align*}
  Dividing this expression by $h$ and taking the limit as $h\rightarrow 0$ gives the complex derivative of $f$ at $z$.
\begin{align*}
  \lim_{h\rightarrow0}
  \fr{f(z+h)-f(z)}{h}
=
  \fr{1}{2}
  \pr{
    \pd{f(z)}{x}
  +
    \fr{1}{i}
    \pd{f(z)}{y}
  }
+
  \fr{1}{2}
  \pr{
    \pd{f(z)}{x}
  -
    \fr{1}{i}
    \pd{f(z)}{y}
  }
  \lim_{h\rightarrow h^*}
  \fr{h^*}{h}
\end{align*}
  If $h$ approaches along the real axis, the limit of $h^*/h$ is $+1$.
  If $h$ approaches along the imaginary axis, the limit of $h^*/h$ is $-1$.
  Therefore, $f$ is differentiable if and only if
  $
    \fr{1}{2}
    \pr{
      \pd{f(z)}{x}
    -
      \fr{1}{i}
      \pd{f(z)}{y}
    }
  =
    0
  $,
  which is equivalent to
  $
    \pd{f(z)}{z^*}
  =
    0
  $.
  If the derivatives with respect to $x$ and $y$ do not exist then $f$ is not differentiable and
  $
    \pd{f(z)}{z^*}
  $
  is undefined.
}
\end{prop}

\begin{ntt}
\thmtitle{Complex integration}
The notation
$
  \int_\g
  dz\,
  f(z)
$
denotes the line integral of $f$ over a path $\g$ in the complex plane, which is known as \textit{contour integration}.
The notation
$
  \oint_\g
  dz\,
  f(z)
$
means that $\g$ is a closed and counterclockwise. 
\end{ntt}

\begin{prop}
\thmstatement{
  If $\g$ is a circular path containing the point $z$, then
$\ds{
  \oint_\g
  dz'\,
  \fr{1}{z'-z}
=
  2\pi i
}$.
}
\thmproof{
  We can assume without loss of generality that $z$ is at the origin and parametrize the path as $z'(\th)=re^{i\th}$.
  Using
  $
    dz'(\th)
  =
    ire^{i\th}
    d\th
  $,
  the integrand simplifies to
  $
    dz'(\th)/
    z'(\th)
  =
    id\th
  $.
  Integrating from $0$ to $2\pi$ concludes the proof.
}
\end{prop}




\newpage
\section{The generalized Stokes' theorem}

\begin{dfn}
\thmtitle{Topological space}
Let $X$ be a set and let $T_X$ be a family of subsets of $X$.
$T_X$ qualifies as a \textit{topology} if
\begin{multicols}{2}%
\begin{enumerate}
\item
  $T_X$ contains the empty set and $X$ itself.
\item
  Any finite or infinite union of sets in $T_X$ is in $T_X$.
\item
  Any finite intersection of sets in $T_X$ is in $T_X$.
\setcounter{enumi}{0}
\item[]
  $\O\in T_X,\ X\in T_X$
\item[]
  $\ds{
    S
  \subseteq
    T_X
  \implies
    \cup_{O\in S}
    O
  \in
    T_X
  }$
\item[]
  $
    \{O_1,\ld,O_n\}
  \subseteq
    T_X
  \implies
    O_1\cap\cd\cap O_n
  \in
    T_X
  $
\end{enumerate}%
\end{multicols}%
\noindent
in which case we call $(X,T_X)$ a \textit{topological space}.
The elements of $X$ are called \textit{points} and the elements of $T_X$ are called \textit{open subsets}.
An arbitrary subset $V\subseteq X$, not necessarily open, qualifies as a \textit{neighborhood of the point $p$} if it contains an open subset containing $p$.\footnotemark
\footnotetext{
  That is, there is some $O\in T_X$ such that $p\in O\subseteq V$.
}
\end{dfn}

\begin{dfn}
\thmtitle{Hausdorff space}
Two points $p$ and $p'$ are \textit{distinct} if $p\neq p'$.
Two points are \textit{separated by neighborhoods} if they have any neighborhoods that are disjoint from each other.
A \textit{Hausdorff space} is a topological space in which all distinct points are separated by neighborhoods.
\end{dfn}

\begin{dfn}
\thmtitle{Base}
A collection of open subsets $B\subset T_X$ is termed a \textit{base} for the topology if every member of $T_X$ can be written as a union of the elements of $B$.
We say that $T_X$ is the topology \textit{generated} by $B$.
\end{dfn}

\begin{dfn}
\thmtitle{Euclidean space}
Given a positive number $r$ and a point $\bo{x}$ in $\mb{R}^n$, the \textit{open ball of radius $r$ at $\bo{x}$} is defined as $\mb{B}_r(\bo{x})\equiv\{\bo{x}'\in\mb{R}^n\,|\,\|\bo{x}'-\bo{x}\|<r\}$, which is an open subset of $\mb{R}^n$.
The \textit{Euclidean topology} on $\mb{R}^n$ is given by
\begin{align}
  T_{\mb{R}^n}
\equiv
  \{
    \cup_{O\in S}
    O
  \,|\,
    S
  \subseteq
    B
  \}
&&
  B
=
  \{
    \mb{B}_r(\bo{x})
  \,|\,
    r>0,
    \bo{x}\in\mb{R}^n
  \}
\end{align}
where the set of open balls $B$ serves as a base.
With this topology, $\mb{R}^n$ is called the \textit{Euclidean $n$-space}.
\end{dfn}

\begin{ex}
We can easily show that Euclidean space is a Hausdorff space:
Two points, $\bo{x}$ and $\bo{x}'$, are distinct if and only if they are separated by some distance non-zero distance, $\|\bo{x}-\bo{x}'\|$.
If so, the open balls $B_r(\bo{x})$ and $B_r(\bo{x}')$ are disjoint for $r=\|\bo{x}-\bo{x}'\|/2$, which proves that they are separated by neighborhoods.
\end{ex}

\begin{dfn}
\thmtitle{Differentiability class}
A mapping  $\varphi:V\rightarrow V'$ between Euclidean subspaces $V\in \mb{R}^n$ and $V'\in\mb{R}^{n'}$ belongs to \textit{differentiability class $C^k$} if it has continuous $k\eth$ derivatives.
\end{dfn}


\begin{dfn}
\thmtitle{Homeomorphism}
An invertible map $\mu:X\rightarrow X'$ that takes open subsets of $X$ into open subsets of $X'$ and vice versa is called \textit{bicontinuous}.
A bicontinuous map whose image covers the codomain is called a \textit{homeomorphism}.
\end{dfn}

\begin{dfn}
\thmtitle{Local homeomorphism}
A \textit{local homeomorphism} is map $\la:X\rightarrow X'$ for which every point in $X$ is a member of at least one open subset $O$ whose restriction $\la|_O$ is a homeomorphism onto an open subset of $X'$.
\end{dfn}


\begin{dfn}
\thmtitle{Chart}
If the image of a local homeomorphism is Euclidean, $\la(X)\subseteq\mb{R}^n$, we say $X$ is \textit{locally~Euclidean}.
For each homeomorphic restriction $\xi=\la|_O$, we call the pair $(O, \xi)$ a \textit{chart}.
A chart is also called a \textit{local coordinate frame} because it relates each point $p$ in a region of the manifold to a coordinate vector $\bo{x}=\xi(p)$ in $\mb{R}^n$.
\end{dfn}

\begin{dfn}
\thmtitle{Topological manifold}
A \textit{topological manifold} is a locally Euclidean Hausdorff space, $(M, T_M)$.
In particular, a \textit{topological manifold of dimension $n$} is locally homeomorphic to Euclidean $n$-space.
\end{dfn}

\begin{dfn}
\thmtitle{Atlas}
An \textit{atlas} is a collection of charts $\mc{A}\equiv\{(O_\a, \xi_\a)\}$ that cover a manifold,
$
  \cup_\a
  O_\a
=
  M
$.
\end{dfn}

\begin{dfn}
\thmtitle{Transition map}
The \textit{transition map}
$
  \ta_{\a\b}(\bo{x})
=
  \xi_\b(\xi_\a^{-1}(\bo{x}))
$
converts between overlapping frames,
$
  \xi_\a(O_\a\cap O_\b)
$
and
$
  \xi_\b(O_\a\cap O_\b)
$.
We say that $\mc{A}$ is a \textit{$C^k$-differentiable atlas} if all of its transition maps are $C^k$ differentiable.
\end{dfn}

\begin{dfn}
\thmtitle{Differentiable manifold}
Two $C^k$-differentiable atlases are considered \textit{compatible}, $\mc{A}\sim\mc{A}'$, if their union is also $C^k$ differentiable.
An equivalence class of compatible $C^k$-differentiable atlases, $[\mc{A}]\equiv\{\mc{A}'\,|\,\mc{A}\sim\mc{A}'\}$, is called a \textit{$C^k$-differentiable structure}.
A manifold equipped with a $C^k$-differentiable structure is a \textit{$C^k$-differentiable manifold}.
\end{dfn}

\begin{dfn}
\thmtitle{Smooth manifold}
A manifold with an infinitely differentiable structure is called \textit{smooth}.
\end{dfn}

\begin{dfn}
\thmtitle{Smooth function}
Real-valued functions on a manifold $f:M\rightarrow\mb{R}$ are classified as \textit{$C^k$-differentiable} according to the differentiability class of $f(\xi_a^{-1}(\bo{x}))$ for the charts $\xi_\a$ in its differential structure.
The set of these functions is denoted $C^k(M)$.
The space of \textit{smooth functions on $M$} is $C^\infty(M)$, which is also called the space of \textit{differential $0$-forms}. 
\end{dfn}

\begin{dfn}
\thmtitle{Derivation}
A \textit{derivation at $p$} is a linear functional on $C^\infty(M)$ that satisfies the \textit{product rule}.
\begin{align}
  (c\mc{D}_p + c'\mc{D}_p')(f)
=
  c\mc{D}_p(f)
+
  c'\mc{D}_p'(f)
&&
  \mc{D}_p(fg)
=
  \mc{D}_p(f)\cdot g(p)
+
  f(p) \cdot \mc{D}_p(g)
\end{align}
The set of derivations at $p$ forms a vector space called the \textit{tangent space at $p$}, denoted $T_pM$.
\end{dfn}

\begin{ex}
A simple example of a derivation is derivative $(\pt_{x_i})_{\bo{x}}$ along one of the coordinate axes $\bo{e}_i$ at the point $\bo{x}$ in Euclidean $n$-space.
This map $(\pt_{x_i})_{\bo{x}}:C^\infty(\mb{R}^n)\rightarrow\mb{R}$ is defined as follows.
\begin{align}
\label{eq:coordinate-directional-derivative}
  (\pt_{x_i})_{\bo{x}}(f)
\equiv
  \pr{\pd{f}{x_i}}_\bo{x}
\end{align}
The coordinate derivatives at a point $\{(\pt_{x_i})_{\bo{x}}\,|\,1\leq i\leq n\}$ make up the \textit{standard basis} for its tangent space $T_\bo{x}\mb{R}^n$.
\end{ex}

\begin{dfn}
\thmtitle{Differential 1-forms}
The \textit{cotangent space at $p$}, denoted $T_p^*M$ is the space of linear functionals on $T_pM$.
That is, its elements $\a_p:T_pM\rightarrow\mb{R}$ eat derivations and spit out real numbers.
These are called \textit{differential $1$-forms}.
\end{dfn}

\begin{ex}
An example of a $1$-form is $(dx_i)_\bo{x}:T_\bo{x}\mb{R}^n\rightarrow\mb{R}$ which acts on derivations as $(dx_i)_\bo{x}(\mc{D}_\bo{x})\equiv\mc{D}_\bo{x}(x_i)$.
As a linear functional, this map is characterized by its action on a basis for $T_\bo{x}\mb{R}^n$.
Acting on $(\pt_{x_j})_\bo{x}$ gives the following.
\begin{align}
\label{eq:coordinate-independence}
  (dx_i)_\bo{x}
  (\pt_{x_j})_\bo{x}
=
  \pr{\pd{x_i}{x_j}}_\bo{x}
=
  \d_{ij}
\end{align}
These tangent functionals $\{(dx_i)_\bo{x}\,|\,1\leq i\leq n\}$ are called \textit{coordinate differentials} and form the \textit{standard basis} for $T_\bo{x}^*\mb{R}^n$.
\end{ex}

\begin{ex}
\label{ex:tangent-cotangent-expansions}
General derivations and differential 1-forms over Euclidean $n$-space can be expanded as follows
\begin{align}
  \mc{D}_\bo{x}
=
  D_\bo{x}^1\,
  (\pt_{x_1})_\bo{x}
+
\cd
+
  D_\bo{x}^n\,
  (\pt_{x_n})_\bo{x}
&&
  \a_\bo{x}
=
  A_\bo{x}^1\,
  (dx_i)_\bo{x}
+
\cd
+
  A_\bo{x}^n\,
  (dx_n)_\bo{x}
&&
  D_\bo{x}^i
\in
  \mb{R}
,\,
  A_\bo{x}^i
\in
  \mb{R}
\end{align}
in terms of the standard bases for $T_\bo{x}\mb{R}^n$ and $T_\bo{x}^*\mb{R}^n$.
\end{ex}

\begin{dfn}
\label{eq:differential-map-on-0-forms}
\thmtitle{Differential map}
The \textit{differential map at $p$} maps $0$-forms into $1$-forms, $(d\,\cdot)_p:C^\infty(M)\rightarrow T_p^*(M)$.
The image of $f$ under this map is the tangent-space functional $(df)_p:T_p(M)\rightarrow\mb{R}$ that acts on derivations in the tangent space as $(df)_p(\mc{D}_p)\equiv\mc{D}_p(f)$.
From the properties of derivations, we can show that
\begin{align}
  d(cf + c'f')_p
=
  c(df)_p
+
  c'(df')_p
&&
  d(fg)
=
  (df)_p\cdot g(p)
+
  f(p)\cdot (dg)_p
\end{align}
where pointwise multiplication by a 0-form is defined as $(g(p)\cdot(df)_p)(\mc{D}_p)\equiv g(p)\cdot(df)_p(\mc{D}_p)$ which equals $g(p)\cdot\mc{D}_p(f)$.
In words, the differential map on 0-forms is linear and satisfies its own product rule.
\end{dfn}

\begin{dfn}
\label{eq:exact-differential}
\thmtitle{Exact differential}
A $1$-form $\a_p\in T_p^*M$ which can be written as a differential map of a 0-form, $\a_p=(df)_p$, is called an \textit{exact differential}.
All other 1-forms are \textit{inexact differentials}.
\end{dfn}

\begin{ex}
An example of an exact differential is a 1-form $\a_\bo{x}\in T_\bo{x}^*\mb{R}^n$ with coefficients of the form $A_\bo{x}^i=(\pd{f}{x_i})_\bo{x}$ for some 0-form $f$.
We can prove that this is an exact differential by acting it on $\mc{D}_\bo{x}$ and expanding both according to \cref{ex:tangent-cotangent-expansions}
\begin{align}
  \a_\bo{x}(\mc{D}_\bo{x})
=
  \sum_{ij}
  \pr{\pd{f}{x_i}}_\bo{x}
  D_\bo{x}^j
  (dx_i)_\bo{x}(\pt_{x_j})_\bo{x}
=
  \sum_i
  \pr{\pd{f}{x_i}}_\bo{x}
  D_\bo{x}^i
=
  \mc{D}_\bo{x}(f)
=
  (df)_\bo{x}(\mc{D}_\bo{x})
\end{align}
which proves that
$
  \a_\bo{x}
=
  (df)_\bo{x}
$.
Here we have made use of equations~\ref{eq:coordinate-directional-derivative} and~\ref{eq:coordinate-independence}.
\end{ex}


\begin{samepage}
\begin{dfn}
\thmtitle{Exterior power}
The \textit{$k\eth$ exterior power} $\bigwedge^k(V)$ of a vector space $V$ a vector space with elements $\{\sum v_1\wedge\cd\wedge v_k\,|\,v_i\in V\}$.
The \textit{wedge product}, $\wedge$, is antisymmetric, 
$
  v_1\wedge\cd\wedge v_k
=
  \e_\pi
  v_{\pi(1)}\wedge\cd\wedge v_{\pi(k)}
$,
and multilinear
\begin{align}
  v_1\wedge\cd \wedge(cv_i + d w_i)\wedge\cd\wedge v_k
= 
  c(
  v_1\wedge\cd \wedge v_i\wedge\cd\wedge v_k
  )
+
  d(
  v_1\wedge\cd \wedge w_i\wedge\cd\wedge v_k
  )
\end{align}
where $c$ and $d$ are scalars, making $\bigwedge^k(V)$ isomorphic to the antisymmetric subspace of $V^{\otimes k}$.
\end{dfn}
\end{samepage}

\begin{dfn}
\thmtitle{Differential $k$-form}
A \textit{differential $k$-form at $p$} is an element of $\bigwedge^k(T_p^*M)$.
We have already defined $0$-forms $\bigwedge^0(T_p^*M)=\left.C^\infty(M)\right|_p=\mb{R}$ and $1$-forms $\bigwedge^1(T_p^*M)=T_p^*M$.
If the manifold has dimension $n$ and $\{(\e_1)_p, \ld, (\e_n)_p\}$ is a basis for $T_p^*M$, then a general differential $k$-form $\b_p$ can be expanded as
\begin{align}
  \b_p
=
  \sum_{i_1\cd i_n}
  \b_p^{i_1\cd i_n}
  \,
  (\e_{i_1})_p
  \wedge
  \cd
  \wedge
  (\e_{i_n})_p
\end{align}
where the $\b_p^{i_1\cd i_n}$ forms a real-valued coordinate array.
\end{dfn}

\begin{dfn}
\thmtitle{Exterior derivative}
The differential map of \cref{eq:differential-map-on-0-forms} is a specific case of the \textit{exterior derivative} of a differential $k$-form, which is a map $d$ characterized by the following properties.
\begin{multicols}{2}%
\begin{enumerate}
\item
  The exterior derivative of a $k$-form is a $(k+1)$-form.
\item
  The exterior derivative of a $0$-form is a differential map.
\item
  The exterior derivative of an exact differential vanishes.
\item
  The exterior derivative of a wedge product satisfies the generalize product rule.
\setcounter{enumi}{0}
\item[]
  $d:\bigwedge^k(T_p^*M)\rightarrow\bigwedge^{k+1}(T_p^*M)$
\item[]
  $\ds{
  }$
\item[]
  $\ds{
  }$
\item[]
  $
    \{O_1,\ld,O_n\}
  \subseteq
    T_X
  \implies
    O_1\cap\cd\cap O_n
  \in
    T_X
  $
\end{enumerate}%
\end{multicols}%
\noindent
blasdf
\end{dfn}


\begin{dfn}
\thmtitle{Differential $k$-form}
Smooth functions in $C^\infty(M)$ are called \textit{differential $0$-forms}.
Higher differential forms are defined by a map $d$ which takes \textit{differential $k$-forms} $\a, \b$ into $(k+1)$-forms according to the following axioms.
\begin{align}
  df
\equiv
  \text{differential of $f$}
&&
  d^2
=
  0
&&
  d(\a \wedge \b)
=
  d\a \wedge \b
+
  (-)^k
  \a \wedge d\b
\end{align}
The wedge product of a $k$-form and a $0$-form is defined as $f\wedge \a=\a\wedge f=f\a$ for all $f\in C^\infty(M)$.
\end{dfn}




\newpage

\begin{dfn}
\thmtitle{Compact subset}
A subset $V\subseteq X$ is \textit{compact} if every infinite family $S\subseteq T_X$ of open subsets whose union contains $V$ has a finite subfamily $\{O_1,\ld,O_n\}\subseteq S$ whose union still contains $V$.
\end{dfn}

\begin{dfn}
\thmtitle{Compact support}
The \textit{support} of a smooth function $f$ is the set of points on which the function is non-zero, $\mr{supp}(f)\equiv\{p\in M\,|\,f(p)\neq0\}$.
A function has \textit{compact support} if $\mr{supp}(f)$ forms a compact subset on $M$.
\end{dfn}


\begin{ex}
Consider $\si=u dx^1\wedge dx^2$ on the 1-form basis $dx^1,\ld,dx^n$. Its exterior derivative is
\begin{align}
  d\si
=
  du
  \wedge
  dx^1
  \wedge
  dx^2
=
  \sum_{i=1}^3
  \pd{u}{x^i}
  dx^i
  \wedge
  dx^1
  \wedge
  dx^2
\end{align}
\end{ex}

\begin{align}
  \w
=
  \sum_{i=1}^3
  F_i(\bo{x})
  dx_i
&&
  d\w
=
  \sum_{i=1}^3
  dF_i(\bo{x})
  \wedge
  dx_i
=
  \sum_{i=1}^3
  \sum_{j=1}^3
  \pd{F_i(\bo{x})}{x_j}
  dx_j
  \wedge
  dx_i
\end{align}



\begin{rmk}
A manifold is \textit{orientable} if it has an atlas whose transition functions all have positive Jacobian determinants.
\begin{align}
  \bo{J}_{\a\b}
\equiv
  \pd{\bm{\ta}_{\a\b}(\bo{x})}{\bo{x}}
&&
  \mr{det}(\bo{J}_{\a\b})
>
  0
\end{align}
A maximal atlas is the union of its own equivalence class $\mc{A}=\cup_{\mc{A}'\in[\mc{A}]}\mc{A}'$.
A \textit{volume form} $\w$ on $M$ gives rise to a natural orientation, which is the atlas of coordinate charts on $M$ that set $\w$ into a positive multiple of the Euclidean volume form $dx^1\wedge\cd\wedge dx^n$.
Call a basis of \textit{tangent vectors} $\{X_1,\ld,X_n\}$ \textit{right-ahnded} if
\begin{align}
  \w(X_1,\ld,X_n)
>
  0
\end{align}
\end{rmk}

\begin{dfn}
\thmtitle{Pullback}
Suppose that $\f:M\rightarrow M'$ is a smooth map between smooth manifolds $M$ and $M'$.
The \textit{pullback} of a smooth function $f'\in C^\infty(M')$ is the function $\f^*f'\in C^\infty(M)$ defined by $(\f^*f)(p)=f(\f(p))$.
If $\a'$ is a one-form on $M'$, the \textit{pullback} of $\a'$ by $\f$ is the one-form on $M$ defined by
\begin{align}
  (\f^*\a')_p(D)
\equiv
  \a'_{\f(p)}
  (d\f_p(D))
=
  \a'_{\f(p)}
  (\mc{D}_p(\f))
\end{align}
The line integral of $\a$ over $\W$ is defined as
\begin{align}
  \int_\W
  \a
=
  \int_{\xi(O)}
  (\xi^{-1})^*\a
\end{align}
where $(O, \xi)$ is the local coordinate chart and $\xi:O\rightarrow \mb{R}^n$ and $(\xi^{-1})^*$ is the pullback to $\mb{R}^n$.
\end{dfn}


\begin{rmk}
\begin{align}
  d\xi_1
  \wedge
  \cd
  \wedge
  d\xi_n
=
  \sum_{i_1\cd i_n}
  \pd{\xi_1}{x_{i_1}}
  \cd
  \pd{\xi_n}{x_{i_n}}
  dx_{i_1}
  \wedge
  \cd
  \wedge
  dx_{i_n}
=
  \pr{
    \sum_{i_1\cd i_n}
    \e_{i_1\cd i_n}
    \pd{\xi_1}{x_{i_1}}
    \cd
    \pd{\xi_n}{x_{i_n}}
  }
  dx_1
  \wedge
  \cd
  \wedge
  dx_n
=
\left|
  \pd{\bm{\xi}}{\bo{x}}
\right|
  dx_1
  \wedge
  \cd
  \wedge
  dx_n
\end{align}
\end{rmk}

\newpage
\begin{dfn}
\thmtitle{Derivation}
A \textit{derivation} is a map between smooth functions $D:C^\infty(M)\rightarrow C^\infty(M)$ which satisfies
\begin{align}
  (cD + c'D')(f)
=
  cD(f)
+
  c'D'(f)
&&
  D(fg)
=
  D(f)\cdot g
+
  f\cdot D(g)
\end{align}
where $\cdot$ denotes the pointwise function multiplication, $(f\cdot g)(p)\equiv f(p)g(p)$.
The condition on the right is called the \textit{product rule}.
The set of derivations for a given manifold forms a vector space, which is denoted $\mr{Der}(C^\infty(M))$.
A \textit{derivation at the point $p$} is a real-valued functional
$
  \mc{D}_p
:
  C^\infty(M)
\rightarrow
  \mb{R}
$
defined by
$
  \mc{D}_p(f)
\equiv
  \pr{D(f)}(p)
$.
\end{dfn}


\begin{dfn}
\thmtitle{Tangent bundle}
The set of derivations as $p$ forms a vector space called the \textit{tangent space at $p$}, denoted $T_pM$.
The elements of $T_pM$ are \textit{tangent vectors}.
Associating each point in the manifold with its tangent space yields the \textit{tangent bundle}, which is written as
$
  TM
\equiv
  \{(p,\mc{D}_p)\,|\,p\in M, \mc{D}_p\in T_pM\}
$
\end{dfn}


\begin{dfn}
\thmtitle{Differential}
A \textit{differential} maps a derivation into a function, 
$
  df
:
  \mr{Der}(C^\infty(M))
\rightarrow
  C^\infty(M)
$.
This is expressed in terms of a smooth function $f$ as
$
  df(D)
\equiv
  D(f)
$.
From the properties of derivations, one finds that
\begin{align}
  d(cf + c'f')
=
  cdf
+
  c'df'
&&
  d(fg)
=
  df\cdot g
+
  f\cdot dg
\end{align}
where pointwise multiplication with a smooth function is defined as
$
  (df\cdot g)(D)
\equiv
  df(D)\cdot g
$.
The \textit{differential at the point~$p$}
is a real-valued function on the tangent space at that point,
$
  df_p
:
  T_pM
\rightarrow
  \mb{R}
$
where
$
  df_p(D)
\equiv
  \pr{D(f)}(p)
$.
\end{dfn}

\begin{dfn}
\thmtitle{Cotangent bundle}
The set of differentials at $p$ forms a vector space called the \textit{cotangent space at $p$}, denoted
$
  T_p^*M
$.
The elements of $T_p^*M$ are called \textit{cotangent vectors} or \textit{one-forms}.
Associating each point in the manifold with its cotangent space yields the \textit{cotangent bundle},
$
  T^*M
\equiv
\{
  (p, df_p)
\,|\,
  p\in M,
  f\in C^\infty(M)
\}
$.
\end{dfn}




\begin{dfn}
\thmtitle{Differential}
\begin{enumerate}
\item
the differential of $f$ is
$
  df
:
  T_pM
\rightarrow
  C^\infty(M)
$
\item
given by
$
  df(D)
=
  D(f)
$

\item
Can show that
whereointwise scalar multiplication

\item
the differential of $f$ at $p$ is
$
  df_p
:
  T_pM
\rightarrow
  \mb{R}
$
\item
$
  df_p(D)
=
  \pr{D(f)}(p)
$
\item
The set $\{df_p\,|\,f\in C^\infty(M)\}$ of differentials at $p$ forms a vector space called the \textit{cotangent space at $p$}, denoted $T_p^*M$, whose elements are \textit{cotangent vectors}.

\item
Cotangent vectors are alos called \textit{one-forms}.

\item
Associating each point in the manifold with its cotangent space yields the \textit{cotangent bundle},
$
  T^*M
\equiv
\{
  (p, df_p)
\,|\,
  p\in M,
  df_p\in T_p^*M
\}
$

\end{enumerate}
\end{dfn}


\begin{dfn}
\thmtitle{Differential $k$-form}
A \textit{differential $k$-form at the point $p$} is an element of the $k\eth$ exterior power of the cotangent bundle at that point, $\a_p\in\bigwedge^k T_p^*M$.
\end{dfn}


\begin{ex}
An example of a derivation is the \textit{directional derivative at the point $p(t)$} along a curve $p:\mb{R}\rightarrow M$
\begin{align}
  D_{p(t)}(f)
=
  \fd{f(p(t))}{t}
\end{align}
Which is equal to the differential $df_{p(t)}$ evaluated at $D$.
\end{ex}


\end{document}