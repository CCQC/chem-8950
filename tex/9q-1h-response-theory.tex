\documentclass[11pt]{article}
\usepackage[cm]{fullpage}
%%AVC PACKAGES
\usepackage{avcgreek}
\usepackage{avcfonts}
\usepackage{avcmath}
\usepackage[numberby=section,skip=9pt plus 2pt minus 7pt]{avcthm}
\usepackage{qcmacros}
\usepackage{goldstone}
%%MACROS FOR THIS DOCUMENT
\numberwithin{equation}{section}
\usepackage[
  margin=1.5cm,
  includefoot,
  footskip=30pt,
  headsep=0.2cm,headheight=1.3cm
]{geometry}
\usepackage{fancyhdr}
\pagestyle{fancy}
\fancyhf{}
\fancyhead[LE,RO]{Quiz 9, Handout 1: Response theory}
\fancyfoot[CE,CO]{\thepage}
\usepackage{url}
\makeatother
\newcommand{\resolventline}[2][1]{
  \tikz[overlay]{
      \draw[thick,flexdotted] (0,-1ex) to ++(0,#1*4.5ex) node[above,inner sep=1pt] {#2};
  }
}
\usepackage{accents}
\newcommand{\oc}[1]{\ensuremath{\accentset{\circ}{#1}}}
\newcommand{\wtl}[1]{\ensuremath{\widetilde{#1}}}
\usepackage{multicol}

\begin{document}

\setlength{\abovedisplayskip}{5pt}
\setlength{\belowdisplayskip}{5pt}


\setcounter{section}{8}
\section{Response theory}

\begin{rmk}
In the presence of a time-varying field, a molecule's electronic wavefunction is no longer simply an eigenfunction of the Hamiltonian.
Instead, its electronic structure is described by the \textit{time-dependent Schr\"odinger equation}
\begin{align}
\label{eq:schrodinger-equation}
  H(t)
  \Y(t)
=
  i
  \pd{\Y(t)}{t}
&&
  H(t)
=
  H
+
  V(t)
\end{align}
where $H$ is the usual electronic Hamiltonian and $V(t)$ is an \textit{interaction Hamiltonian} describing the energetic influence of the field.
A general series solution to equation~\ref{eq:schrodinger-equation}, known as the \textit{Dyson series}, is derived in \cref{appendix:dyson-series}.
The interaction Hamiltonian can be expressed as a sum over one-electron operators $V_\b$, representing the electronic degrees of freedom which couple to the external field, scaled by \textit{time-envelopes} $f_\b(t)$ which control the strength of the applied field over time.
\begin{align}
\label{eq:interaction-time-envelopes}
  V(t)
=
\ts{
  \sum_\b
  V_\b
  f_\b(t)
}
\end{align}
One of the most important examples is the Hamiltonian of a dipole in an electric field, which is discussed in \cref{ex:dipole-approximation} below.
The zeroth order solutions of equation~\ref{eq:schrodinger-equation} are termed \textit{stationary states}, which have the following form.\footnotemark
\footnotetext{
When $\bm{f}=\bo{0}$, the Hamiltonian loses its time-dependence and we can write $\left.\Y(t)\right|_{\bm{f}=\bo{0}}=\f(t)\Y$ where $\f(t)$ is independent of the electronic coordinates.
Substituting this into eq~\ref{eq:schrodinger-equation} and rearranging gives
$
  H\Y/\Y
=
  i\dot{\f}(t)/\f(t)
$, which equals a constant $E$ since each side depends in different variables.
Therefore, $H\Y=E\Y$ and $i\dot{\f}(t)=E\f(t)$.  Integrating the latter gives $\f(t)=e^{-iEt}$.
}
\begin{align}
  \left.
  \Y(t)
  \right|_{\bm{f}=\bo{0}}
=
  e^{-iE_kt}
  \Y_k
&&
  H\Y_k
=
  E_k\Y_k
\end{align}
As a boundary condition we assume that $V(t)$ vanishes in the past, where $\Y(t)$ is initially in the ground stationary state.
\begin{align}
\label{eq:boundary-conditions}
  \lim_{t\rightarrow-\infty}
  f_\b(t)
=
  0
&&
  \lim_{t\rightarrow-\infty}
  e^{+iHt}
  \Y(t)
=
  \Y_0
\end{align}
This limiting behavior can be enforced by introducing a complex shift in the frequency domain of $f_\b(t)$'s Fourier expansion.\footnotemark
\footnotetext{
  This is a slightly unusual convention for the Fourier transform.
  A useful mnemonic for checking these is
  $
    \int_{-\infty}^\infty
    dk\,
    e^{ikx}
  =
    2\pi\,
    \d(x)
  $.
}
\begin{align}
\label{eq:frequency-envelopes}
  f_\b(t)
=
  \int_{-\infty}^\infty
  d\w\,
  f_\b(\w_\ev)
  e^{-i\w_\ev t}
&&
  f_\b(\w_\ev)
\equiv
  (2\pi)^{-1}
  \int_{-\infty}^\infty
  dt\,
  f_\b(t)
  e^{+i\w_\ev t}
&&
  \w_\ev
\equiv
  \w
+
  i\ev
&&
  \ev
=
  |\ev|
\end{align}
This has the effect of scaling the time envelope by a damping factor $e^{\ev t}$.
For sufficiently small $\ev$, this scaled envelope will match the original one to arbitrary precision in an arbitrarily wide window about the time origin.
The fact that the interaction Hamiltonian and the coupling operators $\{V_\b\}$ are Hermitian implies the following identities.
\begin{align}
  f_\b^*(t)
=
  f_\b(t)
&&
  f_\b^*(\w_{\ev})
=
  f_\b(-\w_{-\ev})
\end{align}
\end{rmk}


\begin{ex}
\label{ex:dipole-approximation}
The dominant coupling of an electronic system to an external electric or magnetic field is mediated through its dipoles, leading to \textit{the dipole approximation}.
Quantizing the classical formulae for these interaction energies gives
\begin{align}
\begin{array}{r@{\ }l@{\ }l@{\hspace{2cm}}r@{\ }l@{\hspace{2cm}}r@{\ }l}
  V_{\bo{E}}(t)
\approx&
-\,
  \bm{\mu}\cdot\bo{E}(t)
&=
-
  \sum_\b
  \mu_\b
  \mc{E}_\b(t)
&
  \bm{\mu}
&=
  \sum_{pq}
  \ip{\y_p|\op{\bm{\mu}}\,|\y_q}
  a_p\dg a_q
&
  \op{\bm{\mu}}
&=
-
  \op{\bo{r}}
\\
  V_{\bo{B}}(t)
\approx&
-
  \bm{m}\cdot\bo{B}(t)
&=
-
  \sum_\b
  m_\b
  \mc{B}_\b(t)
&
  \bm{m}
&=
  \sum_{pq}
  \ip{\y_p|\op{\bm{m}}|\y_q}
  a_p\dg a_q
&
  \op{\bm{m}}
&=
-
  \tfr{1}{2}
  (
    \op{\bm{l}}
  +
    2\,
    \op{\bo{s}}
  )
\end{array}
\end{align}
where $\op{\bm{\mu}}$ and $\op{\bm{m}}$ are the first-quantized electric and magnetic dipole operators.\footnote{
More generally, these expressions are
$
  \op{\bm{\mu}}
=
  q_e\,\op{\bo{r}}
$,
where $q_e=-e$ is the charge of an electron,
and
$
  \op{\bm{m}}
=
  \mu_\mr{B}
  (
    g_l\,
    \op{\bm{l}}
  +
    g_{\mr{s}}\,
    \op{\bo{s}}
  )
$
where $\mu_{\mr{B}}=\tfr{1}{2}\cdot\tfr{e\hbar}{m_e}$ is the Bohr magneton and $g_l=-1$, $g_\mr{s}=-2$ are the spin and orbital \textit{$g$-factors}.
Note that the exact $g_\mr{s}$ actually deviates very slightly from $2$ due to effects arising in quantum field theory.
The orbital angular momentum operator is given by $\op{\bm{l}}=\op{\bo{r}}\times\op{\bo{p}}$ and $\op{\bo{s}}$ is the intrinsic spin angular momentum operator.
}
The leading terms neglected by the dipole approximation are quadratic in the field amplitudes.
These weaker interactions are mediated through the higher moments (quadrupole, octupole, etc.) of the charge and current distributions and may become important in symmetric molecules where certain dipole interactions are ``symmetry forbidden''.
\end{ex}





\begin{dfn}
\thmtitle{Quasi-energy}
\begin{align}
  \Y(t)
=
  e^{-i\th(t)}
  \bar{\Y}(t)
&&
  \left.
  \th(t)
  \right|_{\bm{f}=\bo{0}}
=
  E_0
  t
&&
  \lim_{t\rightarrow-\infty}
  \bar{\Y}(t)
=
  \Y_0
\end{align}
\begin{align}
  (
    H(t)
  -
    i
    \tpd{}{t}
  )
  \bar{\Y}(t)
=
  \dot{\th}(t)
  \bar{\Y}(t)
\end{align}
\begin{align}
  \dot{\th}(t)
=
  \int_0^t
  dt'
  \ip{\bar{\Y}(t')|
    H(t')
  -
    i\tpd{}{t'}
  |\bar{\Y}(t')}
\end{align}
\begin{align}
  \br{\d\bar{\Y}(t)}
    H(t)
  -
    i
    \tpd{}{t}
  \kt{\bar{\Y}(t)}
=
  \dot{\th}(t)
  \ip{\d\bar{\Y}(t)|\bar{\Y}(t)}
\end{align}
\begin{align}
  \ip{\d\bar{\Y}(t)|\bar{\Y}(t)}
+
  \ip{\bar{\Y}(t)|\d\bar{\Y}(t)}
=
  0
\end{align}
\begin{align}
  \d
  \ip{\bar{\Y}(t)|
    H(t)
  -
    i
    \tpd{}{t}
  |\bar{\Y}(t)}
+
  i
  \tpd{}{t}
  \ip{\bar{\Y}(t)|\d\bar{\Y}(t)}
=
  0
\end{align}
\end{dfn}


\newpage
\appendix

\section{Dyson series}
\label{appendix:dyson-series}

\begin{dfn}
\thmtitle{Time-evolution operator}
If we know the wavefunction at a particular time $t_0$, we can express the wavefunction at any other time as a unitary transformation of this initial state, $\Y(t)=U(t,t_0)\Y(t_0)$.
This unitary transformation is called the \textit{time-evolution operator}.
\end{dfn}


\begin{dfn}
\thmtitle{Interaction picture}
The \textit{interaction picture} results from to the following similarity transformation.
\begin{align}
  \tl{\Th}(t)
\equiv
  e^{+iHt}
  \Th(t)
&&
  \tl{W}(t)
\equiv
  e^{+iHt}
  W(t)
  e^{-iHt}
\end{align}
Expanding the Schr\"odinger equation in the interaction picture yields the the \textit{Schwinger-Tomonaga equation}.
\begin{align}
\label{eq:schwinger-tomonaga}
  \tl{V}(t)
  \tl{\Y}(t)
=
  i
  \pd{\tl{\Y}(t)}{t}
\end{align}
Multiplying both sides by $-i$ and integrating from $t_0$ to $t$ yields a recursive equation for the time-evolution operator
\begin{align}
\label{eq:integrated-schwinger-tomonaga-equation}
  \tl{\Y}(t)
-
  \tl{\Y}(t_0)
=
-
  i
  \int_{t_0}^t
  dt'\,
  \tl{V}(t')
  \tl{\Y}(t')
&&
\implies
&&
  \tl{U}(t,t_0)
=
  1
-
  i
  \int_{t_0}^t
  dt'
  \tl{V}(t')\,
  \tl{U}(t',t_0)
\end{align}
and infinite recursion of this identity leads to the following expansion.
\begin{align}
\label{eq:time-evolution-infinite-recursion}
  \tl{U}(t,t_0)
=
  \sum_{n=0}^\infty
  (-i)^n
  \int_{t_0}^t
  dt_1
  \int_{t_0}^{t_1}
  dt_2
  \cd
  \int_{t_0}^{t_{n-1}}
  dt_n
  \,
  \tl{V}(t_1)
  \cd
  \tl{V}(t_n)
\end{align}
\end{dfn}

\begin{dfn}
\label{dfn:time-ordering}
\thmtitle{Time-ordering}
Let 
$
  \tl{q}_1(t_1)\cd \tl{q}_n(t_n)
$
be a string of particle-hole operators in the interaction picture.\footnotemark
\footnotetext{
As in
$
  \tl{q}(t)
\equiv
  e^{+iHt}
  q
  e^{-iHt}
$
for some
$q\in\{a_p\}\cup\{a_p\dg\}$.
}
The \textit{time-ordering map} takes this string into
$
  \mc{T}\{
  \tl{q}_1(t_1)\cd \tl{q}_n(t_n)
  \}
\equiv
  \e_\pi \,\tl{q}_{\pi(1)}(t_{\pi(1)})\cd \tl{q}_{\pi(n)}(t_{\pi(n)})
$,
where $\pi\in\mr{S}_n$ is a permutation that puts the time arguments in chronological order, $t_{\pi(1)}>\cd>t_{\pi(n)}$.
\end{dfn}

\begin{ntt}
Let us define the following notation for multivariate integrals by analogy with multi-index summations.\footnotemark
\footnotetext{
Compare these integrals to
$
  \sum_{i_1i_2i_3\cd}^{\{n_0,\ld,n\}}
$
and
$
  \sum_{i_1>i_2>i_3\cd}^{\{n_0,\ld,n\}}
$.
The subscript defines the summation variables, along with any conditions restricting their values, and the superscript indicates the allowed range of values for each variable.
}
\begin{align}
  \int_{t_1t_2t_3\ld}^{[t_0,t]}
  dt_1dt_2dt_3\cd
\equiv
  \int_{t_0}^t
  dt_1
  \int_{t_0}^t
  dt_2
  \int_{t_0}^t
  dt_3
  \cd
&&
  \int_{t_1>t_2>t_3>\cd}^{[t_0,t]}
  dt_1dt_2dt_3\cd
\equiv
  \int_{t_0}^t
  dt_1
  \int_{t_0}^{t_1}
  dt_2
  \int_{t_0}^{t_2}
  dt_3
  \cd
\end{align}
This notation should elucidate the following identity, 
which breaks an unrestricted integral into all possible chronologies.\footnotemark
\footnote{
  The corresponding summation identity would be
$
  \sum_{i_1\neq i_2\neq i_3\neq\cd}^{\{n_0,n\}}
=
  \sum_{\pi}^{\mr{S}_n}
  \sum_{i_{\pi(1)}> i_{\pi(2)}> i_{\pi(3)}>\cd}^{\{n_0,n\}}
$.
  The unrestricted integral is equivalent to an integral over $t_1\neq t_2\neq t_3\neq\cd$ because individual integrand values have ``measure zero'':
$
  \int_{t_j}^{t_j}
  dt_i
=
  0
$.
}
\begin{align}
\label{eq:integral-identity}
  \int_{t_1\cd t_n}^{[t_0,t]}
  dt_1\cd t_n\,
  f(t_1\cd t_n)
=
  \sum_\pi^{\mr{S}_n}
  \int_{t_{\pi(1)}>\ld>t_{\pi(n)}}^{[t_0,t]}
  dt_1\cd t_n\,
  f(t_1\cd t_n)
\end{align}
\end{ntt}


\begin{prop}
\thmtitle{The Dyson series}
\thmstatement{
If $\tl{V}(t)$ is particle-number consering, then
$
  \tl{U}(t,t_0)
=
  \mc{T}\{
    e^{
    -
      i
      \int_{t_0}^t
      dt'\,
      \tl{V}(t')
    }
  \}
$.
}\vspace{5pt}
\thmproof{
   Expanding the time-ordered exponential in a Taylor series and applying equation~\ref{eq:integral-identity} gives the following
\begin{align}
  \sum_{n=0}^\infty
  \fr{(-i)^n}{n!}
  \int_{t_1\cd t_n}^{[t_0,t]}
  dt_1\cd dt_n\,
  \mc{T}\{
    \tl{V}(t_1)
    \cd
    \tl{V}(t_n)
  \}
=
  \sum_{n=0}^\infty
  \fr{(-i)^n}{n!}
  \sum_\pi^{\mr{S}_n}
  \int_{t_{\pi(1)}>\ld>t_{\pi(n)}}^{[t_0,t]}
  dt_1\cd dt_n\,
  \mc{T}\{
    \tl{V}(t_1)
    \cd
    \tl{V}(t_n)
  \}
\end{align}
which simplifies to equation~\ref{eq:time-evolution-infinite-recursion} because all $n!$ terms in the sum over chronologies are equal by \cref{dfn:time-ordering}.
}
\end{prop}

\begin{rmk}
Assuming the boundary conditions of eq~\ref{eq:boundary-conditions}, the Dyson series for the wavefunction is
\begin{align}
\label{eq:wavefunction-dyson-series}
  \tl{\Y}(t)
=
  \lim_{t_0\rightarrow-\infty}
  \tl{U}(t,t_0)
  \Y(t_0)
=
  \sum_{n=0}^\infty
  \fr{(-i)^n}{n!}
  \int_{\mb{R}^n}
  dt_1\cd dt_n\,
  \th(t-t_1)
  \cd
  \th(t-t_n)\,
  \mc{T}\{
    \tl{V}(t_1)
    \cd
    \tl{V}(t_n)
  \}
  \Y_0
\end{align}
where
$
  \th(x)
=
  \int_{-\infty}^x
  dx'
  \d(x')
$
is the Heaviside step function, which here enforces the upper limits of integration.
\end{rmk}



\newpage
\section{Response functions}
\label{appendix:response-functions}


\begin{dfn}
\thmtitle{Response functions}
Any quantity $X(t)$ which depends on the time-envelopes $\{f_\b(t)\}$ can be expanded in a Taylor series.
The expansion coefficients in this series are called the \textit{response functions} of $X(t)$.
\begin{align}
\label{eq:general-perturbation-expansion}
  X(t)
=
  \sum_{n=0}^\infty
  \fr{1}{n!}
  \sum_{\b_1,\ld,\b_n}
  \int_{\mb{R}^n}
  dt_1\cd t_n\,
  f_{\b_1}(t_1)
  \cd
  f_{\b_n}(t_n)\,
  X^{\b_1\cd \b_n}_{t;t_1\cd\,t_n}
&&
  X^{\b_1\cd \b_n}_{t;t_1\cd\,t_n}
\equiv
  \left.
  \fd{^n
    X(t)
  }{
    f_{\b_1}(t_1)
    \cd
    df_{\b_n}(t_n)
  }
  \right|_{\bm{f}=\bo{0}}
\end{align}
\end{dfn}

\begin{ex}
Substituting equation~\ref{eq:interaction-time-envelopes} into equation~\ref{eq:wavefunction-dyson-series} and comparing the result to equation~\ref{eq:general-perturbation-expansion} implies the following.
\begin{align}
\label{eq:general-wavefunction-response}
  \tl{\Y}^{\b_1\cd \b_n}_{t; t_1\cd t_n}
=
  (-i)^n
  \th(t-t_1)
  \cd
  \th(t-t_n)\,
  \mc{T}\{
    \tl{V}_{\b_1}(t_1)
  \cd
    \tl{V}_{\b_n}(t_n)
  \}
  \Y_0
\end{align}
Defining $\ta_i\equiv t_i-t$, we find that wavefunction responses transform as follows when we move the time origin to $t$.\footnotemark
\footnotetext{
  This follows from $\th(t-t_i)=\th(0-\ta_i)$ and
  $
    \tl{V}_{\b_i}(\ta_i)
  =
    e^{-iHt}\tl{V}_{\b_i}(t_i)e^{+iHt}
  \implies
    \tl{V}_{\b_1}(\ta_1)
    \cd
    \tl{V}_{\b_n}(\ta_n)
  =
    e^{-iHt}
    \tl{V}_{\b_1}(t_1)
    \cd
    \tl{V}_{\b_n}(t_n)
    e^{+iHt}
  $.
}
\begin{align}
\label{eq:wavefunction-time-shift}
  \tl{\Y}^{\b_1\cd \b_n}_{0; \ta_1\cd \ta_n}
=
  e^{-i(H - E_0)t}\,
  \tl{\Y}^{\b_1\cd \b_n}_{t; t_1\cd t_n}
\end{align}
\end{ex}


\begin{dfn}
\thmtitle{Property response functions}
Response functions for the expectation value of an observable property $W$ are usually denoted with the following double-brackets notation.
\begin{align}
  \iip{\tl{W}(t); \tl{V}_{\b_1}(t_1),\ld,\tl{V}_{\b_n}(t_n)}
\equiv
  \left.
  \fd{^n
    \ip{\Y(t)|W|\Y(t)}
  }{
    f_{\b_1}(t_1)
    \cd
    df_{\b_n}(t_n)
  }
  \right|_{\bm{f}=\bo{0}}
\end{align}%
In some contexts, these \textit{property response functions} are known as \textit{retarded propagators} or \textit{retarded Green's functions}.

\end{dfn}

\begin{ex}
Substituting the response-function expansion of the wavefunction into $\ip{\Y(t)|W|\Y(t)}=\ip{\tl{\Y}(t)|\tl{W}(t)|\tl{\Y}(t)}$ and grouping powers of $\bm{f}$ gives the following expression for property response functions.
\begin{align}
\label{eq:general-linear-response}
  \iip{\tl{W}(t); \tl{V}_{\b_1}(t_1),\ld,\tl{V}_{\b_n}(t_n)}
=
  \sum_{p=0}^n
  \fr{1}{p!(n-p)!}
  \sum_\pi^{\mr{S}_n}
  \ip{
    \tl{\Y}_{t;t_{\pi(1)}\cd t_{\pi(p)}}
           ^{\b_{\pi(1)}\cd\b_{\pi(p)}}
  |
    \tl{W}(t)
  |
    \tl{\Y}_{t;t_{\pi(p+1)}\cd t_{\pi(n)}}
           ^{\b_{\pi(p+1)}\cd \b_{\pi(n)}}
  }
\end{align}
Using equation~\ref{eq:wavefunction-time-shift} and $\tl{W}(t)=e^{-iHt}\tl{W}(0)e^{+iHt}$, we can show that the property responses are invariant to time translation.
\begin{align}
  \iip{\tl{W}(0); \tl{V}_{\b_1}(\ta_1),\ld,\tl{V}_{\b_n}(\ta_n)}
=
  \iip{\tl{W}(t); \tl{V}_{\b_1}(t_1),\ld,\tl{V}_{\b_n}(t_n)}
\end{align}
\end{ex}

\begin{prop}
\label{prop:linear-response-commutator-expression}
\thmtitle{Linear property response function}
\thmstatement{
$\ds{
  \iip{\tl{W}(t);\tl{V}_\b(t')}
=
-
  i
  \th(t-t')
  \ip{\Y_0|[\tl{W}(t), \tl{V}_\b(t')]|\Y_0}
}$
}\vspace{6pt}
\thmproof{
This follows from equations~\ref{eq:general-wavefunction-response} and \ref{eq:general-linear-response} with $n=1$.
}
\end{prop}

\begin{samepage}
\begin{cor}
\thmstatement{
Defining $\w_k\equiv E_k-E_0$ and $\ta\equiv t'-t$, the linear property reponse can be expressed as follows.
\begin{align*}
  \iip{\tl{W}(t);\tl{V}_\b(t')}
=
-
  i
  \th(-\ta)
  \sum_{k=0}^\infty
  (
    e^{+i\w_k \ta}
    \ip{\Y_0|W|\Y_k}
    \ip{\Y_k|V_\b|\Y_0}
  -
    e^{-i\w_k \ta}
    \ip{\Y_0|V_\b|\Y_k}
    \ip{\Y_k|W|\Y_0}
  )
\end{align*}
}\thmproof{
Expanding the interaction-picture operators of \cref{prop:linear-response-commutator-expression} in the Schr\"odinger picture yields the following
\begin{align}
  \iip{\tl{W}(t);\tl{V}_\b(t')}
=&\
-
  i
  \th(t-t')
  (
    \ip{\Y_0|We^{-i(H-E_0)(t-t')}V_\b |\Y_0}
  -
    \ip{\Y_0|V_\b e^{-i(H-E_0)(t'-t)}W|\Y_0}
  )
\end{align}
since $H\Y_0=E_0\Y_0$.
The proposition follows from a spectral resolution of $e^{\mp(H-E_0)(t-t')}$ in each term.
}
\end{cor}
\end{samepage}

\begin{dfn}
\thmtitle{Response functions (frequency domain)}
The frequency-domain response functions of $X(t)$ at $t=0$ are defined as $\ev$-shifted Fourier transforms of the time-domain response functions with respect to $\ta_1,\ld,\ta_n$.
\begin{align}
  X^{\b_1\cd \b_n}_{0;\ta_1\cd \ta_n}
=
  (2\pi)^{-n}
  \int_{\mb{R}^n}
  d\w_1\cd d\w_n\,
  X^{\b_1\cd \b_n}_{\w_{\ev,1}\cd\w_{\ev,n}}
  e^{+i\sum_j\w_{\ev,j}\ta_j}
&&
  X^{\b_1\cd \b_n}_{\w_{\ev,1}\cd\w_{\ev,n}}
\equiv
  \int_{\mb{R}^n}
  d\ta_1\cd d\ta_n\,
  X^{\b_1\cd \b_n}_{0;\ta_1\cd \ta_n}
  e^{-i\sum_j\w_{\ev,j}\ta_j}
\end{align}
From equations~\ref{eq:frequency-envelopes} and~\ref{eq:general-perturbation-expansion}, we find that these are coefficients in the frequency-envelope Taylor expansion of $X(0)$.
\begin{align}
  X(0)
=
  \sum_{n=0}^\infty
  \fr{1}{n!}
  \sum_{\b_1\cd\b_n}
  \int_{\mb{R}^n}
  d\w_1\cd d\w_n\,
  f_{\b_1}(\w_{\ev,1})\cd
  f_{\b_n}(\w_{\ev,n})\,
  X^{\b_1\cd \b_n}_{\w_{\ev,1}\cd \w_{\ev,n}}
&&
  X^{\b_1\cd \b_n}_{\w_{\ev,1}\cd \w_{\ev,n}}
=
  \left.
  \fd{^n\,X(0)}{f_{\b_1}(\w_{\ev,1})\cd df_{\b_n}(\w_{\ev,n})}
  \right|_{\bm{f}=\bo{0}}
\end{align}
Property response functions in the frequency domain are denoted by $\iip{W; V_{\b_1},\ld,V_{\b_n}}_{\w_{\ev,1}\cd\w_{\ev,n}}$, which can be written as a Fourier transform of
$
  \iip{\tl{W}(t); \tl{V}_{\b_1}(t_1),\ld,\tl{V}_{\b_n}(t_n)}
$
itself due to its translational invariance.
\Cref{prop:property-response-frequency-expansion} shows that these frequency-domain functions can be used to expand $\ip{\Y(t)|W|\Y(t)}$ away from the time origin.
\end{dfn}


\begin{prop}
\label{prop:property-response-frequency-expansion}
\thmstatement{
The expectation value of an observable $W$ at time $t$ is given by the following.
\begin{align*}
  \ip{\Y(t)|W|\Y(t)}
=
  \sum_{n=0}^\infty
  \fr{1}{n!}
  \sum_{\b_1,\ld,\b_n}
  \int_{\mb{R}^n}
  d\w_1\cd d\w_n\,
  f_{\b_1}(\w_{\ev,1})
  \cd
  f_{\b_n}(\w_{\ev,n})\,
  \iip{W; V_{\b_1},\ld,V_{\b_n}}_{\w_{\ev,1}\cd\w_{\ev,n}}\,
  e^{-i\sum_j\w_{\ev,j}t}
\end{align*}
}%
\thmproof{
This follows from substituting equation~\ref{eq:frequency-envelopes} into the time-envelope expansion and inserting $e^{-i\sum_j\w_{\ev,j}t} e^{+i\sum_j\w_{\ev,j}t}$.
}
\end{prop}


\begin{rmk}
\begin{align*}
\end{align*}
\end{rmk}

\begin{align}
  \iip{\tl{W}(t);\tl{V}_\b(t')}
=&\
  \sum_{k=0}^\infty
  (
    g_k^+(\ta)
    \ip{\Y_0|W|\Y_k}
    \ip{\Y_k|V_\b|\Y_0}
  -
    g_k^-(\ta)
    \ip{\Y_0|V_\b|\Y_k}
    \ip{\Y_k|W|\Y_0}
  )
&&
  g_k^{\pm}(\ta)
\equiv
-
  i
  \th(-\ta)
    e^{\pm i\w_k \ta}
\end{align}
\begin{align}
  g_k^\pm(\w_\ev)
=&\
  \int_{-\infty}^\infty
  d\ta\,
  g_k^\pm(\ta)
  e^{-i\w_\ev \ta}
=
-
  i
  \int_{-\infty}^0
  d\ta\,
  e^{-i(\w_\ev \mp \w_k) \ta}
=
  \fr{
    1
  }{
    \w_\ev \mp \w_k
  }
\\
  g_k^{\pm}(\ta)
=&\
  \fr{1}{2\pi}
  \int_{-\infty}^{\infty}
  d\w\,
  g_k^\pm(\w_\ev)
  e^{+i\w_\ev \ta}
=
  \fr{1}{2\pi}
  \int_{-\infty}^{\infty}
  d\w\,
  \fr{
    e^{+i\w_\ev \ta}
  }{
    \w_\ev \mp \w_k
  }
\end{align}

\newpage
\section{Complex analysis}

\begin{dfn}
\thmtitle{Continuity}
A complex-valued function $f$ is said to be \textit{continuous} at $z\in\mb{C}$ if for any positive real number $\ev$ we can choose a radius $\d>0$ such that all complex values $z'$ within $\d$ of $z$ satisfy $|f(z')-f(z)|<\ev$.
That is, we can always choose a circle small enough that all function values lie within some threshold.
\end{dfn}

\begin{dfn}
\label{dfn:holomorphic-function}
\thmtitle{Holomorphic function}
The function $f(z)$ is \textit{differentiable} at $z$ if the following limit exists and
has the same value with $h$ approaching from any direction in the complex plane.
\begin{align}
  \pd{f(z)}{z}
\equiv
  \lim_{h\rightarrow0}
  \fr{f(z+h)-f(z)}{h}
\end{align}
A \textit{holomorphic function} is a complex-valued function which is differentiable everywhere on $\mb{C}$.
\end{dfn}

\begin{dfn}
\thmtitle{Wirtinger derivatives}
Denoting the real and imaginary components of $z$ by $x$ and $y$ we find
\begin{align}
  z
=
  x
+
  iy
&&
\implies
&&
  dz
=
  dx
+
  idy
,\ \ 
  dz^*
=
  dx
-
  idy
&&
\implies
&&
  dx
=
  \tfr{1}{2}
  \pr{
    dz
  +
    dz^*
  }
,\ \ 
  dy
=
  \tfr{1}{2i}
  \pr{
    dz
  -
    dz^*
  }
\end{align}
by adding and subtracting differentials.
Comparing these to the total derivative expansion for each variable, we find
\begin{align}
  \pd{x}{z}
=
  \fr{1}{2}
&&
  \pd{x}{z^*}
=
  \fr{1}{2}
&&
  \pd{y}{z}
=
  \fr{1}{2i}
&&
  \pd{y}{z^*}
=
-
  \fr{1}{2i}
\end{align}
which can lead to the following formulas for derivatives with respect to $z$ and $z^*$, known as \textit{Wirtinger derivatives}.
\begin{align}
  \pd{}{z}
=
  \pd{x}{z}
  \pd{}{x}
+
  \pd{y}{z}
  \pd{}{y}
=
  \fr{1}{2}
  \pr{
    \pd{}{x}
  -
    i
    \pd{}{y}
  }
&&
  \pd{}{z^*}
=
  \pd{x}{z^*}
  \pd{}{x}
+
  \pd{y}{z^*}
  \pd{}{y}
=
  \fr{1}{2}
  \pr{
    \pd{}{x}
  +
    i
    \pd{}{y}
  }
\end{align}
These can be used to show that $\pd{z^*}{z}=\pd{z}{z^*}=0$, confirming that $z$ and $z^*$ are independent variables.
\end{dfn}

\begin{prop}
\thmstatement{
  The function $f$ is differentiable at $z$ if and only if
  $
    \dpd{f(z)}{z^*}
  =
    0
  $.
}
\thmproof{
  Let $z=x+iy$ and assume the derivatives with respect to $x$ and $y$ exist.
  Then we can express $f(z+h)-f(h)$ as a bivariate Taylor expansion in $\mr{Re}(h)$ and $\mr{Im}(h)$, whose linear term is given by the following.
\begin{align*}
  \pd{f(z)}{x}
  \mr{Re}(h)
+
  \pd{f(z)}{y}
  \mr{Im}(h)
=
  \pd{f(z)}{x}
  \fr{
    h + h^*
  }{
    2
  }
+
  \pd{f(z)}{y}
  \fr{
    h - h^*
  }{
    2i
  }
\end{align*}
  Dividing this expression by $h$ and taking the limit as $h\rightarrow 0$ gives the complex derivative of $f$ at $z$.
\begin{align*}
  \lim_{h\rightarrow0}
  \fr{f(z+h)-f(z)}{h}
=
  \fr{1}{2}
  \pr{
    \pd{f(z)}{x}
  +
    \fr{1}{i}
    \pd{f(z)}{y}
  }
+
  \fr{1}{2}
  \pr{
    \pd{f(z)}{x}
  -
    \fr{1}{i}
    \pd{f(z)}{y}
  }
  \lim_{h\rightarrow h^*}
  \fr{h^*}{h}
\end{align*}
  If $h$ approaches along the real axis, the limit of $h^*/h$ is $+1$.
  If $h$ approaches along the imaginary axis, the limit of $h^*/h$ is $-1$.
  Therefore, $f$ is differentiable if and only if
  $
    \fr{1}{2}
    \pr{
      \pd{f(z)}{x}
    -
      \fr{1}{i}
      \pd{f(z)}{y}
    }
  =
    0
  $,
  which is equivalent to
  $
    \pd{f(z)}{z^*}
  =
    0
  $.
  If the derivatives with respect to $x$ and $y$ do not exist then $f$ is not differentiable and
  $
    \pd{f(z)}{z^*}
  $
  is undefined.
}
\end{prop}

\begin{ntt}
\thmtitle{Complex integration}
The notation
$
  \int_\g
  dz\,
  f(z)
$
denotes the line integral of $f$ over a path $\g$ in the complex plane, which is known as \textit{contour integration}.
The notation
$
  \oint_\g
  dz\,
  f(z)
$
means that $\g$ is a closed and counterclockwise. 
\end{ntt}

\begin{prop}
\thmstatement{
  If $\g$ is a circular path containing the point $z$, then
$\ds{
  \oint_\g
  dz'\,
  \fr{1}{z'-z}
=
  2\pi i
}$.
}
\thmproof{
  We can assume without loss of generality that $z$ is at the origin and parametrize the path as $z'(\th)=re^{i\th}$.
  Given that
  $
    dz'(\th)
  =
    ire^{i\th}
    d\th
  $,
  we have 
  $
    dz'(\th)\,
    z'(\th)^{-1}
  =
    id\th
  $.
  Integrating from $0$ to $2\pi$ concludes the proof.
}
\end{prop}


\newpage
\section{The generalized Stokes' theorem}

\begin{dfn}
\thmtitle{Topological space}
Let $X$ be a set and let $T_X$ be a family of subsets of $X$.
$T_X$ qualifies as a \textit{topology} if
\begin{multicols}{2}%
\begin{enumerate}
\item
  $T_X$ contains the empty set and $X$ itself.
\item
  Any finite or infinite union of sets in $T_X$ is in $T_X$.
\item
  Any finite intersection of sets in $T_X$ is in $T_X$.
\setcounter{enumi}{0}
\item[]
  $\O\in T_X,\ X\in T_X$.
\item[]
  $\ds{
    \{O_i\,|\,i\in I\}
  \subseteq
    T_X
  \implies
    \cup_{i\in I}
    O_i
  \in
    T_X
  }$
\item[]
  $
    \{O_i\,|\,1\leq i\leq n\}
  \subseteq
    T_X
  \implies
    \cap_{i=1}^n
    O_i
  \in
    T_X
  $
\end{enumerate}%
\end{multicols}%
\noindent
in which case we call $(X,T_X)$ a \textit{topological space}.
The elements of $X$ are called \textit{points} and the elements of $T_X$ are called \textit{open subsets}.
The points $x$ and $x'$ are said to be \textit{distinct} if $x\neq x'$.
For an arbitrary subset $V\subseteq X$, not necessarily open, we say that $V$ is a \textit{neighborhood of the point $x$} if it contains an open subset containing $x$.\footnotemark
\footnotetext{
  That is, there is some $O\in T_X$ such that $x\in O\subseteq V$.
}
We say that two points are \textit{separated by neighborhoods} if they have neighborhoods that are disjoint from each other.
A topological space in which all distinct points are separated by neighborhoods called a \textit{Hausdorff space}.
\end{dfn}


\begin{dfn}
\thmtitle{Base}
A collection of open subsets $B\subset T_X$ is termed a \textit{base} for the topology if every member of $T_X$ can be written as a union of the elements of $B$.
We say that $T_X$ is the topology \textit{generated} by $B$.
\end{dfn}

\begin{dfn}
\thmtitle{Euclidean space}
At any point $\bo{r}$ in $\mb{R}^n$ the set $\{\bo{r}'\in\mb{R}^n\,|\,\|\bo{r}'-\bo{r}\|<r\}$ is an \textit{open ball of radius~$r$}, denoted $B_r(\bo{r})$.
The \textit{standard} or \textit{Euclidean topology} on $\mb{R}^n$ is the topology generated by the set of open balls $\{B_r(\bo{r})\,|\,\bo{r}\in\mb{R}^n, r>0\}$.
When $\mb{R}^n$ is equipped with this topology we call it the \textit{Euclidean space of dimension $n$}, or \textit{Euclidean $n$-space}.
\end{dfn}

\begin{ex}
We can easily show that Euclidean space is a Hausdorff space:
Two points, $\bo{r}$ and $\bo{r}'$, are distinct if and only if they are separated by some distance non-zero distance, $\|\bo{r}-\bo{r}'\|$.
If so, the open balls $B_r(\bo{r})$ and $B_r(\bo{r}')$ are disjoint for $r=\|\bo{r}-\bo{r}'\|/2$, which proves that they are separated by neighborhoods.
\end{ex}


\begin{dfn}
\thmtitle{Homeomorphism}
A function $f:X\rightarrow X'$ between topological spaces $(X,T_X)$ and $(X',T_{X'})$ is said to be \textit{surjective} if it maps onto all of $X'$.\footnotemark
\footnotetext{
  In other words every $x'\in X$ can be written as $x'=f(x)$ for some $x\in X$.
}
This function is \textit{continuous} if the preimage of any open subset in $T_X$ is an open subset in $T_{X'}$.\footnotemark
\footnotetext{
The preimage image of the open set $O'\in T_{X'}$ is $f^{-1}(O')=\{x\in X\,|\, f(x)\in O'\}$.
}
If it has a continuous inverse, we say that $f$ is \textit{bicontinuous}.
A surjective, bicontinuous function is called a \textit{homeomorphism}.
If there is a homeomorphism between two topological spaces, we call them \textit{homeomorphic}.
More generally, $f$ is a \textit{local homeomorphism} if every point is a member of some open subset $O$ in $T_X$ such that $f(O)$ is an open subset in $T_{X'}$ and the restriction, $f|_O:O\rightarrow f(O)$, is a \textit{homeomorphism}.
Then $X$ and $X'$ are \textit{locally homeomorphic}.
We say that $X$ is \textit{locally Euclidean} if it is locally homeomorphic to Euclidean $n$-space for some $n$.
\end{dfn}

\begin{dfn}
\thmtitle{Topological manifold}
A \textit{topological manifold} is a locally Euclidean Hausdorff space, $(M, T_M)$.
In particular, a \textit{topological manifold of dimension $n$} is locally homeomorphic to Euclidean $n$-space.
\end{dfn}

\begin{dfn}
\thmtitle{Transition map}
For any open subset $O$ of an $n$-dimensional topological manifold, we can define a homeomorphism $\f:O\rightarrow\f(O)$ onto an open subset of $\mb{R}^n$, which is known a s \textit{chart} or \textit{local frame} and denoted $(O, \f)$.
A collection of charts $\{(O_\a,\f_\a)\}$ that ``patch'' together to form the full space, $\cup_\a O_\a=M$, is called an \textit{atlas}.
Two overlapping charts in an atlas for $M$ can be compared through the \textit{transition map},
$
  \ta_{\a\b}
:
  \f_\a(O_\a\cap O_\b)\rightarrow \f_\b(O_\a\cap O_\b)
$,
which is a homeomorphism between open subsets on $\mb{R}^n$ defined by
$
  \ta_{\a\b}(\bo{r})
=
  \f_\b(\f_\a^{-1}(\bo{r}))
$.
\end{dfn}

\begin{dfn}
\thmtitle{Differentiable manifold}
A \textit{differentiable manifold} is a topological manifold whose transition maps continuous and differentiable.
An infinitely differentiable manifold is called a \textit{smooth manifold}.
\end{dfn}

\begin{dfn}
\thmtitle{Derivation}
The \textit{smooth functions} $C^\infty(M)$ on a smooth manifold include all $f:M\rightarrow\mb{R}$ for which $f(\f^{-1}(\bo{r}))$ is infinitely differentiable on every chart $(\f,O)$ containing $\bo{r}$.
A \textit{derivation} at $x$ is a map $D:C^{\infty}(M)\rightarrow \mb{R}$ with
\begin{align}
\label{eq:derivation-properties}
  (cD + c'D')(f(x))
=
  c D(f(x))
+
  c' D'(f(x))
&&
  D(f(x)g(x))
=
  D(f(x))\cdot g(x)
+
  f(x)\cdot D(g(x))
\end{align}
where $D'$ is another derivation.
In words, derivations are real functionals that are \textit{linear} and satisfy the \textit{product rule}.
\end{dfn}

\begin{ex}
\thmtitle{Directional derivative}
Given a smooth function $f\in C^\infty(M)$ on manifold and a curve $\g:\mb{R}\rightarrow M$ containing the point $x=\g(t)$, the \textit{directional derivative of $f$ along $\g$ at $x$} is defined as follows.
\begin{align}
  D_\g(f(x))
=
  \fd{f(\g(t))}{t}
\end{align}
As can be verified with basic calculus, the derivative satisfies the properties of a derivation given in equation~\ref{eq:derivation-properties}.
\end{ex}

\begin{dfn}
\thmtitle{Directional derivative}
A continuous function $\g:\mb{R}\rightarrow M$ forms a \textit{curve} in the manifold.
Let $\g(t)=x$ be a particular point on the curve.
Given a function of the manifold $f$, its \textit{directional derivative} along $\g$ at $x$ is
$
  \fd{}{t}
  f(\g(t))
$.
\end{dfn}

\begin{dfn}
\thmtitle{Tangent bundle}
The set of derivations at $x$ forms a vector space known as the \textit{tangent space at $x$}, which is denoted $T_xM$.
Associating each point in $M$ with its tangent space yields the \textit{tangent bundle}, which is written as
$
  TM
\equiv
  \{
    (x,D_x)
  \,|\,
    x\in M,
    D_x\in T_xM
  \}
$
\end{dfn}


\begin{thm}
\thmtitle{
}
\end{thm}

\end{document}