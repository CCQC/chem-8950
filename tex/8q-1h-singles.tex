\documentclass[11pt]{article}
\usepackage[cm]{fullpage}
%%AVC PACKAGES
\usepackage{avcgreek}
\usepackage{avcfonts}
\usepackage{avcmath}
\usepackage[numberby=section,skip=9pt plus 2pt minus 7pt]{avcthm}
\usepackage{qcmacros}
\usepackage{goldstone}
%%MACROS FOR THIS DOCUMENT
\numberwithin{equation}{section}
\usepackage[
  margin=1.5cm,
  includefoot,
  footskip=30pt,
  headsep=0.2cm,headheight=1.3cm
]{geometry}
\usepackage{fancyhdr}
\pagestyle{fancy}
\fancyhf{}
\fancyhead[LE,RO]{Quiz 8, Handout 1: Singles}
\fancyfoot[CE,CO]{\thepage}
\usepackage{url}
\makeatother
\newcommand{\resolventline}[2][1]{
  \tikz[overlay]{
      \draw[thick,flexdotted] (0,-1ex) to ++(0,#1*4.5ex) node[above,inner sep=1pt] {#2};
  }
}

\begin{document}

\setlength{\abovedisplayskip}{5pt}
\setlength{\belowdisplayskip}{5pt}


\setcounter{section}{7}
\section{Singles}

\begin{rmk}
\thmtitle{Orbital relaxation}
According to the Thouless theorem (\cref{appendix:thouless}), the effect of the singles CC operator is to transform the orbitals of the reference determinant into a new set $\{\tilde{\y}_i\}$ by ``mixing in'' some of the virtual orbitals.
\begin{align}
  \Y_\mr{CC}
=
  \mr{exp}(T_2 + T_3 + \cd)
  \widetilde{\F}
&&
  \widetilde{\F}
\equiv
  \mr{exp}(T_1)
  \F
=
  \tfr{1}{\sqrt{n!}}\,
  \mr{det}(\tilde{\y}_1\cd \tilde{\y}_n)
&&
  \tilde{\y}_i
=
  \y_i
+
  \sum_a
  \y_a
  t_a^i
\end{align}
This can be thought of as ``relaxing'' the orbitals in the presence of electron correlation.
The size of this \textit{orbital relaxation effect} can be monitored as the root mean square difference from the reference orbitals, which is known as the \textit{$\mc{T}_1$ diagnostic}.
\begin{align}
  \mc{T}_1
\equiv
  \sqrt{
  \fr{1}{n}
  \sum_{i=1}^n
  \|\tilde{\y}_i - \y_i\|^2
  }
=
  \fr{\|\bo{t}_1\|}{\sqrt{n}}
\end{align}
A significant orbital relaxation effect generally indicates that the reference determinant forms a poor approximation to the wavefunction, suggesting the convergence of the CC hierarchy\footnote{CCSD, CCSDT, CCSDTQ, etc.} to the FCI limit may be slow.
\end{rmk}



\begin{ntt}
\label{ntt:orbital-transformation}
Let $\bm{\y}$ be a row vector of orthonormal spin-orbitals, $(\bm{\y})_p=\y_p$, which can be split into occupied and virtual blocks as $\bm{\y}=[\bm{\y}_\mr{o}\ \bm{\y}_\mr{v}]$.
Other sets of spin-orbitals are related to this one by a transformation
$
  \kt{\bm{\y}'}
=
  \kt{\bm{\y}}\,
  \bo{U}
$
which is unitary if the primed orbitals are also orthonormal.
If $\bm{\y}$ is constructed from a basis set $\bm{\x}=[\x_\mu]$ as $\bm{\y}=\bm{\x}\,\bo{C}$ where $\bo{C}$, then the coefficients of $\bm{\y}'$ are given by $\bo{C}'=\bo{C}\bo{U}$.
The transformed reference determinant $\F'$ is constructed from the first $n$ primed orbitals, which are considered occupied in the transformed space.
\end{ntt}

\begin{rmk}
\label{rmk:spectral-theorem}
\thmtitle{Some consequences of the spectral theorem}
A matrix $\bo{M}$ is termed \textit{normal} if it satisfies $\bo{M}\dg\bo{M}=\bo{M}\bo{M}\dg$.
This includes \textit{Hermitian}, \textit{anti-Hermitian}, and \textit{unitary matrices}, which satisfy $\bo{H}\dg=\bo{H}$, $\bo{A}\dg=-\bo{A}$, and $\bo{U}\dg=\bo{U}^{-1}$, respectively.
Hermitian matrices and anti-Hermitian can always be written as $\bo{H}=\bo{X}+\bo{X}\dg$ and $\bo{A}=\bo{X}-\bo{X}\dg$ for some matrix $\bo{X}$.
The Spectral Theorem\footnote{See \url{https://en.wikipedia.org/wiki/Spectral_theorem}} says that a normal matrix is diagonalizable by a unitary transformation.
A corollary of the Spectral Theorem is that Hermitian, anti-Hermitian, and unitary eigenvalues can be written as follows
\begin{align}
  h^*
=
  h
\implies
  h
=
  \f
&&
  a^*
=
-
  a
\implies
  a
=
  i\f
&&
  u^*
=
  u^{-1}
\implies
  u
=
  e^{i\f}
\end{align}
where $\f$ is a real number.
In words, Hermitian eigenvalues are real, anti-Hermitian eigenvalues are pure imaginary, and unitary eigenvalues lie on the unit circle.
The form of its eigenvalues implies that a unitary matrix $\bo{U}$ can be written as $\mr{exp}(\bo{A})$, where $\bo{A}$ is anti-Hermitian.
As a result, unitary matrices are often written in exponential form as
$
  \bo{U}
=
  \mr{exp}(\bo{X} - \bo{X}\dg)
$.
\end{rmk}

\begin{prop}
\label{prop:creation-operator-similarity-transform}
\thmstatement{
The identity\ \
$\ds{
  \mr{exp}(G)\,a_p\dg\,\mr{exp}(-G)
=
  \sum_q
  a_q\dg\,
  (\mr{exp}(\bo{G}))_{qp}
}$\ 
holds for any
$
  G
=
  \sum_{pq}
  (\bo{G})_{pq}\,
  a_p\dg a_q
$.
}\vspace{3pt}
\thmproof{
  This follows from
  $
    [G,\cdot\,]^m(a_p\dg)
  =
    \sum_q
    a_q\dg
    (\bo{G}^m)_{qp}
  $,
  which we will prove by induction.
  For $m=0$ the statement is trivially true.
  If we assume it holds for $m$, then the following shows that it also holds for $m+1$,\footnote{
  The second equality here follows from expanding $G$ and using
$
  [a_r\dg a_s, a_q\dg]
=
  \no{
    a_r\dg
    \ctr{}{a}{_s}{}
    a_s a_q\dg
  }
=
  a_r\dg\,
  \d_{sq}
$.
}
\begin{align*}
\ts{
  [G,\cdot\,]^{m+1}(a_p\dg)
=
  \sum_q
  [G, a_q\dg]\,
  (\bo{G}^m)_{qp}
=
  \sum_{qr}
  a_r\dg
  (\bo{G})_{rq}
  (\bo{G}^m)_{qp}
=
  \sum_r
  a_r\dg
  (\bo{G}^{m+1})_{rp}
}
\end{align*}
  which completes the induction.
  Substituting this result into the Hausdorff expansion of
$
  \mr{exp}(G)\,a_p\dg\,\mr{exp}(-G)
$
  recognizing the Taylor expansion of $\mr{exp}(\bo{G})$ completes the proof.
}
\end{prop}

\begin{rmk}
\thmtitle{Orbital rotations}
Unitary transformations of the spin-orbital basis are termed \textit{orbital rotations}.
According to \cref{rmk:spectral-theorem}, orbital rotations can be written as
$
  \bm{\y'}
=
  \bm{\y}\,
  \mr{exp}(\bm{\Th})
$
where
$
  \bm{\Th}
\equiv
  \bo{X}
-
  \bo{X}\dg
$.
The transformed creation operators are
$
  a_p^{\prime\,\dagger}
=
  \sum_q
  a_q\dg
  (\mr{exp}(\bm{\Th}))_{qp}
$\,\footnote{Since $a_p\dg\kt{\vac}=\kt{\y_p}$, the creation operators always transform like orbitals.}
which, using \cref{prop:creation-operator-similarity-transform}, leads to the following.
\begin{align}
\begin{array}{r@{\ }l}
  a_p^{\prime\,\dagger}
&=
  \mr{exp}(\Th)
  a_p\dg\,
  \mr{exp}(\Th)\dg
\\[5pt]
  a_p^{\prime}
&=
  \mr{exp}(\Th)
  a_p\,
  \mr{exp}(\Th)\dg
\end{array}
&&
  \Th
\equiv
  X
-
  X\dg
&&
  X
\equiv
  \sum_{pq}
  x_p^q\,
  a^p_q
&&
  x_p^q
\equiv
  (\bo{X})_{pq}
\end{align}
Substituting this into 
$
  \kt{\F'}
=
  a_1^{\prime\,\dagger}
  \cd
  a_n^{\prime\,\dagger}
  \kt{\vac}
$
gives a convenient expression for the transformed reference determinant\,\footnote{This follows from 
$
  \mr{exp}(\Th)\dg
  \mr{exp}(\Th)
=
  1
$
and
$
  \mr{exp}(\Th)\dg
  \kt{\vac}
=
  0
$.
}
\begin{align}
  \kt{\F'}
=
  \mr{exp}(\Th)
  \kt{\F}
\end{align}
which generalizes to 
$
  \kt{\F'_{(p_1\cd p_n)}}
=
  \mr{exp}(\Th)
  \kt{\F_{(p_1\cd p_n)}}
$
for other determinants.
\end{rmk}

\begin{rmk}
\thmtitle{Orbital stationarity condition}
\end{rmk}

\begin{align}
  H
=
  h_p^q
  a^p_q
+
  \tfr{1}{4}
  \ol{g}_{pq}^{rs}
  a^{pq}_{rs}
\end{align}

\begin{align}
  E
=
  h_p^q
  \g^p_q
+
  \tfr{1}{4}
  \ol{g}_{pq}^{rs}
  \g^{pq}_{rs}
\end{align}

\begin{align}
  \mc{L}
=
  \mr{Re}
  (
    h_p^q
    \ol{\g}^p_q
  +
    \tfr{1}{4}
    \ol{g}_{pq}^{rs}
    \ol{\g}^{pq}_{rs}
  )
&&
\begin{array}{r@{\ }l}
  \ol{\g}^p_q
&\equiv
  \ip{\F|(1+\La)\,a^p_q\,\mr{exp}(T)|\F}_\mr{C}
\\[4pt]
  \ol{\g}^{pq}_{rs}
&\equiv
  \ip{\F|(1+\La)\,a^{pq}_{rs}\,\mr{exp}(T)|\F}_\mr{C}
\end{array}
\end{align}

\begin{align}
  \mc{L}
=
  h_p^q
  \g^p_q
+
  \tfr{1}{4}
  \ol{g}_{pq}^{rs}
  \g^{pq}_{rs}
&&
\begin{array}{r@{\ }l}
  \g^p_q
&\equiv
  \tfr{1}{2}
  (
    \ol{\g}^p_q
  +
    \ol{\g}^{q*}_p
  )
\\[4pt]
  \g^{pq}_{rs}
&\equiv
  \tfr{1}{2}
  (
    \ol{\g}^{pq}_{rs}
  +
    \ol{\g}^{rs*}_{pq}
  )
\end{array}
\end{align}






\newpage


Brueckner ``best-overlap'' condition
\begin{align}
  |\ip{\F|\Y}|
\end{align}


BCCD
\begin{align}
  0
&=
  \ip{\F_i^a|
    R_0
    V_\mr{c}\,
    \mr{exp}(T)
  |\F}_\mr{C}
\\
  t_{ab}^{ij}
&=
  \ip{\F_{ij}^{ab}|
    R_0
    V_\mr{c}\,
    \mr{exp}(T)
  |\F}_\mr{C}
\end{align}

\begin{align}
  \ip{\F|\ol{H}|\F}
+
  \sum_{ia}
  \la_i^a
  \ip{\F_i^a|\ol{V}_\mr{c}|\F}
+
  \tfr{1}{4}
  \sum_{ijab}
  \la_{ij}^{ab}
  \ip{\F_{ij}^{ab}|\ol{H}_\mr{c}|\F}
\end{align}






\newpage
\appendix
\section{The Thouless theorem}
\label{appendix:thouless}

\begin{thm}
\label{thm:thouless}
\thmtitle{The Thouless theorem}
\begin{enumerate}
\item
\label{item:thouless-part-1}
\thmstatement{
  The function
  $e^{T_1}\F$
  is an intermediately normalized determinant
  $
    \tfr{1}{\sqrt{n!}}
    \mr{det}(\tilde{\y}_1\cd\tilde{\y}_n)
  $
  with orbitals
  $
    \tilde{\y}_i
  =
    \y_i
  +
    \sum_a
    \y_a
    t_a^i
  $.
}
\thmproof{
  Intermediate normalization follows from
  $
    \ip{\F|e^{T_1}\F}
  =
    1
  $.
  This function has the form of a determinant
\begin{align*}
  e^{T_1}
  \kt{\F}
=
  e^{\sum_{a}t_a^1 a_1^a + \cd + \sum_a t_a^n a_n^a}
  a_1\dg
  \cd
  a_n\dg
  \kt{\vac}
=
  \tilde{a}_1\dg
  \cd
  \tilde{a}_n\dg
  \kt{\vac}
=
  \kt{\widetilde{\F}}
&&
  \tl{a}_i\dg
\equiv
  \mr{exp}(\ts{\sum_{a}t_a^i a_i^a})\,
  a_i\dg
\end{align*}
  since $\sum_a t_a^ia_i^a$ commutes with all creation operators except $a_i\dg$.
  The corresponding orbitals are given by
\begin{align*}
  \kt{\tilde{\y}_i}
=
  \tl{a}_i\dg
  \kt{\vac}
=
  \mr{exp}(\ts{\sum_{a}t_a^i a_a\dg a_i})\,
  a_i\dg
  \kt{\vac}
=
  (\ts{
    1
  +
    \sum_a
    t_a^i
    a_a\dg
    a_i
  })\,
    a_i\dg
  \kt{\vac}
=
  \kt{\y_i}
+
  \ts{\sum_a}
  t_a^i
  \kt{\y_a}
\end{align*}
 using $a_i^2=0$ as well as $a_ia_i\dg\kt{\vac}=\kt{\vac}$.
}

\item
\thmstatement{
  Any intermediately normalized determinant
  $
    \widetilde{\F}
  =
    \tfr{1}{\sqrt{n!}}
    \mr{det}(\tilde{\y}_1\cd\tilde{\y}_n)
  $
  can be written as $e^{T_1}\,\F$.
}
\thmproof{
  Intermediate normalization is only possible if $\widetilde{\F}$ has non-zero overlap with the reference determinant.
  Therefore, $\widetilde{\F}$ can be written as
  $\F'/\ip{\F|\F'}$
  where $\F'$ is a Slater determinant.
  The overlap of $\ip{\F|\F'}$ is given by
\begin{align*}
\ts{
  \ip{\F|\F'}
=
  \tfr{1}{n!}
  \sum_{\pi,\si}^{\mr{S}_n}
  \e_\pi
  \e_\si
  \ip{\y_{\pi(1)}|\y'_{\si(1)}}
  \cd
  \ip{\y_{\pi(n)}|\y'_{\si(n)}}
=
  \sum_{\si}^{\mr{S}_n}
  \e_\si
  \ip{\y_{1}|\y'_{\si(1)}}
  \cd
  \ip{\y_{n}|\y'_{\si(n)}}
=
  \mr{det}(\bo{U}_{\mr{oo}})
}
\end{align*}
  where $\bo{U}_\mr{oo}$ is the occupied block of the orbital transformation matrix $\bo{U}$, defined according to \cref{ntt:orbital-transformation}.
  Therefore, 
  $
    \widetilde{\F}
  =
    \F'/
    \mr{det}(\bo{U}_\mr{oo})
  =
    \F'\,
    \mr{det}(\bo{U}_\mr{oo}^{-1})
  $
  and the rows of $\widetilde{\F}$ are given by the following.
\begin{align*}
\ts{
  \kt{\bm{\tilde{\y}}_\mr{o}}
=
  \kt{\bm{\y}'_\mr{o}}\,
  \bo{U}_\mr{oo}^{-1}
=
  \kt{\bm{\y}_\mr{o}}\,
+
  \kt{\bm{\y}_\mr{v}}\,
  \bo{U}_\mr{vo}
  \bo{U}_\mr{oo}^{-1}
\implies
  \tilde{\y}_i
=
  \y_i
+
  \sum_a
  (\bo{U}_\mr{vo}\bo{U}_\mr{oo}^{-1})_{ai}\,
  \y_a
}
\end{align*}
  The second equality follows from
  $
    \kt{\bm{\y}_\mr{o}'}
  =
    \kt{\bm{\y}_\mr{o}}\,
    \bo{U}_\mr{oo}
  +
    \kt{\bm{\y}_\mr{v}}\,
    \bo{U}_\mr{vo}
  $
  where $\bo{1}_\mr{o}$ is an identity matrix in the occupied block.
  According to \cref{item:thouless-part-1}, this implies that
  $\widetilde{\F}$
  equals
  $
    e^{T_1}\F
  $
  with amplitudes 
  $
    t_a^i
  \equiv
    (\bo{U}_\mr{vo}\bo{U}_\mr{oo}^{-1})_{ai}
  $.
}
\end{enumerate}
\end{thm}




\end{document}