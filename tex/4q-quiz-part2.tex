\documentclass[11pt]{article}
\usepackage[cm]{fullpage}
%%AVC PACKAGES
\usepackage{avcgreek}
\usepackage{avcfonts}
\usepackage{avcmath}
\usepackage[numberby=section]{avcthm}
\usepackage{qcmacros}
\usepackage{goldstone}
%%MACROS FOR THIS DOCUMENT
\usepackage[
  margin=1.5cm,
  includefoot,
  footskip=30pt,
  headsep=0.2cm,headheight=1.3cm
]{geometry}
\usepackage{fancyhdr}
\pagestyle{fancy}
\fancyhf{}
\fancyhead[LE,RO]{\textbf{Quiz 4, Part 2}}
\fancyfoot[CE,CO]{\thepage}
\usepackage{url}

\begin{document}

\begin{enumerate}
\item
  Expand the electronic Hamiltonian $H$ in terms of $\F$-normal-ordered operators using Wick's theorem for graphs, writing the core Hamiltonian and electron repulsion operators as
$
\diagram{
  \draw (-0.5,0) node[squarex] (h) {} -- (0,0) node[dot=white] (h1) {};
  \draw[->-] (h1) to ++(0,+0.35);
  \draw[-<-] (h1) to ++(0,-0.35);
}
\equiv
  h_p^qa_q^p
$
and
$
\diagram{
  \interaction{2}{g}{(0,0)}{dot=white}{sawtooth};
  \draw[->-] (g1) to ++(0,+0.35);
  \draw[-<-] (g1) to ++(0,-0.35);
  \draw[->-] (g2) to ++(0,+0.35);
  \draw[-<-] (g2) to ++(0,-0.35);
}
\equiv
  \textstyle\frac{1}{4}\overline{g}_{pq}^{rs}a_{rs}^{pq}
$.
  \begin{align*}
    H
  =
  \diagram{
    \draw (-0.5,0) node[squarex] (h) {} -- (0,0) node[dot=white] (h1) {};
    \draw[->-] (h1) to ++(0,+0.5);
    \draw[-<-] (h1) to ++(0,-0.5);
  }
  +
  \diagram{
    \interaction{2}{g}{(0,0)}{dot=white}{sawtooth};
    \draw[->-] (g1) to ++(0,+0.5);
    \draw[->-] (g2) to ++(0,+0.5);
    \draw[-<-] (g1) to ++(0,-0.5);
    \draw[-<-] (g2) to ++(0,-0.5);
  }
  =
  \,\,?
  \end{align*}


\newpage
\item
  Evaluate the following using Wick's theorem for graphs.
  Fully simplify your answer assuming the indices refer to a basis of canonical Hartree-Fock spin-orbitals.
  \begin{align*}
    \ip{\F_{ijk}^{abc}|F_\mr{c}\,C_3|\F}
  =
    \,\,?
  &&
  \begin{array}{r@{\ }l}
    F_\mr{c}
  &\equiv
    f_p^q
    \tl{a}^p_q
  \\[5pt]
    C_3
  &\equiv
    (\tfr{1}{3!})^2
    c_{def}^{lmn}
    \tl{a}^{def}_{lmn}
  \end{array}
  \end{align*}


\newpage
\item
  \begin{enumerate}
  \item
    Explain how to get from the projected CCD Schr\"odinger equation
    \begin{align}
      E_\mr{c}\,t_{ab}^{ij}
    =
      \ip{\F_{ij}^{ab}|H_\mr{c}\,\mr{exp}(T_2)|\F}
    &&
        H_\mr{c}
      &=
        F_\mr{c}
      +
        V_\mr{c}
    &&
      \begin{array}{r@{\ }l}
        F_\mr{c}
      &\equiv
        f_p^q
        \tl{a}^p_q
      \\[5pt]
        V_\mr{c}
      &\equiv
        \tfr{1}{4}
        \ol{g}_{pq}^{rs}
        \tl{a}^{pq}_{rs}
      \end{array}
    \end{align}
    to the working equation for CCD amplitudes
    \begin{align}
    \label{eq:ccd-working-equation}
      t_{ab}^{ij}
    =
      (\mc{E}_{ab}^{ij})^{-1}
      \ip{\F_{ij}^{ab}|V_\mr{c}\,\mr{exp}(T_2)|\F}_\mr{L}
    &&
      \mc{E}_{ab}^{ij}
    \equiv
      \ev_i
    +
      \ev_j
    -
      \ev_a
    -
      \ev_b
    \end{align}
    assuming a canonical Hartree-Fock reference.\footnote{Hint:
    You only need to evaluate three diagrams to answer this question.}

  \item
    Write out an algorithm to numerically solve equation~\ref{eq:ccd-working-equation}.
  \end{enumerate}


\newpage
\item[]
  \textbf{Extra Credit}:
  Derive the following interpretation rule in your own words:
  \begin{enumerate}
  \item[]
    Each open cycle in a graph contributes $(-)^{h_i}a^p_q$ to the normal-ordered product of operators, where $p$ and $q$ label the free ends and $h_i$ is the number of hole contractions in the cycle.
  A closed cycle (loop) contributes $(-)^{h_i+1}$.
  \end{enumerate}
\end{enumerate}

\end{document}
