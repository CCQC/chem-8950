\documentclass[11pt]{article}
\usepackage[cm]{fullpage}
%%AVC PACKAGES
\usepackage{avcgreek}
\usepackage{avcfonts}
\usepackage{avcmath}
\usepackage[numberby=section]{avcthm}
\usepackage{qcmacros}
\usepackage{goldstone}
%%MACROS FOR THIS DOCUMENT
\usepackage[
  margin=1.5cm,
  includefoot,
  footskip=30pt,
  headsep=0.2cm,headheight=1.3cm
]{geometry}
\usepackage{fancyhdr}
\pagestyle{fancy}
\fancyhf{}
\fancyhead[LE,RO]{Quiz 7, Suggested Problems 1: Perturbative analysis}
\fancyfoot[CE,CO]{\thepage}
\usepackage{url}
\usepackage{multicol}

\begin{document}

\begin{enumerate}
\item
Explain why the maximum excitation level of the wavefunction increases by $+2$ with each order in perturbation theory.

\item
Prove that the leading contribution to the $k$-tuples CI operator has order $\ceil{k/2}$ in perturbation theory.

\item
Write down the CISD energy and singles and doubles equations in terms of $C_k$ operators.  Identify the leading order of each term in perturbation theory with and without Brillouin's theorem.

\item
Prove that $\mr{CIS}{\cd}m$ is correct to order $\floor{m/2}$ in the wavefunction and $2\floor{m/2} + 1$ in the energy.

\item
Write down the CCSD energy and singles and doubles equations in terms of $T_k$ operators.  Identify the leading order of each term in perturbation theory with and without Brillouin's theorem.

\item
Explain why the leading contribution to the $k$-tuples cluster operator has order $k-1$.

\item
Prove that $\mr{CCS}{\cd}m$ is correct to order $m-1$ in the wavefunction and $m + \floor{m/2}$ in the energy.

\item
``Derive'' the [T] correction and evaluate it, showing both the diagrams and their algebraic interpretation.\footnote{``Derive'' here means ``give detailed motivation for''.}\,\footnote{The operator $Q_3$ here simply projects onto the space of triply substituted determinants.}
You may write your answer in terms of ${}\bord{2}t_{abc}^{ijk}$ amplitudes and evaluate those separately.
\begin{align}
  E_{[\mr{T}]}
=
  \ip{\F|T_2\dg V_\mr{c} T_3\bord{2}|\F}
&&
  T_3\bord{2}
=
  (\tfr{1}{3!})^2
  \sum_{\substack{abc\\ijk}}
  \tl{a}_{ijk}^{abc}
  \ip{\F_{ijk}^{abc}|
    R_0
    V_\mr{c}
    T_2
  |\F}
\end{align}

\item
Prove that the left and right EOM-CC wave operators are given by
\begin{align}
  {}^kR
=
  \mr{exp}(-T)
  (
    {}^kC_0
  +
    \,{}^kC
  )
&&
  {}^kL
=
  (
    {}^kC_0
  +
    {}^kC
  )\dg\,
  \mr{exp}(T)
\end{align}
where $\,{}^kC_0+\,{}^kC$ is the CI wave operator for the $k\eth$ state.
Use this to show that we can determine excited state expectation values and transition matrix elements for an observable $W$ as
$
  \ip{\Y_k|W|\Y_k}
=
  \ip{\F|L_k\ol{W}R_k|\F}
$
and
$
  \ip{\Y_k|W|\Y_l}
=
  \ip{\F|L_k\ol{W}R_l|\F}
$
where
$
  \ol{W}
\equiv
  \mr{exp}(-T)
  W
  \mr{exp}(T)
$.

\item
Evaluate the CCD lambda equations, showing both the diagrams and their interpretation.

\setcounter{enumi}{12}
\item
Derive the $(m+1)_\La$ correction from the coupled-cluster L\"owdin functional.

\item
``Derive'' the (T) correction as an approximation to (T)$_\La$ correction and evaluate it.
\begin{align}
  E_{(\mr{T})}
=
  E_{[\mr{T}]}
+
  \br{\F}
    T_1\dg
    V_\mr{c}
    T_3\bord{2}
  \kt{\F}
\end{align}


\end{enumerate}


\end{document}