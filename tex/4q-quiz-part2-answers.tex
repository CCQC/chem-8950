\documentclass[11pt]{article}
\usepackage[cm]{fullpage}
%%AVC PACKAGES
\usepackage{avcgreek}
\usepackage{avcfonts}
\usepackage{avcmath}
\usepackage[numberby=section]{avcthm}
\usepackage{qcmacros}
\usepackage{goldstone}
%%MACROS FOR THIS DOCUMENT
\usepackage[
  margin=1.5cm,
  includefoot,
  footskip=30pt,
  headsep=0.2cm,headheight=1.3cm
]{geometry}
\usepackage{fancyhdr}
\pagestyle{fancy}
\fancyhf{}
\fancyhead[LE,RO]{\textbf{Quiz 4, Part 2}}
\fancyfoot[CE,CO]{\thepage}
\usepackage{url}

\begin{document}

\begin{enumerate}
\item
  Expand the electronic Hamiltonian $H_e$ in terms of $\F$-normal-ordered operators using Wick's theorem for graphs, writing the core Hamiltonian and electron repulsion operators as 
$
\diagram{
  \draw (-0.5,0) node[squarex] (h) {} -- (0,0) node[dot=white] (h1) {}; 
  \draw[->-] (h1) to ++(0,+0.35);
  \draw[-<-] (h1) to ++(0,-0.35);
}
\equiv
  h_p^qa_q^p
$
and
$
\diagram{
  \interaction{2}{g}{(0,0)}{dot=white}{sawtooth};
  \draw[->-] (g1) to ++(0,+0.35);
  \draw[-<-] (g1) to ++(0,-0.35);
  \draw[->-] (g2) to ++(0,+0.35);
  \draw[-<-] (g2) to ++(0,-0.35);
}
\equiv
  \textstyle\frac{1}{4}\overline{g}_{pq}^{rs}a_{rs}^{pq}
$.
  \begin{align*}
    H_e
  =
  \diagram{
    \draw (-0.5,0) node[squarex] (h) {} -- (0,0) node[dot=white] (h1) {};
    \draw[->-] (h1) to ++(0,+0.5);
    \draw[-<-] (h1) to ++(0,-0.5);
  }
  +
  \diagram{
    \interaction{2}{g}{(0,0)}{dot=white}{sawtooth};
    \draw[->-] (g1) to ++(0,+0.5);
    \draw[->-] (g2) to ++(0,+0.5);
    \draw[-<-] (g1) to ++(0,-0.5);
    \draw[-<-] (g2) to ++(0,-0.5);
  }
  =
  \,\,?
  \end{align*}
\vspace{10pt}
\textbf{Answer}:
\begin{align*}
\diagram{
  \draw (-0.5,0) node[squarex] (h) {} -- (0,0) node[dot=white] (h1) {};
  \draw[->-] (h1) to ++(0,+0.5);
  \draw[-<-] (h1) to ++(0,-0.5);
}
+
\diagram{
  \interaction{2}{g}{(0,0)}{dot=white}{sawtooth};
  \draw[->-] (g1) to ++(0,+0.5);
  \draw[->-] (g2) to ++(0,+0.5);
  \draw[-<-] (g1) to ++(0,-0.5);
   \draw[-<-] (g2) to ++(0,-0.5);
}
=&\
\diagram{
  \draw (-0.5,0) node[squarex] (h) {} -- (0,0) node[ddot=white] (h1) {};
  \draw[->-] (h1) to ++(0,+0.5);
  \draw[-<-] (h1) to ++(0,-0.5);
}
+
\diagram{
  \draw (-0.5,0) node[squarex] (h) {} -- (0,0) node[ddot=white] (h1) {};
  \draw[-<-] (h1) arc (0:360:-0.25);
}
+
\diagram{
  \interaction{2}{g}{(0,0)}{ddot=white}{sawtooth};
  \draw[->-] (g1) to ++(0,+0.5);
  \draw[-<-] (g1) to ++(0,-0.5);
  \draw[->-] (g2) to ++(0,+0.5);
  \draw[-<-] (g2) to ++(0,-0.5);
}
+
\diagram{
  \interaction{2}{g}{(0,0)}{ddot=white}{sawtooth};
  \draw[->-] (g1) arc (0:360:+0.25);
  \draw[->-] (g2) to ++(0,+0.5);
  \draw[-<-] (g2) to ++(0,-0.5);
}
+
\diagram{
  \interaction{2}{g}{(0,0)}{ddot=white}{sawtooth};
  \draw[->-] (g1) arc (0:360:+0.25);
  \draw[-<-] (g2) arc (0:360:-0.25);
}
\\=&\
  h_p^q
  \tl{a}^p_q
+
  h_i^i
+
  \tfr{1}{4}
  \ol{g}_{pq}^{rs}
  \tl{a}^{pq}_{rs}
+
  \ol{g}_{ip}^{iq}
  \tl{a}^p_q
+
  \tfr{1}{2}
  \ol{g}_{ij}^{ij}
\end{align*}
The degeneracy factors were derived as follows:
The third term has two pairs of equivalent linesI', yielding a degeneracy of $2^2=4$.
The fifth term has just one pair of equivalent lines, yielding a degeneracy of $2$.

\newpage
\item
  Evaluate the following using Wick's theorem for graphs.
  Fully simplify your answer assuming the indices refer to a basis of canonical Hartree-Fock spin-orbitals.
  \begin{align*}
    \ip{\F_{ijk}^{abc}|F_\mr{c}\,C_3|\F}
  =
    \,\,?
  &&
  \begin{array}{r@{\ }l}
    F_\mr{c}
  &\equiv
    f_p^q
    \tl{a}^p_q
  \\[5pt]
    C_3
  &\equiv
    (\tfr{1}{3!})^2
    c_{def}^{lmn}
    \tl{a}^{def}_{lmn}
  \end{array}
  \end{align*}
\vspace{10pt}
\textbf{Answer}:
\begin{align*}
  \ip{\F_{ijk}^{abc}|F_\mr{c}\,C_3|\F}
=
\diagram[top,bottom]{
%top
  \interaction{3}{a}{(0,1)}{ddot}{densely dotted};
  \draw[->-] (a1) to ++(-0.25,-0.5);
  \draw[-<-] (a1) to ++(+0.25,-0.5);
  \draw[->-] (a2) to ++(-0.25,-0.5);
  \draw[-<-] (a2) to ++(+0.25,-0.5);
  \draw[->-] (a3) to ++(-0.25,-0.5);
  \draw[-<-] (a3) to ++(+0.25,-0.5);
%midddle
  \draw (0.5,0) node[circlex](f){} -- (1,0) node[ddot=white](f1){};
  \draw[->-] (f1) to ++(0,+0.5);
  \draw[-<-] (f1) to ++(0,-0.5);
%bottom
  \interaction{3}{c}{(0,-1)}{ddot}{overhang};
  \draw[->-] (c1) to ++(-0.25,+0.5);
  \draw[-<-] (c1) to ++(+0.25,+0.5);
  \draw[->-] (c2) to ++(-0.25,+0.5);
  \draw[-<-] (c2) to ++(+0.25,+0.5);
  \draw[->-] (c3) to ++(-0.25,+0.5);
  \draw[-<-] (c3) to ++(+0.25,+0.5);
}
=&\
\diagram{
  \interaction{3}{c}{(0,-0.5)}{ddot}{overhang};
  \draw[->-=0.25,->-=0.75] (c1)
    to node[midway,ddot] (f1) {} ++(-0.25,+1) node[smalldot] {};
  \draw (f1) to ++(-0.5,0) node[circlex] (f) {};
  \draw[-<-] (c1) to ++(+0.25,1) node[smalldot] {};
  \draw[->-] (c2) to ++(-0.25,1) node[smalldot] {};
  \draw[-<-] (c2) to ++(+0.25,1) node[smalldot] {};
  \draw[->-] (c3) to ++(-0.25,1) node[smalldot] {};
  \draw[-<-] (c3) to ++(+0.25,1) node[smalldot] {};
}
+
\diagram{
  \interaction{3}{c}{(0,-0.5)}{ddot}{overhang};
  \draw[-<-=0.25,-<-=0.75] (c1)
    to node[midway,ddot] (f1) {} ++(-0.25,+1) node[smalldot] {};
  \draw (f1) to ++(-0.5,0) node[circlex] (f) {};
  \draw[->-] (c1) to ++(+0.25,1) node[smalldot] {};
  \draw[-<-] (c2) to ++(-0.25,1) node[smalldot] {};
  \draw[->-] (c2) to ++(+0.25,1) node[smalldot] {};
  \draw[-<-] (c3) to ++(-0.25,1) node[smalldot] {};
  \draw[->-] (c3) to ++(+0.25,1) node[smalldot] {};
}
\\=&\
  \op{P}_{(a/bc)}
  f_a^d
  c_{dbc}^{ijk}
-
  \op{P}^{(i/jk)}
  f_l^i
  c_{abc}^{ljk}
\end{align*}
  These are the reduced antisymmetrizers, which means we have cancelled the degeneracy factors coming from equivalent coefficient lines (a set of three and a set of two in each graph).
  For canonical orbitals, Brillouin's theorem holds and this can be further simplified.
\begin{align*}
  \ip{\F_{ijk}^{abc}|F_\mr{c}\,C_3|\F}
=&\
  \op{P}_{(a/bc)}
  \ev_a
  c_{abc}^{ijk}
-
  \op{P}^{(i/jk)}
  \ev_i
  c_{abc}^{ijk}
\\=&\
  \ev_a
  c_{abc}^{ijk}
-
  \ev_b
  c_{bac}^{ijk}
-
  \ev_c
  c_{cba}^{ijk}
-
  \ev_i
  c_{abc}^{ijk}
+
  \ev_j
  c_{abc}^{jik}
+
  \ev_k
  c_{abc}^{kji}
\\=&\
  (\ev_a + \ev_b + \ev_c - \ev_i - \ev_j - \ev_k)
  c_{abc}^{ijk}
\end{align*}
  The last step follows from the fact that $c_{abc}^{ijk}$ is antisymmetric in its upper and lower indices.

\newpage
\item
  \begin{enumerate}
  \item
    Explain how to get from the projected CCD Schr\"odinger equation
    \begin{align}
    \label{eq:projected-ccd-schrodinger-equation}
      E_\mr{c}\,t_{ab}^{ij}
    =
      \ip{\F_{ij}^{ab}|H_\mr{c}\,\mr{exp}(T_2)|\F}
    &&
        H_\mr{c}
      &=
        F_\mr{c}
      +
        V_\mr{c}
    &&
      \begin{array}{r@{\ }l}
        F_\mr{c}
      &\equiv
        f_p^q
        \tl{a}^p_q
      \\[5pt]
        V_\mr{c}
      &\equiv
        \tfr{1}{4}
        \ol{g}_{pq}^{rs}
        \tl{a}^{pq}_{rs}
      \end{array}
    \end{align}
    to the working equation for CCD amplitudes
    \begin{align}
    \label{eq:ccd-working-equation}
      t_{ab}^{ij}
    =
      (\mc{E}_{ab}^{ij})^{-1}
      \ip{\F_{ij}^{ab}|V_\mr{c}\,\mr{exp}(T_2)|\F}_\mr{L}
    &&
      \mc{E}_{ab}^{ij}
    \equiv
      \ev_i
    +
      \ev_j
    -
      \ev_a
    -
      \ev_b
    \end{align}
    assuming a canonical Hartree-Fock reference.\footnote{Hint: 
    You only need to evaluate three diagrams to answer this question.}\\[10pt]
\textbf{Answer}:
The only way to form a unlinked diagram from the operators in
$\ip{\F_{ij}^{ab}|H_\mr{c}\,\mr{exp}(T_2)|\F}$ is
\begin{align*}
  \ip{\F_{ij}^{ab}|H_\mr{c}\,\mr{exp}(T_2)|\F}_{\mr{U}}
=&\
\diagram[top, bottom]{
%top
  \node at (0,+0.75) {
  $
  \diagram{
    \interaction{2}{c}{(0,+0.25)}{ddot}{densely dotted};
    \draw[->-] (c1) to ++(-0.25,-0.5);
    \draw[-<-] (c1) to ++(+0.25,-0.5);
    \draw[->-] (c2) to ++(-0.25,-0.5);
    \draw[-<-] (c2) to ++(+0.25,-0.5);
  }$
  };
%middle
  \node at (0,0) {
  $\left(
  \diagram{
    \draw (-0.5,0) node[circlex](f){} -- (0,0) node[ddot=white](f1){};
    \draw[->-] (f1) to ++(0,+0.5);
    \draw[-<-] (f1) to ++(0,-0.5);
  }
  +
  \diagram{
    \interaction{2}{g}{(0,0)}{ddot=white}{sawtooth};
    \draw[->-] (g1) to ++(0,+0.5);
    \draw[-<-] (g1) to ++(0,-0.5);
    \draw[->-] (g2) to ++(0,+0.5);
    \draw[-<-] (g2) to ++(0,-0.5);
  }\right)$
  };
%bottom
  \node at (0,-0.8) {
  $
  \mr{exp}\left(
  \diagram{
    \interaction{2}{c}{(0,-0.25)}{ddot}{overhang};
    \draw[->-] (c1) to ++(-0.25,+0.5);
    \draw[-<-] (c1) to ++(+0.25,+0.5);
    \draw[->-] (c2) to ++(-0.25,+0.5);
    \draw[-<-] (c2) to ++(+0.25,+0.5);
  }
  \right)
  $
  };
}_{\mr{U}}
=
\diagram{
  \interaction{2}{ta}{(0,-0.5)}{ddot}{overhang};
  \interaction{2}{tb}{(2,-0.5)}{ddot}{overhang};
  \interaction{2}{g}{(0,+0.5)}{ddot}{sawtooth};
  \draw[->-,bend left]  (ta1) to (g1);
  \draw[-<-,bend right] (ta1) to (g1);
  \draw[->-,bend left]  (ta2) to (g2);
  \draw[-<-,bend right] (ta2) to (g2);
  \draw[->-] (tb1) to ++(-0.25,1) node[smalldot] {};
  \draw[-<-] (tb1) to ++(+0.25,1) node[smalldot] {};
  \draw[->-] (tb2) to ++(-0.25,1) node[smalldot] {};
  \draw[-<-] (tb2) to ++(+0.25,1) node[smalldot] {};
}
\end{align*}
because we need at least two operators on bottom to form a disconnected part, and one of them must be fully contracted with either $V_{\mr{c}}$ or $\tl{a}_{ab}^{ij}$.
The first factor on the right is the only diagram arising from
$\ip{\F|H_{\mr{c}}\,\mr{exp}(T_2)|\F}=E_\mr{c}$.
Therefore,
\begin{align*}
  0
=
  E_\mr{c}\,
  t_{ab}^{ij}
-
  \ip{\F_{ij}^{ab}|H_\mr{c}\,\mr{exp}(T_2)|\F}_{\mr{U}}
=&\
  \ip{\F_{ij}^{ab}|H_\mr{c}\,\mr{exp}(T_2)|\F}_{\mr{L}}
\end{align*}
which implies the following.
\begin{align}
\label{eq:1}
-
  \ip{\F_{ij}^{ab}|F_\mr{c}\,\mr{exp}(T_2)|\F}_{\mr{L}}
=
  \ip{\F_{ij}^{ab}|V_\mr{c}\,\mr{exp}(T_2)|\F}_{\mr{L}}
\end{align}
The term on the left evaluates as follows.
\begin{align*}
  \ip{\F_{ij}^{ab}|F_\mr{c}\,\mr{exp}(T_2)|\F}_{\mr{L}}
=
\diagram[top,bottom]{
%top
  \interaction{2}{a}{(0,1)}{ddot}{densely dotted};
  \draw[->-] (a1) to ++(-0.25,-0.5);
  \draw[-<-] (a1) to ++(+0.25,-0.5);
  \draw[->-] (a2) to ++(-0.25,-0.5);
  \draw[-<-] (a2) to ++(+0.25,-0.5);
%midddle
  \draw (0,0) node[circlex](f){} -- (0.5,0) node[ddot=white](f1){};
  \draw[->-] (f1) to ++(0,+0.5);
  \draw[-<-] (f1) to ++(0,-0.5);
%bottom
  \interaction{2}{c}{(0,-1)}{ddot}{overhang};
  \draw[->-] (c1) to ++(-0.25,+0.5);
  \draw[-<-] (c1) to ++(+0.25,+0.5);
  \draw[->-] (c2) to ++(-0.25,+0.5);
  \draw[-<-] (c2) to ++(+0.25,+0.5);
}
=&\
\diagram{
  \interaction{2}{c}{(0,-0.5)}{ddot}{overhang};
  \draw[->-=0.25,->-=0.75] (c1)
    to node[midway,ddot] (f1) {} ++(-0.25,+1) node[smalldot] {};
  \draw (f1) to ++(-0.5,0) node[circlex] (f) {};
  \draw[-<-] (c1) to ++(+0.25,1) node[smalldot] {};
  \draw[->-] (c2) to ++(-0.25,1) node[smalldot] {};
  \draw[-<-] (c2) to ++(+0.25,1) node[smalldot] {};
}
+
\diagram{
  \interaction{2}{c}{(0,-0.5)}{ddot}{overhang};
  \draw[-<-=0.25,-<-=0.75] (c1)
    to node[midway,ddot] (f1) {} ++(-0.25,+1) node[smalldot] {};
  \draw (f1) to ++(-0.5,0) node[circlex] (f) {};
  \draw[->-] (c1) to ++(+0.25,1) node[smalldot] {};
  \draw[-<-] (c2) to ++(-0.25,1) node[smalldot] {};
  \draw[->-] (c2) to ++(+0.25,1) node[smalldot] {};
}
\\=&\
  \op{P}_{(a/b)}
  f_a^c
  t_{cb}^{ij}
-
  \op{P}^{(i/j)}
  f_k^i
  t_{ab}^{kj}
\\=&\
  \ev_a
  t_{ab}^{ij}
-
  \ev_b
  t_{ba}^{ij}
-
  \ev_i
  t_{ab}^{ij}
+
  \ev_j
  t_{ab}^{ji}
\\=&\
-
  \mc{E}_{ab}^{ij}
  t_{ab}^{ij}
\end{align*}
Substituting this into equation \ref{eq:1} and dividing both sides by $\mc{E}_{ab}^{ij}$ leads to equation \ref{eq:ccd-working-equation}.

  \item
    Write out an algorithm to numerically solve equation~\ref{eq:ccd-working-equation}.\\[10pt]
\textbf{Answer}:
  Starting from ${}^{[0]}t_{ab}^{ij}=0$ and $n=1$, do the following.
  \begin{enumerate}
  \item\label{item:ccd-amplitude-update}
    Update amplitudes as
    $
      {}^{[n]}
      t_{ab}^{ij}
    =
      (\mc{E}_{ab}^{ij})^{-1}
      \ip{\F_{ij}^{ab}|V_\mr{c}\,\mr{exp}({}^{[n-1]}T_2)|\F}_\mr{L}
    $.
  \item
    If $|{}^{[n]}\bm{t}_2-{}^{[n-1]}\bm{t}_2|$ is less than the convergence threshold, increment $n$ and return to step i.
  \end{enumerate}
  \end{enumerate}


\newpage
\item[]
  \textbf{Extra Credit}:
  Derive the following interpretation rule in your own words:
  \begin{enumerate}
  \item[]
    Each open cycle in a graph contributes $(-)^{h_i}a^p_q$ to the normal-ordered product of operators, where $p$ and $q$ label the free ends and $h_i$ is the number of hole contractions in the cycle.
  A closed cycle (loop) contributes $(-)^{h_i+1}$.
  \end{enumerate}
\vspace{10pt}
\textbf{Answer}:
The single-excitation operators involved in a contraction can be brought together with no sign change, since they each contain two operators.
The contracted operators can then be eliminated as follows.
\begin{align*}
  \gno{
  \cd
    a^r_{a^\ptcl}
    a^{a^\ptcl}_s
  \cd
  }
=&\
-
  \gno{
  \cd
    a^r_s
    (-\h_a^a)
  \cd
  }
=
+\,
  \gno{
  \cd
    a^r_s
  \cd
  }
\hspace{10pt}
  \text{or}
\\
  \gno{
  \cd
    a^{i^\hole}_s
    a^r_{i^\hole}
  \cd
  }
=&\
-
  \gno{
  \cd
    a^r_s
    (+\g_i^i\,)
  \cd
  }
=
-\,
  \gno{
  \cd
    a^r_s
  \cd
  }
\end{align*}
Applying this result to each contraction line in an open cycle yields
$(-)^{h_i}a^p_q$.
Applying it to all but one particle line in a loop yields
$
  (-)^{h_i}a^{a^\ptcl}_{a^\ptcl}
=
  (-)^{h_i}(-\h^a_a)
=
  (-)^{h_i + 1}
$.
\end{enumerate}

\end{document}