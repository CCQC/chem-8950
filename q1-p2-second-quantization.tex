\documentclass[11pt]{article}
\usepackage[cm]{fullpage}
%%AVC PACKAGES
\usepackage{avcgreek}
\usepackage{avcfonts}
\usepackage{avcmath}
\usepackage[numberby=section]{avcthm} % 
\usepackage{qcmacros}
\usepackage{goldstone}
%%MACROS FOR THIS DOCUMENT
\numberwithin{equation}{section}
\usepackage[
  margin=1.5cm,
  includefoot,
  footskip=30pt,
  headsep=0.2cm,headheight=1.3cm
]{geometry}
\usepackage{fancyhdr}
\pagestyle{fancy}
\fancyhf{}
\fancyhead[LE,RO]{Quiz 1, Suggested Problems 2: Second Quantization}
\fancyfoot[CE,CO]{\thepage}
\usepackage{url}

\begin{document}

\begin{enumerate}
\item
  Show why the following relationships hold for antisymmetric functions $\Y,\Y'\in\mc{F}_n$.
  \begin{align*}
    \sum_{i=1}^n
    \ip{\Y|\op{h}(i)\Y'}
  =
    n\ip{\Y|\op{h}(1)\Y'}
  &&
    \sum_{i<j}^n
    \ip{\Y|\op{g}(i,j)\Y'}
  =
    \tfr{n(n-1)}{2}
    \ip{\Y|\op{g}(1,2)\Y'}
  \end{align*}

\item
  Show how to make the following rearrangement.
  \begin{align*}
    \tfr{1}{2}
    \sum_{pqrs}
    \ip{pq|rs}
    a_p\dg a_q\dg a_s a_r
  =
    \tfr{1}{4}
    \sum_{pqrs}
    \ip{pq||rs}
    a_p\dg a_q\dg a_s a_r
  \end{align*}

\item
  For the integral-operator definition of $\op{a}_p$, show that the following relationships hold.
  \begin{enumerate}
  \item
    $
      (\op{a}_p\F_{(p_1\cd p_n)})(2,\cd,n)
    =
    \left\{
    \ar{
      (-)^{k-1}\F_{(p_1\cd \cancel{p_k}\cd p_n)}(2,\cd,n) & p=p_k\in(p_1\cd p_n)\\[5pt]
      0 & \text{otherwise}
    }
    \right.
    $

  \item
    $\op{a}_p\op{a}_q = - \op{a}_q\op{a}_p$

  \item
    $\ds{
      \Y(1,\cd,n)
    =
      \tfr{1}{\sqrt{n}}
      \sum_p^\infty\y_p(1)\pr{\op{a}_p\Y}(2,\cd,n)
    }$

  \item
    $\ds{
      \Y(1,\cd,n)
    =
      \tfr{1}{\sqrt{n(n-1)}}
      \sum_{pq}^\infty
      \y_p(1)\y_q(2)(\op{a}_q\op{a}_p\Y)(3,\cd,n)
    }$
  \end{enumerate}


\item
  Using your own words, prove that $c_p=a_p\dg$.
  You may use either the determinant formalism or the occupation number formalism.

\item
  Verify the anticommutator relation $[q,q']_+=\d_{q'q\dg}$ by proving each of the following cases.
  \begin{align*}
    [a_p, a_q]_+
  =
    0
  &&
    [a_p\dg, a_q\dg]_+
  =
    0
  &&
    [a_p, a_q\dg]_+
  =
    0
  \hspace{5pt} (p\neq q)
  &&
    [a_p, a_p\dg]_+
  =
    1
  \end{align*}

\item
  Show that the occupation-number definition of $a_p$ and $c_p$ is consistent with the determinant definition.

\item
  Do Problem 1.4 in Helgaker's big purple book.

\item
  Derive Slater's rules using second quantization.
  Where necessary, explain in words why a given term vanishes.
  (Hint: Use particle-hole isomorphism.)
  \begin{enumerate}
  \item
    $\ds{
      \ip{\F|H_e|\F}
    =
      \sum_i
      h_{ii}
    +
      \tfr{1}{2}
      \sum_{ij}
      \ip{ij||ij}
    }$
  \item
    $\ds{
      \ip{\F|H_e|\F_i^a}
    =
      h_{ia}
    +
      \sum_j
      \ip{ij||aj}
    }$
  \item
    $\ds{
      \ip{\F|H_e|\F_{ij}^{ab}}
    =
      \ip{ij||ab}
      \vphantom{\sum_j}
    }$
  \item
    $\ds{
      \ip{\F|H_e|\F_{ijk}^{abc}}
    =
      0
    }$
  \end{enumerate}
  For extra credit, show how to derive them without using particle-hole isomorphism.
\end{enumerate}

\end{document}