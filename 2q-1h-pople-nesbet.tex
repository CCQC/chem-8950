\documentclass[11pt]{article}
\usepackage[cm]{fullpage}
%%AVC PACKAGES
\usepackage{avcgreek}
\usepackage{avcfonts}
\usepackage{avcmath}
\usepackage[numberby=section]{avcthm} % 
\usepackage{qcmacros}
\usepackage{goldstone}
%%MACROS FOR THIS DOCUMENT
\numberwithin{equation}{section}
\usepackage[
  margin=1.5cm,
  includefoot,
  footskip=30pt,
  headsep=0.2cm,headheight=1.3cm
]{geometry}
\usepackage{fancyhdr}
\pagestyle{fancy}
\fancyhf{}
\fancyhead[LE,RO]{Quiz 2, Handout 1: Roothaan-Hall/Pople-Nesbet}
\fancyfoot[CE,CO]{\thepage}
\usepackage{url}

\begin{document}

\setlength{\abovedisplayskip}{3pt}
\setlength{\belowdisplayskip}{3pt}

\setcounter{section}{1}
\section{The Roothaan-Hall and Pople-Nesbet equations}

\subsection{Spin and spin-orbitals}


A spin-orbital $\y$ can be expressed as $\y(\bo{r},s)=\f(\bo{r})\w(s)$, where $\f(\bo{r})$ is a spatial orbital and $\w(s)$ is a spin function.
Spin functions live in a two-dimensional space spanned by $\{\a(s),\b(s)\}$, which are orthonormal eigenfunctions of one component of $\op{\bo{S}}$, the spin angular momentum operator.
\begin{align}
  \op{S}_z\a
=
  +\fr{1}{2}\a
&&
  \op{S}_z\b
=
  -\fr{1}{2}\b
&&
  \ip{\a|\a}=\ip{\b|\b}=1
&&
  \ip{\a|\b}=\ip{\b|\a}=0
\end{align}
The ``spin-coordinate'' $s$ is either $+$ or $-$, identifying the component of $\w(s)$ along $\a(s)$ and $\b(s)$.
You can think of it as the index of a coordinate vector in spin space
\begin{align*}
  \ma{\w(+)\\\w(-)}
=
  \ma{\ip{\a|\w}\\\ip{\b|\w}}
\end{align*}
so that $\a(+)=\b(-)=1$ and $\a(-)=\b(+)=0$.
Using this spin coordinate, the inner product in spin space can be defined explicitly as $\ip{\w|\w'}=\sum_s\w^*(s)\w'(s)$.
This is commonly referred to as a ``spin integration'', even though it looks more like a dot product.

A complete set of one-electron functions (spin-orbitals) comes in $\a,\b$-pairs.
\begin{align}\label{eq:spin-orb-general}
	\y_{2p-1}(\bo{r},s)
=
  \f_{p_\a}(\bo{r})\a(s)
&&
  \y_{2p}(\bo{r},s)
=
  \f_{p_\b}(\bo{r})\b(s)
\end{align}
The spatial components of these functions can be expanded in terms of a set of AO basis functions $\{\x_\nu\}$ as 
\begin{align}\label{eq:space-orb}
  \f_{p_\a}
=
  \sum_\mu \x_\mu C_{\mu p_\a}
&&
  \f_{p_\b}
=
  \sum_\mu \x_\mu C_{\mu p_\b}
\end{align}
where the AOs basis set consists of atom-centered Gaussian functions (cc-pVTZ, 6-31G, etc.).



\subsection{Spin-integration of the canonical Hartree-Fock equations}

For a system with $n_\a$ spin-up and $n_\b$ spin-down electrons, the spin-orbital canonical Hartree-Fock equation takes the following form.
\begin{align}\label{eq:canonical-hf}
&
  \op{f}\y_p
=
  \ev_p\y_p
&&
  \op{f}
=
  \op{h}
+
  \sum_{i_\a}^{n_\a}
  (\op{J}_{i_\a}-\op{K}_{i_\a})
+
  \sum_{i_\b}^{n_\b}
  (\op{J}_{i_\b}-\op{K}_{i_\b})
\end{align}
This equation can be expanded in the spin basis as
\begin{align*}
  \ma{
    \ip{\a|\op{f}|\a}&\ip{\a|\op{f}|\b}\\
    \ip{\b|\op{f}|\a}&\ip{\b|\op{f}|\b}}
  \ma{
    \ip{\a|\y_p}\\
    \ip{\b|\y_p}}
=
  \ev_p
  \ma{
    \ip{\a|\y_p}\\
    \ip{\b|\y_p}}
\end{align*}
where we are integrating over the spin coordinate.
The operators in $\op{f}$ then become
\begin{align*}
&
  \op{h}
\mapsto
  \ma{
    \ip{\a|\op{h}|\a}&0\\
    0&\ip{\b|\op{h}|\b}}
&&
\begin{array}{r@{\ }l@{\hspace{1cm}}r@{\ }l}
  \op{J}_{i_\a}
&\mapsto
  \ma{
    \ip{\a|\op{J}_{i_\a}|\a}&0\\
    0&\ip{\b|\op{J}_{i_\a}|\b}}
&
  \op{K}_{i_\a}
&\mapsto
  \ma{
    \ip{\a|\op{K}_{i_\a}|\a}&0\\
    0&\makebox[\widthof{\ip{\b|\op{K}_{i_\b}|\b}}]{0}}
\\[10pt]
  \op{J}_{i_\b}
&\mapsto
  \ma{
    \ip{\a|\op{J}_{i_\b}|\a}&0\\
    0&\ip{\b|\op{J}_{i_\b}|\b}}
&
  \op{K}_{i_\b}
&\mapsto
  \ma{
    \makebox[\widthof{\ip{\a|\op{K}_{i_\a}|\a}}]{0}&0\\
    0&\ip{\b|\op{K}_{i_\b}|\b}}
\end{array}
\end{align*}
where the core and Coulomb operators can be evaluated as $\ip{\w|\op{h}|\w'}=\op{h}\ip{\w|\w'}$, since they don't act on spin coordinates.
The exchange operator, however, does act on spin coordinates by its coordinate-swapping operation.
\begin{align*}
  \op{K}_{i_\a}(\bo{r})\a(s)
=&\
  \sum_{s'}
  \ip{\f_{i_\a}(\bo{r}')\a(s')|\op{g}(\bo{r},\bo{r}')|\cdot(\bo{r}')\a(s')}\,
  \f_{i_\b}(\bo{r})\a(s)
=
  \ip{\f_{i_\a}(\bo{r}')|\op{g}(\bo{r},\bo{r}')|\cdot(\bo{r}')}\,
  \f_{i_\b}(\bo{r})\a(s)
\\
  \op{K}_{i_\b}(\bo{r})\a(s)
=&\
  \sum_{s'}
  \ip{\f_{i_\b}(\bo{r}')\b(s')|\op{g}(\bo{r},\bo{r}')|\cdot(\bo{r}')\a(s')}\,
  \f_{i_\b}(\bo{r})\b(s)
=
  0
\end{align*}
Here, the $\cdot$ means ``fill in spatial function here'' and the matrix elements are integrated over $\bo{r}'$.
This shows why $\ip{\a|\op{K}_{i_\b}|\a}=\ip{\b|\op{K}_{i_\b}|\a}=\ip{\b|\op{K}_{i_\a}|\a}=0$ and $\ip{\a|\op{K}_{i_\a}|\a}\neq0$.
The remaining components can be derived in the same way.

Since all of the off-diagonal elements in the spin basis vanish, we can separate \cref{eq:canonical-hf} into two equations.
\begin{align}\label{eq:canonical-uhf-equation-alpha}
&
  \op{f}_\a \f_{p_\a}
=
  \ev_{p_\a} \f_{p_\a}
&&
  \op{f}_\a
=
  \op{h}
+
  \sum_{i_\a}^{n_\a}
  (\op{J}_{i_\a} - \op{K}_{i_\a})
+
  \sum_{i_\b}^{n_\b}
  \op{J}_{i_\b}
\\\label{eq:canonical-uhf-equation-beta}
&
  \op{f}_\b \f_{p_\b}
=
  \ev_{p_\b} \f_{p_\b}
&&
  \op{f}_\b
=
  \op{h}
+
  \sum_{i_\a}^{n_\a}
  \op{J}_{i_\a}
+
  \sum_{i_\b}^{n_\b}
  (\op{J}_{i_\b} - \op{K}_{i_\b})
\end{align}




\subsection{RHF: The Roothaan-Hall Equations}

Assuming a closed-shell system with $n_\a=n_\b=n/2$, we can impose the restriction that $\f_{i_\a}=\f_{i_\b}$ for each pair of electrons.
Then equations \ref{eq:canonical-uhf-equation-alpha} and \ref{eq:canonical-uhf-equation-beta} collapse into a single expression.
\begin{align*}
  \op{f}_\textsc{r}\f_p
=
  \ev_p\f_p
&&
  \op{f}_\textsc{r}
=
  \op{h}
+
  \sum_i^{n/2}
  (2\op{J}_i - \op{K}_i)
\end{align*}
where the R stands for ``restricted''.
Expanding $\f_p$ in the AO basis and projecting by $\x_\mu$, we get
\begin{align*}
  \sum_\nu
  \ip{\x_\mu|\op{f}_\textsc{r}|\x_\nu}
  C_{\nu p}
=
  \sum_\nu
  \ip{\x_\mu|\x_\nu}C_{\mu p}\ev_p
\end{align*}
which are the \textit{Roothaan-Hall equations}.
In matrix notation, these can be written as follows.
\begin{align}\label{eq:roothaan-hall}
&
  \bo{F}\bo{C}
=
  \bo{S}\bo{C}\bm{\ev}
&&
  F_{\mu\nu}
=
  \ip{\x_\mu|\op{f}_\textsc{r}|\x_\nu}
&&
  S_{\mu\nu}
=
  \ip{\x_\mu|\x_\nu}
&&
  (\bm{\ev})_{pq}
=
  \ev_p\d_{pq}
\end{align}
Note that if the AO basis contains $m$ functions, then $\bo{C}=[C_{\mu p}]$ is an $m\times m$ matrix with each column vector containing the expansion coefficients for an orbital $\f_p$.
Also, note that only the $n/2$ MOs of lowest energy ($\ev_p$) will be ``occupied'' -- the remaining virtual orbitals will not enter into the Coulomb and exchange parts of $\op{f}_\textsc{R}$.
Expanding $\op{f}_\textsc{R}$ in its core, Coulomb, and exchange parts, we find
\begin{align*}
  F_{\mu\nu}
=&\
  \ip{\x_\mu|\op{h}|\x_\nu}
+
  \sum_i^{n/2}
  (2\ip{\x_\mu\f_i|\x_\nu\f_i}-\ip{\x_\mu\f_i|\f_i\x_\nu})
\\
=&\
  \ip{\x_\mu|\op{h}|\x_\nu}
+
  \sum_i^{n/2}
  \sum_{\rho\si}^m
  C_{\rho i}^*C_{\si i}
  (2\ip{\x_\mu\x_\rho|\x_\nu\x_\si}-\ip{\x_\mu\x_\rho|\x_\si\x_\nu})
\end{align*}
which is conveniently given in terms of a ``density matrix'' $D_{\mu\nu}$.
\begin{align}\label{eq:rhf-ao-basis-fock}
  F_{\mu\nu}
=&\
  \ip{\x_\mu|\op{h}|\x_\nu}
+
  \sum_{\rho\si}^m
  D_{\rho\si}
  (2\ip{\x_\mu\x_\rho|\x_\nu\x_\si}-\ip{\x_\mu\x_\rho|\x_\si\x_\nu})
&&
  D_{\mu\nu}
=
  \sum_i^{n/2}
  C_{\mu i}^*C_{\nu i}
\end{align}




\subsubsection{Solving the Roothaan-Hall Equations}

Equation \ref{eq:roothaan-hall} would look like an ordinary eigenvalue problem if $\bo{S}$ were an identity matrix.
This would be the case if the AO basis were orthogonal, but that generally isn't the case.
We can get around this problem using the following algebraic trick:
if we multiply both sides of equation \ref{eq:roothaan-hall} by $\bo{S}^{-\frac{1}{2}}$ and insert $\bo{I}=\bo{S}^{-\frac{1}{2}}\bo{S}^{\frac{1}{2}}$ between $\bo{F}$ and $\bo{C}$, we can write
\begin{align}
\label{orthogonalized-roothaan-hall}
&
  \tl{\bo{F}}
  \tl{\bo{C}}
=
  \tl{\bo{C}}\bm\ev
&&
  \tl{\bo{F}}
=
  \bo{S}^{-\frac{1}{2}}\bo{F}\bo{S}^{-\frac{1}{2}}
&&
  \tl{\bo{C}}
=
  \bo{S}^{\frac{1}{2}}\bo{C}\ .
\end{align}
This is equivalent to a transformation to an orthogonalized AO basis $\tl\x_\mu=\sum_\nu\x_\nu (\bo{S}^{-\frac{1}{2}})_{\nu\mu}$ which satisfies $\ip{\tl\x_\mu|\tl\x_\nu}=\d_{\mu\nu}$.
After diagonalizing the transformed Fock matrix, $\tl{\bo{F}}$, the MO coefficients with respect to the original non-orthogonal AO basis can be recovered as $\bo{C}=\bo{S}^{-\frac{1}{2}}\tl{\bo{C}}$.

\begin{samepage}
Equation \ref{orthogonalized-roothaan-hall} is still not exactly an ordinary eigenvalue problem because $\bo{F}$ depends on the MOs via the density matrix $\bo{D}$.
The standard procedure for solving the Roothan-Hall equations repeatedly diagonalizes $\tl{\bo{F}}$ and feeds in the new density until the equation becomes self-consistent.
The algorithm looks as follows.
\begin{enumerate}
  \item Get integrals $\ip{\x_\mu|\x_\nu}, \ip{\x_\mu|\op{h}|\x_\nu}, \ip{\x_\mu\x_\nu|\x_\rho\x_\si}$ and form orthogonalizer $\bo{S}^{-\frac{1}{2}}$
  \item Guess $\bo{D}=\bo{0}$
  \item\label{loop} Build $\bo{F}$ (equation \ref{eq:rhf-ao-basis-fock})
  \item Diagonalize $\tl{\bo{F}}=\bo{S}^{-\frac{1}{2}}\bo{F}\bo{S}^{-\frac{1}{2}}$ to get $\tl{\bo{C}}$ and $\bm\ev$
  \item Backtransform to original basis: $\bo{C}=\bo{S}^{-\frac{1}{2}}\tl{\bo{C}}$
  \item Form new density matrix: $D_{\mu\nu}=\sum_i^{n/2} C_{\mu i}^*C_{\nu i}$
  \item If the new $\bo{D}$ differs from the old $\bo{D}$ by more than some threshold, return to step \ref{loop}
\end{enumerate}
\end{samepage}


\subsection{UHF: The Pople-Nesbet Equations}

For open-shell systems of arbitrary $n_\a, n_\b$, we can solve equations \ref{eq:canonical-uhf-equation-alpha} and \ref{eq:canonical-uhf-equation-beta} without requiring $\f_{i_\a}=\f_{i_\b}$.
By exactly the same procedure as was used for the Roothaan-Hall equations, this leads to \textit{Pople-Nesbet equations}
\begin{align*}
&
  \bo{F}^\a\bo{C}^\a
=
  \bo{S}\bo{C}^\a\bm\ev^\a
&&
  F_{\mu\nu}^\a
=
  \ip{\x_\mu|\op{f}_\a|\x_\nu}
&&
  S_{\mu\nu}
=
  \ip{\x_\mu|\x_\nu}
&&
  (\bm\ev^\a)_{p_\a q_\a}
=
  \ev_{p_\a}\d_{p_\a q_\a}
\\
&
  \bo{F}^\b\bo{C}^\b
=
  \bo{S}\bo{C}^\b\bm\ev^\b
&&
  F_{\mu\nu}^\b
=
  \ip{\x_\mu|\op{f}_\b|\x_\nu}
&&
  S_{\mu\nu}
=
  \ip{\x_\mu|\x_\nu}
&&
  (\bm\ev^\b)_{p_\b q_\b}
=
  \ev_{p_\b}\d_{p_\b q_\b}
\end{align*}
where $\bo{C}^\a=[C_{\mu p_\a}]$ is an $m\times m$ matrix of MO coefficients for $\{\f_{p_\a}\}$ and $\bo{C}^\b=[C_{\mu p_\b}]$ is an $m\times m$ matrix of MO coefficients for $\{\f_{p_\b}\}$.
Expanding the $\a$ and $\b$ Fock matrices as in equation \ref{eq:rhf-ao-basis-fock}, we find
\begin{align*}
&
  F_{\mu\nu}^\a
=
  \ip{\x_\mu|\op{h}|\x_\nu}
+
  \sum_{\rho\si}
  D_{\rho\si}^\a
  \ip{\x_\mu\x_\rho||\x_\nu\x_\si}
+
  \sum_{\rho\si}
  D_{\rho\si}^\b
  \ip{\x_\mu\x_\rho|\x_\nu\x_\si}
&&
  D_{\mu\nu}^\a
=
  \sum_{i_\a}^{n_\a} C_{\mu i_\a}^*C_{\nu i_\a}
\\
&
  F_{\mu\nu}^\b
=
  \ip{\x_\mu|\op{h}|\x_\nu}
+
  \sum_{\rho\si}
  D_{\rho\si}^\a
  \ip{\x_\mu\x_\rho|\x_\nu\x_\si}
+
  \sum_{\rho\si}
  D_{\rho\si}^\b
  \ip{\x_\mu\x_\rho||\x_\nu\x_\si}
&&
  D_{\mu\nu}^\b
=
  \sum_{i_\b}^{n_\b} C_{\mu i_\b}^*C_{\nu i_\b}
\end{align*}
where $\bo{D}^\a$ and $\bo{D}^\b$ are the $\a$ and $\b$ density matrices.
The procedure for solving the Pople-Nesbet equations is identical to the one given for RHF.
However, note that one must solve the $\a$ and $\b$ equations simultaneously because each Fock operator depends on both $\bo{C}^\a$ and $\bo{C}^\b$ (via $\bo{D}^\a$ and $\bo{D}^\b$).



\end{document}